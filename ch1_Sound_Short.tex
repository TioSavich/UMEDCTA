\chapter{The Sound of Time}

\begin{abstract}
This chapter introduces the concept of \textit{determinate negation} through the metaphor of \textit{the sound of time}, connecting embodied experience to the rhythm of compression and expansion that underlies self-consciousness. The core of this exploration is a guided meditation called \textit{The Exercise}, which cultivates an awareness of \textit{self-certainty} through proprioceptive awareness---the felt sense of bodily presence that subtends all conceptual thought. The Exercise is used throughout the text to provide readers with direct access to the experiences being theorized. The chapter explores the interplay between direct, non-conceptual experience and philosophical reflection, arguing that self-certainty cannot be reached through thought alone but requires a kind of surrender to the immediacy of embodied being. Drawing on phenomenology and Hegelian philosophy, the analysis utilizes polarized modal logic to articulate the rhythm of embodied experience---the oscillation between expansion and contraction, presence and absence. The chapter also examines the limitations of representational thinking and the role of artistic expression in articulating truths that exceed conceptual grasp. By connecting the individual experience of the Exercise to broader structures of intersubjective understanding, this chapter grounds the project of critical mathematics in embodied practice.
\end{abstract}

\section{Introduction: The Misrecognition of Certainty}
In the prelude, a type of negation was alluded to that is distinct from the abstract negation of formal logic. In this chapter, the concept of \textit{determinate negation} will be elaborated through a metaphor called \textit{the sound of time}. This metaphor connects the rhythm of inner, felt experience to the physical nature of sound. The core of this exploration is an embodied practice: \textit{The Exercise}.

Before abstract philosophy, before symbolic logic, there is the body. The Exercise attempts to cultivate direct awareness of how consciousness moves through the fundamental relationship between space and time---not as metaphors, but as the primordial structure of experience itself. To approach this phenomenologically requires first acknowledging how abstraction becomes meaningful at all.

\subsection*{A Starting Point: Conceptual Metaphor and Embodied Cognition}

The work of cognitive linguists George Lakoff and Mark Johnson provides a helpful entry point \parencite*{lakoff_metaphors_1980}. Their research argues that our conceptual system is grounded in recurring patterns of sensory-motor experience---a perspective known as \textit{embodied cognition}. This framework demystifies abstraction by revealing how even sophisticated concepts arise from bodily experience. Their central thesis is that \textbf{conceptual metaphor} is not mere decoration but a fundamental cognitive mechanism structuring our understanding. Metaphor works by ``understanding and experiencing one kind of thing in terms of another,'' allowing us to reason about abstract domains by importing the inferential structure of more concrete, familiar domains grounded in physical interaction.

Consider the conceptual metaphor AN ARGUMENT IS A JOURNEY \parencite[90-91]{lakoff_metaphors_1980}. It establishes systematic mappings: the arguer is a \emph{traveler}, the argument's goal is the \emph{destination}, the means of argumentation are \emph{paths}, and logical progression defines \emph{steps along the way}. This metaphorical structure gives rise to everyday expressions like ``We have set out to prove that...,'' ``When we get to the next point...,'' ``So far, we've seen that...,'' and ``We have arrived at a disturbing conclusion.'' The metaphor has internal systematicity through entailments: since A JOURNEY DEFINES A PATH, then AN ARGUMENT DEFINES A PATH (``He strayed from the line of argument,'' ``Do you follow my argument?''). Since THE PATH OF A JOURNEY IS A SURFACE, then THE PATH OF AN ARGUMENT IS A SURFACE (``We have covered a lot of ground,'' ``Let's go back over the argument again''). Similarly, THEORIES ARE BUILDINGS maps physical structures onto argumentation. A theory's assumptions are its \emph{foundations}, its logical structure is its \emph{frame}, its strength is its \emph{structural integrity}.

Lakoff and Nuñez extend this framework to mathematics itself, demonstrating that basic arithmetic is structured by metaphors based on collecting objects, constructing things from parts, using measuring sticks, and moving along paths. For instance, the most fundamental grounding metaphor treats numbers as object collections:

\begin{table}[H]
\centering
\title{\textit{Arithmetic as Object Collection}}
\begin{tabular}{p{0.45\textwidth} p{0.45\textwidth}}
\hline
\textbf{Source Domain: Object Collections} & \textbf{Target Domain: Natural Numbers} \\
\hline
Collections of objects of the same size & Numbers \\
The size of a collection & The value of a number \\
An individual object & A unit (the basis for counting) \\
The smallest possible collection (one object) & The number 1 \\
The empty collection & The number 0 \\
Putting two collections together & Addition \\
Taking a smaller collection from a larger one & Subtraction \\
A larger/smaller collection & A greater/lesser number \\
\hline
\end{tabular}
\caption{\textit{Note.} The foundational metaphor grounding natural number concepts in the bodily experience of collecting discrete objects. Adapted from Lakoff \& Nuñez \parencite*[55]{Lakoff2000}.}
\label{tab:object_collection_metaphor}
\end{table}

This cognitive science perspective dismantles what Lakoff and Nuñez call the ``Romance of Mathematics''---the belief that mathematics is transcendent, disembodied truth existing independently of human minds \parencite*[339]{Lakoff2000}. Mathematics is not a disembodied language we discover; it is a magnificent conceptual system human beings \emph{create}, making ``extraordinary use of the ordinary tools of human cognition.''\parencite*[377]{Lakoff2000}

\subsection*{Going Deeper: From Cognitive Science to Phenomenology}

This embodied cognition framework provides an accessible starting point, but The Exercise attempts to go deeper. Lakoff and Johnson, as cognitive scientists, articulate embodiment in terms of the mind as it exists in the brain---neural mappings, sensory-motor schemas, cognitive mechanisms. Their account remains within what might be called the natural attitude, treating consciousness as an object of scientific investigation.

The Exercise, however, seeks a phenomenological investigation of the subject-object relation itself. Rather than explaining how the mind constructs metaphors, I am attempting to cultivate direct awareness of the movement between differentiation and undifferentiation, between the bounded and the unbounded, between contraction and expansion. This is not metaphorical language imported from spatial experience to understand something else. This \textit{is} the primordial rhythm of spatial and temporal determination itself---what Hegel calls \textit{absolute negativity}.

Where cognitive science describes metaphorical mappings, phenomenology attempts to bracket those descriptions and return to the experience from which they arise. The goal is not to understand how we \emph{think about} space and time through bodily metaphors, but to experience directly how consciousness \emph{is} the movement of spatial and temporal determination. The Exercise cultivates awareness of this movement not as concept but as the pre-conceptual ground from which concepts emerge.

Lakoff and Nuñez articulate how embodied experience grounds abstract thought. The Exercise attempts to reverse that movement---to de-metaphorize, to dissolve the abstractions back into the embodied ground, and then to recognize that ground itself as the self-negating movement of consciousness. This is why the phenomenology of self-certainty cannot be captured by cognitive science. Self-certainty is not a brain state or a cognitive mechanism. It is the implicit awareness that subtends and enables all such objectifications.

Still, the cognitive science framework provides valuable scaffolding. Our shared embodiment does provide common source domains making intersubjective meaning possible. We can agree on validity claims or inference correctness precisely because abstract concepts involved are grounded in shared physical experience. This establishes the bridge between the biological and the social, between individual cognition and collective normativity. But that bridge must ultimately be crossed phenomenologically---through direct cultivation of the awareness The Exercise articulates.

\subsection*{Proprioception and the Feeling Body}

The Exercise guides you into direct experience of this embodied foundation. It explores \textit{self-certainty} through \textit{proprioception}---bodily awareness. Proprioception is a sense, like smell or sight, except it concerns not the `outside world' but bodily awareness itself. Common proprioceptive knowledge includes the implicit awareness that your head is above your shoulders. Making that explicit requires communicative norms; you must know words like ``above, head, shoulders'' to articulate these subjective/objective regions.

Less common ways of talking about proprioception include \textit{proprioceptive expansion} ($\uparrow$) or \textit{contraction} ($\downarrow$). Consider an experience of feeling `one with nature,' gazing at a beautiful valley. You may recall such moments of \textit{unity} followed by contraction in bodily awareness ($\uparrow \rightarrow \downarrow$): ``ah! Bee!'' Alternatively, movement from point-like to diffuse awareness is proprioceptive expansion ($\downarrow \rightarrow \uparrow$): ``oh, just a fly, mmm...'' This rhythm of expansion and contraction is what I call \textit{the sound of time}.

The desire for self-certainty is a deep, primordial drive---the desire for unmediated experience of the self, a direct, irrefutable experience settling, once and for all, that ``I am.'' Descartes' formula ``I think therefore I am'' has an unnecessary antecedent; saying ``I am'' is unfalsifiable, proved by the utterance. To say otherwise---``I ain't''---is self-defeating, disproved by its utterance. Still, simply saying ``I am'' has never satisfied my desire for self-certainty. Grammatically unfalsifiable statements can fuse with the I-feeling, but they are not self-certainty itself.

The Exercise is a guided meditation whose form I adumbrate from Phil Carspecken \parencite[169-184]{Carspecken1999}. Readers frustrated by the somewhat cartoonish nature of what follows should consult his work for more thorough treatment. The cartoonishness is intentional (I wish to make understanding accessible) and unintentional (I am not a trained meditation teacher). I will do my best not to lead you astray, but encourage you to seek more experienced guides if you wish to pursue this practice more deeply.

You do not need to master concepts before beginning. You already implicitly know how to do what The Exercise makes explicit. My expressions are framed as distractions from the purpose, even as mistakes, for two reasons. First, none of what follows the experience of self-certainty \emph{is} self-certainty; self-certainty is necessarily implicit. Second, it is methodologically useful for the critical ethnographer to understand how self-certainty is recollected in different traditions. Engage with the meditation. The reflections plant seeds that grow throughout the book. Move between feeling and thinking without judgment. They are two sides of a single process of coming to understand.

\subsection*{The Rhythm of Determinate Negation}

When I say ``no'' to something, what am I doing? In formal logic, negation is simple: ``not-$p$'' means everything that $p$ excludes. If I say ``This shape is not-circular,'' I have told you almost nothing---the shape could be square, triangular, or any of infinitely many possibilities. This \textit{abstract negation} creates an empty void, a sheer absence.

But there is another kind of negation. When you determinately negate something, you do not simply erase it. You negate it \textit{by replacing it with something specific}. If I say ``This shape is square,'' I am implicitly negating circular, triangular, and every other incompatible shape---not abstractly, but by asserting something with its own positive content. Square and triangular are \textit{materially incompatible}. You cannot have both simultaneously. One excludes the other not by formal contradiction, but by virtue of what they mean.

Spinoza declared \textit{omnis determinatio est negatio}---all determination is negation. To be one thing is to \textit{not be} its incompatible others. For Hegel, determinate negation is \textit{prior} to formal negation. You cannot understand what ``not-red'' means unless you already understand what red's contraries are: green, blue, yellow \parencite{Brandom:2019aa}.

The Exercise is an embodied enactment of this principle. When you bring awareness to your toes, name them (``I am my toes''), and then let them go (\cancel{toes}), you are not performing abstract negation. You are not erasing your toes into an empty void. You are \textit{sublating} them---preserving, negating, and uplifting them simultaneously.

The toes are \textbf{preserved}: still there, still part of your body, still contributing to your proprioceptive wholeness. They are \textbf{negated}: you are no longer fixated on them; you have released your attention. And they are \textbf{uplifted}: now part of an accumulating totality, a growing awareness of the body as a unified whole.

Moreover, the very act of naming your toes as ``I am my toes'' negates itself. The name is inadequate. You are \textit{more} than your toes. The assertion negates itself in its very utterance, pushing you forward to the next region. I am my toes. No, I am more than my toes. I am my feet. No, I am more than my feet. This rhythm---fixation and release, naming and sublating---is the sound of time, the pulse of determinate negation in lived experience.

If you only understood abstract negation, you might think the goal of meditation is to achieve an empty void, a sheer nothingness. But this exercise aims for a ``determinate nothingness,'' a nothingness that retains the content of what it came from. When you let go of your toes, you do not forget them. They accumulate as background, as \textit{forestructure}, as the implicit context for what comes next.

Figure \ref{fig:yoga_of_dialectic_reasoning} illustrates how this dialectical process maps onto the bodily rhythm of tension and release. I recently led a class examining social justice case studies. When I encouraged students to push back on each other's assumptions, they mistook me as asking them to ``play the devil's advocate.'' But understanding isn't the devil's work. Through pushing, I hoped they would separate from themselves to analyze their assumptions, then reflect and return to themselves with deeper understanding. The point wasn't to change their conclusions but to understand themselves and others more deeply. I drew this cartoon to repair our misunderstandings.

The process begins with \textbf{Tension and Inhalation}---the first negation (`No'). This is the focused effort required to `Push' and Determine a concept, followed by the sustained effort to `Separate' and Analyze its assumptions. We hold our breath, metaphorically, as we fixate and differentiate.

The turning point arrives with \textbf{Relaxation and Exhalation}---the second negation (\cancel{No}). It begins as we `Reflect,' releasing the fixation by adopting a receptive, second-person perspective. This letting go allows us to `Grow' and Sublate the analysis, integrating understanding into more expansive comprehension. The entire `full-circle' movement is felt as a single wave of compression and decompression.

\begin{figure}[h]
 \centering
 \title{The `Yoga' of Dialectic Reasoning}
 \includegraphics[width=0.8\textwidth]{/Users/tio/Documents/GitHub/September_UMEDCA/images/dialectic_yoga_poses.pdf}
 \caption{\textit{Note.} The body feels the rhythms of determinate negation—the self-determining concept. The body-feelings of tension and relaxation that accompany determinate negation are important for understanding dialectical reasoning. The poses are metaphorical cartoons, designed to communicate complex Hegelian concepts for young readers.}
 \label{fig:yoga_of_dialectic_reasoning}
\end{figure}

Pay attention to this rhythm as you move through The Exercise. Notice how each act of naming is also an act of exclusion. Notice how each act of letting-go is not erasure but transformation. Notice how the self-negating movement of awareness allows you to expand from toes to feet to legs to the whole body. This is determinate negation in action. This is the dialectical soul of experience. This is how thought \textit{moves}.




\subsection*{The Semantics of Self-Certainty}
Self-certainty is, for me, not about the strength of one's convictions or \textit{commitments}. The desire for self-certainty is a deep, primordial drive. It is the desire for an unmediated experience of the self; a direct, irrefutable experience that would settle, for once and for all, that ``I am.'' Descartes' meditations that resulted in the formula ``I think therefore I am'' has an unnecessary antecedent, as saying ``I am'' is unfalsifiable. It is proved by the utterance. To say otherwise, ``I ain't,'' for example, is an example of a self-defeating assertion: it is disproved by its utterance. Still, simply saying ``I am'' has not, for me, ever resulted in the satisfaction of the desire for self-certainty. Grammatically unfalsifiable statements can be fused with the I-feeling, but they are not self-certainty itself.

Self-certainty emerges from the practice of embodied reason below as a negative; it is always a not-this to whatever 'this' attempts to define it. Determinate negation is named, or \textit{reified} ($\ulcorner \text{determinate negation} \urcorner$) as \textit{the negative}. I use the Quine corners ($\ulcorner \text{\,} \urcorner$) to indicate the name of the concept thus enclosed \parencite{Gaifman2005}. The negative took a philosophical journey from Spinoza to Hegel, where Spinoza said, ``omnis determinatio est negatio'' (all determination is negation) and then Hegel worked to demonstrate how determinate negation determines itself. 

The first negation constrains, bounds, or limits in its finite moment. The second negation (\cancel{no}) dissolves those exact boundaries \textit{exactly}. The more determined the concept is, the more precise are the dissolutions of those determinations. In everyday speech practices, our concepts are usually either over- or under-determined. That means that our experiences of the second negation are often experienced as an imprecise loosening. The Exercise articulates a space where the self-similarity between the no and the \cancel{no} is their absolute difference. By virtue of how the negative is necessarily other than itself, it is a concept that both produces and undermines itself. It builds and it breaks.


From Hegel, the negative continued its journey through existentialism and phenomenology, and was picked up again by the post-structuralist Jacques Derrida. Derrida's work is challenging to parse, so I will select one interesting idea that Carspecken \parencite*{Carspecken1999} develops more fully than I can here. The phrase is that there is a concept, specifically a norm or rule in the system that I am developing, that erases its own name. The negative is one name for that concept, captured when \textit{sous rature} is interpreted as a necessary expression followed by a necessary erasure (\cancel{no}). When I wrote that self-certainty is necessarily implicit, I was drawing on Derrida's concept. Self-certainty is a \cancel{$\ulcorner \text{concept} \urcorner$}, as is the negative. 

The idea of a concept that erases its own name is a challenging one, especially since I will not be deriving it from Derrida's work! Readers will have to do the exercise to recognize that every attempt to pin down self-certainty as a concept is necessarily a misstep. As you practice, approach this as an ethnographer-in-training. You will encounter groups of people with different ways of talking about their convictions. Perhaps they believe in the Trinity, a God who stands outside of creation, the Sutras, or the Four Noble Truths. Perhaps you worship at the altar of Rock and Roll, taking Bruce Springsteen as your saint. What unites these diverse beliefs and practices must be implicit. You \textit{can} understand their beliefs and practices as oriented around the intersubjective practices that cultivate self-certainty. But if you try to pin their beliefs down to $\ulcorner \text{self-certainty} \urcorner$, and fit them into a logical system that captures that concept, the resulting texts you produce will likely result in the subjects of your analysis rejecting your interpretations: ``I don't care about self-certainty, I care about avoiding damnation!'' 

I do not intend to suggest that self-certainty, as I explore it below, is devoid of logical structure. The `grooves' or `folds' in intersubjective space through which I move when conducting simple calculations like $5+7=5+5+2=12$ are actions oriented around self-certainty. There, self-certainty metaphorically functions like a hole, singularity, or gravitational well in representational space. I will introduce a polarized modal logic in \ref{ch:algorithmic-elaboration-and-history} that formalizes the text below in an attempt to make those troubled metaphorical expressions less transient. But none of that formalization and none of the text below \textit{is} self-certainty. Cultivating an openness towards otherness requires an emptying that might read like solipsism or nihilism, but the intersubjective vessel is never seriously in doubt. 



\subsection*{Readers' Guide}
This chapter is structured as a guided journey. Before beginning, a note on how to read it. The text moves between two modes: the \textit{experience} of a guided meditation and the \textit{reconstruction} of that experience through philosophical reflection and logic. You do not need to master all the concepts on the first pass. You definitely do not need to master the exercises before moving on. Wild as it may seem, I suspect you already have mastered the exercises; my role is to express what you already implicitly know how to do. Those expressions are all framed as distractions from the purpose of the exercise. I frame my work as a mistake, specifically a distraction, for two reasons. First, none of what follows the experience of self-certainty \textit{is} self-certainty. Self-certainty is necessarily implicit. Second, it is methodologically useful for the critical ethnographer to understand how self-certainty is recollected in different traditions. The goal is to engage with the meditation. The reflections are there to plant seeds that will grow throughout the book. Allow yourself to move between the feeling of the exercise and the thinking of the reflections without judgment. One is not a failure of the other; they are two sides of a single process of coming to understand.


\section{The Exercise: Negating Fixation through Practice}
\subsection*{Exercise 0: Dragging Yourself to The Exercise}
Are you ready to practice? \textit{I don't want to practice!} Come on, you know you'll feel better once you start...\textit{NO!} Are you saying ``no'' to the practice or to the claim that you'll feel better? \textit{NO!} Hmmm... Without knowing what you are resisting, I do not have much to say. I take it that you and I are, at our core, no-sayers. ``NO!'' on its own is not \textit{determinate}. But I assume you are resisting the practice itself, not my claim about the benefits of practice. Let me give you a secret for how I change my mind without changing what I am. It's a way to stay true to yourself (what you are) while still allowing you to change how you are (how you feel) in the world. I take that first ``NO!''---the one that says I will not practice, the one that says, ``I will not be moved.''---and I direct a second ``NO!'' towards \textit{it}. This second ``no,'' this determinate negation \textit{of} determinate negation (\cancel{no}), does not necessarily mean ``yes'' to the practice. You are free to close the book or skip whatever sections you wish. I don't want to change what I am or what you are to change us into yes-sayers. That is dangerous. There are so many excellent reasons to be disagreeable. I don't want you to change what you are, you delightful contrarian. The second negation functions as a transformation. The initial ``no'' is honored through your acknowledgement of its material content. Through that recognition, the resistance is softened. As a negation of the negation, recognition releases the initial blockage. 



\subsection*{Exercise 1: Let-Go of Particularity}

\textit{Begin by settling into a comfortable position, either laying down or sitting in a dignified manner. Whether your eyes are open or shut, soften your gaze. You may not realize the tension in your forehead that pulls your lower eyelids back towards your ears---among a thousand other regions in the face---until you attend to those regions as you soften your gaze.} 

\textit{Turn that attention to your toes and notice how they feel. Are they tight and curled? Extended and rigid? Neutral? In pain? Take an easy in-breath and imagine the breath traveling down into your lungs, nerves, and arteries, into the capillaries in your toes. You do not have to try to relax, just attend to them and breathe. Recognize them. Before you began The Exercise, they were likely an implicit aspect of your being, but they weren't unfamiliar. You may feel this recognition as a re-sensitization.}

\textit{As soon as you recognize them, you may find that they uncurl or soften into a neutral position without effort or intention, perhaps warming up as the breath travels through them. You may experience a sense of proprioceptive expansion---a sensation that feels good. As soon as you become aware of that sensation, you may want more of it. If you try to hold onto---to fix---the sensation, you may find that it slips away. The mind moves; you must find a way to move with it.}

\subsection*{Reflection: The First Temptation}

What you have just experienced in your toes reveals the fundamental structure of all conceptual thinking. As you continue the exercise, your thinking mind may reassert itself. This is not a failure; it is an opportunity to observe how concepts emerge from raw sensation. 

\textit{You might think: ``I am my toes.'' No, that's not quite right. I am more than just my toes. But I am still my toes somehow, just not in the same way that I felt when I was aware only of my toes when they just relaxed...hmmm...I let go of my toes.''}

Initially, it's very tempting to try to grasp the experience by naming it: ``I am my $\ulcorner \text{toes}\urcorner$.'' Because the words on the page already name concepts, I will use corners to indicate the act of naming. However, once stated, you might realize very quickly that the name is inadequate. It may happen so quickly that it feels silly to bring it into explicitness, but I want to slow things down. You are not \textit{just} your toes, so I cross out the name: I am my $\ulcorner \cancel{toes} \urcorner$. This naming and crossing out is a way to describe what happens when you bring awareness to your toes and then let them go. 

To understand why it's important to slow this first movement down, let me name what this first part of the exercise is doing. It is a \textit{somatic sublation}. Somatics relate to the body, while \textit{sublation} is Hegel's term that means conceptual content has been preserved, negated, and uplifted. I find that tripartite definition contradictory in three ways: preservation opposes negation, and both oppose uplifting. So, Hegel's concept is very challenging to understand. But in this first part of the exercise, you did not chop off your toes when you `let them go.' They were preserved. They were also negated, in the sense that their name was found to be inadequate to contain your identity. I have not yet described what the uplifting part means, but will get there soon. Hegel was not particularly interested in implicit embodied sensations as such. For him, sublation is always in the normative, conceptual domain. He would probably vigorously disagree that what is happening with your toes is a `sublation,' but I think it is pedagogically useful to start with the toes. 

This move from pure feeling to a named---and then sublated---concept is the first step out of subjective immediacy and into the world of normative claims. It is a glimpse of what Carspecken calls the ``primordial rhythm'' of \enquote{sensation inclusive of `I-feeling,' awareness of this, letting go, repeated awareness of sensation as inclusive of `I-feeling,' and so on} \parencite[171]{Carspecken1999}. This rhythm, as I will go on to explain, is what I mean by ``the sound of time.''


\textit{You start again. This time, you decide to follow the spine of Ram Dass' book: Be Here Now. You want to be present for the practice, but this resolution and desire are already separated from the sensation. You are kicked out again.}

This common experience points to a deeper philosophical problem. I grew up with Ram Dass' book on my parents' shelf, and I always struggled with the imperative. You tried to grasp the pure, immediate present, and in doing so, it vanished. It's tempting to think of words like ``this, here, and now'' as fixed, as if by pinning ourselves to a coordinate in space and time, we could find some certainty. 

But the moment you try to name it---``it is 11:45 AM''---that ``now'' has already become a ``then.'' The act of observation and labeling is always a step behind the reality it seeks to capture. This analysis of the universal nature of \textit{indexicals} (words like ``here'' and ``now'') and the self-defeating attempt to capture the particular through the universals of language is the core argument of Hegel's chapter on ``Sense-Certainty'' \parencite[\S 90-110]{hegel1977phenomenology}. The same slippage occurs with ``here.'' It feels specific and solid, but if you take a step, the old ``here'' is now ``there,'' and you are in a new ``here.'' These words don't function like pins on a map; they are more like universal placeholders. Hegel defines the universal \{now\} as not-now. When trying to capture a unique, particular experience with a general term, the uniqueness you seek to express slips through your fingers.

Brandom argues that that the \enquote{authority of any immediate sensory episode depends on its being situated in a larger relational structure containing elements that are not immediate in the same sense,} such as \textit{anaphoric structures} that allow recollection \parencite*[149]{Brandom:2019aa}. In the analytic tradition, \textit{anaphora} is basically pronoun use. The referent of an indexical is fixed by a subsequent anaphoric term like ``it.'' So, when I say ``now is day, it is hot, and it is sunny'' the chain of ``now, it, it'' fixes the prior referent even as the particular now marches ever on \parencite[150]{Brandom:2019aa}. Deictic expressions like ``this book'' fall to the same problem. The ``this'' is not a fixed object but a universal that can apply to any thing. The attempt to express the purely particular inevitably results in expressing a universal.

So if these words are such poor tools for capturing a fleeting moment, what are they truly for?

Perhaps ``now'' is less like a label for a point on a timeline, and more like a trigger for action: an \textit{impetus to act}. When you tell someone, ``I'm coming home now,'' you are not just describing your temporal location; you are making a commitment to act. When I tell my kids, ``Brush your teeth. Now.'' I am signaling that the time for deliberation is over. Start doing. Start acting. ``Now'' is the point where a general commitment crystallizes into a specific, concrete act. It's less a tool for analysis and more of a practical tool for acknowledging a commitment. This pragmatic reframing of indexicals, particularly ``now,'' as tools for undertaking and acknowledging practical commitments is central to Robert Brandom's work. It shifts the focus from semantics (what words mean) to pragmatics (what we do with words) \parencite[56-69]{Brandom2008}.

This brings us back to that feeling of being blocked, of trying to ``be present'' and failing. Maybe the blockage comes from treating the present as a static object to be observed, when its nature is more about holding ourselves and each other to account for our actions and commitments. The unease I feel in the ``now'' is often the friction of some unfulfilled obligation.

To get unstuck, the solution isn't to try harder to observe the present. Instead, it may be to consciously set aside a time where you can release the striving. Cultivating a space where ``now'' is not about what must be done, but just another name for the sensation may help relieve the blockage. This requires giving yourself permission to be free from the normative weight of your commitments, even if only for a moment. I have not yet addressed the directive to ``be,'' but the here and now are now sublated: \{\cancel{$\ulcorner \text{now}\urcorner$}, \cancel{$\ulcorner \cancel{here}\urcorner$}\}. The idea that individual agency is realized when a person identifies with and actualizes universal norms within a particular situation is a key theme in Brandom's reading of Hegel \parencite*[533]{Brandom:2019aa}. The choice to step outside of that cycle of obligation is the practical advice derived from this understanding. 





\textit{Sinking back in... You may notice that the arches of your feet are bent like bows. Find the 'keystone' of that arch, the point where the tension is greatest, and breathe into it. On the exhalation, allow your attention and breath to spread from that keystone, radiating through your foot. You can let go of their image. I am my \cancel{feet}. Consider each thought as it emerges from the horizon as a temptation to take a gift and flatten it. Instead of doing, you may try receiving the gift without opening it, letting the un-opened packages accumulate.} 

That last bit may be surprising; the violence of smashing gifts may be a jarring image that kicks you out of the sensation. You may conclude that I'm not a very good meditation teacher. But I, perhaps like you, am strongly opposed to surrender. I fear, and in that fear, I seek control. What better way to feel in control of someone or something than by smashing their gifts? Consuming, negating, and destroying are powerful ways to feel certain of one's existence, though that route only leads to more and more destructive consumption. This dynamic of seeking control through negation draws on the initial stage of Hegel's dialectic of self-consciousness \parencite*[\S 178-196]{hegel1977phenomenology}. Hegel argues that self-consciousness first attempts to achieve certainty of itself by negating an ``other''---by consuming and destroying it to prove its own independence. The impulse described here to ``smash'' or ``flatten'' emergent thoughts to feel ``self-certain'' illustrates this initial stage of desire. Self-certainty is not the I-feeling; the I-feeling may accompany such acts of destruction, but the feeling tends to fade to empty regret when consumption is taken as the grounds for self-certainty. For Hegel, this is a dead-end strategy; by destroying the other, the self proves its power only over a lifeless thing and fails to achieve the recognition it truly seeks, leading to the endless cycle of destructive consumption. The meditative instruction to ``receive the gift without opening it'' represents a move beyond this combative, negating posture toward a more mature form of consciousness. 


\section{Systematic Analysis: The Structure of Embodied Reason}
\textit{When you were with your toes, your feet were, loosely speaking, a kind of `\textit{forestructure}.' They existed as implicitnesses, working on your behalf in the background of experience. As you bring them into awareness, they may become explicit. They are no longer forestructures, but structures. When you let them go, they do not become unreal: they are retained in sensation but are no longer foregrounded. As you continue, these once implicit regions accumulate: I am my \{\cancel{toes}, \cancel{feet}, \cancel{legs}...\}.}

The mind, seeking to grasp this process, might generate representations like the one in figure \ref{fig:forestructure_image}.

\begin{figure}[h]
 \centering
 \title{A Thinker's Representation}
 \includegraphics[width=0.8\textwidth]{/Users/tio/Documents/GitHub/September_UMEDCA/images/toes_legs_forestructure.pdf}
 \caption{\textit{Note. }The body as an accumulation of sublated regions, moving from the explicit focus (toes) to the implicit background (legs as forestructure).}
 \label{fig:forestructure_image}
\end{figure}


\textit{You proceed, but now the thought might arise ``Don't I want the contents of those gifts?''} Here things turn subtle again. Each `gift' is oriented towards the desire to enhance the sensation. Each thought, as its hard edge emerges from the horizon, is an attempt to represent the enhancement of the sensation. It feels so good to experience self-certainty that you desperately \textit{want} to remember how you got to where you just were. Each movement in The Exercise was regular, a constant relaxation against different as-yet-unacted-upon thoughts. And now you have thought before, so you know that these thoughts are structured around the desire to continually enhance the sensation. 




Each thought, prior to its explication, is what Carspecken calls a \textit{forestructure}—necessarily implicit, yet structured by the discipline The Exercise cultivates. These accumulate as vast dimensionality when recollected, though self-certainty itself feels non-dimensional in the gentle rhythm of awareness. 


What does this accumulating, unspoken knowledge look like when one gives in to the temptation to represent it? For me, this usually isn't felt as failure---at least not at first. First, it is experienced as excitement. All the energy that was building gets released into the representation. When the desire to share the understanding of the experience bubbles over, I find myself representing it with images. This is what Carspecken identifies as the moment where ``representation is an impetus to act; here to 'act in thought''' \parencite[172]{Carspecken1999}. 

Another, more abstract temptation might be to conceptualize the structure of this accumulation itself. In this book, the \textit{null representation} ($\emptyset$) functions across the subjective, normative, and objective modes of validity. It is both absence and potential: the erasure from which elaboration begins. Stored senses---\{\cancel{toes}, \cancel{feet}, \cancel{I}\}---coalesce into a divaded set, a collection that resists completion. This nullity is the pragmatic metavocabulary of embodiment: always already there, and always transcending the determinate names it erases. The thinking mind might attempt to formalize this process—but these representations, however elegant, are just thoughts. Let these images go. Return to the feeling. As the forestructures accumulate, they might feel like `potential' energy. As that energy builds, it eventually reaches a threshold where it transforms into `kinetic' energy. Each practioner has their own threshold that moves with practice. Perhaps the energy built in the toes is so exciting you MUST draw a representation of your toes! But you can probably build more energetic forestructures than just the toes.

What does this rhythm of tension and release feel like? Well, I've been talking about it a lot, but is there a way to express more and tell less? It's not a purely logical process; it's partially mechanical and partially subjective. I introduce the Zeeman Catastrophe Machine (ZCM) as a model (\ref{fig:ZCM}). The \textit{Sound of Time} is the overarching metaphor for the continuous rhythm of consciousness. The \textit{ZCM}, however, provides a model for a specific moment within that rhythm: the \textit{threshold} where accumulated tension (potential energy) suddenly releases into a new state. It models the `snap' of understanding or the `whoops' of falling into thought.


\begin{figure}[h]
    \centering
    \includegraphics[width=0.7\textwidth]{/Users/tio/Documents/GitHub/September_UMEDCA/images/Mechanization_of_Determinate_Negation.pdf}
    \caption{\textit{Note.} A sketch of the \textit{ZCM}. Tension (potential energy) builds until a critical point, where the system snaps into a new state.}
    \label{fig:ZCM}
\end{figure}


The ZCM models this threshold: as tension builds, the system eventually snaps catastrophically to a new stable position. The gradual tension represents the accumulation of forestructure; the sudden release is either collapse into thought or deeper relaxation. It's a machine that runs on tension and release, not symbols. I will return to this machine in chapter 6.

The machine consists of a rotating disc with elastic bands attached. Part A of figure \ref{fig:ZCM} shows simple feedback: as you pull the control parameter, the disc rotates smoothly and predictably. Part B demonstrates catastrophic feedback: the folded surface represents how the same gradual increase in tension can result in sudden, discontinuous changes in the system's state. The `fold' in the surface models the critical threshold where accumulated potential energy suddenly `snaps' the system to a new stable configuration—what catastrophe theory calls the `catastrophe' or what I experience in meditation as the `whoops' moment when forestructures collapse into thought.

\section{The Dialectical Turn: The Risk of Dissolution}
If those assurances are too weak to open the next movement towards self-certainty, consider the risk of de-committing. The Exercise is a practice of de-committing from the normative self, the ``me'' that is recognized by others. It is a risk to let go of the commitments that define you, even temporarily. But in that risk lies the potential for profound transformation.

It may seem odd to put such an emphasis on \textit{risk} in the context of a guided meditation. But there may be negative consequences for de-committing, even if those commitments will lock back in place as soon as you think. Perhaps those who hold your commitments for you will be upset if you do not fulfill them. But they might forgive you. They may experience de-commitment, too, with the deliciousness of mutually cancelled plans. 

One of the blocks I have with meditation is that surrendering is so close to death. While I can't guarantee it, you probably won't biologically die from relaxing. But what you may routinely experience is an \textit{existential death}; the death of the social self, or ``me.'' This is precisely the risk of learning something new. Our social identities can feel like a house of cards, where pulling one out because it led to the experience of error risks a full collapse of social identity. 


\textit{Consider the cresting tsunami that you fear. It is a single wave rising from the ocean. It has its own unique journey, its shape, its crest, its fall. And then, it returns to the vastness from which it came. Was it ever separate from the ocean? When you identify with the thought---with the wave---you may fear its repercussions. But you may also fear its inevitable return to the whole. Both are real. But remember: each thought is not the sensation. The sensation feels good. Let go of the wave before it crashes. Let it slosh gently back into the vast sea of implicitness. What you fear is social death, which you have died a thousand times before and been reborn from each time, recollecting the experience as growth. Death, in three dimensions, need not be feared, for who knows what mysteries lie beyond the horizon? Perhaps when you meet it, its first moment may feel like the deliciousness of reciprocally cancelled plans.} 



That last thought might kick you out of The Exercise. I feel it as a `zap! No! I do NOT know what I am writing about.' Death feels too big, my words too small. It's like the metaphors for the experience become so cloying that they suffocate the sensation. So, I drop the metaphorical facade. 

Before the divisions of subject, object, and rule, there is the primordial \textit{No}. The ``no'' or its erasure come unbidden with varying strengths. In the embodied logic, this is a temporal claim: these are more primordial terms than the differentiation into subject, object, and rules. The NO! shouted in abhorrence or the one that cradles the broken pieces of our lives are undefinable in print. They are like topological holes in the space of reasons---singularities that defy articulation.

The normative distinction between subject and object dissolves in the feeling-body. This happens outside the practice, too, the practice just affords a recognition of those moments that are often ephemeral. Pregnancy, birth, death, intimacy, teaching, learning, etc. evoke a deep sense of being that transcends the limits of lexical articulation. 

The terms ``No'' and \cancel{no} (determinate negation) are meta-systemic. As the text unfolds, I will claim they are a \textit{pragmatic metavocabulary} for the system of critical mathematics. Objectively, they accrue senses: \{\cancel{toes}, \cancel{feet}, \cancel{I},...\}. Subjectively, they are the source of action recollected as \{I\}. Normatively, they are the concept that erases its own name: \cancel{$\name{concept}$}. This nuance can be compressed into the null representation ($\emptyset$). The ``no'' or its erasure are primordial terms, more fundamental than the differentiation into subject, object, and rules. The normative distinction between subject and object dissolves in the feeling-body.

\textit{You start again, but now you know that the desire for the grand experience---the desire for self-certainty---is itself hindering the ability to let go. As Carspecken notes, ``To have what is desired and anticipated one must let go of desire'' \parencite[174]{Carspecken1999}. Perhaps to get to non-thinking, you must say ``no'' to the desire for the grand experience. And if thinking is saying ``no'' to the flow, then maybe non-thinking is saying ``no'' to that no-saying.}







\subsection*{Exercise 5: Iterate and Integrate (The Upward Spiral)}
At this point, I lose some feeling of entitlement to proceed in The Exercise, and rely on Phil Carspecken's text. 

\textit{``It is as if the forestructures of desire are accumulating each time you let one go... Letting go is what allows the forestructures to accumulate; the implicit portions of each anticipation appear to contain more and more possibilities. Thus the situation is contradictory. To have what is desired and anticipated one must let go of desire''}\parencite[173]{Carspecken1999}.

\textit{``... Letting go results in a repetition. It results in the repetition of (i) sensation, (ii) awareness-of-sensation, (iii) letting-go, (iv) intensified-sensation. It is not a repeated `act' because the only agency involved is the choice not to act, `letting go.' This allows an anonymous repetition to act of itself; a sort of vibration that throws one out in the second phase (`awareness of') but waits for one to fall back in [to sensation]''}\parencite[174]{Carspecken1999}

Each cycle of this process doesn't leave things as they were; it enriches the whole. As Carspecken describes it, ``letting go is what allows the forestructures to accumulate; the implicit portions of each anticipation appear to contain more and more possibilities'' \parencite[173]{Carspecken1999}. If the cycles are numbered, one can say that from state $S_n$ plus a new provocation $D$, a letting-go yields $S_{n+1}$. The state $S_{n+1}$ carries an accumulated, assimilated content from all the prior releases. Each repetition is not a mere repeat but an intensification: not a circle so much as an upward spiral.




\subsection*{Exercise 6: Reading Philosophy---Being, Nothing, and Becoming}
You have been learning to navigate the rhythm of thought in The Exercise: compression into awareness and decompression through letting go. A recurring temptation arises: perhaps this rhythm can be stabilized by deliberately directing it? Could thinking a thought and then intentionally thinking its opposite return experience to unified immediacy?

The following extends The Exercise into a controlled abstraction experiment.

\paragraph{Embodied Experiment} \textit{Settle into the relaxed, open awareness cultivated previously. Now intentionally introduce the most abstract thought available: ``Being.'' Attempt to fixate on it—pure, undifferentiated presence. The felt quality is not the expansive unity of the I-feeling; it is intense Temporal Compression ($\downarrow$). Attention strains to grasp totality as a static object.}

\textit{Yet ``Being,'' devoid of inner difference, is unstable. It gives nothing determinate for attention to grip. The fixation collapses into ``Nothing.'' This, too, is compression—the negation of the prior abstraction. An oscillation arises:}
\[
 	\text{Being} \; \longleftrightarrow \; \text{Nothing}.
\]
\textit{This loop is restless and draining—a bad infinite of endless alternation without qualitative transformation. The strategy of control through opposition fails because it remains imprisoned in mutual compression.}

Pause. What occurs between the poles? In each transition there is a micro-movement—a \textit{temporal singularity} where one fixation dissolves and the other has not yet stabilized. Shift attention toward that passage. Breathe into \textit{the in between}. The movement itself is what Hegel names \textit{Becoming}.

To remain with Becoming, apply the earlier lesson: Let Go of fixation on the oscillation as content. This letting go is Temporal Decompression ($\uparrow$). The rhythm becomes becoming and the awareness of becoming. As a temporal compression, $\ulcorner \text{becoming} \urcorner$ paradoxically compresses both compression and decompression into a single process.

This is an exciting thought—so much so that I began my dissertation claiming ``I am a constant becoming.'' Yet there is peril: for becoming to avoid falling back into simple oscillation, it must move beyond itself. When movement (becoming) is compressed into a static concept ($\ulcorner \text{becoming}\urcorner$), we get a kind of recursion: $\text{becoming}(\ulcorner \text{becoming}\urcorner)$. If becoming becomes itself, it achieves what mathematicians call a \textit{fixed point}—yet it cannot be itself if it is itself. No movement would be expressed; it would slide back into nothingness.

Hegel's whole \textit{Science of Logic} falls out from this beginning. The exercise attempts a somatic sublation of being/nothing into \cancel{being/nothing} into becoming, where becoming names their opposition. I imagine dense philosophical texts annotated with breath marks, like musical scores telling wind instrument players when to breathe.

The oscillation that the sound of time metaphorizes is ultimately between the finite name ($\ulcorner \text{becoming}\urcorner$) and the \textit{in}finite movement it purports to represent. This returns us to determinate negation in its two moments: the first negation produces finite representation; the second determinately dissolves those determinations. You find yourself learning again what you already knew, but from a conceptual descent that folds back to the original ``no.'' 




Hegel's \textit{Science of Logic} begins with the most abstract categories: Being, Nothing, and Becoming. Pure Being, completely indeterminate, gives nothing for thought to grasp—and thus reveals itself as Nothing. Yet this Nothing is not mere absence; it is the dynamic movement between Being and Nothing that Hegel calls Becoming.

Mapping this onto The Exercise reveals the connection. The moment of unified presence (I-feeling in full bloom) is like a moment of pure Being in experience---so pure and featureless that it cannot be held onto. The collapse of that experience when the thought ``I am this'' arises is akin to that Being turning into Nothing---the fullness evaporates into an absence (the feeling is ``lost,'' and one is left only with an empty thought that doesn't deliver the goods). Yet out of that nothing, a new determination immediately begins to form: perhaps a new approach, a new aspect of the body to focus on, or a new resolve (``I'll try again''). In other words, the Nothing gives rise to a new Becoming---a coming-to-be of another state of being (for instance, restarting the cycle and reaching a new unified feeling, which will again have no determinate content and thus again risk collapsing). The cycle of recognition described is essentially a cycle of Becoming. The moments of pure undifferentiated sensation (Being) inevitably pass away (Nothing), yet from that void a richer unity can emerge if allowed (the next Being). Each ``loss'' of the experience, when handled by letting-go, is actually the engine of a deeper recognition of the experience. Each negation (loss of being) is followed by a negation of that negation (the letting go of the thought, which re-establishes a new being).


\subsection*{Exercise 7: A Symbol for Silencing}
As you begin again, you might notice that \cancel{no} is a symbol. It contains both the first negation and the second negation within itself, so it represents both fixity and movement. While paradoxical: stay with its movement.

\textit{\cancel{no}}



\section{Space, Time, and the Limits of Picture-Thinking}


The metaphor of sound arises from the body feelings associated with compression and decompression. Time may be grasped as vibration: the unit circle rotates, the sine wave oscillates, and a resonance emerges. These mathematical gestures are ornamental to The Exercise but are useful for reconstructing mathematics and may be helpful for qualitative researchers. They anchor thought in motion, making rational reconstruction audible as rhythm and recurrence. To feel the wave crest and trough is already to participate in temporality; to recognize it mathematically is to render that temporality communicable.

The central metaphor can now be fully articulated. In air, sound is the regular compression and rarefaction (decompression) of air particles. The ``sound of time'' is the rhythm of embodied reason, a dialectical pulse of compression and decompression.
\begin{itemize}
 \item \textit{Temporal Compression:} The act of focusing, judging, or thinking. This is the first ``no.'' It corresponds to the compression phase of a sound wave. It is felt as proprioceptive contraction.
 \item \textit{Temporal Decompression:} The act of releasing and re-integrating. This is the second ``no,'' the sublation. It corresponds to the rarefaction phase of a sound wave. It is felt as proprioceptive expansion.
\end{itemize}


I came up with the metaphor from considering the hermeneutic circle: the idea that the whole of some text must be understood through its parts and the parts must be understood in relation to the whole. This way of interpreting texts was originally developed by biblical scholars, but hermeneutics has taken on a broader significance in philosophy and social theory. 

Since I know a bit of trigonometry, I considered what it would mean to unwrap the hermeneutic circle as if it were the unit circle. Take a circle and imagine yourself as a single point on its circumference. Roll the circle forward in time, as in figure \ref{fig:sound_of_time_wave}, and your distance from the ground will trace out a sine wave. So, the sine wave is, metaphorically, a way of unwrapping the hermeneutic circle into a temporal wave. So, metaphorically speaking, hermeneutic meaning is the sound of time. Time is the medium in which meaning occurs. At least within this metaphor. 

\begin{figure}[h]
 \centering
 \title{\textit{The Sound of Time}}
 \includegraphics[width=.8\textwidth]{/Users/tio/Documents/GitHub/September_UMEDCA/images/Unit_Circle_Elaboration.pdf}
 \caption{\textit{Note.} Circular, unified experience (A) is ``unrolled'' through time into a linear, oscillating wave (B). This wave, like sound, alternates between compression (``No'') and decompression (``\cancel{No}''), representing the rhythm of determinate negation (C).}
 \label{fig:sound_of_time_wave}
\end{figure}

Recognizing this rhythm is the first step to synthesizing the concept of \textit{becoming}. 


One of the implications of the sound of time metaphor is its connection to space. I have been working on developing the `space of reasons' but a more primordial kind of space has been left at the margins. Take an analog watch, one of the ones where the second-hand moves smoothly around the dial. That spatial movement is a very basic way to understand empirical time. This may feel very abstract, but consider the experience of driving a young child to a town that is 50 miles away. The answer ``50 miles'' has no meaning when they ask, ``how long until we get there?'' They have yet to compress temporally extended experiences into the abstract concept of a spatial distance. Distance, as an abstract concept, does not arise until a temporally extended experience is recollected. It often takes many such trips before a child will have a strong sense for what ``50 miles'' means. I've made the drive from Bloomington to Indianapolis hundreds of times. When my daughters and I were returning from Indianapolis one time, after having made about 98 of the 100 mile round trip journey, they asked ``are we home yet?'' I said ``just a few more miles,'' and one of them complained that we had to drive a whole hour longer. Space and time are confusing! We tend to work very hard to teach their separation, then only a few who study relativity ever get to consider their unity again. 

The concept of \textit{distance} is a temporal compression. Temporally extended experiences are recollected and compressed into a spatial form. To fit this idea into the sound of time metaphor, the two fields of space and time are orthogonal to one another. Temporal compression induces spatial predication when a temporally extended experience is recollected. Those spatial predications can be decompressed into temporally extended \textit{histories} for how those thoughts came to be. The relationship mirrors electromagnetic radiation: just as a changing electric field induces a magnetic field, and a changing magnetic field induces an electric field, it is the \textit{movement} or \textit{change} itself—the act of compression or decompression—that creates the mutual relationship between spatial and temporal fields. Static spatial forms cannot induce temporal histories; only the dynamic process of spatializing (temporal compression) or temporalizing (spatial decompression) creates this reciprocal induction. 

So, when I consider a spatial form like ``triangle,'' I can temporally decompress that form into a history for how the form came to be. My story for triangles begins in Dr. Laughlin's office because I remember playing with a toy, trying to cram a triangular shape through a circular hole.

This relationship between space and time is inherently intersubjective. The temporal experience of the Exercise is spatialized when the desire to communicate the experience is acted upon in representation. 

I make no claim about the purely objective relationship between space and time that cosmologists explore. There are implications that relate the first moment of determinate negation to observability in particle physics, as waveforms collapse into particles. There are other implications for relativistic interpretations of spacetime, too. But we have not learned how to add or subtract yet. Those diversions must wait until critical mathematics has developed the tools to express those notions. 

Spacetime is not a backdrop for events but an active participant in the unfolding of reality. The compression and decompression of spacetime can be likened to the rhythmic oscillations of sound waves, where moments of high density (compression) alternate with moments of lower density (decompression). Highly spatialized representations (picture thinking) miss how space emerges from the recollection of temporal processes.


Suppose you try to capture self-certainty. You might try to indicate the particular now, this, or here through a deictic pointing gesture. This fails to capture self-certainty, as ``this'' refers to many different ``thises.'' To indicate the passage of time, you might use a deictic ranging gesture (see figure \ref{fig:deictic_gestures}). It is very hard to represent temporal difference without relying on spatial difference. Carspecken writes, ``Space, not time, is used to represent differences: when time is used to express a difference, like `I am totally different now than I was back then,' the temporal difference has a spatial representation---in the cultures I am familiar with, the past is to the left and the future to the right, or the past is behind and the future is ahead'' \parencite*[19]{Carspecken2018}. 

\begin{figure}[h]
 \centering
 \title{\textit{Deictic Gestures}}
 \includegraphics[width=0.8\textwidth]{/Users/tio/Documents/GitHub/September_UMEDCA/images/deictic_gestures.pdf}
 \caption{\textit{Note.} Deictic gestures that are important for teaching mathematics as they arise in The Exercise.}
 \label{fig:deictic_gestures}
\end{figure}


In quoted passage, Carspecken is analyzing the problem of picture thinking. In Hegel's philosophy, \textit{Vorstellung}, often translated as ``picture thinking,'' ``representation,'' or ``figurative thought,'' occupies a crucial intermediate stage between sensory intuition and pure conceptual thought (\textit{Begriff}). Carspecken takes up the problem with picture thinking in several works \parencite*{carspecken_limits_2016,Carspecken2018,Carspecken1999}. When an essentially temporal phenomenon, like \textit{becoming}, is rendered in spatial form, as if knowledge is limited to that which can be conceptualized, we risk getting caught up in what Hegel calls the \textit{bad infinity} of endless spatial predications. Each may be progressively more adequate, but it is impossible to picture the \{I\} because the \{I\} is not-a-thing. Picturing it as an object essentially misrecognizes it. Spatializing thought as images, metaphors, and scientistic models artificially limits the horizons that phenomenology explores.

It is a mode of thinking that relies on images, metaphors, and sensuous or spatial-temporal forms to grasp philosophical and religious truths. Even \textit{rhythm} misses the Concept, as it requires the notion of empirical spacetime to be explicated. 

The ``problem'' with picture thinking lies in its inherent limitations.
\begin{enumerate}
 \item \textit{Externality and Separation}: \textit{Vorstellung} tends to present the content of thought as a collection of distinct images or representations that stand in an external relation to one another and to the thinking subject. For instance, the canonical texts in math education (think textbooks or the Common Core Standards) may be represented as a series of symbolic manipulations (i.e., `procedural knowledge), rather than as expressions of divaded conceptual relationships.
 \item \textit{Inadequacy to the Universal}: Because it relies on particular images, \textit{Vorstellung} struggles to capture the true universality and necessity of conceptual determinations. A representation is always a specific instance, and while it can symbolize a universal, it cannot fully embody or express its dynamic, self-mediating nature. The content of the Concept is distorted when forced into the static and finite forms of representation.
 \item \textit{Resistance to Dialectical Movement}: The fixed nature of images in picture thinking can make it difficult to grasp the fluid, dialectical movement of thought, where concepts pass into their opposites and are sublated into higher unities. Picture thinking tends to hold onto its representations as fixed and separate, hindering the transition to speculative, conceptual understanding. A series of images or representations can give the illusion of completeness, but they do not allow for the dynamic interplay of concepts that characterizes true philosophical understanding. 
\end{enumerate}

Neither Hegel nor Carspecken simply dismiss \textit{Vorstellung}. It is a necessary stage in the development of consciousness and spirit. The philosophical task of this chapter, now that a robust system of representations has been articulated, is to transcend picture thinking. We must break the metaphor to overcome the limitations of its pictorial form. ``Picture-thinking is totally taken for granted in mainstream methodologies for social research, but criticisms of mainstream methodologies mostly substitute one representation for another, or take the basic representation underlying empiricism or positivism and merely remove and/or redefine its specific components. Critical theory in the original sense begins consciously with the full transcendence of picture-thought in general, not with a new representation'' \parencite[19]{Carspecken2018}. This requires exploring alternative modes of expression, such as music, that capture the dynamism of lived experience beyond the limits of static representation.

\section{Integration: Breath and Kindling}
Representations and machines fail, structurally, to represent the totality of experience. But logic also fails; recall that the logic includes unspeakable symbols. The text I've written excludes phonemic pronunciation. So, music is entirely absent from its pages. Music also can't represent the totality of experience. It provides yet another orthogonal direction for reconstructing the experience.

Brandom offers a bit of wisdom for how to take the logics that spring forth from his project of \textit{logical expressivism}. He writes, \enquote{Acknowledging the value of the unique clarity afforded by algebraic understanding accordingly does not entail commitment to this sort of understanding being available in every case, even in principle. It does not oblige one to embrace the shaky method of the drunk who looks for his keys under the streetlamp, not because they are likely to be there, but just because the light is better there. We should admit that, sometimes, algebraic understanding is not available} \parencite[215]{Brandom2008}. I riff on this idea in the following song, \textit{Breath and Kindling}, calling such logics `white-knuckled halos'---though the phrase also reflects personal struggles with goodness. To backfill what is lost in the flattening of experience into text, I gesture toward the \textit{in}finite that formal representation cannot contain:

\textit{Breath and Kindling}

\begin{verse}
A moment too late, a moment too soon,\\
Bleeds into a cinnamon moon.\\
Red, silver snow, breath frozen dew.\\
The sun died and a thousand stars bloomed.\\

Good morning, dear Venus,\\
Good morning to all my evening stars.\\
Though the glow is faint between us,\\
Your light slices through the dark.

Strung street lights, small sodium moons\\
White-knuckled halos cut holes in the gloom.\\
Are you always so ate up, man?\\
Just give me some room.\\
You cut to the quick,\\
I get a weeklong wound.\\

Good morning, dear Venus,\\
Good morning to all my evening stars.\\
Though the glow is faint between us,\\
Your light slices through the dark.\\

I try to walk the line,\\
But miss the flight.\\
Hit the curb to get some lift inside.\\
Start to drift and fall for a while out of time.\\

Easy blows the listening breeze\\
In breath caress\\
Wind over wings.\\
That's the way that you move me --\\

-- From white to green\\
To hollow dream,\\
Hatchet through the sycamore tree.\\
It'll do for the kindling --\\

-- Easy blows the listening breeze\\
In breath caress\\
Wind over wings.\\
That's the way that you move me --\\

The care you take\\
With the moves you make\\
To ease the world to sing for its own sake --\\
Frankly, it's moving --\\

-- Easy blows the listening breeze\\
In breath caress\\
Wind over wings.\\
That's the way that you move me --\\

-- In grief I've worn\\
The cloth of storm\\
Of warp and weft\\
Bereft of words of words reborn\\
Moving toward \cancel{divinity}--\\

-- Easy blows the listening breeze\\
In breath caress\\
Wind over wings.\\
That's the way that you move.
\end{verse}

The song opens with haiku-like images that blend time, color, and sensation: \enquote{A moment too late, a moment too soon / Bleeds into a cinnamon moon / Red, silver snow, breath frozen dew.} These lines recollect the setting sun in winter, where I try to establish a crepuscular, transitional mood while introducing dialectical concepts. The evanescent \enquote{now}---a moment that is both too late and too soon---reflects indexicals like \enquote{now,} which vanish the moment they are named, highlighting the failure of sense-certainty to grasp the present. The chords begin with George Harrison's progression from \textit{Something}, hanging on A major 7 to A flat 7. I play the song on an 8-string baritone guitar tuned to drop A, so there is a round warmth with subtle shimmer from the doubled strings, mirrored in the metaphor of a \enquote{cinnamon moon}. That metaphor combines sight with taste and smell to create a `spicy' warmth that contrasts with \enquote{breath frozen dew}. The death of the sun gives way to the blooming of stars, setting the implicit `lesson' of the song: loss of a fixation (losing the sun) through the experience of error can feel profoundly bad but enables subtler understandings to shine through. 

I address Venus directly, whose dawn arises with the setting sun. There are shadows of past romantic relationships embedded in the imagery, but the invocation of \enquote{Dear Venus,} \enquote{morning stars,} and \enquote{evening stars} also alludes to Frege's distinction, where a single referent (Venus) is grasped through different senses (morning star/evening star). This suggests an underlying unity through different representations. The imagery of light continues, shifting from the celestial to the mundane and back again. \enquote{Strung street lights, small sodium moons} elevates ordinary urban infrastructure to something cosmic, while \enquote{White-knuckled halos cut holes in the gloom} serves as metaphor for rigid, algebraic logic---systems that provide clarity through constraint but are necessarily limited and exclusive.

The beauty is contrasted with a moment of sharp emotional pain: \enquote{You cut to the quick and I get a week long wound.} The idiom \enquote{cutting to the quick} refers to the sensitive area under a fingernail, but also suggests a rush to judgment. The I-You/me I use in this line is a form of self-address that includes otherness. 

The rhythm shifts from gently swept guitar strumming to more of a grove that mimics a heartbeat. Simultaneously, the gentleness of the Harrison progression is exfoliated in favor of a simpler I-IV-V progression. This stability represents temporal compression or the \enquote{first negation}---a fixation, the creation of determinate boundaries. The lines, \enquote{I try to walk the line, but miss the flight / Hit the curb to get some lift inside / Start to drift and fall for a while out of time,} use enjambment to enact the feeling of losing control within that stability. The thought spills over the line break, mimicking the act of drifting and falling. 

There is a brief pause in the music after \enquote{fall for awhile, out of time}---a cessation that represents a threshold where stability reaches a critical point necessitating release. But this release occurs without tutelage or grounding. As I wait through a bar of silence when I play the song, I take the opportunity to breathe deeply. 

Upon the phrase \enquote{Easy blows,} the three chords start to move. The song's structural and thematic core becomes the anaphoric refrain: \enquote{Easy blows the listening breeze in breath caress wind over wings / That's the way that you move me.} This line creates sonic texture through sibilance and liquid consonants that create a soft, flowing sound mimicking the movement of gentle breeze. The phrase \enquote{breath caress} links the movement of breath to an intimacy. It is hard to avoid sexual connotations, but the intimacy that I am trying to establish is less freighted than that. I was after both the spiritual center of self-certainty and the teachers who have listened to me in a way that invited free movement around that core. 

On that turn, the key changes in minor thirds to move all the way around the circle of fourths---a kind of homage to John Coltrane's \textit{Giant Steps}, though simpler than his movements. The harmony modulates systematically through a cycle of ascending minor thirds: A Major, C Major, Eb Major, F\# Major, returning to A Major. This cycle divides the 12-semitone octave symmetrically into four equal parts, creating abrupt shifts between distant keys. The constant, structured shifting embodies the concept of becoming---not a static state but a dynamic process where each key is destabilized by the modulation.

Relative to the key, the melody simply repeats. The core melodic motifs are repeated identically \textit{relative} to the new key, using the same scale degrees relative to each new tonic. But as the key modulates up, the melody also rises. This enacts a dialectic of sameness and difference: the melody's internal structure is preserved while the absolute pitch constantly rises as the underlying harmonic structure shifts. With each change, the last note of the melody, which happens to be the tonic, becomes the first note of the melody in the new key, the fifth. In the first shift, the last note is A (the root) and the first note in the key of C is A (the sixth). However, since the key has changed behind the melody, those samenesses feel different. The underlying structure moves, while its representation (the melody) stays the same.

This process, like the somatic sublations above, is a kind of implicit sublation: the form of the melody is preserved, the harmonic ground and absolute pitch are negated, and the melody is transformed into a higher unity. As the keys ascend, the preserved melodic shape is realized at progressively higher pitches. This is not a simple repetition but an upward spiral where each repetition is an intensification. Like poetic/rhetorical anaphora (``I have a dream''), which involves linguistic repetition and intensification, the song adds a layer of musical anaphora. The transitions between keys are mediated by a pivot mechanism where the pitch remains the same while its harmonic function is redefined by the shift, ensuring the transition is a determinate negation where the dissolution of the old state defines the beginning of the new one.

This is reflected in the sycamore tree metaphor: \enquote{From white to green / To hollow dream, hatchet through the sycamore tree / It'll do for the kindling.} The sycamore trees of southern Indiana grow with both white bark and green leaves. The particular tree I recall was an old friend who was planted too close to the house when I was a kid. My parents had to chop it down.



Metaphorically, it is about the creative process. Beginning with a white page, which turns lush with prose in my obsessive writing practice, but then those prose fall apart when I realize a core error in my assumptions thus negating the validity of those expressions on the page. I usually feel sad and disappointed when those errors are made explicit. But the words in error don't become unwritten or useless. The kindling metaphor ties back to the song's title, suggesting that from the destruction of old forms (the sycamore) arises the potential for new life and understanding (the kindling for fire). So, the metaphor recollects a necessary destruction of old certainties to provide material for new growth (the kindling). The hatchet's action is (to my abundant disappointment) both violent and purposeful. I wish my words were always gentle, but each one holds the potential for the pain of division. 

The confusion about intimacy is, I hope, clarified in reference to those who \enquote{ease the world to sing for its own sake.} Phil Carspecken's gentle, catalytic presence as educator taught me a lot the role of listening. The `active' listener serves as a kind of vacuum, or negative withdraw, implicitly providing direction for me, the student, who filled that space with words. When I was working on my dissertation, I wrote a song I repeat later in this manuscript called \textit{Love's Memory.} I did not know if I \textit{could} or \textit{should} include it in the dissertation. I wrote him an email asking if it would be okay. He responded ``Oh, my. Thank you! Thank you, and thank you!!! Phew. I'm crying. Ohhh! The heart; loving, mourning, bowing, stopping: no doing for moments in the second position. I'm grateful for what you have written, and for what your father wrote. Of course you must keep this in.'' I have kept that email `pinned' in my messages for three years, to serve as a constant reminder and validation of the those works I might otherwise burn for kindling. The `hatchet' is dropped to ease the world to sing for its own sake.

The lyric then circles back to a personal state of grief. The line, \enquote{In grief I've worn the cloth of storm / Of warp and weft / Bereft of words of words reborn}: this will make more sense after the Eulogy and \textit{Love's Memory}. 

The physicality of performing the song is also theoretically relevant, as the limitations of my body are reflected in the ZCM. The rising pitch increases potential energy, pushing me toward my vocal limits. With \enquote{In grief...} I reach near the top of my vocal range. Energy has been building as I approach my limit, then, with the last chorus, it `snaps' down, resolving to the original key. I relax as the song ends. The ZCM suddenly downshifts in energy. The tension dissolves, and the music returns to the initial stability. I enjoy singing the song. What is left, after moving through the circle, is not just a return to the beginning for me. Going nowhere---often derided by academics through the phrase ``there is no `there' there''---has a deeply satisfying quality. A glow, easily missed and often forgotten, accompanies the physical, emotional, and theoretical act of performing the song. Lately, in the mornings when I wake up feeling anxious, I like to sit outside my daughters' room, singing this song as an invitation to them and myself to wake up gently with the breath and kindling of a new day. 

\section{Prefatory Analysis: Two Readings of Determinate Negation}

Before concluding, I want to articulate an alternative to the interpretation of determinate negation that will be taken up in the next chapter. This prefatory analysis serves as a bridge, clarifying the philosophical terrain we have been traversing.

\subsection*{Brandom's Interpretation: Other-Exclusive Material Incompatibility}

Robert Brandom, whose work profoundly shapes the next chapter's analysis, argues that determinate negation should be understood as \textit{other-exclusive} material incompatibility. Something is what it is because it excludes what it is not. A red surface excludes green, blue, and yellow; a square surface excludes triangular, circular, and pentagonal. The identity of each concept lies in its relations of exclusive difference to other concepts.

This means conceptual content is inherently \textit{relational}. You cannot understand what ``red'' means in isolation. You understand it by grasping the entire family of color concepts and recognizing how red contrasts with green, blue, yellow. The meaning of ``red'' is constituted by what it is \textit{not}.

For Brandom, this is Hegel's radicalization of the law of non-contradiction. Far from rejecting that law, Hegel places exclusion at the very center of his metaphysics. Everything is what it is by virtue of what it excludes. Determinateness is exclusion. To be is to contrast.

This interpretation explained why I could never pin down the meaning of mathematical concepts by definitions alone. The meaning of ``2'' is not captured by saying ``the successor of 1.'' Rather, ``2'' gets its meaning from its position in a web of incompatibilities and inferences: it is not 1, not 3, not 7; it is even, not odd; it is prime, not composite (depending on convention); it is the result of 1+1, not 1+2 or 2+2. The concept is alive in this network of relations.

\subsection*{Bordignon's Critique: Self-Exclusive Absolute Negation}

Some Hegel scholars argue Brandom's account, while insightful, does not go far enough. Michela Bordignon claims Brandom ``stops short'' of Hegel's full concept of determinate negation. For Hegel, determinate negation is not merely \textit{other-exclusive}; it is \textit{self-exclusive}. It is what Hegel calls ``absolute negation'' or ``negation of negation.''

What does it mean for a negation to be self-exclusive? The determination does not just exclude \textit{other} things; it excludes \textit{itself}. It negates itself, turns into what is other than itself, and thereby propels the dialectical movement forward.

Consider Hegel's famous example of the finite and the infinite. The finite is defined as that which has limits, which ends. But to grasp what it means to be finite, you must think its contrast: the infinite, that which has no limits. However, the moment you define the infinite as ``that which excludes the finite,'' you have \textit{limited} the infinite by its exclusion of the finite. The infinite is now finite in a new sense: bounded by what it excludes. This is the self-negating movement of the concept. The infinite negates the finite, but in doing so, negates \textit{itself} as truly infinite.

The finite, in its ceasing-to-be, in its coming to an end, realizes its own finitude. It is itself insofar as it is no longer itself. The infinite is itself in this very process of self-negation. This reveals that some concepts are inherently self-contradictory, and that this self-contradiction is not a defect but the engine of their development.

When I first read this, I resisted. How can something be itself by being not-itself? But then I thought about my own experience of self-alienation. I often feel that I am not myself---that the ``me'' others recognize is not the \{I\} that I experience myself to be. Yet that very alienation is constitutive of who I am. The \{I\} becomes a ``me'' through recognition by others, and that ``me'' is never adequate to the \{I\} that I feel myself to be. The self is constituted through this movement of self-externalization and return. I am myself by becoming other than myself.

This self-exclusive negation is what Hegel calls the ``dialectical soul'' of every determination. It is the ``innermost source of all activity, of all animate and spiritual self-movement.'' It is why concepts \textit{move}, why they develop, why they do not stay put.

\subsection*{The Two Interpretations in Practice}

Let me make the contrast concrete. Imagine you are in The Exercise, and you have just brought awareness to your toes.

\textit{Brandomian interpretation:} ``I am my toes'' excludes ``I am my feet,'' ``I am my head,'' and every other body region. The identity of ``toes'' is constituted by its material incompatibility with these other regions. When I shift to ``I am my feet,'' I am replacing one exclusive determination with another. The movement is a sequence of mutually exclusive determinations: toes $\rightarrow$ feet $\rightarrow$ legs. Each excludes the others, but all are preserved in memory as part of the accumulating totality.

\textit{Bordignonian interpretation:} ``I am my toes'' is a self-negating assertion. The very act of identifying with my toes reveals that I am \textit{more} than my toes. The identification negates itself by pointing beyond itself to the larger whole. The toes are not just excluded by the feet; they negate \textit{themselves} by being inadequate to the \{I\} that asserts them. The movement is not a sequence of static replacements, but a living, self-propelling dynamic where each determination undermines itself and thereby generates the next.

Both interpretations are valuable. Brandom helps articulate the relational structure of conceptual content. Bordignon helps articulate the dynamic, self-propelling nature of consciousness. I do not have to choose. Hegel's determinate negation has these two aspects: other-exclusion and self-exclusion, static structure and dynamic movement.

The Exercise allows you to feel both. You feel the exclusive difference between toes and feet (Brandom). And you feel the self-negating inadequacy of each identification (Bordignon). Together, these two aspects constitute the full rhythm of determinate negation, the sound of time as it pulses through embodied experience.

In the next chapter, we will take up Brandom's inferentialism in detail, exploring how material incompatibility grounds the entire structure of reason-giving. But before that, I want to acknowledge this alternative reading---one that emphasizes the self-moving, self-negating character of absolute negativity that cannot be fully captured by Brandom's relational account. Both are needed. Both are true.

\section{Conclusion: The Safeguard of Experience}
This chapter began with the simple act of attending to the breath and the body. From this embodied practice, a theoretical tapestry has unfolded. My great worry is that readers might misunderstand the purpose of denying the certainty of words as if I am advocating for solipsism. To be clear, I am trying to entrain you (and myself) into the second-person \textit{listener's} position in communicative action. \cancel{Self-certainty}, or any certainty, is not found in the words that describe it. Words flatten what listening expands. What communicative practices make explicit must still answer to what must remain implicit. Communication is, in essence, dyadic: the speaker and the listener, the author and the reader, the first negation and the second. Uncertainty in words does not imply a lonely consciousness. I intend no solipsism. By preparing you to listen, through practice, I aim to create a space for shared understanding. 

In essence, what The Exercise and reflections provide is what Brandom calls a \textit{pragmatic metavocabulary}. The phenomenological vocabulary describes the embodied practice, but since the practice is sufficient to understand the vocabulary, they work in tandem to deepen understanding. This kind of reciprocal sufficiency between words and practices creates what might be called \textit{expressive pragmatic bootstrapping}---a process where notation and experience mutually enrich each other.

The unspeakability of certain notational practices like Quine corners ($\ulcorner \cdot \urcorner$) and sous rature (\cancel{·}) is a feature, not a limitation. For those who struggle with meditation because they are thinkers, ruminators, or rehearsers of thoughts, this differentiation between speakable and unspeakable elements helps eliminate the temptation to think. The `sound' in your head---the \textit{sotto voce} voice that narrates experience---can be so loud that it challenges focus on The Exercise. These notational practices purposefully eliminate inner speech, perhaps enhancing the sensation by silencing the discursive mind.

When the concept of negation is encountered, the felt experience of ``no'' and ``\cancel{no}'' will be remembered. When the \textit{in}finite is discussed, the horizon of the ``Grand Experience'' will be recalled. And when working with the null representation, $\emptyset$, it will be recognized not as an empty symbol, but as the mark of the silent, unrepresentable ground of thinking itself. The ultimate aim is to explicate a critical mathematics, and understanding how mathematical objects might be seen as recollections invites an unwinding of these objects back into the subjective, embodied experience from which they arise. Uniting the subjective and objective poles of validity claims is deeply related to the project of emancipatory knowledge. The sound of time will continue to resonate, reminding the reader that even the most abstract mathematics is, at its heart, a profoundly human song. 

\printbibliography[heading=subbibliography]