\chapter{Thoughts for Two---Who Are You?}

\begin{abstract}
This chapter examines the structures of intersubjective recognition, building on the previous chapters' explorations of how the ``Who are you?'' question reveals the dynamics of mathematical communication and human development. A central case study follows the evolution of ``Big Gorilla,'' a character created by preschoolers, whose transformations illustrate the processes of communicative breakdown and repair. The analysis applies insights from George Herbert Mead's symbolic interactionism, Jürgen Habermas's theory of communicative action, and Robert Brandom's analytic pragmatism to understand how shared meanings emerge through social practice. A key contribution is the distinction between the transactional ``you''---which addresses a particular, concrete other---and the ``CUSP you'' (Claimed Universal Subject Position)---which invokes a generalized, normative subject position. This distinction, grounded in Sebastian Rödl's work on self-consciousness and the first person, reveals how mathematical discourse often presupposes a universal subject while failing to recognize the particularity of actual learners. The chapter develops an algorithmic model that frames intersubjective development as a recursive process involving gesture formation, communicative action, breakdown, and repair. This model will be extended in later chapters to understand the development of numerical concepts.
\end{abstract}

\section{The CUSP You: A Misrecognition of the Particular}

The last chapter argued that the tension between existential needs is resolved not in solitude, but in a developmental journey toward reciprocal recognition. This resolution, embodied in the surrender to the ``Beast of Love,'' is fundamentally a social and communicative achievement. But what is the structure of this recognition? Who is the ``You'' that meets the \{I\}?

When I was working on my first research project in graduate school which explored how gestures and speech relate in teaching algebraic concepts, I found that preservice teachers often used the word ``you'' in ways that I felt were curious. I was the only other subject in the room as a preservice teacher said things like ``The first thing you do is distribute the 2.'' When factoring an equation like $2(p^2+5p+6)$, the first thing I would do is definitely not distribute the 2, as I knew I would just have to factor it out again later. They used the second-person pronoun in a way that I felt missed me as a referent of that pronoun.

I called such uses of ``you'' the \textit{CUSP you}---the \textit{Claimed Universal Subject Position}---because it appeared to refer to a generalized other, rather than the particular other with whom they were speaking.\label{def:CUSP} The CUSP you is a kind of universalized subject position that is not tied to any particular individual, but rather to a normative expectation of how one ought to act in a given situation. It is a way of speaking that assumes a shared understanding of what is appropriate or correct, without necessarily addressing the specific person being spoken to.

When I recognized others as using ``you'' as a CUSP, I also recognized that I used the term in similar ways in teaching. However, I also had privileged access to my thoughts, those inner rehearsals of speech acts. In particular, I had access to a very judgmental voice that took every opportunity to say things like ``You're stupid! You suck!'' A form of liberation seemed available if I could just figure out how to soften that hectoring voice. Finding oneself in that universalized field---not hating oneself to prove the other is loveable but still holding oneself and others to account---has been a spiritual process for me.
\section{Defining the ``You'' in Context}

CUSPs are distinguished from the \textit{transactional} \parencite{Rodl:2014aa} use of ``you'' that refers anaphorically to the specific person being addressed in a face-to-face encounter. Sebastian Rödl's work on \textit{intentional transaction} provides a useful framework for understanding this distinction. In my reading of that work, he distinguishes between monadic and dyadic forms of speech actions, where the monadic form has one subject: $Fa$, like ``Peter is painting the wall'' \parencite[307]{Rodl:2014aa}, where $a$ is ``Peter'' and $F$ is the predicate ``$\Box$ is painting the wall.'' Dyadic predication is formally $aRb$, where $a$ is ``Peter,'' $b$ is ``Paul'' and $R$ encodes the relation between them ``$\Box$ is handing the brush to $\Box$'' in a sentence like ``Peter is handing the brush to Paul'' \parencite[307]{Rodl:2014aa}.

Transactions have both \textit{actor} and \textit{patient} roles. His paradigmatic example is gift-giving, where the actor gives a gift to the patient who receives it. He argues that removing the patient's contribution to a transaction means that it has not occurred. Communicative action is essentially dyadic, not monadic.

When I taught a seminar on that research project, participants learned to recognize when they were using ``you'' to refer to the generalized other, rather than the particular other with whom they were speaking. By the way they would quickly try to claw back some usages of the general and clarify that the other individual was their referent, I gathered that they felt like using ``you'' in that way was a kind of violation. Emmanuel Levinas' work on the face-to-face encounter with an absolute other suggests that subsuming the other into what I have internalized without leaving room for the other's dissent does have an ethical implication.

To explore these questions, I want to share a story that began after I defended my dissertation. When I began dating Amalia, I entered into a rich context for exploring development and intersubjectivity, particularly through my interactions with her twin daughters, $\mathcal{M}$ and $\exists$, who were $3 \frac{1}{2}$ years old at the time.
\section{Analyzing Communicative Practice}

Initially, I approached my role as a kind of playful participant-observer. But when we all caught COVID and ended up quarantining together, I realized that the dynamic would need to change if we were going to share living space. I would speak in elliptical parentheticals---around and around---and sometimes they would smile like ``huh?!,'' sometimes they would try to teach me how to speak more competently, and sometimes communication broke down in spectacular fits. Rather than an impasse wherein the assumption of competency transformed into the assumption of incompetency, such moments led me to invent the character of Big Gorilla, who eventually became entitled the name Daddy Gorilla. Big Gorilla had a couple of baby gorillas who were always getting in trouble. He didn't know how they got there or why they were hanging around him so much. He would walk around with his fists on the ground and say, ``Big Gorilla tired, but Baby Gorilla still awake. Baby Gorilla sleep now, then Big Gorilla sleep.''

Part of Big Gorilla's behavioral regimen was to thump his chest when he was angry about something. ``Baby Gorilla hit other Baby Gorilla. Hooohoohoo thumpthumpthump.'' I tried to keep anger in a space of play, not fully identifying with it. The kids were very small and easily frightened. At night they would often get ornery about going to bed. Big Gorilla would say ``hoohoohoo'' and thump his chest, and the Baby Gorillas would smile and thump on their bellies and go ``hoohoohoo.'' It was very cute for a while.

To understand how the Big Gorilla story illustrates the emergence of meaningful communication, the analysis begins with George Herbert Mead's foundational argument that the self is not a pre-given entity but an achievement that unfolds through social interaction. Mead begins his analysis with what he calls the \textit{conversation of gestures}---the kind of interaction we might observe between two dogs preparing to fight. For Mead's dogs, stimulus and response form a feedback loop where each dog's act serves as a stimulus that provokes a response from the other dog, which in turn modifies the first dog's behavior.

Once those cycles have been compressed to the extent that a growl or aggressive stance results in the other dog backing off, there is no need to bite. The growl has the significance of the bite as a proto-inferential consequence. This is what Mead calls a \textit{significant symbol}. The transformation from gesture to a significant gesture or symbol occurs when the gesture-maker becomes conscious of the response their gesture evokes in the other. When I thump my chest as Big Gorilla and am delighted by the twins' delighted imitation, I experience my own gesture from their perspective. The chest-thump becomes a significant symbol because it means the same thing to me as it does to them: playful anger that invites rather than threatens.
\section{The Breakdown: A Call for Rational Discourse}

But one time, the anger got pretty close to me, and I thumped my chest until it hurt and hoohoohoo'ed in all capital letters. Tears bloomed and Amalia looked at me in a way that said ``back off,'' and so I said, ``Oh, that was too loud. Big Gorilla sorry. Big Gorilla get frustrated when Baby Gorillas don't listen.''

This moment of breakdown requires a shift from Mead's framework to Jürgen Habermas's theory of communicative action. While Mead gives us the foundation for understanding how selves emerge through symbolic interaction, Habermas extends this understanding to explain how rational communication itself is structured. Habermas distinguishes between two modes of social interaction that are crucial for understanding what happened in the Big Gorilla story.

Most everyday interactions proceed smoothly through what Habermas calls \textit{communicative action}. Communicative action unfolds under the assumption of a shared desire to reach understanding. When I first began thumping my chest as Big Gorilla, the twins and I quickly developed a shared understanding about what this gesture meant. The twins learned that Big Gorilla's chest-thumping was playful, not threatening, not through explicit instruction but through the accumulated context of those interactions.

Communicative action can break down. When the background consensus is challenged or disrupted, a shift to what Habermas calls \textit{rational discourse} is required. This is exactly what happened when I thumped too loudly and Amalia gave me that look. The taken-for-granted understanding that Big Gorilla was safe and playful was called into question.

His theory of communicative action distinguishes between different validity claims which participants in communication implicitly or explicitly raise and seek to redeem: (1) \textbf{Truth claims} about objective reality (``The chest-thumping was too loud''), (2) \textbf{Rightness claims} about social norms and appropriateness (``Adults shouldn't frighten children''), and (3) \textbf{Sincerity claims} about one's own subjective states (``I was genuinely frustrated''). When I said, ``Oh, that was too loud. Big Gorilla sorry. Big Gorilla get frustrated when Baby Gorillas don't listen to Momma Gorilla,'' I was engaging in rational discourse. I was making explicit claims about truth/facts (too loud), acknowledging the inappropriateness of my action (rightness), and revealing my internal state (sincere apology). This verbal response served to repair the breach in our communicative relationship and reestablish the normative boundaries of our shared practice.
\section{Reflexivity and the Limits of Formalization}

While Mead and Habermas provide powerful insights into the social nature of selfhood and communication, Robert Brandom's analytic pragmatism offers precise tools for understanding how meaning and use relate to each other. Brandom's approach is grounded in a separation that governs the methodology of \textit{meaning use analysis}: \textit{vocabularies} ($V$) and \textit{practices-or-abilities} ($P$). A vocabulary $V$ is what is \textit{said}. A practice $P$ is what is \textit{done}, expressed as a pattern of activity. For Big Gorilla, this included the ability to thump my chest, but also the rules for how to put together strings of `hoohoohoo's.

Brandom contributes to action theory by using theoretical automata to express what it means to conduct an action with words. These machines read and write formal languages. An act begins with an impetus to act. As the act unfolds, the actor engages in self-monitoring, reflecting on their act to determine whether it will satisfy the desire or need that prompted the act. This is a movement from the first-person position to the second-person position, indicating that reflection is part of the act.

Brandom formalizes the relationship between practices and vocabularies through two complementary notions of sufficiency: \textbf{PV-Sufficiency (Practice-Vocabulary Sufficiency)}: A practice $P$ is sufficient for deploying a vocabulary $V$ if mastering that practice essentially equips you to use that vocabulary. \textbf{VP-Sufficiency (Vocabulary-Practice Sufficiency)}: A vocabulary $V$ is sufficient to specify a practice $P$ if it affords the ability to explicitly formulate the rules of the practice. A \textbf{pragmatic metavocabulary} $V'$ is the language used to make the implicit rules of practice $P$ explicit.

But here it is necessary to acknowledge a crucial limitation of the formal apparatus. The theoretical frameworks we have explored---powerful as they are---cannot capture the full depth of what occurred in the Big Gorilla story. The transition from understanding (\textit{Verstand}) to reason (\textit{Vernunft}), from mechanical rule-following to genuine recognition, involves dimensions that resist algorithmic elaboration.
\section{Integration: The Recursive Structure of Intersubjectivity}

Let me continue the story to reflect this point. The Beast of Love poem that I referenced earlier was both an apology and a reflection on the complexities of love and anger--the beast within that can both harm and heal. It acknowledged my fears that my own struggles might impact these tender lives before they had a chance to understand their own depths. The idea that my fear--expressed as anger--could be something others were afraid of was a sad--but important--moment in my development.

A year or so after I invented the character, I came down with a terrible cold. We had just moved in together, and my poor old body was not conditioned for the kinds of communicable diseases that float around daycares. I was sick for months. In the middle of that period, we were eating dinner in our new house. The kids were ornery about something or other, and I SLAMMED my fists on the table, shouting ``ENOUGH!'' The kids started crying and they and Amalia decamped to visit Amalia's parents a few doors down. While they were away, I wrote them a note expressing my apology and left it on the stairs for them to find when they returned. Later that evening, Amalia came into the guest room where I had been isolating. She said, ``We read the note, and the girls wanted to come in here and tell you that they forgive you and love you.''

It didn't stick. I needed forgiveness from the kids. Then $\exists$ banged her fist on the bed, and then started thumping her belly like Baby Gorilla. She said something about how strong I am. I felt as though she was expressing a deep understanding---validating that part of my self, that I was loathing in bed, that needed to be understood as in control. I read her as saying ``oh, look at how strong you are, you be Daddy Gorilla and I'll be Baby Gorilla, hoohoohoo.'' In banging my fist on the table, I was violating our norms, but in thumping her chest as a response, it felt like she folded that action back into the norms of the family. Daddy Gorilla just got a little too wild. It felt like my anger as a parent, which I can't seem to fully transcend, wasn't as dangerous to express. Her act of recognition broke my heart.

What strikes me most profoundly about $\exists$'s gesture is that it demonstrates something that goes beyond our formal analysis. Her chest-thumping was not just a response within the established practice---it was a creative, forgiving, and ultimately transformative act that exceeded the boundaries of our existing vocabularies and practices. It was a moment of genuine recognition that created new possibilities for meaning rather than simply deploying existing ones.
\section{Conclusion: Beyond the Algorithm}

So, who are you? The answer this chapter proposes is that ``You'' are not a static entity but a dynamic and necessary position in communicative action. Speaking assumes a listener, a ``you'' who might possibly understand. Such assumptions are the bedrock of intersubjectivity, though it is a shifting ground, subject to negotiation when the assumption is no longer viable. You are the particular other whose face-to-face encounter grounds my speech in a concrete reality. You are also the universal other whose internalized norms I appeal to for justification and shared meaning. And you are the interlocutor with whom the boundary between these two poles can blur, revealing the fluid, negotiated nature of a shared world.

This dyadic structure is precisely what allows for the developmental synthesis described in the previous chapter. The ``Beast of Love'' is tamed not in solitude, but in the reciprocal I-You encounter where participants learn to balance the need for \textit{in}finite, authentic expression with the need for finite, normative recognition. This process is not a logical deduction but a lived, recursive, and often challenging practice. It is the practice of building a ``we''---a shared space of meaning, a common history, a recognitive community.

In understanding the different roles the ``you'' can play in communication fosters greater attunement to the ethical dimensions of dialogue. It becomes easier to recognize when a particular person is being addressed and when universal norms are being invoked. This sensitivity helps ensure speech opens space for the other's response rather than foreclosing it.

Most importantly, becoming a self is not a solitary achievement but a collaborative one. The \{I\} emerges only in relation to a ``You,'' and the ``You'' exists only in the context of shared practices and mutual recognition. Selves take shape through the patient, recursive, often difficult work of building understanding with others. In this work, there are no final answers, only an ongoing commitment to meet each other with care, curiosity, and a willingness to be changed by what is created together.

This is why the algorithm must be recursive--why it must be prepared to start over, to revise its fundamental assumptions. The ``you'' I am addressing at the end of our interaction is not the same ``you'' I addressed at the beginning, just as the \{I\} who speaks these final words has been shaped by the process of trying to articulate what it means to recognize and be recognized by another. This recursive, constitutive character of the I-You relation suggests that intersubjectivity has what we might call a \textit{transcendental-like} structure. I cannot point beyond the assumption of communicative competency, for doing so just assumes communicative competency at some other level. What makes intersubjectivity transcendental-like is that it is possible to tell stories about how it arises. But it must be assumed for any of those stories to have force.

In the next chapter, I will continue to explore the ``you'' in the context of \textit{limits of thought}. The kind of creative responsiveness that $\exists$ demonstrated invites reciprocal grace to the family in ways that make all kinds of `modeling' feel obtuse, no matter how sophisticated those models are. I will explore the limits of knowledge and argue for why poetic and musical forms of expression are necessary to capture such moments of terrible, ferocious beauty.

\printbibliography[heading=subbibliography]