\chapter{Numerals are Pronouns}
\begin{abstract}
This chapter explores a novel, human-centered approach to understanding
mathematics, arguing that numerals and number words function as
first-person pronouns. It begins by recounting the author's experience
with a student struggling with mathematical concepts, leading to a
reconceptualization of number grounded in self-consciousness. The
chapter introduces ``The Exercise,'' a practice of introspective
listening designed to provide an embodied understanding of the core
concepts. This exercise explores the interplay of resistance, naming,
and the desire for presence within subjective experience. The chapter
then connects these experiences to the null representation (\(∅\)) and
determinate negation, arguing that the null representation symbolizes
the unrepresentable ``I think'' -- the pre-conceptual ground of
experience. This ``I'' is not an object of experience but the condition
for having experiences. The chapter argues that this framework aligns
with Brandom's analysis of reference and the implicit recognitive
abilities required for understanding. Finally, it proposes that
grounding mathematical understanding in self-recognition does not lead
to relativism, but rather motivates the pursuit of mathematical
correctness as a form of authentic self-recognition. The reader can
expect to gain a new perspective on the foundations of mathematics,
linking abstract concepts to the lived experience of self-consciousness.

\end{abstract}

\section{Introduction}

What are numbers?  This seemingly simple question became agonizingly complex when voiced by a fifth-year senior, $\mathcal{I}$, in Indianapolis. He couldn't graduate with his friends - couldn't move on with his life - because he could not pass a high-stakes standardized test of algebraic knowledge. I was teaching a test-prep course at a high-needs urban school, and the classroom was full of super-seniors crushed by systemic forces. I felt like a conduit of those forces who had a choice to either enact the expectations of my position or treat the people around me as people. You see, it was the first week of class in my first year as a high school teacher. The night before, one of $\mathcal{I}$'s friends was killed in a car accident. $\mathcal{B}$ was fleeing the police when his SUV rolled over, killing him. The night before beginning his super-senior year, where he would have been taking the same test prep class as $\mathcal{I}$, $\mathcal{B}$ was gone. The students were grieving their friend and bitter about taking another stupid algebra class. I had the choice to listen and grieve, but I chose to jump right into algebra. As I yammered on about $x$ and $y$, $\mathcal{I}$ seemed attentive to the lesson, where other students were not; he was a kind of ally in a hostile room. When he looked up and said ``Mr. Savich, what even is two?,'' I wanted to share my most beautiful truth. I got out my markers and started drawing von Neumann ordinals and lectured about the nothingness embedded everywhere you look. 

Any sense of allyship slipped away as the light of connection withdrew. The connection between numbers and the empty set had been a revelatory moment for me when I studied pure mathematics at Earlham College. I could not `unsee' the nothingness that connected all forms of thought because it was not `there.' But whatever spiritual import that revelation had was lost in the profound disconnection I experienced with $\mathcal{I}$. I thought I should be able to answer such a simple question somewhat simply. It reinforced a disturbing suspicion: mathematical knowledge, far from being a universal language of reason, can appear as an arcane tool of subjugation, a barrier rather than a bridge. As a math teacher, my role became clear: I was the instrument of suppression. Sick at heart with uselessness, what had once been a profound and beautiful practice became hard to sustain. I eventually burned out of teaching, though the fundamental problem of what we are talking about when we talk about numbers lingered. 


I was given the opportunity to reflect on the problem and found an answer that I wish to share with $\mathcal{I}$, though I do not know where he lives or who he has become in the intervening decade. Reflecting deeply on $\mathcal{I}$'s question, I arrived at an answer that offers, if not certainty, then at least a shift in mood: numerals, like ``2,'' and number words like ``two,'' are best understood as \textit{first-person pronouns.}  This is the central idea of this paper, and it offers a radical departure from traditional ways of thinking about the ontology of number, as I propose number exists as \{I\} or \{you\} exists: number is an autonomous subject.  But what does it mean for a numeral to be a pronoun? And why is this perspective significant, especially for mathematics education?

While my answer to the question, ``What is 2?'' is simple enough for someone in $\mathcal{I}$'s position to contemplate and make their own meaning from, for the premise to hold much academic value, I must situate it in some technical language and make some kind of argument with falsifiable premises. I will try to be as gentle as possible with the jargon because the point is to articulate what numerals are for anyone who uses them. I will provide a glossary of the technical terms as an appendix to this paper, even though presenting a static definition of a term damages the term as surely as touching a stalactite arrests its growth. 

Indexicals are context-sensitive words like \{I, here, now\}. `Context' is that river of motility that Heraclitus claimed one could not set foot in twice. But try to grasp the meaning of this ``\textit{now}'' in its immediate particularity.  Say, ``Now it is night'' \parencite[60]{hegel1977phenomenology}.  As soon as we think about this ``now,'' as soon as we try to pin it down, the moment of the ``now'' has already slipped away, washed away in that Heraclitan river.  Whatever time it is right now has passed before the word can leave the speaker's lips. Each now is \cancel{now}. In fact, the desire to talk about the ineffable particularity of the present moment is profoundly frustrated by the fact that each word we might possibly use to refer to that particularity is simply another universal \parencite[58-66]{hegel1977phenomenology}. This is the paradox of indexicality:  to speak of the indexical is already to have moved beyond its pure, present immediacy. While we may wish to be known in our utmost particularity, any attempt to secure that knowledge for oneself or from another is bound by the intrinsic universality of language. 

The same holds for the ``I,'' when that term is taken as the necessarily implicit source of action. \footnote{Empirical utterances of \{I\} are \textit{anaphoric} pronouns recollecting the social being who utters the term, but when discussing human being in the context of theory or philosophy, it is helpful to import George Herbert Mead's distinction between the self-as-recognized, which he calls the ``me,'' and the self-as-source-of-action. I will embrace the \{I\} when speaking of the necessarily implicit source of action. Otherwise, \{I\} shall mean the person who purports to have authored this paper.} As soon as that source is recalled, it moves on to some other activity. This other activity is usually a monitoring activity from the second-person position. As an action unfolds, the \{I\} separates from the action to reflect on whether the impetus to act (a desire) is satisfied through the act. Successful acts fulfill that desire. My act towards $\mathcal{\text{I}}$ was anything but successful, which is one reason why I was so profoundly uneasy about the practice of doing mathematics until I found some way to repair the misrecognition. 

The \{I\} ``never is what it is and always is what it is not'' \parencite[174]{Hyppolite1974}. When we utter ``\textit{I am},'' however unfalsifiable this claim may be, the ``\textit{I}'' we reference, the source of that utterance, is no longer simply \textit{being}, but has moved on, already becoming something else.  \textit{The act of trying to grasp the indexical ``now'' or \{I\} in its immediacy reveals its inherent ungraspability, its fleeting, transient nature.}  Taken as an indexical, there is no `thing' to grasp when we try to fix the raw immediacy of awareness called consciousness. I am no more assured of my existence when I try and catch that shadow than I am when assured that I could capture the \{now\}. This claim may seem to contradict ordinary experience that involves a certain degree of self-monitoring. But that monitoring involves taking a second-person position on conscious awareness; the moment we try to capture awareness - to be conscious of consciousness -  we are no longer in that Heraclitian river but are sitting on its banks. 



Yet, to be communicatively useful, this fleeting indexical \{I\} needs to become repeatable, shareable, something that can be referred back to in discourse.  This is where \textit{anaphora} comes in.  \textit{Pronouns, fundamentally, are anaphoric terms.} They refer back to something previously introduced in the discourse, creating repeatability structures \parencite[549]{brandom1994making} from what would otherwise be fleeting and context-bound.  Consider the ``it'' in ``This chalk is white. \textit{It} is also cylindrical, and if \textit{it} were to be rubbed on the board, \textit{it} would make a mark'' \parencite[150]{Brandom:2019aa}. ``It'' refers back to ``this chalk'' in a similar way that I argue that numerals function anaphorically.  But what do they refer back to?

This opening section has traced the path from a moment of profound pedagogical failure to a radical reconceptualization of numerical reference. $\mathcal{I}$'s question -- ``What even is two?'' -- emerged from grief and educational alienation, exposing how traditional mathematical ontology can become a barrier rather than a bridge to understanding. The analysis of indexicals like \{I\} and \{now\} reveals that attempts to grasp pure immediacy always encounter their own ungraspability, yet this limitation opens the possibility for meaningful communication through anaphoric reference. 

Rather than treating numerals as names for abstract objects -- the approach that failed so dramatically in that Indianapolis classroom -- they can be understood as pronouns that recollect the enabling conditions of thought itself. This shift from asking ``What is 2?'' to asking ``What does '2' recollect?'' prepares us to explore how mathematical reference emerges from the structures of human self-consciousness and recognition.

My claim is that numerals are \textit{first-person recollective pronouns.}  Numerals like ``2'' do not name abstract objects.  Instead, they \textit{anaphorically recollect} the enabling conditions of thought itself, specifically the structure of self-recognition.  They are pronouns for the ``me'' in relation to the ``I,'' the self-as-recognized reflecting back on the self-as-source-of-action.  This act of recollection, structured through language and social inter\-action, pulls structure out of the flux, creating something repeatable and communicable from the inherently ungraspable ``I think.''

Now, to illustrate the traditional way of thinking about numbers that my pronoun thesis challenges, consider a simple joke told by the comedian Hal Roach:

\begin{quote}
This fellow decided to audition for a play put on by his local community
theater. He got the part, and, at the very end of the play, his one line
was to say, ``It is.'' So he walked around town for weeks, practicing
his line, saying, ``It is\ldots{} it is\ldots{} it is\ldots'' When the
day of the play arrived, and it was his time to deliver his line, he
stood up and said, ``Uh\ldots{} is it?''
\end{quote}

The humor is mean-spirited if the joke is on the actor for buffing a line. To me, the humor comes from the fact that I can readily imagine an actor whose one line is to voice agreement. The last play I was ever in, I had one line: ``Lefty's dead.'' I rehearsed it until I was cured of acting. Since we do not know what ``it'' refers to, the humor comes the actor treating ``it'' as if it were a thing in itself, something whose existence or nature can be directly questioned -- ``Is \textit{it}?''.  This ``Is \textit{it}?'' question, I suggest, mirrors a common, yet ultimately misguided, approach in the philosophy of mathematics.  Philosophers and even mathematicians often ask, in various forms, ``Is \textit{2}?''  ``Does \textit{number} exist?'', as if ``2'' or ``number'' were like the ``it'' in the joke -- some pre-given object whose manner of being we need to ascertain.  They delve into complex ontological debates, asking: Do mathematical objects exist in a Platonic heaven? Are they abstract entities? Are they truth values, or something else entirely? \parencite{Linnebo,sep-nominalism-mathematics}  Such questions are generative and produced nuanced debates, but they share a common false start: they treat numerals as if they are fundamentally trying to \textit{point to} or \textit{represent} something external to language itself -- be it an abstract object, a concept, or a truth value.

I include the joke as it gets at the heart of what Wittgenstein calls a peculiar form of philosophical puzzlement: once we recognize that numerals are anaphoric terms recollecting something else, there is no need to ask what manner of existence those terms have. The question becomes: What is the manner of existence of that which numerals recall? That is, instead of asking ``Is \textit{2}?'', we should ask ``What does '2' \textit{recollect}?''  This shift in perspective is crucial, especially for mathematics education.  For while formal meta-mathematics may concern itself with the internal consistency of axiomatic systems, a meta-mathematics relevant to education must engage with the human dimension of mathematical understanding.  It must ask: How do \textit{people} come to understand numbers? How can we make mathematics a bridge, not a barrier, for students like $\mathcal{I}$? What silences them, and what empowers them to speak mathematically?

In this paper, I begin to answer these questions by exploring the idea that numerals are first-person recollective pronouns. They are \textit{anaphoric} terms that, at their most fundamental level, recollect the enabling conditions of thought -- the ``I think'' that makes any judgment possible.  By shifting our focus from the ``being'' of numbers to their function as recollective pronouns, we can begin to move beyond the traditional impasses of the philosophy of mathematics and towards a more pragmatically grounded understanding of numeracy, one that is deeply relevant to the practice and purpose of mathematics education. In what follows, I will elaborate on this idea, drawing on the philosophical resources of inferential pragmatism and transcendental philosophy to unpack the nature of this ``recollection'' and its implications for our understanding of number and human being.

\subsection*{The Null Representation}\label{def:null-representation}

To understand how numerals can be first-person pronouns, I need to introduce a crucial concept: the \textbf{null representation}, symbolized as $\emptyset$ or \{\}.  This might seem counterintuitive. How can ``nothing'' -- the null, the empty -- be a representation, let alone a key to understanding numbers as pronouns?  To share the insight, I need some finite form through which to express the \cancel{finite}. I will start with Kant's transcendental insight, though I hope readers will come to understand that there is nothing specifically Kantian about the following argument. 

Kant argued that the ``I think'' ``must be able to accompany all my representations'' \parencite[77]{Longuenesse2017}.  Think back to our discussion of ``I am.''  For any experience to be \textit{mine}, for there to be a coherent flow of consciousness, there must be a unifying ``I,'' a subject to whom these experiences belong.  This ``I,'' often called the transcendental ego, is not itself an object of experience.  It's not something ``out there'' that we can perceive. Instead, it is the condition that \textit{makes experience possible}.

As explored earlier, this unrepresentability of the ``I think'' leads to a profound paradox: ``I am not myself.''  This paradox reflects the inherent division within self-consciousness, the tension between the \{I\} as source of action and the ``me'' as self-as-recognized, and the potential for the ``violence of misrecognition'' \parencite{Taylor1994}.  Like Grover in ``There's a Monster at the End of This Book,'' we seem perpetually chased by an ungraspable self.

But this inherent implicitness, this necessary inability to fully objectify the subjective ground of thought, is not unique to self-consciousness.  It is, in fact, a fundamental feature of reference itself.  Consider a simple inference based on identity, as Robert Brandom (1996) elucidates.  Suppose we reason:

Premise 1: This paper is green.
Premise 2: This paper is my to-do list.
Conclusion: Therefore, my to-do list is green.

This inference seems straightforward, mirroring the algebraic form: If $\Psi a$, and $a=b$, then $\Psi b$.  However, as Brandom points out, and as we can readily see, there's no guarantee this inference is valid in everyday discourse.  Imagine I point to a green piece of paper when uttering the first premise, and then gesture towards a yellow notepad when uttering the second -- ``This paper is my to-do list.''  My to-do list, written on yellow paper, is \textit{not} green, even if ``this paper'' in the first premise \textit{was} green.

The problem is securing co-reference.  How do we ensure that ``this paper'' in the first premise refers to the \textit{same} ``this paper'' in the second premise, or, more generally, that the first instance of 'a' is truly co-referential with the second 'a'?  As Brandom argues, securing reference, even for seemingly simple indexicals like ``this,'' requires an implicit recognitive ability.  We must be able to \textit{recognize} that we are, in fact, talking about the same referent across different utterances, different contexts.  But this recognitive ability itself cannot be made fully explicit without undermining the act of reference.  There is always an implicit ``background'' of recognitive capacities that underpins our ability to refer at all.

This necessity of implicit recognitive capacity for reference provides a powerful analogy for, and further justification of, the \textit{null representation}.  Just as reference relies on a necessarily implicit recognitive ability that cannot be fully objectified, so too does thought rely on the necessarily implicit, unrepresentable ``I think.''  The \textit{null representation} $\emptyset$ symbolizes this shared structure of implicitness, this groundless ground that underlies both reference and self-consciousness.

Now, it is vital to clarify:  grounding mathematical understanding in self-recognition is emphatically \textit{not} a license for mathematical relativism or the validation of nonsense.  \textit{Self-recognition, in this context, is not about subjectively \textit{inventing} mathematical truths, but about authentically \textit{recognizing} the structures of validity that are inherent in mathematical thought itself.}  Just as misrecognition in social contexts can inflict a ``grievous wound'' \parencite{Taylor1994}, so too can mathematical misrecognition -- getting it wrong, misunderstanding, failing to grasp a valid proof -- be a form of intellectual and even existential discomfort.

Indeed, the possibility of \textit{misrecognition} presupposes that there \textit{is} something to be recognized correctly.  Authenticity in mathematics, therefore, demands a commitment to rigor, to logical coherence, to the careful construction of valid inferences.  The pain of mathematical error, the frustration of misunderstanding, is not merely an intellectual inconvenience; it is a sign that we have, in some sense, misrecognized ourselves as mathematical thinkers.

Thus, the drive for mathematical correctness, the pursuit of valid proofs and sound reasoning, is not just about adhering to external rules or conventions.  It is, at a deeper level, motivated by a fundamental human need for authentic self-recognition, a desire to align our mathematical thinking with the inherent structures of reason and validity that make mathematical understanding possible in the first place.  This desire for authenticity, for avoiding the pain of mathematical self-misrecognition, is a powerful engine for mathematical inquiry and a crucial aspect of what makes mathematics a meaningful human endeavor, not just an arbitrary game.

Therefore, I propose we use the symbol $\emptyset$ to represent this necessarily implicit, yet unrepresentable, ``I think'' -- the groundless ground of self-consciousness and reference, the ``monster at the end of the book'' that is both terrifying and, ultimately, just ourselves.  Think of it as a placeholder for that which is always already there, underpinning every thought and every act of reference, but which recedes from direct conceptual grasp.

To arrive at this symbol, consider the phrase ``I think.'' We can represent it initially as \{``I think''\} -- enclosing it in braces to indicate its implicit nature.  Then, recognizing that the specific \textit{content} ``I think'' is less important than its function as the unrepresentable condition, we can strip away the content, moving towards \{\cancel{I think}\}.  Finally, we arrive at \{\} and then $\emptyset$.  This symbolic move -- substituting \{\} for quotation marks around ``I think,'' while emptying the braces -- aims to capture the sense of unrepresentability.  To summarize the symbolic derivation:

``I think'' $\rightarrow$ \{``I think''\} $\rightarrow$ \{\cancel{``I think''}\} $\rightarrow$ \{\} $\rightarrow$ $\emptyset$.


Thus, the \textit{null representation} $\emptyset$ is not meant to represent ``nothingness'' in a simple sense.  Instead, it symbolizes the unrepresentable, yet necessarily presupposed, transcendental ``I think'' -- the enabling condition of all thought and representation, and the implicit recognitive capacity that makes reference itself possible.  It is, in this sense, the groundless ground of our being as thinking, speaking, and referring subjects, a ground that demands authenticity and correctness in our mathematical self-understanding.

The null representation emerges as a crucial bridge between the ungraspable ``I think'' and the mathematical structures that follow from it. This symbol $\emptyset$ captures the paradoxical nature of consciousness itself: necessarily presupposed yet never directly accessible, the enabling condition that makes all thought possible while remaining beyond full objectification. The parallel with Brandom's analysis of reference shows that this structure of implicitness is fundamental to meaning-making generally -- we cannot secure co-reference without relying on recognitive abilities that themselves remain implicit. In mathematics, this means that numerical understanding cannot be grounded in abstract objects but must acknowledge its roots in the self-recognitive structures of human consciousness. Like Agamben's Voice (recall \ref{def:Voice}) that enables speech through its own silence, the null representation thus becomes not an empty void but a full acknowledgment of the groundless ground from which mathematical meaning emerges, preparing us to see how numerals function as anaphoric pronouns recollecting these enabling conditions.

Now, you might object: isn't this paradoxical?  Representing the unrepresentable with a representation?  Yes, it is paradoxical.  And this paradox is not a flaw, but a crucial feature of the null representation.  It reflects the inherent limitations of any attempt to fully objectify or capture the subjective source of our own thought, the ``monster'' we can never quite face directly.  But it also points towards the possibility of liberation. For when mathematical claims are grounded in a recognition of this inherent self-division, when they acknowledge the unrepresentable ``I think'' at their core, they can become tools not of subjugation, but of self-affirmation and empowerment.  This is the emancipatory potential of a mathematics that embraces its own groundlessness. % [voice-neutralize] 

We return to this paradox and its implications in the discussion section. For now, simply accept the null representation $\emptyset$ as the symbol for the unrepresentable ``I think,'' the implicit ground of all representation, and move on to see how it plays a role in the anaphoric nature of numerals.

\subsection*{Anaphora}\label{def:anaphora}

Having introduced the null representation $\emptyset$ as a symbol for the unrepresentable ``I think,'' we now turn to the concept of \textit{anaphora}.  


As discussed in the introduction, pronouns are fundamentally \textbf{anaphoric} terms.  Understanding anaphora, and specifically recognizing it as a form of \textit{linguistic self-reference}, is crucial to grasping why I argue that numerals are, at their core, first-person pronouns and why mathematics itself can be seen as a product of language reflecting on itself.

In essence, \textit{anaphora} is not just a mechanism for repeatability; it is a fundamental form of \textit{linguistic self-reference}.  It is how language turns back on itself, allowing a word or phrase to refer back to something already present within the discourse itself. Consider the example: ``This chalk is white. \textit{It} is also cylindrical.''  The pronoun ``it'' here is an anaphoric term, and it exemplifies linguistic self-reference in action.  ``It'' doesn't point to something outside of language; instead, it \textit{recollects} the content of the earlier phrase ``this chalk,'' referring back to its \textit{antecedent} within the same linguistic context.  Anaphora is, in this sense, language referring to its own prior utterances, creating a chain of reference within discourse itself.

Robert Brandom (2008) emphasizes that \textit{indexical} terms like ``I,'' ``here,'' and ``now'' are inherently context-sensitive. Their semantic content shifts with each new utterance, like a Heraclitan river constantly flowing.  If we could only use indexicals directly, our discourse would be trapped in a perpetual present, unable to build upon past utterances or establish lasting connections between different parts of a conversation.  Anaphora solves this problem, and it does so precisely through linguistic self-reference.  It provides a way to ``freeze'' the fleeting content of indexical expressions, to lift them out of their immediate context and make them available for re-use within the ongoing flow of language.

Think again about ``This chalk is white. \textit{It} is also cylindrical.''  ``This chalk'' is initially an indexical expression, its reference fixed by a demonstration in a particular context.  But the anaphoric pronoun ``it'' breaks free from that immediate context through self-reference.  It picks up the conceptual content of ``this chalk'' from within the sentence itself and makes it repeatable.  Without anaphora, if we were to say ``This chalk is white'' and then ``This chalk is cylindrical,'' we might be talking about two different pieces of chalk entirely.  Anaphora, as linguistic self-reference, ensures \textit{co-reference}, allowing language to build coherent discourses and reasoned arguments that extend beyond the fleeting moment of utterance by referring back to itself.

Variables in algebra, for instance, are also anaphoric terms, exemplifying linguistic self-reference in mathematical notation.  Once a variable, say '$x$', is introduced and defined (e.g., ``Let $x$ be the number of apples''), it becomes a point of self-reference within the mathematical language.  It can be used repeatedly throughout an equation or a proof, consistently referring back to that initial definition within the symbolic system itself.

Numerals, I argue, function in a profoundly similar way, but at an even more fundamental level of linguistic self-reference. They are not merely indexical terms, nor are they simply variables within a formal system. Instead, they are \textit{anaphoric recollections} of something that precedes and enables all discourse: the enabling conditions of thought symbolized by the null representation $\emptyset$.  They are first-person pronouns that point back, not to a concrete object, but to the structure of self-conscious thought.  And because they are anaphoric, because they are forms of linguistic self-reference, they suggest that \textit{mathematics itself, built upon numerals, arises as language recollects itself, as discourse turns back to capture and structure the conditions of its own possibility.}

This analysis of anaphora reveals why numerals function as pronouns rather than names. Like the pronoun ``it'' that recollects ``this chalk'' within discourse itself, numerals engage in linguistic self-reference, creating repeatability from what would otherwise remain fleeting and context-bound. But numerals operate at a deeper level than ordinary pronouns -- they recollect not just prior content within a conversation, but the enabling conditions that make any meaningful discourse possible. When language turns back on itself through anaphoric recollection, it creates the possibility for mathematical thinking. This suggests that mathematics is not an external realm of abstract objects but an internal development of language's capacity for self-reference, emerging when discourse recognizes and repeats its own fundamental structures. Understanding numerals as anaphoric pronouns thus reveals mathematics as a sophisticated form of linguistic self-consciousness, preparing us to see how specific numbers like ``1,'' ``2,'' ``3'' emerge from the iterative process of quotative embedding and recollection.

The next section explores precisely what numerals like ``1,'' ``2,'' ``3,'' etc., anaphorically recollect in this act of linguistic self-reference, and shows how they build upon the null representation $\emptyset$ while revealing their deep connection to the first-person perspective and the act of counting as a form of self-recognition, a form of language reflecting on the conditions of its own intelligibility.
\subsection*{Number: numerals and ordinals}\label{number-numerals-and-ordinals}

It is now possible to explain how numerals, like ``1,'' ``2,'' ``3,'' etc., function as first-person recollective pronouns.  Building upon the concepts of the null representation $\emptyset$ and anaphora, the core idea is this: \textit{Numerals anaphorically recollect the enabling conditions for thought, specifically as these conditions are revealed through quotative recollection.}  In simpler terms, numerals are pronouns that point back to, and make repeatable, the structures of self-conscious thought that emerge when we reflect on our own and others' judgments.

The formula for this process is as follows:  The process begins with a thought, whose sense depends on another thought, typically in \textit{quotative recollection}.  This process, when reflected upon, reveals a nested structure that corresponds to a von Neumann ordinal.  This von Neumann ordinal, representing the enabling conditions of the initial thought, is then anaphorically recollected by a numeral, a first-person pronoun of number.

Before illustrating this with the ``Telephone'' game, a brief clarification of \textit{von Neumann ordinals} is in order. In set theory, von Neumann ordinals provide a way to define natural numbers using only set-theoretic principles, starting from the empty set.  The ordinal 0 is defined as the empty set $\emptyset$. The successor of any ordinal is then defined as the set containing all preceding ordinals.  Thus, 1 is $\{\emptyset\}$, 2 is $\{\emptyset, \{\emptyset\}\} = \{\emptyset, 1\}$, 3 is $\{\emptyset, \{\emptyset\}, \{\emptyset, \{\emptyset\}\}\} = \{\emptyset, 1, 2\}$, and so on.  While we are not concerned with the set-theoretic details here, the key idea is that von Neumann ordinals represent a nested, iterative structure built from the most basic element -- the empty set.  In this context, I propose that these nested structures, built from the \textit{null representation} $\emptyset$ of the ``I think,'' are precisely what numerals anaphorically recollect.

To illustrate this in a more engaging and everyday way, consider the game of ``Telephone.''  In this game, a message is whispered from person to person, and the fun (and the point) lies in how the message gets distorted as it's passed along.  Imagine playing ``Telephone'' with your family. Let $\mathcal{T}$ be me, $\mathcal{A}$ be my wife, and $\mathcal{M}$ and $\mathcal{E}$ be my daughters.  When it is my turn to start, I utter my thesis as the initial message: ``Numerals are pronouns.''

The game unfolds as follows:

	\textit{$\mathcal{T}$ (initial speaker) starts with the message, intending to convey my core thesis:} ``Numerals are pronouns.''

\textit{$\mathcal{A}$ (hears from $\mathcal{T}$ and whispers to $\mathcal{M}$, mishearing and slightly distorting the message):} $\mathcal{A}$, perhaps slightly mishearing or reinterpreting, whispers to $\mathcal{M}$: ``Numerals are \textit{nouns}.'' (The message subtly shifts, moving closer to a more conventional, object-based understanding of number).

\textit{$\mathcal{M}$ (hears from $\mathcal{A}$ and whispers to $\mathcal{E}$, further distortion and misrecognition):} $\mathcal{M}$, hearing ``Numerals are nouns,'' and perhaps further simplifying or misunderstanding, whispers to $\mathcal{E}$: ``Numerals are \textit{names}.'' (The message undergoes further distortion, arriving at a common, but in my view, ultimately inaccurate, conception of numerals as simply names for mathematical objects).

\textit{$\mathcal{E}$ (final player announces the message they heard aloud, revealing the cumulative distortion):} $\mathcal{E}$, having received the message ``Numerals are names,'' announces to the group: ``The message is: `Numerals are names.' ''

The humor and the point of ``Telephone'' are now evident.  The initial, nuanced claim -- ``Numerals are pronouns,'' -- has, through iterative quotative recollection and subtle misrecognitions, been transformed into its opposite, or at least, into a representation of the view we are critiquing: ``Numerals are names.''  This distortion, however, is not arbitrary.  It reflects a common tendency to understand numerals as simply names for mathematical objects, a tendency that, I argue, obscures their deeper, pronoun-like function.

Now, let's trace back the chain of recollections, asking each player, in reverse order, ``What did \textit{you} hear?''  This process of tracing back reveals the nested structure of quotative recollection and the emergence of von Neumann ordinals:

\textit{$\mathcal{E}$'s utterance:} ``Numerals are names.''  This is the final, announced message.  We can represent the implicit ``I think'' behind this utterance with the null representation: $\emptyset$: ``Numerals are names.''

\textit{$\mathcal{M}$'s recollection of $\mathcal{A}$'s utterance:} ``$\mathcal{A}$ told me `Numerals are nouns.' '' $\mathcal{M}$'s utterance is a quotative recollection of $\mathcal{A}$'s speech act.  We can represent this nested structure as: \{$\emptyset$\}: ``$\mathcal{M}$ says, `$\mathcal{A}$ said ``Numerals are nouns.'' ' ''

\textit{$\mathcal{A}$'s recollection of $\mathcal{T}$'s utterance:} ``$\mathcal{T}$ told me `Numerals are pronouns.' '' $\mathcal{A}$'s utterance is, in turn, a quotative recollection of $\mathcal{T}$'s speech act.  The structure becomes further nested: \{\{$\emptyset$\}\} : ``$\mathcal{A}$ says, `$\mathcal{T}$ said ``Numerals are pronouns.'' ' ''

\textit{$\mathcal{T}$'s original utterance:} ``Numerals are pronouns.'' This is the initial, un-recollected utterance, the starting point of the chain. While we could represent an implicit ``I think'' even here, for the purpose of tracing the \textit{quotative} recollection structure, we can consider this the base case, implicitly contained within the subsequent recollections. However, to be consistent, we can represent the very first implicit ``I think'' as well, arriving at: \{\{\{$\emptyset$\}\}\} : ``$\mathcal{T}$ says, `Numerals are pronouns.' ''


\subsection*{Discerning von Neumann ordinals in ``Telephone'' Game (Family Edition)}
\begin{longtable}[]{@{}
  >{\raggedright\arraybackslash}p{0.341\linewidth}
  >{\raggedright\arraybackslash}p{0.324\linewidth}
  >{\raggedright\arraybackslash}p{0.335\linewidth}@{}}
\toprule\noalign{}
\begin{minipage}[b]{\linewidth}\raggedright
Implicit von Neumann Ordinal
\end{minipage} & \begin{minipage}[b]{\linewidth}\raggedright
``Telephone'' Game Utterances (Family, Traced Backwards)
\end{minipage} & \begin{minipage}[b]{\linewidth}\raggedright
Numeral (Anaphoric Recollection)
\end{minipage} \\
\midrule\noalign{}
\endhead
\bottomrule\noalign{}
\endlastfoot
(None) & $\mathcal{E}$ announces: ``Numerals are names.'' & None \\
$\emptyset$ & $\mathcal{E}$: ``Numerals are names.'' & 0 \\
\{$\emptyset$\} & $\mathcal{M}$: ``$\mathcal{A}$ told me `Numerals are nouns.' '' & 1 \\
\{\{$\emptyset$\}\} & $\mathcal{A}$: ``$\mathcal{T}$ told me `Numerals are pronouns.' '' & 2 \\
\{\{\{$\emptyset$\}\}\} & $\mathcal{T}$: ``Numerals are pronouns.'' & 3 \\

\caption{\textit{Note. }This table organizes the relationship between the ``I think'' as it is recalled with the null representation, the ``Telephone'' game utterances with family characters, and numerals.}
\end{longtable}



Analyzing the ``Telephone'' game example demonstrates how numerals, as first-person pronouns, can be understood as anaphoric recollections of the nested structures of quotative recollection.  The numeral ``3,'' for instance, anaphorically recollects the structure \{\{\{$\emptyset$\}\}\}, which emerges from tracing back three layers of ``said that...'' in the game.  And crucially, the distortion of the message as it's passed along -- from ``pronouns'' to ``nouns'' to ``names'' -- highlights the tendency to misrecognize the nature of numerals that this paper seeks to address.  Numerals are not names for mathematical objects ``out there''; they are pronouns of thought, born from the dynamic, iterative, and sometimes misrecognizing, process of linguistic communication and self-reflection. Moreover, the von Neumann ordinal structure, built iteratively from the null representation, provides a formal counterpart to this iterative process of quotative recollection, showing how even the most abstract mathematical structures can be seen as rooted in the fundamental dynamics of human communication and self-consciousness.

The ``Telephone'' game provides a concrete illustration of how numerals emerge from the dynamics of human communication and recognition. The progression from ``pronouns'' to ``nouns'' to ``names'' mirrors exactly the misrecognition this chapter seeks to correct -- the tendency to treat numerals as object-referring names rather than self-referring pronouns. By tracing the quotative embedding structure backwards (what $\mathcal{T}$ said that $\mathcal{A}$ said that $\mathcal{M}$ said...), this trace shows how von Neumann ordinals capture the nested levels of linguistic self-reference. Each numeral thus becomes an anaphoric recollection of a specific depth of reflexive embedding, showing how mathematical structure emerges from the process of human beings reflecting on their own speech acts. This playful yet rigorous demonstration reveals that what are called ``abstract'' mathematical objects are actually fractal-like patterns of human self-consciousness and communicative practice. Rather than referring to a mysterious Platonic realm, numerals recollect the enabling conditions of human thinking, making them fundamentally first-person pronouns of mathematical self-recognition. 

Perhaps it is worth noting that this entire text is framed as a critical autoethnograpy. While that name is built to break, I claim it reflects the structure of mathematics itself. The method and its topic are not external to one another. They are divaded. 

In the following subsections, the implications of this view are further explored, including its connection to the successor function and the nature of mathematical proof, and the discussion returns to consider the Derridean dimensions of this iterative process of linguistic recollection, now illuminated by the playful yet profound example of the ``Telephone'' game with my family.
\subsection*{Successor function}\label{successor-function}

One of the strengths of understanding numerals as first-person recollective pronouns is that it provides a natural and intuitive definition of the \textit{successor function} -- a concept often taken as a primitive axiom in many mathematical systems.  In this framework, the successor function emerges from the interplay of two distinct, yet related, forms of recollection: \textit{quotative embedding} and \textit{anaphoric recollection}.

A clarification is in order:

\textit{Quotative Embedding:} This is the process of embedding a prior thought or representation within a new layer of quotative context, symbolized by set braces \{\}.  This embedding creates the nested structure of von Neumann ordinals.  For example, moving from the null representation $\emptyset$ to $\{\emptyset\}$ involves quotative embedding.  It's like adding quotation marks around a thought, indicating that it is being considered \textit{as} a thought, as something recalled or reported.

\textit{Anaphoric Recollection:} This is the act of recognizing and ``picking up'' a previously established conceptual content and making it repeatable and referable in discourse, symbolized by the assignment of a numeral (e.g., ``1,'' ``2,'' ``3'').  Numerals are \textit{anaphoric pronouns} that recollect these embedded structures.  For example, assigning the numeral ``1'' to the von Neumann ordinal $\{\emptyset\}$ is an act of anaphoric recollection.

With this distinction in mind, the emergence of the successor function can be re-examined:

	extit{Starting point: The numeral ``1'' as anaphoric recollection of $\{\emptyset\}$:}  I have argued that ``1'' is the \textit{anaphoric recollection} of the von Neumann ordinal $\{\emptyset\}$, which represents the structure resulting from the first act of \textit{quotative embedding} of the null representation (i.e., embedding the ``I think'' within set braces).

	extit{Thinking the successor of ``1'' (i.e., ``2''):} To think ``2,'' proceed with another two-step process:

  1.  \textit{Quotative Embedding of ``1'':} Take the numeral ``1,'' already an anaphoric recollection, and perform \textit{quotative embedding} on it, creating the structure:  \{``1''\}.  In terms of von Neumann ordinals, this means embedding the ordinal $\{\emptyset\}$ within another set of braces: $\{\{\emptyset\}\}$.

  2.  \textit{Anaphoric Recollection of the Embedded Structure:} Then perform \textit{anaphoric recollection} on this newly embedded structure, $\{\{\emptyset\}\}$.  This act of anaphoric recollection is symbolized as the numeral ``2.''

	extit{Formalizing the Successor Function:}  It is now possible to re-formalize the successor function to reflect this two-fold process:

    \textit{\textit{successor}($n$) = \textit{anaphoric-recollection}(\{\textit{quotative-embedding}($n$)\})}

  However, to simplify the notation and emphasize the iterative nature of the process, it can also be expressed recursively, using ``recollection'' to encompass both quotative embedding and anaphoric recollection as they work together:

    \textit{\textit{successor}($n$) = \textit{recollection}($n$) = $n + 1$}

    In this simplified recursive form, ``recollection'' signifies the combined act of quotative embedding a structure (represented by $n$) and then anaphorically recollecting the resulting, newly embedded structure as the successor, $n+1$.

Thinking the successor of any numeral $n$ is thus a two-step process: first, \textit{quotatively embed} $n$ (in its von Neumann ordinal form), and then \textit{anaphorically recollect} this newly embedded structure as $n+1$.  It is worth noting that the practical implementation of number systems in specific bases, such as our familiar base ten system, introduces additional layers of complexity to this fundamental successor function.  For instance, the transition from ``9'' to ``10'' in base ten involves a fascinating process of composing a new unit and ``carrying over,'' which can be understood as a form of \textit{sublation}, where a collection of units is simultaneously negated and preserved in a higher-order representation.  However, a detailed exploration of base systems and their implications for our theory of numerals as pronouns must be reserved for future work.

This definition of the successor function, in its fundamental form, is not imposed as an axiom, but rather emerges from the dynamic interplay of quotative embedding and anaphoric recollection, grounded in this understanding of numerals as first-person recollective pronouns.

This approach also sheds light on the nature of basic arithmetic operations.  Addition, for example, can be understood recursively in terms of recollection and the successor function.  While a full exploration of arithmetic is beyond the scope of this paper, briefly note that judgments like ``$2 + 3 = 5$,'' which often feel definitionally true, can find their grounding in these recursive definitions based on recollection.  Such \textit{analytic} mathematical judgments, in this view, reflect the inherent structure of normativity itself, rather than requiring externally imposed axioms for their justification.  In contrast, \textit{synthetic} mathematical judgments, such as ``two is the only even prime number,'' require material inferential rules and richer conceptual content, as explored in more detail elsewhere \parencite{savich2020towards,savich2022}.

In the next subsection, potential objections to this account are addressed and the role of the null representation in understanding the nature of mathematical systems and human being is further clarified.

\subsection*{Discussion - What is the null representation?}

To understand the ``groundless ground'' of the null representation, it is necessary to move beyond not only objectivism, but also beyond a static interpretation of even the most foundational philosophical insights, such as Descartes' \textit{cogito}.  The problem with Cartesian dualism, in its traditional rendering, is that both ``I think'' and ``I am'' are often treated as static, axiomatic principles, fixed points upon which to ground knowledge.  But the null representation, in its essence as a form of negation, challenges this static, foundationalist impulse. It points towards a more dynamic and rhythmic understanding of grounding itself, one that embraces self-negation and inherent instability, recognizing that even the most fundamental concepts are not fixed endpoints, but points of departure.

\subsubsection*{From Objects: The Limitation of Objectivity and Static Foundations}

Initially, especially within a mathematical context, one might be tempted to interpret the empty set, and thus the null representation, as simply representing \textit{objects} in their most basic, stripped-down form -- a kind of bare, objective existence.  This initial interpretation, however, not only embraces objectivism, but also risks falling into a static, axiomatic mode of thought, akin to a rigid reading of Descartes.  It treats the object as a self-grounding ground, a fixed foundation.  Think of the notion of ``ground'' in some methodological approaches, where ``'ground' grounds ground,'' implying a static, self-sufficient foundation, a fixed point $f(x)=x$ where ground is both function and argument, requiring no further movement or justification.

However, the true power of Descartes' insight, I argue, lies not in two separate, static axioms (``I think'' and ``I am''), but in their dynamic dyadic relation.  The more appropriate way to understand the \textit{cogito} is not as a linear inference (``I think, \textit{therefore} I am''), but as a rhythmic, self-sustaining structure:  \textit{I think $\leftrightarrow$ I am}.  It is a single structure with two inseparable sides: the active, self-positing ``I think'' and the recollected, self-recognized ``I am.''  Neither side is primary; they are mutually constitutive, constantly implying and negating each other in a dynamic rhythm.

Imagine Thomas the Tank Engine chugging along, repeating to himself, ``I think I am, I think I am\ldots''  Except, in this rhythmic self-affirmation, the ``I am'' would often be unspoken, implicit, the silent beat grounding the explicit assertion of ``I think.''  This rhythmic interplay between explicit assertion and implicit grounding, between thought and being, is closer to the dynamic spirit of the \textit{cogito} and to the nature of the null representation itself.

For the null representation, symbolized by $\emptyset$, is precisely this point of negation, this determinate ``no.''  But it is not a simple, static negation.  It is a ``no'' that is directed, in its finality, even at itself -- a self-negating negation, represented in the interplay between determinate negation ``no'' and its sublation \cancel{no}.  This self-negating ``no'' is, in essence, a critique of the limitations of the ``I think'' when taken in isolation, when understood as a self-sufficient, static principle.  The null representation, as this self-negating ``no,'' reveals that the ``I think,'' in its act of positing itself as ground, simultaneously points beyond itself, towards its own groundlessness, its own inherent incompleteness.  It is in this sense that the null representation becomes a symbol for the recollection of our fundamental \textit{not-a-thingness}, the recognition that the ground of our being, and the ground of mathematical understanding, is not a static object, nor a fixed principle like the isolated ``I think,'' but a dynamic, self-recollecting, and ultimately groundless process that constantly transcends any fixed representation.

Just as object-based ontologies, and static interpretations of foundational principles, can obscure the dynamic and rhythmic nature of thought and being, so too can they miss the profound significance of this ``not-a-thingness'' at the heart of mathematical understanding.  But by embracing the null representation in its paradoxical emptiness, by recognizing the ``groundless ground'' it symbolizes -- not as a final answer, but as an opening to further inquiry -- we can begin to move towards a more dynamic, self-reflective, and ultimately more authentic understanding of numerals, mathematics, and ourselves as thinking, speaking, and becoming beings.  This journey, of course, does not end with the null representation, but rather begins there, inviting us to explore the further dimensions of subjectivity, intersubjectivity, and the ever-unfolding process of meaning-making.


\section{Historical Contexts}

Throughout this chapter it has been shown that philosophies of number do not float above human interests; they are often driven by deep methodological and epistemic motives. Habermas identified three broad categories of interests that underlie knowledge: the technical \textit{empirical-analytic interest} (knowledge to predict and control, aiming at objectifying reality), the practical \textit{historical-hermeneutic interest} (knowledge to interpret and understand meanings in context), and the emancipatory \textit{critical interest} (knowledge to reflect and free ourselves from constraints). Each philosophy of number can be interpreted as foregrounding one (or sometimes two) of these interests.

The following chart (Table \ref{tab:philo-of-number}) summarizes the major philosophies of number, their core assumptions about what numbers are, and the primary knowledge-constitutive interest they serve. This schematic overview is modelled after Carspecken's reconstruction of Habermas but adapted to our subject matter (and expanded where necessary). It should be read as a guide to how each view ``sees'' mathematics and what it hopes to achieve or ensure by that view. Of course, any brief summary is an oversimplification -- many philosophers hold nuanced positions that combine elements. Nevertheless, this chart highlights the central thrust of each perspective.
 

\begin{longtable}{>{\raggedright\arraybackslash}p{3.5cm} >{\raggedright\arraybackslash}p{6.8cm} >{\raggedright\arraybackslash}p{4.8cm}}
  \caption{\textit{Note. }Philosophies of Number and Their Knowledge-constitutive Interests.\label{tab:philo-of-number}}\\
  \toprule
  \textit{Philosophy of Number} & \textit{Core Assumptions about Numbers} & \textit{Primary Knowledge Interest} \\
  \midrule
\endfirsthead

  \multicolumn{3}{c}%
  {{\bfseries \tablename\ \thetable{} -- continued from previous page}} \\
  \toprule
  \textit{Philosophy of Number} & \textit{Core Assumptions about Numbers} & \textit{Primary Knowledge Interest} \\
  \midrule
\endhead

  \midrule \multicolumn{3}{r}{{Continued on next page}} \\ \bottomrule
\endfoot

  \bottomrule
\endlastfoot

\textit{Pythagoreanism (Ancient)} & Numbers (whole numbers and ratios) are the fundamental reality and give structure and meaning to the cosmos (mystical harmony). All things can be understood as numbers. & \textit{Empirical-analytic / Metaphysical}: Sought objective cosmic order through numbers, but also had a \textit{proto-critical spiritual interest} (purification of the soul via numerical harmony).\\[1ex]
\textit{Platonism / Realism} & Numbers are abstract, non-physical objects that exist independently of human minds. Mathematical statements are objectively true descriptions of this realm of numbers (e.g. $5$ is prime is true because 5 has that property in itself). & \textit{Empirical-Analytic}: Emphasizes objective truth and discovery of mind-independent facts. (Also satisfies a theoretical interest in certainty.) Knowledge is ``seeing'' the eternal truths of a numerical world, analogous to empirical observation of nature. \\[1ex]
\textit{Logicism} (Frege, Russell) & Numbers are logical objects (e.g. classes of equinumerous sets); mathematics is an extension of logic. Seeks to derive arithmetic from self-evident logical principles. Arithmetic truths are analytic. & \textit{Empirical-Analytic}: Aims for absolute certainty and objectivity by grounding math in logic. The interest is in secure, error-free knowledge (technical control over truth via rigorous proof). It strips away any mystical or intuitive base in favor of neutral logic. (Critical of reliance on intuition\slash experience.)\\[1ex]
\textit{Formalism} (Hilbert) & Numbers are symbols in a formal game; mathematics is manipulation of formulas according to rules. Consistency of the system is the main requirement. No need to interpret symbols as long as the system works. & \textit{Empirical-Analytic}: Strong technical interest -- mathematics is a tool, valued for its utility and consistency in describing the world. Emphasizes instrumental knowledge (proofs as means to derive results). Reflection is limited (the system doesn't examine itself), focusing instead on reliable rule-following.\\[1ex]
\textit{Intuitionism / Constructivism} (Brouwer) & Numbers are constructions of the human mind. Mathematics is a mental activity of constructing sequences and structures. Only mathematical objects that can be constructed explicitly are admitted. Rejects non-constructive methods (no actual infinity or excluded middle in general). & \textit{Historical-Hermeneutic}: Prioritizes the meaning of mathematical statements as tied to our intuitive constructions. Emphasizes understanding ``how we know'' -- knowledge is an activity, not just a result. There is also a \textit{critical} element: it's a self-reflection on what mathematicians are actually doing in proofs (and a rejection of blindly following formal rules). Intuitionism was motivated by a desire for intellectual \textit{integrity} and autonomy in mathematics (no reliance on non-evident principles), aligning with a critical interest in freeing math from what was seen as metaphysical obfuscation.\\[1ex]
\textit{Structuralism} (Shapiro, Resnik) & Numbers have no inherent nature except their relations in a structure. E.g. ``2'' is the second position in the progression of natural numbers, whichever model instantiates it. Mathematics studies structures (patterns) rather than individual objects. The structure (like the Peano structure) exists objectively (in ante rem structuralism) or at least logically. & \textit{Empirical-Analytic} (in ante rem form): Still seeks objective truth, but about structures as a whole. Satisfies the technical interest by providing unique identification of mathematical truth up to isomorphism. Removes irrelevant ontological choices (no need to choose a particular representation), focusing on what objectively \textit{is} the case in all equivalent systems. Also has a \textit{hermeneutic} flavor: meaning of a number comes from context (its place in structure), highlighting interpretation within a relational system. But overall, structuralism aims at stable knowledge of abstract structures, aligning with analytic interest in invariant truths (e.g. truths that hold in any system of a given structure).\\[1ex]
\textit{Fictionalism / Nominalism} (Field, etc.) & Numbers do \textit{not} exist; mathematical entities are useful fictions. Mathematical statements are systematically false or just useful pretenses (e.g. ``3 is prime'' is not literally true because there is no 3, but it's true-in-the-story of math). Mathematics is a language that simplifies reasoning about the physical world, but has no content of its own. & \textit{Empirical-Analytic} (naturalist branch): Guided by an ontology limited to what empirical science needs. Stresses empirical content: any legitimate statement about the world can ultimately be stated without referring to abstract numbers. Also a \textit{Critical} stance: demystifies mathematics, revealing it as a human convention or fiction (preventing reification or blind faith in ``invisible'' entities). By treating math as fiction, it implicitly warns against the tyranny of numbers (don't confuse the tool with reality) -- a faint critical-emancipatory note.\\[1ex]
\textit{Empiricism / Naturalism} (Kant, Mill historic; modern cognitive science) & (Historic empiricism: arithmetic truths are generalizations of experience -- largely discredited by Frege's critique.) Modern naturalized views: Number concepts arise from sensory experiences and neural structures. Math is reliable because it evolved to track aspects of reality (e.g. numerosities). Emphasis on how practice and experience inform mathematics. & \textit{Empirical-Analytic}: Seeks to ground even mathematics in the same framework as empirical knowledge. Kant saw arithmetic as synthetic \textit{a priori} (an unusual blend: requiring intuition of time to construct succession, but necessarily true given that intuition) -- a mix of analytic interest (necessity) and a form of experiential element. Contemporary cognitive views bring mathematical knowledge into the realm of empirical psychology (an analytic interest in explaining reason, and a hermeneutic interest in meaning via bodily experience). If math is seen as an extension of empirical cognition, the interest is in continuity with science and human biology.\\[1ex]
\textit{Social Constructivism} (Hersh, Ernest) & Numbers are cultural-historical constructs. Mathematics is a product of human culture, maintained by social processes (proof, peer review, teaching). Its objectivity is ``socially constituted'' -- once a number system is established, truths about numbers are objective \textit{within} that cultural framework. But they are not eternal Platonic truths, rather intersubjective agreements. & \textit{Historical-Hermeneutic}: Strong focus on the historical context and communal consensus in the creation of mathematical knowledge. The interest here is in understanding mathematics as part of our shared meaning-making practices. It demystifies number by showing its development (e.g. the invention of zero, negatives) in response to societal needs. Also hints at \textit{critical} interest: if we see math as our own construction, we can adapt or change our mathematical practice deliberately (empowerment to change curricula or methods, for instance).\\[1ex]
\textit{Critical Theory of Mathematics} (e.g. critique of quantification, ethnomathematics) & Mathematics (and the concept of number) is not value-neutral; it can carry ideological power (the authority of quantification). Western formal mathematics might suppress other ways of knowing. A critical approach examines how the use of numbers can dominate (e.g. in technocracy) and seeks to humanize or democratize mathematics. & \textit{Critical-Emancipatory}: Explicitly aims for self-reflection and liberation: to free us from the ``spell'' of numbers as infallible or from using numbers in oppressive ways. Encourages reflecting on the role of mathematics in maintaining systems of control (e.g. economic statistics governing policy). By recognizing number as a human tool, we can reclaim agency in how we use or interpret it. In Habermas's terms, it re-incorporates the context of meaning and human purpose into a domain often treated as purely technical.\\
\end{longtable}

As Table \ref{tab:philo-of-number} suggests, the philosophy of number is a rich field where metaphysical, epistemological, and even ethical considerations intersect. Each philosophy provides a different answer to the question ``What is a number?'': a divine principle, a logical class, an unconstrained symbol, a mental construction, a structural position, a convenient fiction, a cultural artifact, etc. Each carries its own ``knowledge-constitutive interest,'' shaping what it considers important: certainty, consistency, constructive meaning, relational invariance, ontological parsimony, human context, or self-reflection.

The knowledge-constitutive interest structure explains the underlying motivations driving these philosophical debates. They are not only about technical points; they are about what people \textit{want} from mathematics. Do some seek absolute, context-free truths? Then Platonism or structuralism will appeal. Is there a priority on secure methods and avoiding unfounded metaphysics? Then formalism or finitism might attract. Do some value human intelligibility and meaning above all? Intuitionism or constructivism could satisfy that. Are there worries about the societal and personal implications of an overly quantified world? Then a critical perspective becomes relevant.

In practice, many mathematicians and philosophers blend elements of these views. For example, a working mathematician might be a formalist Monday through Friday (treating numbers as symbols in proofs), a Platonist at 2am when marveling at an elegant theorem (``these relations must exist independently of me!''), and a social constructivist when teaching students (``historically, negative numbers took time to be accepted, showing mathematical ideas evolve''). This eclecticism is not necessarily inconsistency; it can be seen as pragmatically adopting the stance appropriate to the context -- much as Habermas's theory would suggest different interests can be pursued without denying each other.

Ultimately, the evolution of philosophies of number through immanent critique shows a progressive enrichment of understanding. By questioning ``what is number?'' from various angles, number emerges not as a monolithic concept but as a multi-faceted one: simultaneously a useful fiction, an abstract pattern, a cognitive tool, a social construct, and (for many) an objective reality. Each new philosophy did not simply negate the previous; it often incorporated what was valuable and corrected what was problematic. In this way, the dialogue continues. The concept of number, one of the greatest inventions of the human mind, reflects human desires for certainty, creative imagination, practical needs, and a capacity for critical self-awareness.


\section{Conclusion}

What, then, are numbers?  If they are not objects in Plato's heaven, nor mere abstractions from empirical collections, nor simply truth values, then what remains?  This paper has argued that numerals, and by extension numbers themselves, are fundamentally \textit{first-person pronouns.}  This is not merely a semantic shift, but a radical re-orientation of our understanding of mathematical being.  To see numerals as pronouns is to see them not as names for things, but as \textit{linguistic gestures of self-recognition}, anaphorically recollecting the enabling conditions of thought itself.

Through the lens of the \textit{null representation} $\emptyset$, this paper has explored how numerals emerge from the dynamic interplay of quotative embedding and anaphoric recollection, rooted in the unrepresentable, yet necessarily presupposed, ``I think.''  The ``Telephone'' game, with its playful distortions and iterative recollections, offered a concrete illustration of this process, revealing how even seemingly abstract mathematical concepts are deeply intertwined with the everyday dynamics of human communication and self-understanding.  And by critiquing object-based ontologies and static foundationalism, this analysis begins to glimpse the ``groundless ground'' of the null representation -- a dynamic, self-negating, and ultimately more authentic ground for mathematical being than any fixed, object-like foundation could provide.

This understanding of numerals as pronouns has significant implications, not only for the philosophy of mathematics, but also, and perhaps especially, for mathematics education.  It shifts the focus from mathematics as a body of pre-given, objective truths to mathematics as a \textit{human activity, a form of linguistic and conceptual self-constitution.}  It suggests that learning mathematics is not just about mastering abstract objects and procedures, but about developing a deeper capacity for self-reflection, for recognizing the enabling conditions of our own thought, and for engaging in the dynamic, intersubjective process of mathematical meaning-making.

In a world increasingly dominated by reductive, objectifying systems that seek to ``finitize the \textit{in}finite'' -- to quantify and control human being itself -- the idea that numerals are pronouns offers a subtle but powerful form of resistance.  It serves as a reminder that at the heart of mathematics, there lies not a cold, objective realm of abstract objects, but the pulse of human subjectivity, the ungraspable ``I think'' that constantly transcends any fixed representation.  To embrace this ``groundless ground,'' to recognize the pronoun-like nature of numerals, is not to abandon mathematical rigor or precision.  Rather, it is to ground mathematics in a more profoundly human and ultimately more meaningful foundation: the ongoing, self-recollecting, and inherently \textit{in}finite process of human being itself.  

What are numbers?  Perhaps, finally, it is possible to say with a degree of provisional certainty:  numbers are not \textit{things} at all.  Numbers, in their essence, express human self-understanding.  They are first-person pronouns of thought, reflections of a dynamic, self-recollecting, and ultimately ungrounded, yet persistently self-affirming, human being.  Recognizing this makes it possible to glimpse not only a new philosophy of mathematics, but also a more humanistic and empowering vision for mathematics education, one that truly bridges the gap between the abstract world of numbers and the lived experience of the students who seek to understand them.



\printbibliography[heading=subbibliography]

