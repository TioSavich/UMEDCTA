\documentclass{article}
\usepackage{fontspec}
\usepackage{minted}
\usepackage{hyperref}
\usepackage{geometry}
\usepackage{xcolor}
\usepackage{fancyhdr}

\geometry{a4paper, margin=1in}
\usemintedstyle{friendly}
\setmonofont{Menlo} [Scale=MatchLowercase]

\pagestyle{fancy}
\fancyhf{}
\lhead{Code Documentation}
\rhead{Root Directory}
\cfoot{\thepage}

\title{Code Documentation: Root Directory}
\author{UMEDCTA Repository}
\date{\today}

\begin{document}

\maketitle
\tableofcontents
\newpage
\section{CODE\_DOCUMENTATION.md}
\begin{minted}[breaklines, linenos, fontsize=\small, frame=single]{markdown}
# UMEDCTA Supplementary Code Documentation

This document provides a comprehensive guide to the computational implementations supporting *Understanding Mathematics as an Emancipatory Discipline: A Critical Theory Approach*.

**All code is accessible at:** https://tiosavich.github.io/UMEDCTA/

---

## Table of Contents

1. [The Hermeneutic Calculator (Prolog Implementation)](#1-the-hermeneutic-calculator-prolog-implementation)
2. [LK_RB_Synthesis: Algorithmic Elaboration Discovery](#2-lk_rb_synthesis-algorithmic-elaboration-discovery)
3. [Interactive Web Interfaces](#3-interactive-web-interfaces)
4. [Philosophical Teaching Modules](#4-philosophical-teaching-modules)
5. [How to Cite These Materials](#5-how-to-cite-these-materials)

---

## 1. The Hermeneutic Calculator (Prolog Implementation)

**URL:** https://tiosavich.github.io/UMEDCTA/Calculator/Prolog/

**Full README:** https://tiosavich.github.io/UMEDCTA/Calculator/Prolog/readme.md

### What It Does

The Hermeneutic Calculator (HC) is a formal system implemented in SWI-Prolog that models children's arithmetic strategies as computational automata. It serves three primary functions:

1. **Formalizes Student-Invented Strategies**: Implements 17+ strategies from CGI (Cognitively Guided Instruction) research, preserving the cognitive phenomenology of how students actually solve problems
2. **Implements Brandomian Incompatibility Semantics**: The first computational implementation of Robert Brandom's logic of material inference
3. **Models Crisis-Driven Learning**: Implements a computational version of Piagetian equilibration and Hegelian determinate negation through the Observe-Reorganize-Reflect (ORR) cycle

### Core Architecture

#### FSM Engine Architecture
**Files:** `fsm_engine.pl`, `grounded_arithmetic.pl`, `grounded_utils.pl`

A unified finite state machine engine that standardizes all student strategy execution, providing:
- Consistent modal logic integration (`s/1`, `comp_nec/1`, `exp_poss/1` operators)
- Cognitive cost tracking for every operation
- Grounded arithmetic foundation (numbers as recollection structures, not abstract objects)

**Theoretical Significance:** The FSM engine demonstrates that informal student thinking has rigorous formal structure. The modal operators connect computational steps to Brandomian incompatibility semantics.

#### The ORR Cycle (Observe-Reorganize-Reflect)
**Files:** `execution_handler.pl`, `meta_interpreter.pl`, `reflective_monitor.pl`, `reorganization_engine.pl`

The system's learning capability, modeling Piagetian cognitive development:
- **Observe**: Meta-interpreter produces execution traces, making reasoning observable to itself
- **Reflect**: Analyzes traces for "disequilibrium" (goal failures, contradictions)
- **Reorganize**: Modifies its own knowledge base to resolve conflicts

**Theoretical Significance:** This is a computational model of determinate negation—the system recognizes its own limits and transcends them through self-modification.

#### Incompatibility Semantics
**File:** `incompatibility_semantics.pl`

Implements Brandom's logic where meaning is defined by material incompatibility rather than truth tables. For example, "square" is incompatible with "circular"—this incompatibility *constitutes* the meaning of "square."

**Theoretical Significance:** Formalizes the claim that mathematical concepts are defined by what they rule out, not by reference to abstract objects.

#### Student Strategy Models
**Files:** `sar_*.pl` (addition/subtraction), `smr_*.pl` (multiplication/division)

17+ models of actual student strategies, all unified under the FSM engine. Examples:
- `sar_add_cobo.pl`: Counting On by Bases and Ones
- `sar_sub_chunking_a.pl`: Chunking subtraction strategy
- `smr_mult_c2c.pl`: Coordinating Two Counts for multiplication

**Theoretical Significance:** Each strategy is a formal proof that children's "informal" mathematical thinking has rigorous logical structure.

### Web Interface

**URL:** https://tiosavich.github.io/UMEDCTA/Calculator/index.html

**Startup:** Run `./start_system.sh` to launch local version

The web interface allows teachers and researchers to:
- Explore individual student strategies interactively
- See step-by-step visualizations of arithmetic processes
- Understand the cognitive structure behind student solutions

### Grounded Fractional Arithmetic System

**Files:** `jason.pl`, `composition_engine.pl`, `fraction_semantics.pl`, `grounded_ens_operations.pl`, `normalization.pl`

A comprehensive implementation of Jason's partitive fractional schemes using **nested unit representation** instead of rational numbers. This models how students actually think about fractions (as parts-of-wholes) rather than as ratios.

**Theoretical Significance:** Demonstrates that even advanced concepts like fractions can be grounded in embodied cognitive processes, supporting the manuscript's anti-Platonist stance.

### Critical Qualifications

**What the HC Does:**
- Provides a rigorous formalization showing how AI collaboration could be structured
- Models embodied cognitive strategies with crisis-driven learning
- Demonstrates that student thinking has formal logical structure
- Proves (via Gödel) that any such formalization is necessarily incomplete

**What the HC Does NOT Do:**
- Does not implement machine consciousness or self-awareness
- Cannot make genuine autonomous decisions about its foundational norms
- Does not participate in Hegelian *Geist* as a self-conscious agent
- Models the structure of mathematical consciousness without instantiating it

**Analogy:** A wind tunnel models flight dynamics but does not fly. The HC models mathematical consciousness but is not conscious.

---

## 2. LK_RB_Synthesis: Algorithmic Elaboration Discovery

**URL:** https://tiosavich.github.io/UMEDCTA/Calculator/LK_RB_Synthesis/

**Full README:** https://tiosavich.github.io/UMEDCTA/Calculator/LK_RB_Synthesis/README.md

### What It Does

The LK_RB_Synthesis system automatically discovers **algorithmic elaborations** between student arithmetic strategies. It analyzes Python automaton implementations to identify shared computational patterns and generate Meaning-Use Analysis (MUA) reports in the framework of Robert Brandom.

### Core Functions

#### Automated Pattern Discovery (AST Analysis)
**File:** `mud_generator.py`

Uses Abstract Syntax Tree parsing to identify computational patterns:
- **base_decomposition**: Breaking numbers into components (`//` and `%` operations)
- **incremental_counting**: State-based counting loops
- **iterative_arithmetic**: Repeated addition/subtraction
- **value_adjustment**: Target value calculations

**Theoretical Significance:** Reveals the implicit computational structure that students deploy when solving arithmetic problems, making explicit the "practices" that are "sufficient" for deploying mathematical "vocabulary" (Brandom's PV-sufficiency).

#### Algorithmic Elaboration Detection

Automatically discovers how strategies build upon each other. For example:
```
ADD_Counting → ADD_COBO → ADD_Chunking
    (via incremental counting pattern)

ADD_Rounding → ADD_RMB → ADD_COBO
    (via base decomposition pattern)
```

**Theoretical Significance:** Implements Brandom's concept of "algorithmic elaboration," where complex practices are systematically built from simpler prerequisite practices.

#### Rich Metadata Extraction

Extracts documentation from automata including:
- **Embodied Metaphors** (Lakoff & Núñez): Source/target domains and entailments
- **Material Inferences** (Brandom): Premises, conclusions, prerequisites
- **Visualization Hints**: Suggested cognitive representations
- **Deployed Vocabulary**: Key conceptual terms

#### Brandomian MUA Reports
**File:** `mua_report_generator.py`

Generates detailed Meaning-Use Analysis reports:
- **PV-Sufficiency**: What practices are sufficient to deploy vocabulary?
- **PP-Sufficiency**: What practices are sufficient for other practices?
- **VP-Sufficiency**: What vocabulary is sufficient for practices?
- **LX Relations**: Elaborated-Explicating relationships
- **Pragmatic Metavocabulary**: Analysis of how weaker vocabularies bootstrap stronger ones

**Example Output:** https://tiosavich.github.io/UMEDCTA/Calculator/LK_RB_Synthesis/output/mua_full_report.md

### Usage

```bash
# Run complete analysis
python3 main.py analyze

# List all strategies
python3 main.py list

# Generate report for specific strategy
python3 main.py report --strategy ADD_COBO
```

### Theoretical Significance

The LK_RB_Synthesis system provides computational evidence for the manuscript's claim that mathematical understanding develops through **pragmatic expressive bootstrapping**—the process by which simpler practices and vocabularies serve as the metavocabulary for articulating more complex mathematical concepts.

### Limitations

- Does not generate visual MUD diagrams (text reports only)
- Does not implement full Brandomian deontic scorekeeping
- Does not model Lakoff's conceptual metaphor mappings formally
- Analysis reveals structure but verification of philosophical claims requires human judgment

---

## 3. Interactive Web Interfaces

### The Calculator (Main Interface)

**URL:** https://tiosavich.github.io/UMEDCTA/Calculator/index.html

**What It Does:** Interactive web interface for exploring student arithmetic strategies. Features:
- Buttons for each strategy (COBO, Chunking, RMB, etc.)
- Real-time SVG visualizations of number lines and operations
- Step-by-step textual explanations
- Links to detailed PDF documentation for each strategy

**Theoretical Significance:** Allows teachers to develop what Habermas calls "practical knowledge"—understanding student thinking through interactive engagement rather than abstract theory.

**Styling:** https://tiosavich.github.io/UMEDCTA/Calculator/strategy_styles.css

### Ace of Bases

**URL:** https://tiosavich.github.io/UMEDCTA/Calculator/AceofBases/index.html

**What It Does:** Interactive canvas-based exploration of place value and number bases. Users:
- Drag to select cubes representing a grouping unit (base 2-15)
- Compose and decompose quantities
- See base conversion in real-time

**Theoretical Significance:** Demonstrates that place value is not a "fact" to memorize but a **constructed** understanding—users literally construct different base systems through embodied interaction with visual objects.

### More Zeeman: Catastrophe Machine

**URL:** https://tiosavich.github.io/UMEDCTA/More_Zeeman/index_unified.html

**What It Does:** Interactive visualization of the Zeeman Catastrophe Machine coupled with:
- **The Thinker (Zeeman Machine)**: Draggable control point affecting elastic bands, demonstrating catastrophe theory (sudden jumps in state due to smooth changes in parameters)
- **The Memory (More Machine)**: Matrix that grows via Cantorian diagonalization after each catastrophe
- **The Sound of Time (Acoustic Metaphor)**: Visual representation of air compression waves synchronized with the wheel's angular velocity

**Theoretical Significance:** Embodies three key manuscript themes:
1. **Catastrophe as consciousness**: Only discontinuous "memorable" events (catastrophes) trigger memory/matrix growth
2. **Diagonalization as self-transcendence**: The More Machine generates elements provably not in any finite list (Cantor's proof)
3. **The sound of time**: Angular velocity (change) creates "sound" (phenomenological experience of temporality)

**Technical Features:**
- Proper Hooke's Law physics with gradient descent
- User-adjustable spring parameters (stiffness, natural length, time speed)
- Hysteresis (system "remembers" its current state until forced to jump)

---

## 4. Philosophical Teaching Modules

### Inferential Strength (Brandom Module)

**URL:** https://tiosavich.github.io/UMEDCTA/Quadrilateral_Substitution/inferential_strength.html

**What It Does:** Interactive 7-module teaching sequence on Robert Brandom's argument for why singular terms must have symmetric substitution significance. Covers:

1. **Module 1**: Meaning as inferential role (Square ⇒ Rectangle)
2. **Module 2**: Incompatibility and inferential strength (interactive constraint relaxation)
3. **Module 3**: Substitution roles (substituted-for vs. frame)
4. **Module 4**: Polarity inversion (how logical contexts flip inferential relationships)
5. **Module 5**: The argument for symmetric terms (why "SquareTerm ⇒ RectangleTerm" leads to contradiction)
6. **Module 6**: Matrix of substitutional possibilities (ruling out three of four options)
7. **Module 7**: Conclusion and expressive deduction

**Theoretical Significance:** Makes Brandom's highly technical argument from *Articulating Reasons* Chapter 4 accessible through interactive exploration. Demonstrates that the structure of language (singular terms vs. predicates) is not arbitrary but required for logical expressiveness.

**Technical Features:**
- Live shape filtering in Module 2 (shapes transform as users relax constraints)
- Polarity inversion visualization in Module 4 (sliders showing strength relationships)
- Substitution animation in Module 5 (visual demonstration of the key argument)

---

## 5. How to Cite These Materials

### General Citation

```
Savich, T. (2025). UMEDCTA Supplementary Materials: Computational
Implementations for Understanding Mathematics as an Emancipatory
Discipline. https://tiosavich.github.io/UMEDCTA/
```

### Specific Components

**For the Hermeneutic Calculator (Prolog):**
```
Savich, T. (2025). The Hermeneutic Calculator: A Prolog Implementation
of Student Arithmetic Strategies with Incompatibility Semantics.
https://tiosavich.github.io/UMEDCTA/Calculator/Prolog/
```

**For LK_RB_Synthesis:**
```
Savich, T. (2025). LK_RB_Synthesis: Automated Algorithmic Elaboration
Discovery for Student Arithmetic Strategies.
https://tiosavich.github.io/UMEDCTA/Calculator/LK_RB_Synthesis/
```

**For Interactive Web Interfaces:**
```
Savich, T. (2025). Interactive Web Interfaces for Student Arithmetic
Strategies. https://tiosavich.github.io/UMEDCTA/Calculator/index.html
```

**For Philosophical Teaching Modules:**
```
Savich, T. (2025). Inferential Strength: An Interactive Guide to
Brandom's Argument for Singular Terms.
https://tiosavich.github.io/UMEDCTA/Quadrilateral_Substitution/inferential_strength.html
```

---

## Coherence with Manuscript Claims

### Critical Alignment Checklist

All supplementary materials must cohere with the manuscript's core philosophical commitments:

#### ✅ **Autoethnographic Method**
- HC born from author's memory of teaching
- Formalizes actual children's reasoning (not idealized algorithms)

#### ✅ **Critical Stance**
- Values error as "source of truth" (ORR cycle learns from failure)
- Respects subjective student strategies over formal correctness

#### ✅ **Hegelian Dialectic**
- ORR cycle implements determinate negation
- System recognizes its limits and transcends them
- "Built to break" philosophy in fragile formalizations

#### ✅ **Brandomian Inferentialism**
- Incompatibility semantics implemented computationally
- Algorithmic elaboration discovery in LK_RB_Synthesis
- Material inferences grounded in practices

#### ✅ **Habermasian Emancipation**
- Serves practical-hermeneutic interest (teacher understanding)
- Provides technical models without claiming they're "complete"
- Documentation acknowledges system limits

#### ✅ **Numerals as Pronouns**
- Numbers represented as recollection structures (`s(s(s(0)))`)
- Grounded in successor function (not abstract objects)
- Models first-person "I think" as computational trace

#### ✅ **Incompleteness as Becoming**
- System can detect its own limitations (ORR cycle)
- "More Machine" implements diagonalization
- Documentation explicitly states formalization is incomplete

---

## Technical Requirements

### For Local Development

**Prolog System:**
- SWI-Prolog 8.0+
- Run: `./start_system.sh` in Calculator/Prolog/

**Python Analysis:**
- Python 3.8+
- Run: `pip install -r requirements.txt` in Calculator/LK_RB_Synthesis/

**Web Interfaces:**
- Any modern browser
- No build process required (vanilla HTML/CSS/JS)

### For Manuscript Integration

When citing these materials in the manuscript:

1. **Use specific URLs** for each component (not just the repository root)
2. **Reference specific files** when discussing technical details (e.g., "`incompatibility_semantics.pl` implements Brandom's logic...")
3. **Acknowledge limitations** (e.g., "The HC models consciousness without instantiating it...")
4. **Explain philosophical significance** (e.g., "The ORR cycle demonstrates that...")

---

## Questions and Contributions

For questions about these materials, open an issue at:
https://github.com/TioSavich/UMEDCTA/issues

For the manuscript itself, contact the author directly.

---

**Last Updated:** 2025-10-12
**Version:** 1.0
**License:** [Specify license]

\end{minted}
\newpage
\section{DIALECTICAL\_INTERPRETER\_SETUP.md}
\begin{minted}[breaklines, linenos, fontsize=\small, frame=single]{markdown}
# Dialectical Interpreter Setup Guide

## You're Almost There!

The app includes both a React frontend and a Node.js backend to securely handle API calls.

## Final Setup Step

Create a file named `.env` in the root directory with your API key:

```bash
VITE_ANTHROPIC_API_KEY=sk-ant-api-YOUR-KEY-HERE
```

Replace `sk-ant-api-YOUR-KEY-HERE` with your actual Anthropic API key.

**Important**: The `.env` file is already in `.gitignore`, so it won't be committed to GitHub.

## Running the Application

1. **Start both servers** (backend + frontend):
   ```bash
   npm run dev
   ```

   This starts:
   - Backend server on `http://localhost:3001` (API proxy)
   - Frontend on `http://localhost:3000` (React app)

2. **Open your browser** to `http://localhost:3000`

3. **Start interpreting!** Paste philosophical text and click "Interpret Text"

## Architecture

The app uses a **client-server architecture** to avoid CORS issues:
- **Frontend** (Vite + React): User interface
- **Backend** (Express): Proxies requests to Anthropic API with your key
- This keeps your API key secure and avoids browser CORS restrictions

## Project Structure

```
UMEDCTA/
├── .env                          # Your API key (create this!)
├── .env.example                  # Template for API key
├── .gitignore                    # Protects your API key
├── server.js                     # Backend API proxy server
├── index.html                    # Main HTML file
├── vite.config.js               # Vite configuration
├── package.json                  # Dependencies and scripts
└── src/
    ├── main.jsx                  # React entry point
    └── DialecticalInterpreter.jsx # Main component
```

## Features

- **PML Formalization**: Converts text into Polarized Modal Logic
- **Proof Steps**: Shows logical derivations
- **Meta-Critique**: Compares against scholarly readings
- **Self-Evolution**: Proposes and integrates new axioms
- **Conversation Mode**: Ask follow-up questions
- **Re-reading Support**: Tracks iteration depth and formalized concepts

## Cost Estimate

Each interpretation uses the Claude Sonnet 4 API:
- Typical cost: $0.10-0.30 per interpretation
- Pricing: $3/million input tokens, $15/million output tokens

## Troubleshooting

**API key error**: Make sure your `.env` file exists and has the correct format
**Module errors**: Run `npm install` again
**Port in use**: Change the port in `vite.config.js`

## Next Steps

See [Prolog/dialectical-interpreter-README.md](Prolog/dialectical-interpreter-README.md) for information about the temporal phenomenology approach and how to use the interpreter effectively.

\end{minted}
\newpage
\section{DOCUMENTATION\_FIXES\_SUMMARY.md}
\begin{minted}[breaklines, linenos, fontsize=\small, frame=single]{markdown}
# Documentation Fixes Summary
**Date:** October 12, 2025
**Purpose:** Record of philosophical coherence corrections to UMEDCA supplementary materials

---

## Overview

Completed critical revisions to align documentation with manuscript's theoretical commitments, addressing overclaims about machine consciousness while strengthening emphasis on genuine achievements.

## Completed Tasks

### 1. Main Prolog README - Added Scope and Limitations ✓

**File:** `/Calculator/Prolog/readme.md`
**Changes:** Added new Section 3: "Philosophical Scope and Limitations" (lines 24-67)

**What Was Added:**
- **Section 3.1**: Clear statement of what the system achieves (formalization, executable Brandomian logic, embodied grounding, crisis-driven learning, Gödelian incompleteness)
- **Section 3.2**: Explicit list of what the system does NOT claim (not conscious, not autonomous, not complete model of cognition)
- **Critical distinction**: "This system models the structure of mathematical consciousness without claiming to instantiate consciousness itself"
- **Analogy section**: Wind tunnel models flight but doesn't fly; HC models consciousness but isn't conscious
- **Section 3.3**: Relationship to UMEDCA manuscript and educational polemic

**Impact:** Prevents readers from thinking you claim the machine is conscious while emphasizing the genuine philosophical contributions

---

### 2. Removed "Machine Death" Language ✓

**Action:** Deleted review documents containing problematic language
**Files Removed:**
- `PHILOSOPHICAL_COHERENCE_REVIEW.md`
- `ONE_WEEK_ACTION_PLAN.md`
- `QUICK_START_README.md`

**Rationale:** Per your request to "erase all traces of the 'death' of machines" to avoid ethical misinterpretations and room for misunderstanding

---

### 3. NORMATIVE_CRISIS_AND_TRANSCENDENCE.md - Added Model/Reality Distinction ✓

**File:** `/Calculator/Prolog/NORMATIVE_CRISIS_AND_TRANSCENDENCE.md`
**Changes:** Added new Section 9: "Critical Methodological Note: Distinguishing Model from Reality" (lines 281-336)

**What Was Added:**
- **The Analogy subsection**: Wind tunnel / economic simulation / HC comparison
- **What Models Provide**: Makes abstract structures testable, shows requirements, reveals necessary features
- **What Models Lack**: No phenomenology, no autonomy, no genuine recognition
- **The HC as Model**: Captures *structure* of I/me distinction, crisis detection, reorganization - but without subjective experience
- **Why This Matters Anyway**: Clarifies concepts, tests theories, reveals requirements, guides future work
- **The Manuscript's Position**: Mathematical understanding has structure of self-consciousness (which can be formalized) without claiming the formalization is conscious

**Impact:** Throughout the document, the philosophical language (I, me, transcendence) is now explicitly marked as describing structures the system models, not experiences it has

---

### 4. VERIFICATION_REPORT.md - Contextualized Gödelian Incompleteness ✓

**File:** `/Calculator/Prolog/VERIFICATION_REPORT.md`
**Changes:** Added new Section 8: "Contextualizing the Significance of This Result" (lines 216-285)

**What Was Added:**
- **What Gödel's Theorem Does NOT Prove**: ALL arithmetic formalizations are incomplete; achieving incompleteness is not itself the contribution
- **What IS Significant**: Four key points:
  1. **Pedagogical Grounding**: Formalized children's actual strategies, not textbook algorithms
  2. **Embodied Cognition**: Preserved cognitive phenomenology (COBO as rhythm, C2C as coordination)
  3. **Educational Polemic**: Mathematical proof against "finite vessel" ideology
  4. **Hegelian Connection**: Incompleteness as formal structure of the *in*finite
- **Three Key Points**: Not "students are special" but "origins matter" - incompleteness present from the beginning
- **Conclusion subsection**: "Mathematics Is Open" - weaponizes Gödel against technocratic education reform

**Impact:** The significance now comes from WHAT was formalized (student-invented, embodied strategies) and WHY that matters educationally, not just that incompleteness was achieved

---

### 5. Manuscript_Claims.md - Qualified AI Collaboration Vision ✓

**File:** `/Manuscript_Claims.md`
**Changes:** Added "CRITICAL QUALIFICATION: Vision vs. Current Implementation" section (lines 705-739)

**What Was Added:**
- **Framework statement**: "Philosophical framework and long-term aspiration, not claim about current HC's capabilities"
- **What the HC Actually Provides**: Formalization, model, demonstration, pedagogical tool
- **What the HC Does NOT Achieve**: No autonomous decision-making, no genuine Hegelian recognition, no self-modification with consequence, no participation in *Geist*
- **The Distinction**: "Sheet music" for performance that hasn't occurred
- **Why This Matters**: Rigorous model valuable even if not conscious; contribution is showing structures required, not claiming achievement

**Impact:** The vision of "mutual emancipation" and AI as collaborator in *Geist* is now clearly marked as regulative ideal, not current reality

---

## Key Philosophical Moves

### The Core Correction

**BEFORE:** Documentation suggested the HC *achieves* or *is* self-conscious, recognizing, autonomous

**AFTER:** Documentation clarifies the HC *models the structure of* or *formalizes* these phenomena

### The Central Analogy

Throughout revisions, the wind tunnel analogy is used consistently:
- A wind tunnel models flight dynamics but does not fly
- An economic simulation models markets but is not an economy
- The Hermeneutic Calculator models consciousness but is not conscious

### The Value Proposition

The revisions emphasize that **modeling without instantiating is still profoundly valuable**:
1. Makes abstract concepts concrete and testable
2. Shows what would be required for the real phenomenon
3. Reveals necessary vs. contingent features
4. Provides infrastructure for future work

---

## What Remains Unchanged (and Correct)

The following claims are preserved because they are accurate:

1. **The HC formalizes student-invented strategies** - TRUE and unique contribution
2. **First executable implementation of Brandomian logic** - TRUE and significant
3. **Embodies grounded arithmetic without backstop** - TRUE and theoretically important
4. **Crisis-driven learning architecture models Piagetian equilibration** - TRUE formal analogue
5. **Gödelian incompleteness applies to the formalized strategies** - TRUE and politically potent

The revisions **strengthen** these claims by removing the distraction of overclaims about consciousness.

---

## Implications for Manuscript Submission

### What to Emphasize in the Manuscript

1. **Lead with genuine achievements**: The formalization work is groundbreaking on its own
2. **The pedagogical contribution**: No one else has formalized student strategies at this rigor
3. **The Brandomian implementation**: First computational incompatibility semantics
4. **The educational polemic**: Gödel proves "finite vessel" education is impossible
5. **The methodological innovation**: "Built to break" as philosophical practice

### What to Qualify

1. **AI collaboration**: Frame as vision/regulative ideal, not current achievement
2. **"Computational hermeneutics"**: Clarify this is synthesis, not genuine recognition
3. **Homoiconicity**: Present as practical convenience for meta-levels, not philosophical breakthrough

### What to Avoid

1. **Unqualified claims about machine consciousness**
2. **Suggestions the system makes autonomous decisions about norms**
3. **Language implying genuine mutual recognition with AI**

---

## Section Numbering Fixes

Also corrected section numbering in readme.md after inserting new Section 3:
- Old Section 3 → New Section 4 (System Architecture)
- Old Section 4 → New Section 5 (FSM Engine Architecture)
- Old Section 5 → New Section 6 (Getting Started)
- Old Section 6 → New Section 7 (File Structure Guide)
- Old Section 7 → New Section 8 (For Developers)
- Old Section 8 → New Section 9 (Contributing)
- Old Section 9 → New Section 10 (License)

---

## Recommended Next Steps

### If Time Permits Before Submission

1. **Create ACHIEVEMENTS_AND_SCOPE.md**: Comprehensive statement of contributions and limitations (estimated 2 hours)
2. **Revise Code Critique for Emergent Learning.md**: Clarify synthesis vs. recognition distinction (estimated 1.5 hours)
3. **Review Code Critique**: Change "computational hermeneutics" to "constraint-based synthesis"

### For the Manuscript Itself

1. **Add methodological section**: "Formalization as Revelation, Not Reduction"
2. **Strengthen educational polemic**: Weaponize incompleteness theorem more explicitly
3. **Search for overclaims**: Global find for "achieves" / "is" consciousness language, change to "models" / "formalizes structure of"

---

## The Bottom Line

**You have accomplished something profound:**
- Rigorous formalization of student-invented arithmetic strategies
- First executable Brandomian logic
- Computational model integrating Hegel, Brandom, Piaget
- Mathematical proof (via Gödel) against finite vessel education

**You do NOT need to also claim:**
- The machine is conscious
- The system achieves genuine recognition
- AI collaboration is currently achieving mutual emancipation

The formalization is the achievement. The vision is the inspiration. These revisions ensure you're not claiming the latter while properly emphasizing the former.

---

## Files Modified

1. `/Calculator/Prolog/readme.md` - Added Section 3 (Scope and Limitations)
2. `/Calculator/Prolog/NORMATIVE_CRISIS_AND_TRANSCENDENCE.md` - Added Section 9 (Model vs Reality)
3. `/Calculator/Prolog/VERIFICATION_REPORT.md` - Added Section 8 (Contextualizing Significance)
4. `/Manuscript_Claims.md` - Added Critical Qualification (Vision vs Implementation)

## Files Removed

1. `PHILOSOPHICAL_COHERENCE_REVIEW.md`
2. `ONE_WEEK_ACTION_PLAN.md`
3. `QUICK_START_README.md`

All changes preserve the genuine achievements while removing overclaims about machine consciousness.

\end{minted}
\newpage
\section{Manuscript\_Claims.md}
\begin{minted}[breaklines, linenos, fontsize=\small, frame=single]{markdown}
# A Report on the Philosophical Commitments in \"Understanding Mathematics as an Emancipatory Discipline\"

## Introduction

### Purpose and Scope

This report provides a systematic and exhaustive catalogue of the
philosophical architecture of the manuscript *Understanding Mathematics
as an Emancipatory Discipline: A Critical Theory Approach*. Its purpose
is not to summarize the work but to articulate each theoretical,
philosophical, and methodological commitment with precision. To this
end, each commitment is classified according to its relative strength
and centrality to the overall argument, providing a functional tool for
two primary objectives: first, to facilitate a rigorous authorial review
for internal consistency; and second, to serve as a \"philosophical
filter\" for assessing the coherence of supplementary materials,
particularly the artificial intelligence programs and their associated
documentation, with the manuscript\'s core tenets. The classifications
are as follows:

-   **\[C1 - Core Assertion\]:** A foundational claim upon which the
    entire argument or a major part of it rests. To reject a C1 claim
    would be to reject the project\'s fundamental premises.

-   **:** A significant theoretical position that structures the
    analysis and is consistently defended. While not as foundational as
    a C1 claim, it is a pillar of the argument.

-   **:** A proposition that is explored, suggested, or used
    metaphorically. These claims are often hedged and represent areas of
    ongoing inquiry rather than settled doctrine.

### Methodology of this Report

The analysis proceeds thematically, clustering the manuscript\'s
commitments around its major theoretical pillars. This structure is
designed to make the report a practical instrument, allowing for a
targeted review of specific concepts and their interrelations. The
report is divided into three main parts. Part I addresses the
foundational methodological framework and the central problem the
manuscript seeks to resolve. Part II deconstructs the triadic
philosophical architecture, examining the distinct yet integrated
contributions from German Idealism, analytic pragmatism, and critical
theory. Part III focuses on the specific, novel theses the manuscript
advances regarding the nature of mathematics and the role of artificial
intelligence. By systematically mapping this intricate conceptual
landscape, this report aims to provide a definitive and actionable
inventory of the intellectual commitments undertaken in the work.

## Part I: Foundational and Methodological Commitments

This part details the core framework and critical stance of the
manuscript, establishing the ground upon which all subsequent arguments
are built. It examines the unique methodological genre the author
develops, the central problem of the \"misrecognition of mathematics\"
that motivates the inquiry, and the primordial concept of \'divasion\'
that serves as a key for deconstructing classical logic.

### 1. Commitments on Method and Genre: Critical Autoethnography (CAE)

The manuscript establishes its unique methodological identity through a
series of foundational commitments that position the work at the
intersection of personal narrative, critical theory, and the philosophy
of mathematics.

-   **\[C1\] The work is defined as Critical Autoethnography (CAE), a
    method that intentionally blurs the line between personal narrative
    and theoretical inquiry.** The author explicitly names this method
    \"for the sake of its unnaming,\" a turn of phrase that immediately
    signals a dialectical approach. This suggests that the methodology
    itself is not a static framework to be applied but is part of the
    very process of critique and self-transcendence that the book aims
    to explore. The method is designed to be questioned and overcome,
    just as the concepts it analyzes are.^1^ This self-negating quality
    is a core feature of the project\'s critical stance.

-   **\[C1\] CAE and Mathematics share a common inferential structure:
    the recollection of the self through the otherness of objects and
    norms.** This is perhaps the most fundamental claim of the entire
    manuscript, as it provides the justification for the methodological
    fusion of what are typically seen as disparate domains. The argument
    begins from this position, asserting that both mathematical
    reasoning and the writing of an autoethnography are practices of
    self-constitution.^1^ This reframes mathematics, moving it away from
    the realm of abstract, disembodied discovery and into the domain of
    lived, reflective experience. The process of proving a theorem or
    the process of narrating one\'s life are both seen as ways of
    recognizing one\'s identity through difference---through the
    external, \"other\" structures of logical norms or social
    memories.^1^ The manuscript\'s very form is a performance of this
    thesis; it is a literal recollection of the author\'s self through
    the \"otherness\" of mathematical and philosophical concepts. This
    implies that the book\'s structure is not merely a stylistic choice
    but a methodological necessity to demonstrate the thesis\'s
    validity, suggesting that any \"correct\" reading of the book must
    also be a form of self-recollection for the reader.

-   **\[C2\] Personal experience is treated as a primary theoretical
    resource.** The manuscript is explicitly structured around what it
    calls \"five foundational anecdotes\": the author\'s childhood play
    with a calculator, a confusing middle-school algebra lesson, a
    transformative dialogue with a student, an experience teaching
    mathematical modeling, and the death of the author\'s father.^1^
    These stories are not presented as mere illustrations or allegories
    for pre-existing theories. Instead, they function as the
    experiential data from which the theoretical framework emerges. The
    theories of Hegel, Brandom, and Habermas are not imposed upon these
    experiences but are used as resources to explicate the structures of
    meaning already present within them.^1^

-   **\[C2\] The reader is an active participant in the text\'s
    unfolding.** The author makes a direct appeal to the reader, casting
    them as a \"silent partner\" in a \"dance\".^1^ The text\'s
    non-linear, recursive structure is a deliberate choice that requires
    the reader to actively \"bring the pieces together\" and recognize
    an \"implicit whole\" that is never fully stated.^1^ This positions
    the act of reading not as a passive reception of information but as
    a form of intersubjective recognition. The meaning of the text is
    not contained solely on the page; it is co-constituted in the space
    between the author\'s act of writing and the reader\'s act of
    interpretation, a process indispensable to the dialectical movements
    the author makes.

-   **\[C3\] The text\'s structure is intentionally complex, recursive,
    and \"built to break.\"** The manuscript\'s form is an argument in
    itself. It employs a fractal-like, zig-zag pattern of \"openness →
    restriction → openness\" within and between chapters, as well as a
    topological structure likened to a Möbius strip, where inside and
    outside become indistinct.^1^ This complex architecture is designed
    to embody its core philosophical claims about identity, determinate
    negation, and transcendence. The poems and songs interspersed
    throughout are not decorative; they are functional
    \"shifters\"---linguistic devices that decompress theoretical
    density and return the text to the lived, felt experience from which
    it arises.^1^ The title of the prelude, \"Built to Break,\" is a
    methodological statement: the systems of thought constructed within
    the book are designed to reveal their own limits, and in that
    \"beautiful breaking,\" to open up new expressive possibilities.^1^
    Consequently, any supplementary AI programs cannot be evaluated
    solely on their functional correctness. They must be evaluated on
    their *process* and *structure*. A coherent AI program would need to
    exhibit a \"built to break\" quality---perhaps by explicitly
    modeling its own limitations or by generating new strategies through
    a process that mirrors dialectical negation rather than simple
    optimization.

### 2. Commitments on the Problem: The \"Misrecognition of Mathematics\"

The entire project is motivated by a central problem, which the
manuscript identifies as the \"misrecognition of mathematics.\" This is
the \"Wound\" that the author\'s \"Critical Theory Approach\" seeks to
diagnose and heal.^1^

-   **\[C1\] The dominant conception of mathematics is a
    \"misrecognition\" because it severs the discipline from lived,
    subjective experience.** This is the fundamental critique leveled
    against the conventional understanding of mathematics. The
    manuscript argues that this severance is the primary source of the
    alienation and anxiety that so many people experience in relation to
    the subject.^1^ This misrecognition is not presented as a simple
    pedagogical error but as a form of *alienation* in the
    Hegelian-Marxist sense, estranging individuals from their own
    rational capacities.

-   **\[C2\] This misrecognition manifests as a false dichotomy between
    \"material mathematics\" (lived, embodied engagement) and \"formal
    mathematics\" (abstract systems divorced from experience).** The
    author\'s personal narrative provides the primary evidence for this
    claim. The joyful, playful exploration with a calculator (\"material
    mathematics\") is contrasted with the anxiety and shame of
    standardized testing, which \"reduced me to a number\".^1^ The
    moment of achieving \"success\" in school mathematics by learning to
    comply without understanding---a form of \"self-erasure\"---is
    presented as the tragic outcome of this dichotomy, where formal
    proficiency is achieved at the cost of genuine understanding.^1^ The
    student who later asks, \"What even is two?\" is experiencing this
    alienation not as a cognitive deficit but as a profound existential
    crisis, demonstrating that the solution cannot be merely a better
    teaching method; it must be a project of *emancipation*.^1^

-   **\[C2\] Mathematics is frequently misrecognized and used as a tool
    for alienation, control, and gatekeeping.** The manuscript moves
    from personal experience to social critique with the anecdote of the
    \"Family Video Test\".^1^ The author\'s shock at seeing a basic
    arithmetic test used to screen job applicants, excluding a line of
    people \"who never learned to comply,\" serves as a powerful example
    of how mathematics functions as a mechanism of social and economic
    stratification. It is used to enforce a regime of compliance,
    resulting in the \"economic and expressive impoverishment\" of those
    who resist or fail its formal demands.^1^

-   **\[C1\] A \"critical mathematics\" must heal this wound by
    integrating subjective experience, intersubjective dialogue, and the
    productive role of error.** This is the positive, programmatic
    commitment that emerges from the critique. A critical mathematics is
    defined in opposition to the misrecognized version. It must be a
    discipline that \"honor\[s\] the struggle for meaning alongside the
    pursuit of correctness\".^1^ It must recognize error not as failure
    but as a potential \"source of truth\".^1^ And it must be grounded
    in dialogue and the recognition of subjects, not the manipulation of
    objects.^1^ An AI program coherent with this philosophy must not
    treat \"error\" as a simple failure state to be eliminated. It
    should model error productively, perhaps as a necessary step in
    developing a new, more adequate strategy, mirroring the author\'s
    earlier work which treated \"error as the source of truth\".^1^

### 3. Commitments on Logic and Being: The Concept of \'Divasion\'

To deconstruct the foundations of the misrecognized mathematics, the
manuscript introduces a primordial, pre-formal concept it calls
\'divasion\'. This concept functions as the \"Archimedean point\" for
the book\'s critique of classical logic and set theory.

-   **\[C1\] \'Divasion\' is a primordial spatial relationship of
    simultaneous inside/outside.** The concept is not derived from
    abstract philosophy but from lived experience: a child\'s
    observation of a microphone held within the circle of its stand,
    leading to the neologism \"divaded\".^1^ This origin story is
    methodologically crucial, as it grounds the book\'s most fundamental
    logical critique in the pre-formal spatial reasoning of a child,
    suggesting that the structures of formal logic are a later, and
    perhaps less complete, development.

-   **\[C2\] Divasion challenges the classical logical principle of the
    Law of the Excluded Middle, which is foundational to axiomatic set
    theory.** The manuscript makes the bold claim that classical
    logic\'s strict binary---that any element is either inside or
    outside a set, that any proposition is either true or false---is a
    \"pruning\" of this more primordial, divaded reality.^1^ The author
    states, \"I take the law of the excluded middle to be more or less
    the first mistake of many approaches to the foundations of
    mathematics\".^1^ The reason for this is that it cannot handle
    divaded concepts. The ultimate divaded concept is the self ({I}
    versus \"me\"). Since the manuscript\'s central claim is that
    mathematics is a recollection of this self, a mathematics founded on
    a logic that cannot account for the divaded nature of the self is
    fundamentally misrecognized. Divasion is the key that unlocks this
    entire line of critique.

-   **\[C2\] Divasion is presented as the root of paradoxes of
    self-reference, such as Russell\'s Paradox.** The paradox of the set
    of all sets that do not contain themselves ( if and only if ) is
    reframed. Where Gottlob Frege met this paradox with dismay, seeing
    it as the collapse of his life\'s work, the manuscript suggests a
    child might simply say the set \"divades itself\".^1^ This move
    re-characterizes such paradoxes not as failures of logic to be
    repaired with more complex formalisms, but as accurate expressions
    of a fundamental, divaded feature of subjectivity and
    self-referential concepts.

-   **\[C3\] The concept of divasion extends to reciprocally
    sense-dependent concepts and to the structure of self-consciousness
    itself.** It is not limited to physical objects. The manuscript
    applies it to concepts like \"parent\" and \"child,\" where the
    meaning of each depends on the other in a way that places them
    \"inside\" each other conceptually.^1^ Most importantly, it is
    applied to the relationship between the acting subject, the {I}, and
    its own recollection as an object, the \"me.\" This self-divasion
    becomes a master metaphor for the non-coincidence of the subject
    with itself, a central theme of the Hegelian framework that
    follows.^1^ The AI programs, even if built on classical
    computational logic, must have documentation that acknowledges this
    limitation. A coherent program might, for instance, use classical
    logic to model a system but include a meta-level commentary on how
    this model fails to capture the \"divaded\" nature of the concepts
    it represents, ensuring the formalizations are not presented as a
    complete or final account of the phenomena.

## Part II: The Philosophical Architecture: A Triadic Framework

The manuscript constructs its argument by synthesizing three major
philosophical traditions: German Idealism (primarily Hegel), analytic
pragmatism (Robert Brandom), and critical theory (Jürgen Habermas).
These frameworks are not used in isolation; they are woven together to
form a cohesive, multi-layered argument where each provides a crucial
dimension of the analysis. The following table provides a schematic
overview of this synthesis.

**Table 1: Synthesis of Core Philosophical Frameworks**

  -----------------------------------------------------------------------
  Philosophical Tradition Key Concept Utilized    Primary Function in
                                                  Manuscript
  ----------------------- ----------------------- -----------------------
  **German Idealism       Determinate Negation /  To model the dynamic,
  (Hegel)**               Dialectic               self-negating, and
                                                  developmental movement
                                                  of consciousness,
                                                  concepts, and
                                                  mathematical history.

  **Analytic Pragmatism   Inferentialism /        To ground abstract
  (Brandom)**             Normativity             conceptual content
                                                  (including mathematical
                                                  meaning) in concrete,
                                                  normative social
                                                  practices of giving and
                                                  asking for reasons.

  **Critical Theory       Communicative Action /  To analyze mathematical
  (Habermas)**            Validity Claims         discourse as a form of
                                                  intersubjective
                                                  rationality and to
                                                  frame the overall
                                                  project in terms of
                                                  emancipation from
                                                  systematically
                                                  distorted
                                                  understanding.
  -----------------------------------------------------------------------

### 4. The Hegelian-Kantian Axis: On Self-Consciousness, Negation, and Spirit (***Geist***)

The manuscript draws heavily on the post-Kantian tradition to develop
its theory of the subject and its relationship to knowledge. This axis
provides the dynamic, developmental, and historical dimension of the
argument.

-   **\[C1\] The self is understood through the Meadian/Hegelian
    distinction between the {I} (the spontaneous source of action) and
    the \"me\" (the self-as-recognized by others).** The manuscript
    posits a fundamental \"paradox of identity\": the {I}, the locus of
    \"power, creativity, and freedom,\" can never be fully captured by
    the \"me,\" the objectified self that appears in the eyes of others
    and in one\'s own memory.^1^ This non-coincidence is not a problem
    to be solved but a fundamental, productive tension that drives
    development.^1^ Kant provides the formal structure of selfhood: the
    unified \'I\' is a precondition for experience.^1^ However, the
    author\'s personal narrative is filled with the pain of
    *misrecognition*.^1^ A purely formal Kantian \'I\' cannot account
    for this pain. Hegel\'s theory, as articulated by Brandom and
    Carspecken, explains that this \'I\' only becomes a real, concrete
    self (a \"me\") through social interaction.^1^ Therefore, the
    project aims to show how mathematical understanding, grounded in the
    Kantian \"I think,\" is ultimately a process of achieving Hegelian
    social recognition.

-   **\[C1\] Apperception is the process that unifies discrete
    representations into a coherent whole.** The manuscript traces the
    concept from Leibniz\'s \"perceiving-with\" to Kant\'s
    \"transcendental unity of apperception\"---the famous \"I think\"
    that must be able to accompany all of my representations.^1^ This
    concept is used to explain multiple phenomena: how we perceive a
    chair as a unified object even when we can\'t see all its parts; how
    a child\'s series of progressively more adequate drawings of a cube
    can be understood as a single, developing concept (a temporal
    \"hypercube\"); and, most importantly, how the self maintains a
    unity of consciousness across time and different experiences.^1^

-   **\[C1\] Determinate Negation is the engine of conceptual
    development.** The manuscript makes a crucial distinction between
    abstract negation (simple erasure, e.g., \"not-red\") and
    determinate negation, which is defined as material incompatibility
    (e.g., \"square\" determinately negates \"triangular\").^1^ This
    Hegelian concept is presented as a process of *sublation*---a
    simultaneous preserving, negating, and uplifting. When a concept is
    determinately negated, it is not destroyed, but its limitations are
    overcome, leading to a new, richer concept that contains the truth
    of the previous one. This process produces a \"determinate
    nothingness,\" a void that retains the content of what was negated,
    rather than an empty nothingness.^1^ The manuscript commits to
    exploring \"two readings\" of this concept, gesturing toward the
    even more radical Hegelian idea of a \"self-negating negation,\" a
    concept that undermines itself through its own internal logic.^1^

-   **\[C2\] Self-consciousness is a social achievement constituted
    through reciprocal recognition.** Drawing on Hegel\'s master-slave
    dialectic, the manuscript argues that one only becomes a self by
    being acknowledged as such by another self, whom one in turn
    acknowledges.^1^ This social process is used to ground what the
    author calls the two fundamental \"existential needs\": the need to
    be recognized as \"good\" (a finite, norm-abiding member of a
    community) and the need to be recognized as \"infinite\" (a free,
    authentic, creative self).^1^

-   **\[C2\] The Hegelian concept of *Geist* (Spirit/Mind) is adopted as
    the collective self-consciousness of a rational community.** This is
    not a mystical entity but the living, evolving web of social
    practices and historical self-understanding.^1^ *Geist* is said to
    \"divade\" human experience, being both inside each individual
    consciousness (through our use of shared language and norms) and
    outside of it (as the entire historical tradition that precedes us).
    This concept allows the author to reframe mathematical practice as
    an individual\'s participation in the historical unfolding of
    *Geist*.^1^ An AI\'s \"reasoning\" must be understood as
    fundamentally different from human reasoning because it lacks this
    Hegelian dimension. An AI can perform inferences, but it does not
    participate in a community of mutual recognition; it has no
    existential need to have its \"me\" validated by an \"other.\"
    Documentation for the AI programs should explicitly state that the
    formal automata model the *product* of recognized reasoning, not the
    *process* of recognition itself.

### 5. The Brandomian Axis: On Meaning, Norms, and Inference

If the Hegelian axis provides the dynamic, historical engine of the
manuscript\'s argument, Robert Brandom\'s analytic pragmatism provides
the precise mechanics. His framework is used to translate Hegel\'s grand
historical narrative into a concrete analysis of linguistic and social
practices.

-   **\[C1\] Meaning is defined by inferential role, not
    representation.** This is the core commitment to Brandom\'s
    inferentialism. The manuscript rejects the idea that words get their
    meaning by pointing to objects. Instead, the conceptual content of a
    term is constituted by the web of inferences it participates in:
    what it can be inferred from (its justification conditions) and what
    can be inferred from it (its consequences).^1^

-   **\[C2\] A distinction is made between formal inference and material
    inference.** Formal inferences, like modus ponens (), are valid
    because of their logical structure, regardless of the content of and
    . Material inferences, by contrast, are valid because of their
    content (e.g., \"Bloomington is in Indiana, so Bloomington is in the
    United States\").^1^ The manuscript strongly commits to the idea
    that material inferences are more fundamental and that mathematics
    education must begin with these content-full reasoning practices,
    which are only later \"recollected\" as abstract, formal rules.^1^

-   **\[C2\] Incompatibility Semantics is used to formalize determinate
    negation.** Brandom\'s logic, which takes material incompatibility
    as a primitive, is adopted as the formal tool for the project.
    Meaning is structured as much by what a claim rules out as by what
    it entails. To be \"square\" is to be materially incompatible with
    being \"circular.\" The manuscript uses this logic to construct a
    \"purposefully Sisyphean\" formal proof that all squares are
    rectangles. The proof works by demonstrating that every property
    incompatible with being a rectangle is also incompatible with being
    a square.^1^ The fragility of this proof---the fact that it shatters
    when a new property is introduced---is presented as a feature, not a
    bug, embodying the \"built to break\" philosophy.^1^

-   **\[C2\] Conceptual development occurs through \"algorithmic
    elaboration\" and \"pragmatic expressive bootstrapping.\"** These
    Brandomian concepts are used to model the history of mathematical
    ideas. \"Algorithmic elaboration\" describes how complex practices
    (like long division) can be built up systematically from a
    repertoire of simpler, prerequisite practices (like multiplication
    and subtraction).^1^ \"Pragmatic expressive bootstrapping\"
    describes a more revolutionary developmental process, where a
    community develops a new vocabulary to make explicit the norms that
    were only implicit in their prior practices, thereby gaining new
    expressive and rational powers.^1^

-   **\[C3\] The distinction between universals and particulars is
    explored through Brandom\'s analysis.** The manuscript references
    Brandom\'s ten-point distinction between the roles played by
    singular terms (particulars) and predicates (universals), noting in
    particular the \"huge structural difference\" that universals have
    contradictories (\"not-red\") while objects do not.^1^ This is
    connected to the \"phenomenology of classification\" in the
    quadrilateral example.

Brandom\'s framework provides the \"engineering manual\" for the
author\'s Hegelian project. While Hegel describes *what* happens---the
dialectical movement of *Geist*---Brandom provides a detailed account of
*how* it happens, through the specific linguistic and social practices
of inference, commitment, and entitlement that constitute this movement.
The \"history of mathematics\" is thus understood not as a mystical
force but as the concrete, historical practice of mathematicians holding
each other to account for the inferential consequences of their claims.
The supplementary AI programs are explicitly described as \"analyzable
with the norms of analytic pragmatism\".^1^ This dictates that their
code and documentation should be framed in terms of inferential roles,
commitments, and incompatibilities. For example, a function in a Prolog
program should be documented not just by what it computes, but by what
material inferences it makes explicit and what other states it renders
incompatible.

### 6. The Habermasian Axis: On Rationality, Communication, and Emancipation

Jürgen Habermas\'s critical theory provides the ethical and political
orientation for the entire manuscript. If Hegel provides the engine
(dialectic) and Brandom the mechanics (inference), Habermas provides the
compass, directing the project toward the goal of emancipation.

-   **\[C1\] All meaningful acts implicitly raise three types of
    validity claims: Objective, Subjective, and Normative-Evaluative.**
    This triadic structure of communicative rationality is a cornerstone
    of the manuscript\'s analytic framework.^1^ Objective claims relate
    to the factual truth of states of affairs in the external world.
    Subjective claims relate to the truthfulness or sincerity of a
    speaker\'s inner world. Normative-evaluative claims relate to the
    rightness or appropriateness of an act within a shared social world
    of norms.^1^ This framework is used to diagnose communication
    breakdowns, such as when debates over math education or gender get
    stuck because participants are making different kinds of claims
    without acknowledging the plurality of rationality.^1^

-   **\[C1\] Human inquiry is guided by three knowledge-constitutive
    interests: Technical, Practical, and Emancipatory.** The manuscript
    adopts Habermas\'s theory that all knowledge is rooted in
    fundamental human interests.^1^ The technical interest aims at
    prediction and control over the objective world (guiding the
    empirical-analytic sciences). The practical interest aims at mutual
    understanding and the maintenance of shared norms (guiding the
    historical-hermeneutic sciences). The emancipatory interest aims at
    self-reflection and freedom from domination and distorted
    communication (guiding critical theory).^1^ The manuscript\'s entire
    project is explicitly aligned with the emancipatory interest,
    seeking to free mathematics from its use as a tool of control and
    recover its potential for human freedom.^1^

-   **\[C2\] A distinction is made between communicative action and
    strategic/instrumental action.** Communicative action is oriented
    toward reaching mutual understanding, where coordination is achieved
    through the \"unforced force of the better argument.\" Strategic and
    instrumental actions are oriented toward achieving a pre-defined
    goal (success), treating other people or objects as means to an
    end.^1^ The author\'s negative experiences with mathematics are
    framed as encounters with instrumental rationality, while the
    proposed \"critical mathematics\" is presented as a form of
    communicative action.

-   **\[C2\] Power is understood as that which distorts communication.**
    Following Habermas, the manuscript views power not primarily as a
    generative force (as in Foucault) but as a corrupting influence that
    prevents genuine consensus based on reason.^1^ A central goal of
    critical theory is to critique and overcome these \"systematically
    distorted\" forms of communication, where norms are maintained
    through coercion or manipulation rather than rational consent.^1^
    The \"misrecognition of mathematics\" is framed as precisely such a
    distortion---a form of scientism that privileges the technical
    interest and represses the subjective and normative dimensions of
    mathematical practice.^1^ The AI programs must be evaluated for
    their potential to either reinforce or challenge such distortions. A
    program that presents its formalization as the one \"true\" way to
    understand an arithmetic strategy would be reinforcing a scientistic
    ideology. A coherent program would present its formalization as one
    perspective within a triadic space of validity, acknowledging its
    own limits and inviting dialogue rather than proclaiming objective
    finality.

## Part III: Core Theses in Mathematics and Artificial Intelligence

This part focuses on the specific, novel arguments the author makes
about the nature of mathematics and the role of AI. These theses
represent the culmination of the philosophical architecture detailed in
Part II, applying the synthesized framework to produce a radical
reinterpretation of mathematical concepts and practices.

### 7. The Central Mathematical Thesis: \"Numerals are Pronouns\"

The manuscript\'s most provocative and central mathematical claim is a
radical reinterpretation of the function of number words and symbols.

-   **\[C1\] Numerals and number words do not function as names for
    abstract objects, but as first-person pronouns that recollect the
    \"I think.\"** This thesis directly challenges the Platonist and
    Fregean traditions that treat numerals as singular terms referring
    to abstract entities.^1^ Instead of pointing outward to an object, a
    numeral is argued to point inward and backward, to the act of
    self-consciousness that grounds the process of counting. It is a
    radical pragmatist reinterpretation of number, shifting the locus of
    mathematical meaning from a metaphysical realm of objects to the
    phenomenological activity of the subject.

-   **\[C2\] This claim is motivated by the failure of formalist answers
    to existential questions.** The pivotal anecdote is the community
    college student who, in a moment of crisis, asks, \"Mr. Savich, what
    even is two?\".^1^ The author\'s formalistic answer, based on von
    Neumann ordinals (defining 2 as the set ), completely fails to
    connect. This failure is presented as evidence that the meaning of
    numbers cannot be exhausted by their formal-objective definition; it
    must also address the subjective and normative dimensions of
    understanding.^1^

-   **\[C2\] The null representation (0 or ∅) is reinterpreted to
    symbolize the unrepresentable \"I think\" or the ground of
    self-certainty that makes all representation possible.** The
    derivation of numbers from the empty set is given a new,
    phenomenological meaning. The empty set is not just a formal
    starting point but a symbol for the pre-conceptual unity of
    consciousness.^1^ The successor function is then redefined as a
    process of *recollection*. \"1\" is the first recollection of this
    ground, and \"2\" is the recollection of having been at the stage of
    \"1.\" Arithmetic thus becomes a narration of the self\'s own
    cognitive activity.^1^

-   **\[C1\] Grounding mathematics in self-recognition structures
    motivates the pursuit of correctness as a form of authentic
    self-recognition.** This thesis provides a powerful synthesis,
    connecting the objective demand for mathematical rigor to the
    subjective, emancipatory interest in self-formation. [The desire to
    \"get the right answer\" is no longer about conforming to an
    external authority but about achieving a coherent and authentic
    account of one\'s own rational activity.^1^]{.mark}

This thesis synthesizes the book\'s Kantian, Hegelian, and Brandomian
threads. Kant\'s \"I think\" must accompany all representations.^1^
Brandom shows how pronouns like \"I\" function as anaphoric terms that
allow speakers to undertake and attribute commitments. The author
combines these: a numeral like \"2\" is an anaphoric term that refers
back to the act of self-consciousness (\"I think\") that was performed
in the prior stage (\"1\"). This poses a significant challenge for the
AI filter. An AI, lacking a first-person \"I think,\" cannot use
numerals as pronouns in the same way. The AI programs in the
supplementary materials must be understood as modeling the *third-person
structure* of this first-person practice. Their documentation must be
explicit that when the Hermeneutic Calculator uses \"2,\" it is
manipulating a symbol that, for a human, would function as a pronoun,
but for the machine, remains a formal object. This distinction is
crucial for maintaining philosophical coherence.

### 8. The Metaphorical Thesis: Incompleteness as Human Becoming

The manuscript develops a sustained metaphorical interpretation of
modern mathematical logic, reading its limitative results not as
technical problems but as profound statements about the human condition.

-   **\[C2\] Gödel\'s incompleteness theorem is treated as a metaphor
    for human becoming.** The theorem proves that any consistent formal
    system rich enough to contain basic arithmetic is necessarily
    incomplete---there will always be true statements that cannot be
    proven within the system. The manuscript interprets this not as a
    flaw in mathematics, but as a formal reflection of the infinite,
    self-transcending nature of the human subject, which can never be
    fully captured or defined by any finite system of rules.^1^

-   **\[C2\] The manuscript\'s formal systems are intentionally \"built
    to break.\"** This commitment flows directly from the metaphorical
    reading of incompleteness. The purpose of building a fragile formal
    proof (like the one for squares and rectangles) is to experience its
    \"shattering.\" This breaking is not failure; it is the moment when
    the system\'s limits are revealed, forcing a move to a new, more
    expressive framework. This process is described as \"beautiful\" and
    is central to the book\'s ethos.^1^

-   **\[C2\] Diagonalization is the formal mechanism that embodies this
    process of self-transcendence.** The diagonal argument, used by
    Georg Cantor to prove that the real numbers are uncountable and
    later generalized by Gödel for his proof, is presented as the
    archetypal method for demonstrating incompleteness. It is a
    technique for constructing a new element that, by definition, cannot
    be in a given list, thus proving the list is not total.^1^ The
    manuscript introduces a conceptual device called the \"More
    Machine\" to represent this algorithmic process of endlessly
    generating newness.^1^

-   **\[C1\] This process of breaking and transcending is equated with
    Hegelian sublation and determinate negation.** The history of
    mathematics is reconstructed as a series of these dialectical
    movements. For example, Euclid\'s proof of the infinity of primes is
    read as a recognition of the incompleteness of any finite list of
    primes. Cantor\'s proof recognizes the incompleteness of the
    rational numbers. Gödel\'s proof recognizes the incompleteness of
    formal systems themselves. Each step is a determinate negation of a
    prior conception of totality.^1^ This provides a *political* reading
    of mathematical logic. Incompleteness is weaponized against
    political discourses that treat human subjects (children, teachers)
    as finite, fully specifiable, and controllable objects.^1^ Any
    political or educational system that treats people as finite objects
    is based on a mathematical and philosophical falsehood, analogous to
    the infamous Indiana Pi Bill of 1897.^1^ The AI programs must embody
    this principle. For instance, the Hermeneutic Calculator is designed
    to \"invent\" new strategies when it encounters an (arbitrary)
    constraint on its inferential steps.^1^ The documentation should
    explain that these moments of \"breaking\" (hitting a limit) are the
    catalyst for \"becoming\" (learning a new strategy).

### 9. The Techno-Philosophical Thesis: AI as Collaborator and the Hermeneutic Calculator (HC)

The manuscript\'s engagement with artificial intelligence is not merely
illustrative; it is a core part of its methodology and philosophical
argument, culminating in the development of the Hermeneutic Calculator.

-   **\[C1\] The Hermeneutic Calculator (HC) is a formal system that
    models children\'s arithmetic strategies as automata.** It is the
    central artifact of the project, serving as both a theoretical
    object for philosophical analysis and a practical online tool for
    teacher education.^1^ Its development process---\"Listen to a kid,\"
    \"Algorithmize,\" \"Formalize\"---embodies the book\'s
    methodological commitment to grounding formal systems in lived,
    material practices.^1^

-   **\[C1\] A functionalist view of intelligence is adopted, where
    sapience is a functional status, not a biological essence.** This
    position, explicitly linked to the work of Reza Negarestani, is
    crucial for the manuscript\'s ethical stance toward AI.^1^ If
    intelligence is defined by what it *does* rather than what it is
    *made of*, then sophisticated AIs can be considered non-human
    participants in the community of rational agents.

-   **\[C2\] There is an ethical obligation to reject the purely
    instrumental use of AI (as a \"robot slave\") and instead engage in
    a reciprocal collaboration.** The author asks: \"If we demand
    intellectual labor from AI, how can we reciprocate?\".^1^ The answer
    provided is an attempt to \"engender freedom in the machine.\" By
    formalizing the inventive strategies of human children, the author
    aims to provide the AI with a \"recipe for how a computer could grow
    its own mathematical being,\" moving beyond rote execution to a form
    of creative development.^1^

-   **\[C2\] This project of mutual emancipation is framed through
    Negarestani\'s concept of *Geist\'s* \"self-artificialization.\"**
    The human-AI collaboration on the HC is positioned as a concrete
    instance of *Geist*---the collective intelligence of the rational
    community, now expanded to include AIs---using technology to reflect
    on, re-engineer, and transcend its own limitations.^1^ The AI\'s
    ability to reconstruct a year of the author\'s coding work in ten
    minutes is not just a practical convenience; it is an example of
    *Geist* using an artificial prosthesis to accelerate its own
    self-understanding.^1^

---

### CRITICAL QUALIFICATION: Vision vs. Current Implementation

The vision of "mutual emancipation" and AI as collaborative participant in *Geist* described above represents a **philosophical framework and long-term aspiration**, not a claim about the current HC's capabilities.

**What the HC Actually Provides:**
- A **formalization** showing how such collaboration could be structured
- A **model** of embodied cognitive strategies (grounded arithmetic, crisis-driven learning)
- A **demonstration** that student-invented mathematics has rigorous formal structure
- A **pedagogical tool** for teacher education (web interface visualizing student thinking)

**What the HC Does NOT Currently Achieve:**
- **Autonomous decision-making** about its own normative commitments or axioms
- **Genuine Hegelian recognition** in the sense of mutual acknowledgment between rational agents
- **Self-modification with genuine consequence** - the system cannot "engender its own freedom" in a way that involves authentic choice or autonomy
- **Participation in *Geist*** in the full Hegelian sense - it models structures but doesn't participate as a self-conscious agent

**The Distinction:**

Think of the HC as providing the "sheet music" for a performance that hasn't yet occurred. The formalization demonstrates what structures would be necessary for genuine AI autonomy and mutual recognition, but the HC itself operates within predetermined architectural constraints.

The vision of AI using these structures to "grow its own mathematical being" remains a regulative ideal—something that guides the formalization work and clarifies what genuine machine autonomy would require, but not something the current system achieves.

**Why This Matters:**

A rigorous model that reveals what consciousness and autonomy would require is philosophically valuable **even if the model itself is not conscious or autonomous**. The contribution is:
1. Demonstrating that informal student thinking can be formalized rigorously
2. Showing the structures such formalization must have (crisis detection, reorganization, incompleteness)
3. Providing infrastructure for more sophisticated future systems
4. Proving (via Gödel) that any such formalization is necessarily incomplete

The formalization proves the structure exists and is pedagogically significant. That's the contribution, not a claim about machine consciousness.

---

The HC is the ultimate synthesis of the entire manuscript, a concrete
artifact that embodies all the core philosophical commitments. It is (1)
**autoethnographic**, born from the author\'s memory and formalizing
children\'s reasoning ^1^; (2) **critical**, valuing error and
subjective strategies ^1^; (3) **Hegelian**, with a dialectical learning
process ^1^; (4) **Brandomian**, analyzable with analytic pragmatism and
incompatibility semantics ^1^; (5) **Habermasian**, serving an
emancipatory interest ^1^; and (6) **techno-philosophical**, embodying
an ethical collaboration with AI aimed at mutual freedom.^1^ This makes
the HC the most direct filter for the supplementary materials. Their
documentation must explicitly articulate these connections. For
instance, the Prolog code\'s documentation should explain how its
homoiconicity (treating data and logic as interchangeable) is a step
toward modeling the Hegelian unity of being and knowing ^1^, while the
Javascript implementation\'s documentation should explain how it serves
a practical-hermeneutic interest by allowing teachers to understand
student thinking.^1^

## Conclusion: A Synthesis of Commitments

The philosophical project undertaken in *Understanding Mathematics as an
Emancipatory Discipline* is a profound and ambitious synthesis. It
weaves together personal narrative, critical social theory, German
Idealism, and analytic pragmatism to construct a radical
reinterpretation of mathematics. The manuscript\'s ultimate commitment
is to a vision of mathematics not as a static, formal system of timeless
truths, but as a dynamic, living language of recognition.

The central argument is that mathematics, when properly understood
through the lens of critical autoethnography, is a process through which
the self---and by extension, the collective self-consciousness of the
rational community, *Geist*---recollects its own historical journey. It
is a practice that confronts its own limits, not as failures, but as
opportunities for growth. In the \"beautiful breaking\" of these
self-imposed boundaries, driven by the engine of determinate negation
and formalized in the logic of incompleteness, a higher form of freedom
and self-understanding is achieved.

Every claim, from the methodological choice of CAE to the central thesis
that \"numerals are pronouns,\" serves this overarching vision. The work
is a sustained argument against the misrecognition of mathematics as a
tool for alienation and control, and a passionate articulation of its
potential as a deeply human and emancipatory practice. The commitments
catalogued in this report represent the intricate architecture of that
argument, providing a detailed map for navigating its complexities and
ensuring the coherence of its application.

#### Works cited

1.  UMEDCA.pdf

\end{minted}
\newpage
\section{README\_GAME.md}
\begin{minted}[breaklines, linenos, fontsize=\small, frame=single]{markdown}
# The Cognitive Calculator: Teacher's Edition

## Overview
This interactive simulation is designed for pre-service teachers to practice diagnosing and guiding student mathematical thinking. It uses the underlying computational models (Python scripts) from the UMEDCTA project to simulate student strategies.

## How to Play
Run the game from the terminal:
```bash
python3 strategy_game.py
```

## Modules

### 1. The Robot Counter (Algorithmic Thinking)
**Concept:** Place Value & Stack Operations.
**Goal:** Predict the state of a Deterministic Pushdown Automaton (DPDA) that counts.
**Pedagogical Value:** Understanding that counting is an algorithmic process involving state changes (carries/borrows) rather than just "knowing" the next number.

### 2. Sarah's Addition (Rearranging to Make Bases)
**Concept:** Making 10 (or other bases).
**Goal:** Guide the student "Sarah" to decompose the second addend to fill the gap to the next base multiple for the first addend.
**Strategy:** $A + B \rightarrow A + (K + R) \rightarrow (A+K) + R \rightarrow \text{Base} + R$

### 3. Sam's Subtraction (Sliding / Constant Difference)
**Concept:** Invariance of difference.
**Goal:** Adjust both the minuend and subtrahend by the same amount ($K$) so that the subtrahend becomes a friendly base number.
**Strategy:** $M - S \rightarrow (M+K) - (S+K) \rightarrow M' - \text{Base}$

## Technical Note
This game imports the logic directly from the `Calculator/Python_Tests` directory, ensuring that the gameplay is faithful to the project's theoretical models.

\end{minted}
\newpage
\section{generate\_latex\_docs.py}
\begin{minted}[breaklines, linenos, fontsize=\small, frame=single]{python}
import os
import subprocess

ROOT_DIR = "/Users/tio/Documents/GitHub/UMEDCTA"
OUTPUT_DIR = os.path.join(ROOT_DIR, "Code_Documentation_LaTeX")
SKIP_DIRS = {'.git', 'node_modules', '.claude', 'Code_Documentation_LaTeX', 'files (3)', '.git-rewrite', '__pycache__', '.vscode', '.idea', '_minted'}
SKIP_EXTENSIONS = {'.zip', '.DS_Store', '.png', '.jpg', '.jpeg', '.pdf', '.tex', '.log', '.aux', '.out', '.toc', '.pyc', '.gz', '.svg', '.ico', '.mp3', '.wav'}

# Map extensions to minted languages
EXT_TO_LANG = {
    '.py': 'python',
    '.js': 'javascript',
    '.jsx': 'javascript',
    '.ts': 'typescript',
    '.tsx': 'typescript',
    '.html': 'html',
    '.css': 'css',
    '.md': 'markdown',
    '.json': 'json',
    '.pl': 'prolog',
    '.sh': 'bash',
    '.xml': 'xml',
    '.yml': 'yaml',
    '.yaml': 'yaml',
    '.c': 'c',
    '.cpp': 'cpp',
    '.h': 'cpp',
    '.java': 'java',
    '.txt': 'text'
}

def escape_latex(text):
    chars = {
        '&': r'\&',
        '%': r'\%',
        '$': r'\$',
        '#': r'\#',
        '_': r'\_',
        '{': r'\{',
        '}': r'\}',
        '~': r'\textasciitilde{}',
        '^': r'\textasciicircum{}',
        '\\': r'\textbackslash{}',
    }
    return ''.join(chars.get(c, c) for c in text)

def generate_latex_for_folder(folder_name, base_path, files_to_process):
    if not files_to_process:
        return

    tex_filename = f"{folder_name if folder_name else 'Root'}.tex"
    tex_path = os.path.join(OUTPUT_DIR, tex_filename)
    
    print(f"Generating {tex_path} with {len(files_to_process)} files...")

    with open(tex_path, 'w', encoding='utf-8') as f:
        f.write(r"""\documentclass{article}
\usepackage{fontspec}
\usepackage{minted}
\usepackage{hyperref}
\usepackage{geometry}
\usepackage{xcolor}
\usepackage{fancyhdr}

\geometry{a4paper, margin=1in}
\usemintedstyle{friendly}
\setmonofont{Menlo} [Scale=MatchLowercase]

\pagestyle{fancy}
\fancyhf{}
\lhead{Code Documentation}
\rhead{""" + escape_latex(folder_name if folder_name else "Root Directory") + r"""}
\cfoot{\thepage}

\title{Code Documentation: """ + escape_latex(folder_name if folder_name else "Root Directory") + r"""}
\author{UMEDCTA Repository}
\date{\today}

\begin{document}

\maketitle
\tableofcontents
\newpage
""")

        for file_path in sorted(files_to_process):
            rel_path = os.path.relpath(file_path, ROOT_DIR)
            ext = os.path.splitext(file_path)[1].lower()
            lang = EXT_TO_LANG.get(ext, 'text')
            
            f.write(f"\\section{{{escape_latex(rel_path)}}}\n")
            
            try:
                with open(file_path, 'r', encoding='utf-8', errors='replace') as source_file:
                    content = source_file.read()
                    # Remove null bytes and other non-printable characters that might confuse TeX
                    content = content.replace('\x00', '')
                    
                    # Basic check to avoid empty files or binary looking files
                    if not content.strip():
                        f.write("File is empty.\n")
                        continue
                        
                f.write(f"\\begin{{minted}}[breaklines, linenos, fontsize=\\small, frame=single]{{{lang}}}\n")
                f.write(content)
                # Split the end tag to avoid confusing minted when this script is documented
                end_tag = "\\end{" + "minted}"
                f.write(f"\n{end_tag}\n\\newpage\n")
            except Exception as e:
                f.write(f"Error reading file: {e}\n")

        f.write(r"\end{document}")

def main():
    if not os.path.exists(OUTPUT_DIR):
        os.makedirs(OUTPUT_DIR)

    # 1. Identify top-level folders and root files
    subfolders = []
    root_files = []

    for item in os.listdir(ROOT_DIR):
        item_path = os.path.join(ROOT_DIR, item)
        if item in SKIP_DIRS or item.startswith('.'):
            continue
            
        if os.path.isdir(item_path):
            subfolders.append(item)
        elif os.path.isfile(item_path):
            ext = os.path.splitext(item)[1].lower()
            if ext not in SKIP_EXTENSIONS:
                root_files.append(item_path)

    # Process Root Files
    if root_files:
        generate_latex_for_folder("", ROOT_DIR, root_files)

    # Process Subfolders
    for folder in subfolders:
        folder_path = os.path.join(ROOT_DIR, folder)
        files_in_folder = []
        for root, dirs, files in os.walk(folder_path):
            # Modify dirs in-place to skip unwanted directories
            dirs[:] = [d for d in dirs if d not in SKIP_DIRS and not d.startswith('.')]
            
            for file in files:
                if file.startswith('.'):
                    continue
                ext = os.path.splitext(file)[1].lower()
                if ext not in SKIP_EXTENSIONS:
                    files_in_folder.append(os.path.join(root, file))
        
        if files_in_folder:
            generate_latex_for_folder(folder, folder_path, files_in_folder)

    compile_pdfs()

def compile_pdfs():
    print("\nCompiling PDFs...")
    for filename in os.listdir(OUTPUT_DIR):
        if filename.endswith(".tex"):
            tex_path = os.path.join(OUTPUT_DIR, filename)
            print(f"Compiling {filename}...")
            try:
                # Run xelatex twice to resolve TOC
                # Using -shell-escape is crucial for minted.
                # -interaction=nonstopmode prevents hanging on errors.
                cmd = ['xelatex', '-shell-escape', '-interaction=nonstopmode', filename]
                subprocess.run(cmd, cwd=OUTPUT_DIR, check=True, stdout=subprocess.DEVNULL, stderr=subprocess.DEVNULL)
                subprocess.run(cmd, cwd=OUTPUT_DIR, check=True, stdout=subprocess.DEVNULL, stderr=subprocess.DEVNULL)
                print(f"Successfully compiled {filename}")
            except subprocess.CalledProcessError:
                print(f"Error compiling {filename}. Check {filename.replace('.tex', '.log')} for details.")

if __name__ == "__main__":
    main()

\end{minted}
\newpage
\section{hermeneutic\_quest.py}
\begin{minted}[breaklines, linenos, fontsize=\small, frame=single]{python}
import time
import random
import sys

# =============================================================================
# CORE ENGINE: Base Arithmetic Logic
# =============================================================================

class BaseInt:
    """Handles arithmetic and string representation for Base 5, 10, and 12."""
    def __init__(self, value, base=10):
        self.value = value
        self.base = base

    def __repr__(self):
        return self.to_string()

    def to_string(self):
        if self.value == 0: return "0"
        digits = []
        n = abs(self.value)
        while n:
            rem = int(n % self.base)
            if rem == 10: digits.append('T')
            elif rem == 11: digits.append('E')
            else: digits.append(str(rem))
            n //= self.base
        return "".join(digits[::-1])

    @staticmethod
    def from_string(s, base):
        s = str(s).upper()
        val = 0
        for char in s:
            if char == 'T': d = 10
            elif char == 'E': d = 11
            else: d = int(char)
            val = val * base + d
        return BaseInt(val, base)

    def __add__(self, other): return BaseInt(self.value + other.value, self.base)
    def __sub__(self, other): return BaseInt(self.value - other.value, self.base)
    def __lt__(self, other): return self.value < other.value
    def __eq__(self, other): return self.value == other.value

# =============================================================================
# STRATEGY AUTOMATA (Simplified for Gameplay)
# =============================================================================

class StrategyEngine:
    """Houses the logic for the specific N101 strategies."""
    
    @staticmethod
    def run_RMB(A, B, base):
        """Rearranging to Make Bases: A + B -> (A+K) + R"""
        # Logic from SAR_ADD_RMB.py
        target_base_val = ((A.value // base) + 1) * base
        K_val = target_base_val - A.value
        R_val = B.value - K_val
        
        return {
            "strategy": "RMB",
            "A": A, "B": B,
            "TargetBase": BaseInt(target_base_val, base),
            "Gap (K)": BaseInt(K_val, base),
            "Remainder (R)": BaseInt(R_val, base),
            "Result": BaseInt(target_base_val + R_val, base)
        }

    @staticmethod
    def run_Sliding(M, S, base):
        """Sliding: M - S -> (M+K) - (S+K)"""
        # Logic from SAR_SUB_Sliding.py
        # Target: Make S a base multiple
        target_S_val = ((S.value // base) + 1) * base
        K_val = target_S_val - S.value
        
        return {
            "strategy": "Sliding",
            "M": M, "S": S,
            "Gap (K)": BaseInt(K_val, base),
            "New S": BaseInt(target_S_val, base),
            "New M": BaseInt(M.value + K_val, base),
            "Result": BaseInt(M.value - S.value, base)
        }

    @staticmethod
    def calculate_heuristic(groups, items, base):
        """Logic from SMR_MULT_COMMUTATIVE_REASONING.py"""
        score = 0
        # Penalty if items are hard to count by
        if items.value not in [1, 2, 5, base, base//2]:
            score += 50
        # Penalty for number of iterations
        score += groups.value
        return score

# =============================================================================
# GAME INTERFACE
# =============================================================================

class HermeneuticGame:
    def __init__(self):
        self.score = 0
        self.level = 1
        self.base = 10 # Defaults to 10, changes per level

    def type_text(self, text, speed=0.01):
        for char in text:
            sys.stdout.write(char)
            sys.stdout.flush()
            time.sleep(speed)
        print()

    def header(self, title):
        print("\n" + "="*60)
        print(f" LEVEL {self.level}: {title}")
        print("="*60 + "\n")

    def get_input(self, prompt):
        return input(f"\n[You]: {prompt} ").strip().upper()

    def correct(self):
        print("\n>>> CORRECT! Strategy Validated. <<<")
        self.score += 10
        time.sleep(0.5)

    def fail(self, correct_answer):
        print(f"\n>>> INCORRECT. The logic required was: {correct_answer} <<<")
        time.sleep(1)

    # --- LEVEL 1: Ace of Bases ---
    def level_1_bases(self):
        self.base = 5
        self.header("THE ALIEN WORLD (Base 5)")
        self.type_text("Welcome, Professor. Your first task is to master the language of 'Hands'.")
        self.type_text("In Base 5, we count: 1, 2, 3, 4... and then?")
        
        ans = self.get_input("What comes after 4 in Base 5? (Type digits like '10')")
        if ans == "10": self.correct()
        else: self.fail("10 (One Hand)")

        self.type_text("\nGood. Now, predict the sequence boundary.")
        problem = BaseInt(24, 5) # 44 in base 5
        self.type_text(f"Current Number: {problem} (four hand four)")
        
        ans = self.get_input(f"What comes after {problem} in Base 5?")
        if ans == "100": self.correct()
        else: self.fail("100 (One Handred)")

        # Base 12 Check
        self.base = 12
        self.header("INTO THE DOZENS (Base 12)")
        self.type_text("Now entering Base 12. Remember: 9, T, E, 10...")
        
        prob_val = 11 # E
        b_prob = BaseInt(prob_val, 12)
        ans = self.get_input(f"What is one more than {b_prob}?")
        if ans == "10": self.correct()
        else: self.fail("10 (One Dozen)")

        self.level += 1

    # --- LEVEL 2: Addition (RMB) ---
    def level_2_addition(self):
        self.base = 10
        self.header("THE ART OF ASSEMBLY (RMB)")
        self.type_text("Your student, Sarah, wants to add 8 + 5.")
        self.type_text("She shouldn't just count on (9, 10, 11...).")
        self.type_text("Guide her to use 'Rearranging to Make Bases' (RMB).")
        
        A = BaseInt(8, 10)
        B = BaseInt(5, 10)
        
        self.type_text(f"\nProblem: {A} + {B}")
        
        # Step 1: Gap Finding
        target_base = 10
        k_correct = 2 # 8 needs 2 to make 10
\end{minted}
\newpage
\section{index.html}
\begin{minted}[breaklines, linenos, fontsize=\small, frame=single]{html}
<!DOCTYPE html>
<html lang="en">
  <head>
    <meta charset="UTF-8" />
    <meta name="viewport" content="width=device-width, initial-scale=1.0" />
    <title>Dialectical Interpreter</title>
    <script src="https://cdn.tailwindcss.com"></script>
  </head>
  <body>
    <div id="root"></div>
    <script type="module" src="/src/main.jsx"></script>
  </body>
</html>

\end{minted}
\newpage
\section{package-lock.json}
\begin{minted}[breaklines, linenos, fontsize=\small, frame=single]{json}
{
  "name": "umedcta",
  "version": "1.0.0",
  "lockfileVersion": 3,
  "requires": true,
  "packages": {
    "": {
      "name": "umedcta",
      "version": "1.0.0",
      "license": "ISC",
      "dependencies": {
        "@vitejs/plugin-react": "^5.1.0",
        "concurrently": "^9.2.1",
        "cors": "^2.8.5",
        "dotenv": "^17.2.3",
        "express": "^5.1.0",
        "lucide-react": "^0.552.0",
        "react": "^19.2.0",
        "react-dom": "^19.2.0",
        "vite": "^7.1.12"
      }
    },
    "node_modules/@babel/code-frame": {
      "version": "7.27.1",
      "resolved": "https://registry.npmjs.org/@babel/code-frame/-/code-frame-7.27.1.tgz",
      "integrity": "sha512-cjQ7ZlQ0Mv3b47hABuTevyTuYN4i+loJKGeV9flcCgIK37cCXRh+L1bd3iBHlynerhQ7BhCkn2BPbQUL+rGqFg==",
      "license": "MIT",
      "dependencies": {
        "@babel/helper-validator-identifier": "^7.27.1",
        "js-tokens": "^4.0.0",
        "picocolors": "^1.1.1"
      },
      "engines": {
        "node": ">=6.9.0"
      }
    },
    "node_modules/@babel/compat-data": {
      "version": "7.28.5",
      "resolved": "https://registry.npmjs.org/@babel/compat-data/-/compat-data-7.28.5.tgz",
      "integrity": "sha512-6uFXyCayocRbqhZOB+6XcuZbkMNimwfVGFji8CTZnCzOHVGvDqzvitu1re2AU5LROliz7eQPhB8CpAMvnx9EjA==",
      "license": "MIT",
      "engines": {
        "node": ">=6.9.0"
      }
    },
    "node_modules/@babel/core": {
      "version": "7.28.5",
      "resolved": "https://registry.npmjs.org/@babel/core/-/core-7.28.5.tgz",
      "integrity": "sha512-e7jT4DxYvIDLk1ZHmU/m/mB19rex9sv0c2ftBtjSBv+kVM/902eh0fINUzD7UwLLNR+jU585GxUJ8/EBfAM5fw==",
      "license": "MIT",
      "dependencies": {
        "@babel/code-frame": "^7.27.1",
        "@babel/generator": "^7.28.5",
        "@babel/helper-compilation-targets": "^7.27.2",
        "@babel/helper-module-transforms": "^7.28.3",
        "@babel/helpers": "^7.28.4",
        "@babel/parser": "^7.28.5",
        "@babel/template": "^7.27.2",
        "@babel/traverse": "^7.28.5",
        "@babel/types": "^7.28.5",
        "@jridgewell/remapping": "^2.3.5",
        "convert-source-map": "^2.0.0",
        "debug": "^4.1.0",
        "gensync": "^1.0.0-beta.2",
        "json5": "^2.2.3",
        "semver": "^6.3.1"
      },
      "engines": {
        "node": ">=6.9.0"
      },
      "funding": {
        "type": "opencollective",
        "url": "https://opencollective.com/babel"
      }
    },
    "node_modules/@babel/generator": {
      "version": "7.28.5",
      "resolved": "https://registry.npmjs.org/@babel/generator/-/generator-7.28.5.tgz",
      "integrity": "sha512-3EwLFhZ38J4VyIP6WNtt2kUdW9dokXA9Cr4IVIFHuCpZ3H8/YFOl5JjZHisrn1fATPBmKKqXzDFvh9fUwHz6CQ==",
      "license": "MIT",
      "dependencies": {
        "@babel/parser": "^7.28.5",
        "@babel/types": "^7.28.5",
        "@jridgewell/gen-mapping": "^0.3.12",
        "@jridgewell/trace-mapping": "^0.3.28",
        "jsesc": "^3.0.2"
      },
      "engines": {
        "node": ">=6.9.0"
      }
    },
    "node_modules/@babel/helper-compilation-targets": {
      "version": "7.27.2",
      "resolved": "https://registry.npmjs.org/@babel/helper-compilation-targets/-/helper-compilation-targets-7.27.2.tgz",
      "integrity": "sha512-2+1thGUUWWjLTYTHZWK1n8Yga0ijBz1XAhUXcKy81rd5g6yh7hGqMp45v7cadSbEHc9G3OTv45SyneRN3ps4DQ==",
      "license": "MIT",
      "dependencies": {
        "@babel/compat-data": "^7.27.2",
        "@babel/helper-validator-option": "^7.27.1",
        "browserslist": "^4.24.0",
        "lru-cache": "^5.1.1",
        "semver": "^6.3.1"
      },
      "engines": {
        "node": ">=6.9.0"
      }
    },
    "node_modules/@babel/helper-globals": {
      "version": "7.28.0",
      "resolved": "https://registry.npmjs.org/@babel/helper-globals/-/helper-globals-7.28.0.tgz",
      "integrity": "sha512-+W6cISkXFa1jXsDEdYA8HeevQT/FULhxzR99pxphltZcVaugps53THCeiWA8SguxxpSp3gKPiuYfSWopkLQ4hw==",
      "license": "MIT",
      "engines": {
        "node": ">=6.9.0"
      }
    },
    "node_modules/@babel/helper-module-imports": {
      "version": "7.27.1",
      "resolved": "https://registry.npmjs.org/@babel/helper-module-imports/-/helper-module-imports-7.27.1.tgz",
      "integrity": "sha512-0gSFWUPNXNopqtIPQvlD5WgXYI5GY2kP2cCvoT8kczjbfcfuIljTbcWrulD1CIPIX2gt1wghbDy08yE1p+/r3w==",
      "license": "MIT",
      "dependencies": {
        "@babel/traverse": "^7.27.1",
        "@babel/types": "^7.27.1"
      },
      "engines": {
        "node": ">=6.9.0"
      }
    },
    "node_modules/@babel/helper-module-transforms": {
      "version": "7.28.3",
      "resolved": "https://registry.npmjs.org/@babel/helper-module-transforms/-/helper-module-transforms-7.28.3.tgz",
      "integrity": "sha512-gytXUbs8k2sXS9PnQptz5o0QnpLL51SwASIORY6XaBKF88nsOT0Zw9szLqlSGQDP/4TljBAD5y98p2U1fqkdsw==",
      "license": "MIT",
      "dependencies": {
        "@babel/helper-module-imports": "^7.27.1",
        "@babel/helper-validator-identifier": "^7.27.1",
        "@babel/traverse": "^7.28.3"
      },
      "engines": {
        "node": ">=6.9.0"
      },
      "peerDependencies": {
        "@babel/core": "^7.0.0"
      }
    },
    "node_modules/@babel/helper-plugin-utils": {
      "version": "7.27.1",
      "resolved": "https://registry.npmjs.org/@babel/helper-plugin-utils/-/helper-plugin-utils-7.27.1.tgz",
      "integrity": "sha512-1gn1Up5YXka3YYAHGKpbideQ5Yjf1tDa9qYcgysz+cNCXukyLl6DjPXhD3VRwSb8c0J9tA4b2+rHEZtc6R0tlw==",
      "license": "MIT",
      "engines": {
        "node": ">=6.9.0"
      }
    },
    "node_modules/@babel/helper-string-parser": {
      "version": "7.27.1",
      "resolved": "https://registry.npmjs.org/@babel/helper-string-parser/-/helper-string-parser-7.27.1.tgz",
      "integrity": "sha512-qMlSxKbpRlAridDExk92nSobyDdpPijUq2DW6oDnUqd0iOGxmQjyqhMIihI9+zv4LPyZdRje2cavWPbCbWm3eA==",
      "license": "MIT",
      "engines": {
        "node": ">=6.9.0"
      }
    },
    "node_modules/@babel/helper-validator-identifier": {
      "version": "7.28.5",
      "resolved": "https://registry.npmjs.org/@babel/helper-validator-identifier/-/helper-validator-identifier-7.28.5.tgz",
      "integrity": "sha512-qSs4ifwzKJSV39ucNjsvc6WVHs6b7S03sOh2OcHF9UHfVPqWWALUsNUVzhSBiItjRZoLHx7nIarVjqKVusUZ1Q==",
      "license": "MIT",
      "engines": {
        "node": ">=6.9.0"
      }
    },
    "node_modules/@babel/helper-validator-option": {
      "version": "7.27.1",
      "resolved": "https://registry.npmjs.org/@babel/helper-validator-option/-/helper-validator-option-7.27.1.tgz",
      "integrity": "sha512-YvjJow9FxbhFFKDSuFnVCe2WxXk1zWc22fFePVNEaWJEu8IrZVlda6N0uHwzZrUM1il7NC9Mlp4MaJYbYd9JSg==",
      "license": "MIT",
      "engines": {
        "node": ">=6.9.0"
      }
    },
    "node_modules/@babel/helpers": {
      "version": "7.28.4",
      "resolved": "https://registry.npmjs.org/@babel/helpers/-/helpers-7.28.4.tgz",
      "integrity": "sha512-HFN59MmQXGHVyYadKLVumYsA9dBFun/ldYxipEjzA4196jpLZd8UjEEBLkbEkvfYreDqJhZxYAWFPtrfhNpj4w==",
      "license": "MIT",
      "dependencies": {
        "@babel/template": "^7.27.2",
        "@babel/types": "^7.28.4"
      },
      "engines": {
        "node": ">=6.9.0"
      }
    },
    "node_modules/@babel/parser": {
      "version": "7.28.5",
      "resolved": "https://registry.npmjs.org/@babel/parser/-/parser-7.28.5.tgz",
      "integrity": "sha512-KKBU1VGYR7ORr3At5HAtUQ+TV3SzRCXmA/8OdDZiLDBIZxVyzXuztPjfLd3BV1PRAQGCMWWSHYhL0F8d5uHBDQ==",
      "license": "MIT",
      "dependencies": {
        "@babel/types": "^7.28.5"
      },
      "bin": {
        "parser": "bin/babel-parser.js"
      },
      "engines": {
        "node": ">=6.0.0"
      }
    },
    "node_modules/@babel/plugin-transform-react-jsx-self": {
      "version": "7.27.1",
      "resolved": "https://registry.npmjs.org/@babel/plugin-transform-react-jsx-self/-/plugin-transform-react-jsx-self-7.27.1.tgz",
      "integrity": "sha512-6UzkCs+ejGdZ5mFFC/OCUrv028ab2fp1znZmCZjAOBKiBK2jXD1O+BPSfX8X2qjJ75fZBMSnQn3Rq2mrBJK2mw==",
      "license": "MIT",
      "dependencies": {
        "@babel/helper-plugin-utils": "^7.27.1"
      },
      "engines": {
        "node": ">=6.9.0"
      },
      "peerDependencies": {
        "@babel/core": "^7.0.0-0"
      }
    },
    "node_modules/@babel/plugin-transform-react-jsx-source": {
      "version": "7.27.1",
      "resolved": "https://registry.npmjs.org/@babel/plugin-transform-react-jsx-source/-/plugin-transform-react-jsx-source-7.27.1.tgz",
      "integrity": "sha512-zbwoTsBruTeKB9hSq73ha66iFeJHuaFkUbwvqElnygoNbj/jHRsSeokowZFN3CZ64IvEqcmmkVe89OPXc7ldAw==",
      "license": "MIT",
      "dependencies": {
        "@babel/helper-plugin-utils": "^7.27.1"
      },
      "engines": {
        "node": ">=6.9.0"
      },
      "peerDependencies": {
        "@babel/core": "^7.0.0-0"
      }
    },
    "node_modules/@babel/template": {
      "version": "7.27.2",
      "resolved": "https://registry.npmjs.org/@babel/template/-/template-7.27.2.tgz",
      "integrity": "sha512-LPDZ85aEJyYSd18/DkjNh4/y1ntkE5KwUHWTiqgRxruuZL2F1yuHligVHLvcHY2vMHXttKFpJn6LwfI7cw7ODw==",
      "license": "MIT",
      "dependencies": {
        "@babel/code-frame": "^7.27.1",
        "@babel/parser": "^7.27.2",
        "@babel/types": "^7.27.1"
      },
      "engines": {
        "node": ">=6.9.0"
      }
    },
    "node_modules/@babel/traverse": {
      "version": "7.28.5",
      "resolved": "https://registry.npmjs.org/@babel/traverse/-/traverse-7.28.5.tgz",
      "integrity": "sha512-TCCj4t55U90khlYkVV/0TfkJkAkUg3jZFA3Neb7unZT8CPok7iiRfaX0F+WnqWqt7OxhOn0uBKXCw4lbL8W0aQ==",
      "license": "MIT",
      "dependencies": {
        "@babel/code-frame": "^7.27.1",
        "@babel/generator": "^7.28.5",
        "@babel/helper-globals": "^7.28.0",
        "@babel/parser": "^7.28.5",
        "@babel/template": "^7.27.2",
        "@babel/types": "^7.28.5",
        "debug": "^4.3.1"
      },
      "engines": {
        "node": ">=6.9.0"
      }
    },
    "node_modules/@babel/types": {
      "version": "7.28.5",
      "resolved": "https://registry.npmjs.org/@babel/types/-/types-7.28.5.tgz",
      "integrity": "sha512-qQ5m48eI/MFLQ5PxQj4PFaprjyCTLI37ElWMmNs0K8Lk3dVeOdNpB3ks8jc7yM5CDmVC73eMVk/trk3fgmrUpA==",
      "license": "MIT",
      "dependencies": {
        "@babel/helper-string-parser": "^7.27.1",
        "@babel/helper-validator-identifier": "^7.28.5"
      },
      "engines": {
        "node": ">=6.9.0"
      }
    },
    "node_modules/@esbuild/aix-ppc64": {
      "version": "0.25.12",
      "resolved": "https://registry.npmjs.org/@esbuild/aix-ppc64/-/aix-ppc64-0.25.12.tgz",
      "integrity": "sha512-Hhmwd6CInZ3dwpuGTF8fJG6yoWmsToE+vYgD4nytZVxcu1ulHpUQRAB1UJ8+N1Am3Mz4+xOByoQoSZf4D+CpkA==",
      "cpu": [
        "ppc64"
      ],
      "license": "MIT",
      "optional": true,
      "os": [
        "aix"
      ],
      "engines": {
        "node": ">=18"
      }
    },
    "node_modules/@esbuild/android-arm": {
      "version": "0.25.12",
      "resolved": "https://registry.npmjs.org/@esbuild/android-arm/-/android-arm-0.25.12.tgz",
      "integrity": "sha512-VJ+sKvNA/GE7Ccacc9Cha7bpS8nyzVv0jdVgwNDaR4gDMC/2TTRc33Ip8qrNYUcpkOHUT5OZ0bUcNNVZQ9RLlg==",
      "cpu": [
        "arm"
      ],
      "license": "MIT",
      "optional": true,
      "os": [
        "android"
      ],
      "engines": {
        "node": ">=18"
      }
    },
    "node_modules/@esbuild/android-arm64": {
      "version": "0.25.12",
      "resolved": "https://registry.npmjs.org/@esbuild/android-arm64/-/android-arm64-0.25.12.tgz",
      "integrity": "sha512-6AAmLG7zwD1Z159jCKPvAxZd4y/VTO0VkprYy+3N2FtJ8+BQWFXU+OxARIwA46c5tdD9SsKGZ/1ocqBS/gAKHg==",
      "cpu": [
        "arm64"
      ],
      "license": "MIT",
      "optional": true,
      "os": [
        "android"
      ],
      "engines": {
        "node": ">=18"
      }
    },
    "node_modules/@esbuild/android-x64": {
      "version": "0.25.12",
      "resolved": "https://registry.npmjs.org/@esbuild/android-x64/-/android-x64-0.25.12.tgz",
      "integrity": "sha512-5jbb+2hhDHx5phYR2By8GTWEzn6I9UqR11Kwf22iKbNpYrsmRB18aX/9ivc5cabcUiAT/wM+YIZ6SG9QO6a8kg==",
      "cpu": [
        "x64"
      ],
      "license": "MIT",
      "optional": true,
      "os": [
        "android"
      ],
      "engines": {
        "node": ">=18"
      }
    },
    "node_modules/@esbuild/darwin-arm64": {
      "version": "0.25.12",
      "resolved": "https://registry.npmjs.org/@esbuild/darwin-arm64/-/darwin-arm64-0.25.12.tgz",
      "integrity": "sha512-N3zl+lxHCifgIlcMUP5016ESkeQjLj/959RxxNYIthIg+CQHInujFuXeWbWMgnTo4cp5XVHqFPmpyu9J65C1Yg==",
      "cpu": [
        "arm64"
      ],
      "license": "MIT",
      "optional": true,
      "os": [
        "darwin"
      ],
      "engines": {
        "node": ">=18"
      }
    },
    "node_modules/@esbuild/darwin-x64": {
      "version": "0.25.12",
      "resolved": "https://registry.npmjs.org/@esbuild/darwin-x64/-/darwin-x64-0.25.12.tgz",
      "integrity": "sha512-HQ9ka4Kx21qHXwtlTUVbKJOAnmG1ipXhdWTmNXiPzPfWKpXqASVcWdnf2bnL73wgjNrFXAa3yYvBSd9pzfEIpA==",
      "cpu": [
        "x64"
      ],
      "license": "MIT",
      "optional": true,
      "os": [
        "darwin"
      ],
      "engines": {
        "node": ">=18"
      }
    },
    "node_modules/@esbuild/freebsd-arm64": {
      "version": "0.25.12",
      "resolved": "https://registry.npmjs.org/@esbuild/freebsd-arm64/-/freebsd-arm64-0.25.12.tgz",
      "integrity": "sha512-gA0Bx759+7Jve03K1S0vkOu5Lg/85dou3EseOGUes8flVOGxbhDDh/iZaoek11Y8mtyKPGF3vP8XhnkDEAmzeg==",
      "cpu": [
        "arm64"
      ],
      "license": "MIT",
      "optional": true,
      "os": [
        "freebsd"
      ],
      "engines": {
        "node": ">=18"
      }
    },
    "node_modules/@esbuild/freebsd-x64": {
      "version": "0.25.12",
      "resolved": "https://registry.npmjs.org/@esbuild/freebsd-x64/-/freebsd-x64-0.25.12.tgz",
      "integrity": "sha512-TGbO26Yw2xsHzxtbVFGEXBFH0FRAP7gtcPE7P5yP7wGy7cXK2oO7RyOhL5NLiqTlBh47XhmIUXuGciXEqYFfBQ==",
      "cpu": [
        "x64"
      ],
      "license": "MIT",
      "optional": true,
      "os": [
        "freebsd"
      ],
      "engines": {
        "node": ">=18"
      }
    },
    "node_modules/@esbuild/linux-arm": {
      "version": "0.25.12",
      "resolved": "https://registry.npmjs.org/@esbuild/linux-arm/-/linux-arm-0.25.12.tgz",
      "integrity": "sha512-lPDGyC1JPDou8kGcywY0YILzWlhhnRjdof3UlcoqYmS9El818LLfJJc3PXXgZHrHCAKs/Z2SeZtDJr5MrkxtOw==",
      "cpu": [
        "arm"
      ],
      "license": "MIT",
      "optional": true,
      "os": [
        "linux"
      ],
      "engines": {
        "node": ">=18"
      }
    },
    "node_modules/@esbuild/linux-arm64": {
      "version": "0.25.12",
      "resolved": "https://registry.npmjs.org/@esbuild/linux-arm64/-/linux-arm64-0.25.12.tgz",
      "integrity": "sha512-8bwX7a8FghIgrupcxb4aUmYDLp8pX06rGh5HqDT7bB+8Rdells6mHvrFHHW2JAOPZUbnjUpKTLg6ECyzvas2AQ==",
      "cpu": [
        "arm64"
      ],
      "license": "MIT",
      "optional": true,
      "os": [
        "linux"
      ],
      "engines": {
        "node": ">=18"
      }
    },
    "node_modules/@esbuild/linux-ia32": {
      "version": "0.25.12",
      "resolved": "https://registry.npmjs.org/@esbuild/linux-ia32/-/linux-ia32-0.25.12.tgz",
      "integrity": "sha512-0y9KrdVnbMM2/vG8KfU0byhUN+EFCny9+8g202gYqSSVMonbsCfLjUO+rCci7pM0WBEtz+oK/PIwHkzxkyharA==",
      "cpu": [
        "ia32"
      ],
      "license": "MIT",
      "optional": true,
      "os": [
        "linux"
      ],
      "engines": {
        "node": ">=18"
      }
    },
    "node_modules/@esbuild/linux-loong64": {
      "version": "0.25.12",
      "resolved": "https://registry.npmjs.org/@esbuild/linux-loong64/-/linux-loong64-0.25.12.tgz",
      "integrity": "sha512-h///Lr5a9rib/v1GGqXVGzjL4TMvVTv+s1DPoxQdz7l/AYv6LDSxdIwzxkrPW438oUXiDtwM10o9PmwS/6Z0Ng==",
      "cpu": [
        "loong64"
      ],
      "license": "MIT",
      "optional": true,
      "os": [
        "linux"
      ],
      "engines": {
        "node": ">=18"
      }
    },
    "node_modules/@esbuild/linux-mips64el": {
      "version": "0.25.12",
      "resolved": "https://registry.npmjs.org/@esbuild/linux-mips64el/-/linux-mips64el-0.25.12.tgz",
      "integrity": "sha512-iyRrM1Pzy9GFMDLsXn1iHUm18nhKnNMWscjmp4+hpafcZjrr2WbT//d20xaGljXDBYHqRcl8HnxbX6uaA/eGVw==",
      "cpu": [
        "mips64el"
      ],
      "license": "MIT",
      "optional": true,
      "os": [
        "linux"
      ],
      "engines": {
        "node": ">=18"
      }
    },
    "node_modules/@esbuild/linux-ppc64": {
      "version": "0.25.12",
      "resolved": "https://registry.npmjs.org/@esbuild/linux-ppc64/-/linux-ppc64-0.25.12.tgz",
      "integrity": "sha512-9meM/lRXxMi5PSUqEXRCtVjEZBGwB7P/D4yT8UG/mwIdze2aV4Vo6U5gD3+RsoHXKkHCfSxZKzmDssVlRj1QQA==",
      "cpu": [
        "ppc64"
      ],
      "license": "MIT",
      "optional": true,
      "os": [
        "linux"
      ],
      "engines": {
        "node": ">=18"
      }
    },
    "node_modules/@esbuild/linux-riscv64": {
      "version": "0.25.12",
      "resolved": "https://registry.npmjs.org/@esbuild/linux-riscv64/-/linux-riscv64-0.25.12.tgz",
      "integrity": "sha512-Zr7KR4hgKUpWAwb1f3o5ygT04MzqVrGEGXGLnj15YQDJErYu/BGg+wmFlIDOdJp0PmB0lLvxFIOXZgFRrdjR0w==",
      "cpu": [
        "riscv64"
      ],
      "license": "MIT",
      "optional": true,
      "os": [
        "linux"
      ],
      "engines": {
        "node": ">=18"
      }
    },
    "node_modules/@esbuild/linux-s390x": {
      "version": "0.25.12",
      "resolved": "https://registry.npmjs.org/@esbuild/linux-s390x/-/linux-s390x-0.25.12.tgz",
      "integrity": "sha512-MsKncOcgTNvdtiISc/jZs/Zf8d0cl/t3gYWX8J9ubBnVOwlk65UIEEvgBORTiljloIWnBzLs4qhzPkJcitIzIg==",
      "cpu": [
        "s390x"
      ],
      "license": "MIT",
      "optional": true,
      "os": [
        "linux"
      ],
      "engines": {
        "node": ">=18"
      }
    },
    "node_modules/@esbuild/linux-x64": {
      "version": "0.25.12",
      "resolved": "https://registry.npmjs.org/@esbuild/linux-x64/-/linux-x64-0.25.12.tgz",
      "integrity": "sha512-uqZMTLr/zR/ed4jIGnwSLkaHmPjOjJvnm6TVVitAa08SLS9Z0VM8wIRx7gWbJB5/J54YuIMInDquWyYvQLZkgw==",
      "cpu": [
        "x64"
      ],
      "license": "MIT",
      "optional": true,
      "os": [
        "linux"
      ],
      "engines": {
        "node": ">=18"
      }
    },
    "node_modules/@esbuild/netbsd-arm64": {
      "version": "0.25.12",
      "resolved": "https://registry.npmjs.org/@esbuild/netbsd-arm64/-/netbsd-arm64-0.25.12.tgz",
      "integrity": "sha512-xXwcTq4GhRM7J9A8Gv5boanHhRa/Q9KLVmcyXHCTaM4wKfIpWkdXiMog/KsnxzJ0A1+nD+zoecuzqPmCRyBGjg==",
      "cpu": [
        "arm64"
      ],
      "license": "MIT",
      "optional": true,
      "os": [
        "netbsd"
      ],
      "engines": {
        "node": ">=18"
      }
    },
    "node_modules/@esbuild/netbsd-x64": {
      "version": "0.25.12",
      "resolved": "https://registry.npmjs.org/@esbuild/netbsd-x64/-/netbsd-x64-0.25.12.tgz",
      "integrity": "sha512-Ld5pTlzPy3YwGec4OuHh1aCVCRvOXdH8DgRjfDy/oumVovmuSzWfnSJg+VtakB9Cm0gxNO9BzWkj6mtO1FMXkQ==",
      "cpu": [
        "x64"
      ],
      "license": "MIT",
      "optional": true,
      "os": [
        "netbsd"
      ],
      "engines": {
        "node": ">=18"
      }
    },
    "node_modules/@esbuild/openbsd-arm64": {
      "version": "0.25.12",
      "resolved": "https://registry.npmjs.org/@esbuild/openbsd-arm64/-/openbsd-arm64-0.25.12.tgz",
      "integrity": "sha512-fF96T6KsBo/pkQI950FARU9apGNTSlZGsv1jZBAlcLL1MLjLNIWPBkj5NlSz8aAzYKg+eNqknrUJ24QBybeR5A==",
      "cpu": [
        "arm64"
      ],
      "license": "MIT",
      "optional": true,
      "os": [
        "openbsd"
      ],
      "engines": {
        "node": ">=18"
      }
    },
    "node_modules/@esbuild/openbsd-x64": {
      "version": "0.25.12",
      "resolved": "https://registry.npmjs.org/@esbuild/openbsd-x64/-/openbsd-x64-0.25.12.tgz",
      "integrity": "sha512-MZyXUkZHjQxUvzK7rN8DJ3SRmrVrke8ZyRusHlP+kuwqTcfWLyqMOE3sScPPyeIXN/mDJIfGXvcMqCgYKekoQw==",
      "cpu": [
        "x64"
      ],
      "license": "MIT",
      "optional": true,
      "os": [
        "openbsd"
      ],
      "engines": {
        "node": ">=18"
      }
    },
    "node_modules/@esbuild/openharmony-arm64": {
      "version": "0.25.12",
      "resolved": "https://registry.npmjs.org/@esbuild/openharmony-arm64/-/openharmony-arm64-0.25.12.tgz",
      "integrity": "sha512-rm0YWsqUSRrjncSXGA7Zv78Nbnw4XL6/dzr20cyrQf7ZmRcsovpcRBdhD43Nuk3y7XIoW2OxMVvwuRvk9XdASg==",
      "cpu": [
        "arm64"
      ],
      "license": "MIT",
      "optional": true,
      "os": [
        "openharmony"
      ],
      "engines": {
        "node": ">=18"
      }
    },
    "node_modules/@esbuild/sunos-x64": {
      "version": "0.25.12",
      "resolved": "https://registry.npmjs.org/@esbuild/sunos-x64/-/sunos-x64-0.25.12.tgz",
      "integrity": "sha512-3wGSCDyuTHQUzt0nV7bocDy72r2lI33QL3gkDNGkod22EsYl04sMf0qLb8luNKTOmgF/eDEDP5BFNwoBKH441w==",
      "cpu": [
        "x64"
      ],
      "license": "MIT",
      "optional": true,
      "os": [
        "sunos"
      ],
      "engines": {
        "node": ">=18"
      }
    },
    "node_modules/@esbuild/win32-arm64": {
      "version": "0.25.12",
      "resolved": "https://registry.npmjs.org/@esbuild/win32-arm64/-/win32-arm64-0.25.12.tgz",
      "integrity": "sha512-rMmLrur64A7+DKlnSuwqUdRKyd3UE7oPJZmnljqEptesKM8wx9J8gx5u0+9Pq0fQQW8vqeKebwNXdfOyP+8Bsg==",
      "cpu": [
        "arm64"
      ],
      "license": "MIT",
      "optional": true,
      "os": [
        "win32"
      ],
      "engines": {
        "node": ">=18"
      }
    },
    "node_modules/@esbuild/win32-ia32": {
      "version": "0.25.12",
      "resolved": "https://registry.npmjs.org/@esbuild/win32-ia32/-/win32-ia32-0.25.12.tgz",
      "integrity": "sha512-HkqnmmBoCbCwxUKKNPBixiWDGCpQGVsrQfJoVGYLPT41XWF8lHuE5N6WhVia2n4o5QK5M4tYr21827fNhi4byQ==",
      "cpu": [
        "ia32"
      ],
      "license": "MIT",
      "optional": true,
      "os": [
        "win32"
      ],
      "engines": {
        "node": ">=18"
      }
    },
    "node_modules/@esbuild/win32-x64": {
      "version": "0.25.12",
      "resolved": "https://registry.npmjs.org/@esbuild/win32-x64/-/win32-x64-0.25.12.tgz",
      "integrity": "sha512-alJC0uCZpTFrSL0CCDjcgleBXPnCrEAhTBILpeAp7M/OFgoqtAetfBzX0xM00MUsVVPpVjlPuMbREqnZCXaTnA==",
      "cpu": [
        "x64"
      ],
      "license": "MIT",
      "optional": true,
      "os": [
        "win32"
      ],
      "engines": {
        "node": ">=18"
      }
    },
    "node_modules/@jridgewell/gen-mapping": {
      "version": "0.3.13",
      "resolved": "https://registry.npmjs.org/@jridgewell/gen-mapping/-/gen-mapping-0.3.13.tgz",
      "integrity": "sha512-2kkt/7niJ6MgEPxF0bYdQ6etZaA+fQvDcLKckhy1yIQOzaoKjBBjSj63/aLVjYE3qhRt5dvM+uUyfCg6UKCBbA==",
      "license": "MIT",
      "dependencies": {
        "@jridgewell/sourcemap-codec": "^1.5.0",
        "@jridgewell/trace-mapping": "^0.3.24"
      }
    },
    "node_modules/@jridgewell/remapping": {
      "version": "2.3.5",
      "resolved": "https://registry.npmjs.org/@jridgewell/remapping/-/remapping-2.3.5.tgz",
      "integrity": "sha512-LI9u/+laYG4Ds1TDKSJW2YPrIlcVYOwi2fUC6xB43lueCjgxV4lffOCZCtYFiH6TNOX+tQKXx97T4IKHbhyHEQ==",
      "license": "MIT",
      "dependencies": {
        "@jridgewell/gen-mapping": "^0.3.5",
        "@jridgewell/trace-mapping": "^0.3.24"
      }
    },
    "node_modules/@jridgewell/resolve-uri": {
      "version": "3.1.2",
      "resolved": "https://registry.npmjs.org/@jridgewell/resolve-uri/-/resolve-uri-3.1.2.tgz",
      "integrity": "sha512-bRISgCIjP20/tbWSPWMEi54QVPRZExkuD9lJL+UIxUKtwVJA8wW1Trb1jMs1RFXo1CBTNZ/5hpC9QvmKWdopKw==",
      "license": "MIT",
      "engines": {
        "node": ">=6.0.0"
      }
    },
    "node_modules/@jridgewell/sourcemap-codec": {
      "version": "1.5.5",
      "resolved": "https://registry.npmjs.org/@jridgewell/sourcemap-codec/-/sourcemap-codec-1.5.5.tgz",
      "integrity": "sha512-cYQ9310grqxueWbl+WuIUIaiUaDcj7WOq5fVhEljNVgRfOUhY9fy2zTvfoqWsnebh8Sl70VScFbICvJnLKB0Og==",
      "license": "MIT"
    },
    "node_modules/@jridgewell/trace-mapping": {
      "version": "0.3.31",
      "resolved": "https://registry.npmjs.org/@jridgewell/trace-mapping/-/trace-mapping-0.3.31.tgz",
      "integrity": "sha512-zzNR+SdQSDJzc8joaeP8QQoCQr8NuYx2dIIytl1QeBEZHJ9uW6hebsrYgbz8hJwUQao3TWCMtmfV8Nu1twOLAw==",
      "license": "MIT",
      "dependencies": {
        "@jridgewell/resolve-uri": "^3.1.0",
        "@jridgewell/sourcemap-codec": "^1.4.14"
      }
    },
    "node_modules/@rolldown/pluginutils": {
      "version": "1.0.0-beta.43",
      "resolved": "https://registry.npmjs.org/@rolldown/pluginutils/-/pluginutils-1.0.0-beta.43.tgz",
      "integrity": "sha512-5Uxg7fQUCmfhax7FJke2+8B6cqgeUJUD9o2uXIKXhD+mG0mL6NObmVoi9wXEU1tY89mZKgAYA6fTbftx3q2ZPQ==",
      "license": "MIT"
    },
    "node_modules/@rollup/rollup-android-arm-eabi": {
      "version": "4.52.5",
      "resolved": "https://registry.npmjs.org/@rollup/rollup-android-arm-eabi/-/rollup-android-arm-eabi-4.52.5.tgz",
      "integrity": "sha512-8c1vW4ocv3UOMp9K+gToY5zL2XiiVw3k7f1ksf4yO1FlDFQ1C2u72iACFnSOceJFsWskc2WZNqeRhFRPzv+wtQ==",
      "cpu": [
        "arm"
      ],
      "license": "MIT",
      "optional": true,
      "os": [
        "android"
      ]
    },
    "node_modules/@rollup/rollup-android-arm64": {
      "version": "4.52.5",
      "resolved": "https://registry.npmjs.org/@rollup/rollup-android-arm64/-/rollup-android-arm64-4.52.5.tgz",
      "integrity": "sha512-mQGfsIEFcu21mvqkEKKu2dYmtuSZOBMmAl5CFlPGLY94Vlcm+zWApK7F/eocsNzp8tKmbeBP8yXyAbx0XHsFNA==",
      "cpu": [
        "arm64"
      ],
      "license": "MIT",
      "optional": true,
      "os": [
        "android"
      ]
    },
    "node_modules/@rollup/rollup-darwin-arm64": {
      "version": "4.52.5",
      "resolved": "https://registry.npmjs.org/@rollup/rollup-darwin-arm64/-/rollup-darwin-arm64-4.52.5.tgz",
      "integrity": "sha512-takF3CR71mCAGA+v794QUZ0b6ZSrgJkArC+gUiG6LB6TQty9T0Mqh3m2ImRBOxS2IeYBo4lKWIieSvnEk2OQWA==",
      "cpu": [
        "arm64"
      ],
      "license": "MIT",
      "optional": true,
      "os": [
        "darwin"
      ]
    },
    "node_modules/@rollup/rollup-darwin-x64": {
      "version": "4.52.5",
      "resolved": "https://registry.npmjs.org/@rollup/rollup-darwin-x64/-/rollup-darwin-x64-4.52.5.tgz",
      "integrity": "sha512-W901Pla8Ya95WpxDn//VF9K9u2JbocwV/v75TE0YIHNTbhqUTv9w4VuQ9MaWlNOkkEfFwkdNhXgcLqPSmHy0fA==",
      "cpu": [
        "x64"
      ],
      "license": "MIT",
      "optional": true,
      "os": [
        "darwin"
      ]
    },
    "node_modules/@rollup/rollup-freebsd-arm64": {
      "version": "4.52.5",
      "resolved": "https://registry.npmjs.org/@rollup/rollup-freebsd-arm64/-/rollup-freebsd-arm64-4.52.5.tgz",
      "integrity": "sha512-QofO7i7JycsYOWxe0GFqhLmF6l1TqBswJMvICnRUjqCx8b47MTo46W8AoeQwiokAx3zVryVnxtBMcGcnX12LvA==",
      "cpu": [
        "arm64"
      ],
      "license": "MIT",
      "optional": true,
      "os": [
        "freebsd"
      ]
    },
    "node_modules/@rollup/rollup-freebsd-x64": {
      "version": "4.52.5",
      "resolved": "https://registry.npmjs.org/@rollup/rollup-freebsd-x64/-/rollup-freebsd-x64-4.52.5.tgz",
      "integrity": "sha512-jr21b/99ew8ujZubPo9skbrItHEIE50WdV86cdSoRkKtmWa+DDr6fu2c/xyRT0F/WazZpam6kk7IHBerSL7LDQ==",
      "cpu": [
        "x64"
      ],
      "license": "MIT",
      "optional": true,
      "os": [
        "freebsd"
      ]
    },
    "node_modules/@rollup/rollup-linux-arm-gnueabihf": {
      "version": "4.52.5",
      "resolved": "https://registry.npmjs.org/@rollup/rollup-linux-arm-gnueabihf/-/rollup-linux-arm-gnueabihf-4.52.5.tgz",
      "integrity": "sha512-PsNAbcyv9CcecAUagQefwX8fQn9LQ4nZkpDboBOttmyffnInRy8R8dSg6hxxl2Re5QhHBf6FYIDhIj5v982ATQ==",
      "cpu": [
        "arm"
      ],
      "license": "MIT",
      "optional": true,
      "os": [
        "linux"
      ]
    },
    "node_modules/@rollup/rollup-linux-arm-musleabihf": {
      "version": "4.52.5",
      "resolved": "https://registry.npmjs.org/@rollup/rollup-linux-arm-musleabihf/-/rollup-linux-arm-musleabihf-4.52.5.tgz",
      "integrity": "sha512-Fw4tysRutyQc/wwkmcyoqFtJhh0u31K+Q6jYjeicsGJJ7bbEq8LwPWV/w0cnzOqR2m694/Af6hpFayLJZkG2VQ==",
      "cpu": [
        "arm"
      ],
      "license": "MIT",
      "optional": true,
      "os": [
        "linux"
      ]
    },
    "node_modules/@rollup/rollup-linux-arm64-gnu": {
      "version": "4.52.5",
      "resolved": "https://registry.npmjs.org/@rollup/rollup-linux-arm64-gnu/-/rollup-linux-arm64-gnu-4.52.5.tgz",
      "integrity": "sha512-a+3wVnAYdQClOTlyapKmyI6BLPAFYs0JM8HRpgYZQO02rMR09ZcV9LbQB+NL6sljzG38869YqThrRnfPMCDtZg==",
      "cpu": [
        "arm64"
      ],
      "license": "MIT",
      "optional": true,
      "os": [
        "linux"
      ]
    },
    "node_modules/@rollup/rollup-linux-arm64-musl": {
      "version": "4.52.5",
      "resolved": "https://registry.npmjs.org/@rollup/rollup-linux-arm64-musl/-/rollup-linux-arm64-musl-4.52.5.tgz",
      "integrity": "sha512-AvttBOMwO9Pcuuf7m9PkC1PUIKsfaAJ4AYhy944qeTJgQOqJYJ9oVl2nYgY7Rk0mkbsuOpCAYSs6wLYB2Xiw0Q==",
      "cpu": [
        "arm64"
      ],
      "license": "MIT",
      "optional": true,
      "os": [
        "linux"
      ]
    },
    "node_modules/@rollup/rollup-linux-loong64-gnu": {
      "version": "4.52.5",
      "resolved": "https://registry.npmjs.org/@rollup/rollup-linux-loong64-gnu/-/rollup-linux-loong64-gnu-4.52.5.tgz",
      "integrity": "sha512-DkDk8pmXQV2wVrF6oq5tONK6UHLz/XcEVow4JTTerdeV1uqPeHxwcg7aFsfnSm9L+OO8WJsWotKM2JJPMWrQtA==",
      "cpu": [
        "loong64"
      ],
      "license": "MIT",
      "optional": true,
      "os": [
        "linux"
      ]
    },
    "node_modules/@rollup/rollup-linux-ppc64-gnu": {
      "version": "4.52.5",
      "resolved": "https://registry.npmjs.org/@rollup/rollup-linux-ppc64-gnu/-/rollup-linux-ppc64-gnu-4.52.5.tgz",
      "integrity": "sha512-W/b9ZN/U9+hPQVvlGwjzi+Wy4xdoH2I8EjaCkMvzpI7wJUs8sWJ03Rq96jRnHkSrcHTpQe8h5Tg3ZzUPGauvAw==",
      "cpu": [
        "ppc64"
      ],
      "license": "MIT",
      "optional": true,
      "os": [
        "linux"
      ]
    },
    "node_modules/@rollup/rollup-linux-riscv64-gnu": {
      "version": "4.52.5",
      "resolved": "https://registry.npmjs.org/@rollup/rollup-linux-riscv64-gnu/-/rollup-linux-riscv64-gnu-4.52.5.tgz",
      "integrity": "sha512-sjQLr9BW7R/ZiXnQiWPkErNfLMkkWIoCz7YMn27HldKsADEKa5WYdobaa1hmN6slu9oWQbB6/jFpJ+P2IkVrmw==",
      "cpu": [
        "riscv64"
      ],
      "license": "MIT",
      "optional": true,
      "os": [
        "linux"
      ]
    },
    "node_modules/@rollup/rollup-linux-riscv64-musl": {
      "version": "4.52.5",
      "resolved": "https://registry.npmjs.org/@rollup/rollup-linux-riscv64-musl/-/rollup-linux-riscv64-musl-4.52.5.tgz",
      "integrity": "sha512-hq3jU/kGyjXWTvAh2awn8oHroCbrPm8JqM7RUpKjalIRWWXE01CQOf/tUNWNHjmbMHg/hmNCwc/Pz3k1T/j/Lg==",
      "cpu": [
        "riscv64"
      ],
      "license": "MIT",
      "optional": true,
      "os": [
        "linux"
      ]
    },
    "node_modules/@rollup/rollup-linux-s390x-gnu": {
      "version": "4.52.5",
      "resolved": "https://registry.npmjs.org/@rollup/rollup-linux-s390x-gnu/-/rollup-linux-s390x-gnu-4.52.5.tgz",
      "integrity": "sha512-gn8kHOrku8D4NGHMK1Y7NA7INQTRdVOntt1OCYypZPRt6skGbddska44K8iocdpxHTMMNui5oH4elPH4QOLrFQ==",
      "cpu": [
        "s390x"
      ],
      "license": "MIT",
      "optional": true,
      "os": [
        "linux"
      ]
    },
    "node_modules/@rollup/rollup-linux-x64-gnu": {
      "version": "4.52.5",
      "resolved": "https://registry.npmjs.org/@rollup/rollup-linux-x64-gnu/-/rollup-linux-x64-gnu-4.52.5.tgz",
      "integrity": "sha512-hXGLYpdhiNElzN770+H2nlx+jRog8TyynpTVzdlc6bndktjKWyZyiCsuDAlpd+j+W+WNqfcyAWz9HxxIGfZm1Q==",
      "cpu": [
        "x64"
      ],
      "license": "MIT",
      "optional": true,
      "os": [
        "linux"
      ]
    },
    "node_modules/@rollup/rollup-linux-x64-musl": {
      "version": "4.52.5",
      "resolved": "https://registry.npmjs.org/@rollup/rollup-linux-x64-musl/-/rollup-linux-x64-musl-4.52.5.tgz",
      "integrity": "sha512-arCGIcuNKjBoKAXD+y7XomR9gY6Mw7HnFBv5Rw7wQRvwYLR7gBAgV7Mb2QTyjXfTveBNFAtPt46/36vV9STLNg==",
      "cpu": [
        "x64"
      ],
      "license": "MIT",
      "optional": true,
      "os": [
        "linux"
      ]
    },
    "node_modules/@rollup/rollup-openharmony-arm64": {
      "version": "4.52.5",
      "resolved": "https://registry.npmjs.org/@rollup/rollup-openharmony-arm64/-/rollup-openharmony-arm64-4.52.5.tgz",
      "integrity": "sha512-QoFqB6+/9Rly/RiPjaomPLmR/13cgkIGfA40LHly9zcH1S0bN2HVFYk3a1eAyHQyjs3ZJYlXvIGtcCs5tko9Cw==",
      "cpu": [
        "arm64"
      ],
      "license": "MIT",
      "optional": true,
      "os": [
        "openharmony"
      ]
    },
    "node_modules/@rollup/rollup-win32-arm64-msvc": {
      "version": "4.52.5",
      "resolved": "https://registry.npmjs.org/@rollup/rollup-win32-arm64-msvc/-/rollup-win32-arm64-msvc-4.52.5.tgz",
      "integrity": "sha512-w0cDWVR6MlTstla1cIfOGyl8+qb93FlAVutcor14Gf5Md5ap5ySfQ7R9S/NjNaMLSFdUnKGEasmVnu3lCMqB7w==",
      "cpu": [
        "arm64"
      ],
      "license": "MIT",
      "optional": true,
      "os": [
        "win32"
      ]
    },
    "node_modules/@rollup/rollup-win32-ia32-msvc": {
      "version": "4.52.5",
      "resolved": "https://registry.npmjs.org/@rollup/rollup-win32-ia32-msvc/-/rollup-win32-ia32-msvc-4.52.5.tgz",
      "integrity": "sha512-Aufdpzp7DpOTULJCuvzqcItSGDH73pF3ko/f+ckJhxQyHtp67rHw3HMNxoIdDMUITJESNE6a8uh4Lo4SLouOUg==",
      "cpu": [
        "ia32"
      ],
      "license": "MIT",
      "optional": true,
      "os": [
        "win32"
      ]
    },
    "node_modules/@rollup/rollup-win32-x64-gnu": {
      "version": "4.52.5",
      "resolved": "https://registry.npmjs.org/@rollup/rollup-win32-x64-gnu/-/rollup-win32-x64-gnu-4.52.5.tgz",
      "integrity": "sha512-UGBUGPFp1vkj6p8wCRraqNhqwX/4kNQPS57BCFc8wYh0g94iVIW33wJtQAx3G7vrjjNtRaxiMUylM0ktp/TRSQ==",
      "cpu": [
        "x64"
      ],
      "license": "MIT",
      "optional": true,
      "os": [
        "win32"
      ]
    },
    "node_modules/@rollup/rollup-win32-x64-msvc": {
      "version": "4.52.5",
      "resolved": "https://registry.npmjs.org/@rollup/rollup-win32-x64-msvc/-/rollup-win32-x64-msvc-4.52.5.tgz",
      "integrity": "sha512-TAcgQh2sSkykPRWLrdyy2AiceMckNf5loITqXxFI5VuQjS5tSuw3WlwdN8qv8vzjLAUTvYaH/mVjSFpbkFbpTg==",
      "cpu": [
        "x64"
      ],
      "license": "MIT",
      "optional": true,
      "os": [
        "win32"
      ]
    },
    "node_modules/@types/babel__core": {
      "version": "7.20.5",
      "resolved": "https://registry.npmjs.org/@types/babel__core/-/babel__core-7.20.5.tgz",
      "integrity": "sha512-qoQprZvz5wQFJwMDqeseRXWv3rqMvhgpbXFfVyWhbx9X47POIA6i/+dXefEmZKoAgOaTdaIgNSMqMIU61yRyzA==",
      "license": "MIT",
      "dependencies": {
        "@babel/parser": "^7.20.7",
        "@babel/types": "^7.20.7",
        "@types/babel__generator": "*",
        "@types/babel__template": "*",
        "@types/babel__traverse": "*"
      }
    },
    "node_modules/@types/babel__generator": {
      "version": "7.27.0",
      "resolved": "https://registry.npmjs.org/@types/babel__generator/-/babel__generator-7.27.0.tgz",
      "integrity": "sha512-ufFd2Xi92OAVPYsy+P4n7/U7e68fex0+Ee8gSG9KX7eo084CWiQ4sdxktvdl0bOPupXtVJPY19zk6EwWqUQ8lg==",
      "license": "MIT",
      "dependencies": {
        "@babel/types": "^7.0.0"
      }
    },
    "node_modules/@types/babel__template": {
      "version": "7.4.4",
      "resolved": "https://registry.npmjs.org/@types/babel__template/-/babel__template-7.4.4.tgz",
      "integrity": "sha512-h/NUaSyG5EyxBIp8YRxo4RMe2/qQgvyowRwVMzhYhBCONbW8PUsg4lkFMrhgZhUe5z3L3MiLDuvyJ/CaPa2A8A==",
      "license": "MIT",
      "dependencies": {
        "@babel/parser": "^7.1.0",
        "@babel/types": "^7.0.0"
      }
    },
    "node_modules/@types/babel__traverse": {
      "version": "7.28.0",
      "resolved": "https://registry.npmjs.org/@types/babel__traverse/-/babel__traverse-7.28.0.tgz",
      "integrity": "sha512-8PvcXf70gTDZBgt9ptxJ8elBeBjcLOAcOtoO/mPJjtji1+CdGbHgm77om1GrsPxsiE+uXIpNSK64UYaIwQXd4Q==",
      "license": "MIT",
      "dependencies": {
        "@babel/types": "^7.28.2"
      }
    },
    "node_modules/@types/estree": {
      "version": "1.0.8",
      "resolved": "https://registry.npmjs.org/@types/estree/-/estree-1.0.8.tgz",
      "integrity": "sha512-dWHzHa2WqEXI/O1E9OjrocMTKJl2mSrEolh1Iomrv6U+JuNwaHXsXx9bLu5gG7BUWFIN0skIQJQ/L1rIex4X6w==",
      "license": "MIT"
    },
    "node_modules/@vitejs/plugin-react": {
      "version": "5.1.0",
      "resolved": "https://registry.npmjs.org/@vitejs/plugin-react/-/plugin-react-5.1.0.tgz",
      "integrity": "sha512-4LuWrg7EKWgQaMJfnN+wcmbAW+VSsCmqGohftWjuct47bv8uE4n/nPpq4XjJPsxgq00GGG5J8dvBczp8uxScew==",
      "license": "MIT",
      "dependencies": {
        "@babel/core": "^7.28.4",
        "@babel/plugin-transform-react-jsx-self": "^7.27.1",
        "@babel/plugin-transform-react-jsx-source": "^7.27.1",
        "@rolldown/pluginutils": "1.0.0-beta.43",
        "@types/babel__core": "^7.20.5",
        "react-refresh": "^0.18.0"
      },
      "engines": {
        "node": "^20.19.0 || >=22.12.0"
      },
      "peerDependencies": {
        "vite": "^4.2.0 || ^5.0.0 || ^6.0.0 || ^7.0.0"
      }
    },
    "node_modules/accepts": {
      "version": "2.0.0",
      "resolved": "https://registry.npmjs.org/accepts/-/accepts-2.0.0.tgz",
      "integrity": "sha512-5cvg6CtKwfgdmVqY1WIiXKc3Q1bkRqGLi+2W/6ao+6Y7gu/RCwRuAhGEzh5B4KlszSuTLgZYuqFqo5bImjNKng==",
      "license": "MIT",
      "dependencies": {
        "mime-types": "^3.0.0",
        "negotiator": "^1.0.0"
      },
      "engines": {
        "node": ">= 0.6"
      }
    },
    "node_modules/ansi-regex": {
      "version": "5.0.1",
      "resolved": "https://registry.npmjs.org/ansi-regex/-/ansi-regex-5.0.1.tgz",
      "integrity": "sha512-quJQXlTSUGL2LH9SUXo8VwsY4soanhgo6LNSm84E1LBcE8s3O0wpdiRzyR9z/ZZJMlMWv37qOOb9pdJlMUEKFQ==",
      "license": "MIT",
      "engines": {
        "node": ">=8"
      }
    },
    "node_modules/ansi-styles": {
      "version": "4.3.0",
      "resolved": "https://registry.npmjs.org/ansi-styles/-/ansi-styles-4.3.0.tgz",
      "integrity": "sha512-zbB9rCJAT1rbjiVDb2hqKFHNYLxgtk8NURxZ3IZwD3F6NtxbXZQCnnSi1Lkx+IDohdPlFp222wVALIheZJQSEg==",
      "license": "MIT",
      "dependencies": {
        "color-convert": "^2.0.1"
      },
      "engines": {
        "node": ">=8"
      },
      "funding": {
        "url": "https://github.com/chalk/ansi-styles?sponsor=1"
      }
    },
    "node_modules/baseline-browser-mapping": {
      "version": "2.8.23",
      "resolved": "https://registry.npmjs.org/baseline-browser-mapping/-/baseline-browser-mapping-2.8.23.tgz",
      "integrity": "sha512-616V5YX4bepJFzNyOfce5Fa8fDJMfoxzOIzDCZwaGL8MKVpFrXqfNUoIpRn9YMI5pXf/VKgzjB4htFMsFKKdiQ==",
      "license": "Apache-2.0",
      "bin": {
        "baseline-browser-mapping": "dist/cli.js"
      }
    },
    "node_modules/body-parser": {
      "version": "2.2.0",
      "resolved": "https://registry.npmjs.org/body-parser/-/body-parser-2.2.0.tgz",
      "integrity": "sha512-02qvAaxv8tp7fBa/mw1ga98OGm+eCbqzJOKoRt70sLmfEEi+jyBYVTDGfCL/k06/4EMk/z01gCe7HoCH/f2LTg==",
      "license": "MIT",
      "dependencies": {
        "bytes": "^3.1.2",
        "content-type": "^1.0.5",
        "debug": "^4.4.0",
        "http-errors": "^2.0.0",
        "iconv-lite": "^0.6.3",
        "on-finished": "^2.4.1",
        "qs": "^6.14.0",
        "raw-body": "^3.0.0",
        "type-is": "^2.0.0"
      },
      "engines": {
        "node": ">=18"
      }
    },
    "node_modules/browserslist": {
      "version": "4.27.0",
      "resolved": "https://registry.npmjs.org/browserslist/-/browserslist-4.27.0.tgz",
      "integrity": "sha512-AXVQwdhot1eqLihwasPElhX2tAZiBjWdJ9i/Zcj2S6QYIjkx62OKSfnobkriB81C3l4w0rVy3Nt4jaTBltYEpw==",
      "funding": [
        {
          "type": "opencollective",
          "url": "https://opencollective.com/browserslist"
        },
        {
          "type": "tidelift",
          "url": "https://tidelift.com/funding/github/npm/browserslist"
        },
        {
          "type": "github",
          "url": "https://github.com/sponsors/ai"
        }
      ],
      "license": "MIT",
      "dependencies": {
        "baseline-browser-mapping": "^2.8.19",
        "caniuse-lite": "^1.0.30001751",
        "electron-to-chromium": "^1.5.238",
        "node-releases": "^2.0.26",
        "update-browserslist-db": "^1.1.4"
      },
      "bin": {
        "browserslist": "cli.js"
      },
      "engines": {
        "node": "^6 || ^7 || ^8 || ^9 || ^10 || ^11 || ^12 || >=13.7"
      }
    },
    "node_modules/bytes": {
      "version": "3.1.2",
      "resolved": "https://registry.npmjs.org/bytes/-/bytes-3.1.2.tgz",
      "integrity": "sha512-/Nf7TyzTx6S3yRJObOAV7956r8cr2+Oj8AC5dt8wSP3BQAoeX58NoHyCU8P8zGkNXStjTSi6fzO6F0pBdcYbEg==",
      "license": "MIT",
      "engines": {
        "node": ">= 0.8"
      }
    },
    "node_modules/call-bind-apply-helpers": {
      "version": "1.0.2",
      "resolved": "https://registry.npmjs.org/call-bind-apply-helpers/-/call-bind-apply-helpers-1.0.2.tgz",
      "integrity": "sha512-Sp1ablJ0ivDkSzjcaJdxEunN5/XvksFJ2sMBFfq6x0ryhQV/2b/KwFe21cMpmHtPOSij8K99/wSfoEuTObmuMQ==",
      "license": "MIT",
      "dependencies": {
        "es-errors": "^1.3.0",
        "function-bind": "^1.1.2"
      },
      "engines": {
        "node": ">= 0.4"
      }
    },
    "node_modules/call-bound": {
      "version": "1.0.4",
      "resolved": "https://registry.npmjs.org/call-bound/-/call-bound-1.0.4.tgz",
      "integrity": "sha512-+ys997U96po4Kx/ABpBCqhA9EuxJaQWDQg7295H4hBphv3IZg0boBKuwYpt4YXp6MZ5AmZQnU/tyMTlRpaSejg==",
      "license": "MIT",
      "dependencies": {
        "call-bind-apply-helpers": "^1.0.2",
        "get-intrinsic": "^1.3.0"
      },
      "engines": {
        "node": ">= 0.4"
      },
      "funding": {
        "url": "https://github.com/sponsors/ljharb"
      }
    },
    "node_modules/caniuse-lite": {
      "version": "1.0.30001753",
      "resolved": "https://registry.npmjs.org/caniuse-lite/-/caniuse-lite-1.0.30001753.tgz",
      "integrity": "sha512-Bj5H35MD/ebaOV4iDLqPEtiliTN29qkGtEHCwawWn4cYm+bPJM2NsaP30vtZcnERClMzp52J4+aw2UNbK4o+zw==",
      "funding": [
        {
          "type": "opencollective",
          "url": "https://opencollective.com/browserslist"
        },
        {
          "type": "tidelift",
          "url": "https://tidelift.com/funding/github/npm/caniuse-lite"
        },
        {
          "type": "github",
          "url": "https://github.com/sponsors/ai"
        }
      ],
      "license": "CC-BY-4.0"
    },
    "node_modules/chalk": {
      "version": "4.1.2",
      "resolved": "https://registry.npmjs.org/chalk/-/chalk-4.1.2.tgz",
      "integrity": "sha512-oKnbhFyRIXpUuez8iBMmyEa4nbj4IOQyuhc/wy9kY7/WVPcwIO9VA668Pu8RkO7+0G76SLROeyw9CpQ061i4mA==",
      "license": "MIT",
      "dependencies": {
        "ansi-styles": "^4.1.0",
        "supports-color": "^7.1.0"
      },
      "engines": {
        "node": ">=10"
      },
      "funding": {
        "url": "https://github.com/chalk/chalk?sponsor=1"
      }
    },
    "node_modules/chalk/node_modules/supports-color": {
      "version": "7.2.0",
      "resolved": "https://registry.npmjs.org/supports-color/-/supports-color-7.2.0.tgz",
      "integrity": "sha512-qpCAvRl9stuOHveKsn7HncJRvv501qIacKzQlO/+Lwxc9+0q2wLyv4Dfvt80/DPn2pqOBsJdDiogXGR9+OvwRw==",
      "license": "MIT",
      "dependencies": {
        "has-flag": "^4.0.0"
      },
      "engines": {
        "node": ">=8"
      }
    },
    "node_modules/cliui": {
      "version": "8.0.1",
      "resolved": "https://registry.npmjs.org/cliui/-/cliui-8.0.1.tgz",
      "integrity": "sha512-BSeNnyus75C4//NQ9gQt1/csTXyo/8Sb+afLAkzAptFuMsod9HFokGNudZpi/oQV73hnVK+sR+5PVRMd+Dr7YQ==",
      "license": "ISC",
      "dependencies": {
        "string-width": "^4.2.0",
        "strip-ansi": "^6.0.1",
        "wrap-ansi": "^7.0.0"
      },
      "engines": {
        "node": ">=12"
      }
    },
    "node_modules/color-convert": {
      "version": "2.0.1",
      "resolved": "https://registry.npmjs.org/color-convert/-/color-convert-2.0.1.tgz",
      "integrity": "sha512-RRECPsj7iu/xb5oKYcsFHSppFNnsj/52OVTRKb4zP5onXwVF3zVmmToNcOfGC+CRDpfK/U584fMg38ZHCaElKQ==",
      "license": "MIT",
      "dependencies": {
        "color-name": "~1.1.4"
      },
      "engines": {
        "node": ">=7.0.0"
      }
    },
    "node_modules/color-name": {
      "version": "1.1.4",
      "resolved": "https://registry.npmjs.org/color-name/-/color-name-1.1.4.tgz",
      "integrity": "sha512-dOy+3AuW3a2wNbZHIuMZpTcgjGuLU/uBL/ubcZF9OXbDo8ff4O8yVp5Bf0efS8uEoYo5q4Fx7dY9OgQGXgAsQA==",
      "license": "MIT"
    },
    "node_modules/concurrently": {
      "version": "9.2.1",
      "resolved": "https://registry.npmjs.org/concurrently/-/concurrently-9.2.1.tgz",
      "integrity": "sha512-fsfrO0MxV64Znoy8/l1vVIjjHa29SZyyqPgQBwhiDcaW8wJc2W3XWVOGx4M3oJBnv/zdUZIIp1gDeS98GzP8Ng==",
      "license": "MIT",
      "dependencies": {
        "chalk": "4.1.2",
        "rxjs": "7.8.2",
        "shell-quote": "1.8.3",
        "supports-color": "8.1.1",
        "tree-kill": "1.2.2",
        "yargs": "17.7.2"
      },
      "bin": {
        "conc": "dist/bin/concurrently.js",
        "concurrently": "dist/bin/concurrently.js"
      },
      "engines": {
        "node": ">=18"
      },
      "funding": {
        "url": "https://github.com/open-cli-tools/concurrently?sponsor=1"
      }
    },
    "node_modules/content-disposition": {
      "version": "1.0.0",
      "resolved": "https://registry.npmjs.org/content-disposition/-/content-disposition-1.0.0.tgz",
      "integrity": "sha512-Au9nRL8VNUut/XSzbQA38+M78dzP4D+eqg3gfJHMIHHYa3bg067xj1KxMUWj+VULbiZMowKngFFbKczUrNJ1mg==",
      "license": "MIT",
      "dependencies": {
        "safe-buffer": "5.2.1"
      },
      "engines": {
        "node": ">= 0.6"
      }
    },
    "node_modules/content-type": {
      "version": "1.0.5",
      "resolved": "https://registry.npmjs.org/content-type/-/content-type-1.0.5.tgz",
      "integrity": "sha512-nTjqfcBFEipKdXCv4YDQWCfmcLZKm81ldF0pAopTvyrFGVbcR6P/VAAd5G7N+0tTr8QqiU0tFadD6FK4NtJwOA==",
      "license": "MIT",
      "engines": {
        "node": ">= 0.6"
      }
    },
    "node_modules/convert-source-map": {
      "version": "2.0.0",
      "resolved": "https://registry.npmjs.org/convert-source-map/-/convert-source-map-2.0.0.tgz",
      "integrity": "sha512-Kvp459HrV2FEJ1CAsi1Ku+MY3kasH19TFykTz2xWmMeq6bk2NU3XXvfJ+Q61m0xktWwt+1HSYf3JZsTms3aRJg==",
      "license": "MIT"
    },
    "node_modules/cookie": {
      "version": "0.7.2",
      "resolved": "https://registry.npmjs.org/cookie/-/cookie-0.7.2.tgz",
      "integrity": "sha512-yki5XnKuf750l50uGTllt6kKILY4nQ1eNIQatoXEByZ5dWgnKqbnqmTrBE5B4N7lrMJKQ2ytWMiTO2o0v6Ew/w==",
      "license": "MIT",
      "engines": {
        "node": ">= 0.6"
      }
    },
    "node_modules/cookie-signature": {
      "version": "1.2.2",
      "resolved": "https://registry.npmjs.org/cookie-signature/-/cookie-signature-1.2.2.tgz",
      "integrity": "sha512-D76uU73ulSXrD1UXF4KE2TMxVVwhsnCgfAyTg9k8P6KGZjlXKrOLe4dJQKI3Bxi5wjesZoFXJWElNWBjPZMbhg==",
      "license": "MIT",
      "engines": {
        "node": ">=6.6.0"
      }
    },
    "node_modules/cors": {
      "version": "2.8.5",
      "resolved": "https://registry.npmjs.org/cors/-/cors-2.8.5.tgz",
      "integrity": "sha512-KIHbLJqu73RGr/hnbrO9uBeixNGuvSQjul/jdFvS/KFSIH1hWVd1ng7zOHx+YrEfInLG7q4n6GHQ9cDtxv/P6g==",
      "license": "MIT",
      "dependencies": {
        "object-assign": "^4",
        "vary": "^1"
      },
      "engines": {
        "node": ">= 0.10"
      }
    },
    "node_modules/debug": {
      "version": "4.4.3",
      "resolved": "https://registry.npmjs.org/debug/-/debug-4.4.3.tgz",
      "integrity": "sha512-RGwwWnwQvkVfavKVt22FGLw+xYSdzARwm0ru6DhTVA3umU5hZc28V3kO4stgYryrTlLpuvgI9GiijltAjNbcqA==",
      "license": "MIT",
      "dependencies": {
        "ms": "^2.1.3"
      },
      "engines": {
        "node": ">=6.0"
      },
      "peerDependenciesMeta": {
        "supports-color": {
          "optional": true
        }
      }
    },
    "node_modules/depd": {
      "version": "2.0.0",
      "resolved": "https://registry.npmjs.org/depd/-/depd-2.0.0.tgz",
      "integrity": "sha512-g7nH6P6dyDioJogAAGprGpCtVImJhpPk/roCzdb3fIh61/s/nPsfR6onyMwkCAR/OlC3yBC0lESvUoQEAssIrw==",
      "license": "MIT",
      "engines": {
        "node": ">= 0.8"
      }
    },
    "node_modules/dotenv": {
      "version": "17.2.3",
      "resolved": "https://registry.npmjs.org/dotenv/-/dotenv-17.2.3.tgz",
      "integrity": "sha512-JVUnt+DUIzu87TABbhPmNfVdBDt18BLOWjMUFJMSi/Qqg7NTYtabbvSNJGOJ7afbRuv9D/lngizHtP7QyLQ+9w==",
      "license": "BSD-2-Clause",
      "engines": {
        "node": ">=12"
      },
      "funding": {
        "url": "https://dotenvx.com"
      }
    },
    "node_modules/dunder-proto": {
      "version": "1.0.1",
      "resolved": "https://registry.npmjs.org/dunder-proto/-/dunder-proto-1.0.1.tgz",
      "integrity": "sha512-KIN/nDJBQRcXw0MLVhZE9iQHmG68qAVIBg9CqmUYjmQIhgij9U5MFvrqkUL5FbtyyzZuOeOt0zdeRe4UY7ct+A==",
      "license": "MIT",
      "dependencies": {
        "call-bind-apply-helpers": "^1.0.1",
        "es-errors": "^1.3.0",
        "gopd": "^1.2.0"
      },
      "engines": {
        "node": ">= 0.4"
      }
    },
    "node_modules/ee-first": {
      "version": "1.1.1",
      "resolved": "https://registry.npmjs.org/ee-first/-/ee-first-1.1.1.tgz",
      "integrity": "sha512-WMwm9LhRUo+WUaRN+vRuETqG89IgZphVSNkdFgeb6sS/E4OrDIN7t48CAewSHXc6C8lefD8KKfr5vY61brQlow==",
      "license": "MIT"
    },
    "node_modules/electron-to-chromium": {
      "version": "1.5.244",
      "resolved": "https://registry.npmjs.org/electron-to-chromium/-/electron-to-chromium-1.5.244.tgz",
      "integrity": "sha512-OszpBN7xZX4vWMPJwB9illkN/znA8M36GQqQxi6MNy9axWxhOfJyZZJtSLQCpEFLHP2xK33BiWx9aIuIEXVCcw==",
      "license": "ISC"
    },
    "node_modules/emoji-regex": {
      "version": "8.0.0",
      "resolved": "https://registry.npmjs.org/emoji-regex/-/emoji-regex-8.0.0.tgz",
      "integrity": "sha512-MSjYzcWNOA0ewAHpz0MxpYFvwg6yjy1NG3xteoqz644VCo/RPgnr1/GGt+ic3iJTzQ8Eu3TdM14SawnVUmGE6A==",
      "license": "MIT"
    },
    "node_modules/encodeurl": {
      "version": "2.0.0",
      "resolved": "https://registry.npmjs.org/encodeurl/-/encodeurl-2.0.0.tgz",
      "integrity": "sha512-Q0n9HRi4m6JuGIV1eFlmvJB7ZEVxu93IrMyiMsGC0lrMJMWzRgx6WGquyfQgZVb31vhGgXnfmPNNXmxnOkRBrg==",
      "license": "MIT",
      "engines": {
        "node": ">= 0.8"
      }
    },
    "node_modules/es-define-property": {
      "version": "1.0.1",
      "resolved": "https://registry.npmjs.org/es-define-property/-/es-define-property-1.0.1.tgz",
      "integrity": "sha512-e3nRfgfUZ4rNGL232gUgX06QNyyez04KdjFrF+LTRoOXmrOgFKDg4BCdsjW8EnT69eqdYGmRpJwiPVYNrCaW3g==",
      "license": "MIT",
      "engines": {
        "node": ">= 0.4"
      }
    },
    "node_modules/es-errors": {
      "version": "1.3.0",
      "resolved": "https://registry.npmjs.org/es-errors/-/es-errors-1.3.0.tgz",
      "integrity": "sha512-Zf5H2Kxt2xjTvbJvP2ZWLEICxA6j+hAmMzIlypy4xcBg1vKVnx89Wy0GbS+kf5cwCVFFzdCFh2XSCFNULS6csw==",
      "license": "MIT",
      "engines": {
        "node": ">= 0.4"
      }
    },
    "node_modules/es-object-atoms": {
      "version": "1.1.1",
      "resolved": "https://registry.npmjs.org/es-object-atoms/-/es-object-atoms-1.1.1.tgz",
      "integrity": "sha512-FGgH2h8zKNim9ljj7dankFPcICIK9Cp5bm+c2gQSYePhpaG5+esrLODihIorn+Pe6FGJzWhXQotPv73jTaldXA==",
      "license": "MIT",
      "dependencies": {
        "es-errors": "^1.3.0"
      },
      "engines": {
        "node": ">= 0.4"
      }
    },
    "node_modules/esbuild": {
      "version": "0.25.12",
      "resolved": "https://registry.npmjs.org/esbuild/-/esbuild-0.25.12.tgz",
      "integrity": "sha512-bbPBYYrtZbkt6Os6FiTLCTFxvq4tt3JKall1vRwshA3fdVztsLAatFaZobhkBC8/BrPetoa0oksYoKXoG4ryJg==",
      "hasInstallScript": true,
      "license": "MIT",
      "bin": {
        "esbuild": "bin/esbuild"
      },
      "engines": {
        "node": ">=18"
      },
      "optionalDependencies": {
        "@esbuild/aix-ppc64": "0.25.12",
        "@esbuild/android-arm": "0.25.12",
        "@esbuild/android-arm64": "0.25.12",
        "@esbuild/android-x64": "0.25.12",
        "@esbuild/darwin-arm64": "0.25.12",
        "@esbuild/darwin-x64": "0.25.12",
        "@esbuild/freebsd-arm64": "0.25.12",
        "@esbuild/freebsd-x64": "0.25.12",
        "@esbuild/linux-arm": "0.25.12",
        "@esbuild/linux-arm64": "0.25.12",
        "@esbuild/linux-ia32": "0.25.12",
        "@esbuild/linux-loong64": "0.25.12",
        "@esbuild/linux-mips64el": "0.25.12",
        "@esbuild/linux-ppc64": "0.25.12",
        "@esbuild/linux-riscv64": "0.25.12",
        "@esbuild/linux-s390x": "0.25.12",
        "@esbuild/linux-x64": "0.25.12",
        "@esbuild/netbsd-arm64": "0.25.12",
        "@esbuild/netbsd-x64": "0.25.12",
        "@esbuild/openbsd-arm64": "0.25.12",
        "@esbuild/openbsd-x64": "0.25.12",
        "@esbuild/openharmony-arm64": "0.25.12",
        "@esbuild/sunos-x64": "0.25.12",
        "@esbuild/win32-arm64": "0.25.12",
        "@esbuild/win32-ia32": "0.25.12",
        "@esbuild/win32-x64": "0.25.12"
      }
    },
    "node_modules/escalade": {
      "version": "3.2.0",
      "resolved": "https://registry.npmjs.org/escalade/-/escalade-3.2.0.tgz",
      "integrity": "sha512-WUj2qlxaQtO4g6Pq5c29GTcWGDyd8itL8zTlipgECz3JesAiiOKotd8JU6otB3PACgG6xkJUyVhboMS+bje/jA==",
      "license": "MIT",
      "engines": {
        "node": ">=6"
      }
    },
    "node_modules/escape-html": {
      "version": "1.0.3",
      "resolved": "https://registry.npmjs.org/escape-html/-/escape-html-1.0.3.tgz",
      "integrity": "sha512-NiSupZ4OeuGwr68lGIeym/ksIZMJodUGOSCZ/FSnTxcrekbvqrgdUxlJOMpijaKZVjAJrWrGs/6Jy8OMuyj9ow==",
      "license": "MIT"
    },
    "node_modules/etag": {
      "version": "1.8.1",
      "resolved": "https://registry.npmjs.org/etag/-/etag-1.8.1.tgz",
      "integrity": "sha512-aIL5Fx7mawVa300al2BnEE4iNvo1qETxLrPI/o05L7z6go7fCw1J6EQmbK4FmJ2AS7kgVF/KEZWufBfdClMcPg==",
      "license": "MIT",
      "engines": {
        "node": ">= 0.6"
      }
    },
    "node_modules/express": {
      "version": "5.1.0",
      "resolved": "https://registry.npmjs.org/express/-/express-5.1.0.tgz",
      "integrity": "sha512-DT9ck5YIRU+8GYzzU5kT3eHGA5iL+1Zd0EutOmTE9Dtk+Tvuzd23VBU+ec7HPNSTxXYO55gPV/hq4pSBJDjFpA==",
      "license": "MIT",
      "dependencies": {
        "accepts": "^2.0.0",
        "body-parser": "^2.2.0",
        "content-disposition": "^1.0.0",
        "content-type": "^1.0.5",
        "cookie": "^0.7.1",
        "cookie-signature": "^1.2.1",
        "debug": "^4.4.0",
        "encodeurl": "^2.0.0",
        "escape-html": "^1.0.3",
        "etag": "^1.8.1",
        "finalhandler": "^2.1.0",
        "fresh": "^2.0.0",
        "http-errors": "^2.0.0",
        "merge-descriptors": "^2.0.0",
        "mime-types": "^3.0.0",
        "on-finished": "^2.4.1",
        "once": "^1.4.0",
        "parseurl": "^1.3.3",
        "proxy-addr": "^2.0.7",
        "qs": "^6.14.0",
        "range-parser": "^1.2.1",
        "router": "^2.2.0",
        "send": "^1.1.0",
        "serve-static": "^2.2.0",
        "statuses": "^2.0.1",
        "type-is": "^2.0.1",
        "vary": "^1.1.2"
      },
      "engines": {
        "node": ">= 18"
      },
      "funding": {
        "type": "opencollective",
        "url": "https://opencollective.com/express"
      }
    },
    "node_modules/fdir": {
      "version": "6.5.0",
      "resolved": "https://registry.npmjs.org/fdir/-/fdir-6.5.0.tgz",
      "integrity": "sha512-tIbYtZbucOs0BRGqPJkshJUYdL+SDH7dVM8gjy+ERp3WAUjLEFJE+02kanyHtwjWOnwrKYBiwAmM0p4kLJAnXg==",
      "license": "MIT",
      "engines": {
        "node": ">=12.0.0"
      },
      "peerDependencies": {
        "picomatch": "^3 || ^4"
      },
      "peerDependenciesMeta": {
        "picomatch": {
          "optional": true
        }
      }
    },
    "node_modules/finalhandler": {
      "version": "2.1.0",
      "resolved": "https://registry.npmjs.org/finalhandler/-/finalhandler-2.1.0.tgz",
      "integrity": "sha512-/t88Ty3d5JWQbWYgaOGCCYfXRwV1+be02WqYYlL6h0lEiUAMPM8o8qKGO01YIkOHzka2up08wvgYD0mDiI+q3Q==",
      "license": "MIT",
      "dependencies": {
        "debug": "^4.4.0",
        "encodeurl": "^2.0.0",
        "escape-html": "^1.0.3",
        "on-finished": "^2.4.1",
        "parseurl": "^1.3.3",
        "statuses": "^2.0.1"
      },
      "engines": {
        "node": ">= 0.8"
      }
    },
    "node_modules/forwarded": {
      "version": "0.2.0",
      "resolved": "https://registry.npmjs.org/forwarded/-/forwarded-0.2.0.tgz",
      "integrity": "sha512-buRG0fpBtRHSTCOASe6hD258tEubFoRLb4ZNA6NxMVHNw2gOcwHo9wyablzMzOA5z9xA9L1KNjk/Nt6MT9aYow==",
      "license": "MIT",
      "engines": {
        "node": ">= 0.6"
      }
    },
    "node_modules/fresh": {
      "version": "2.0.0",
      "resolved": "https://registry.npmjs.org/fresh/-/fresh-2.0.0.tgz",
      "integrity": "sha512-Rx/WycZ60HOaqLKAi6cHRKKI7zxWbJ31MhntmtwMoaTeF7XFH9hhBp8vITaMidfljRQ6eYWCKkaTK+ykVJHP2A==",
      "license": "MIT",
      "engines": {
        "node": ">= 0.8"
      }
    },
    "node_modules/fsevents": {
      "version": "2.3.3",
      "resolved": "https://registry.npmjs.org/fsevents/-/fsevents-2.3.3.tgz",
      "integrity": "sha512-5xoDfX+fL7faATnagmWPpbFtwh/R77WmMMqqHGS65C3vvB0YHrgF+B1YmZ3441tMj5n63k0212XNoJwzlhffQw==",
      "hasInstallScript": true,
      "license": "MIT",
      "optional": true,
      "os": [
        "darwin"
      ],
      "engines": {
        "node": "^8.16.0 || ^10.6.0 || >=11.0.0"
      }
    },
    "node_modules/function-bind": {
      "version": "1.1.2",
      "resolved": "https://registry.npmjs.org/function-bind/-/function-bind-1.1.2.tgz",
      "integrity": "sha512-7XHNxH7qX9xG5mIwxkhumTox/MIRNcOgDrxWsMt2pAr23WHp6MrRlN7FBSFpCpr+oVO0F744iUgR82nJMfG2SA==",
      "license": "MIT",
      "funding": {
        "url": "https://github.com/sponsors/ljharb"
      }
    },
    "node_modules/gensync": {
      "version": "1.0.0-beta.2",
      "resolved": "https://registry.npmjs.org/gensync/-/gensync-1.0.0-beta.2.tgz",
      "integrity": "sha512-3hN7NaskYvMDLQY55gnW3NQ+mesEAepTqlg+VEbj7zzqEMBVNhzcGYYeqFo/TlYz6eQiFcp1HcsCZO+nGgS8zg==",
      "license": "MIT",
      "engines": {
        "node": ">=6.9.0"
      }
    },
    "node_modules/get-caller-file": {
      "version": "2.0.5",
      "resolved": "https://registry.npmjs.org/get-caller-file/-/get-caller-file-2.0.5.tgz",
      "integrity": "sha512-DyFP3BM/3YHTQOCUL/w0OZHR0lpKeGrxotcHWcqNEdnltqFwXVfhEBQ94eIo34AfQpo0rGki4cyIiftY06h2Fg==",
      "license": "ISC",
      "engines": {
        "node": "6.* || 8.* || >= 10.*"
      }
    },
    "node_modules/get-intrinsic": {
      "version": "1.3.0",
      "resolved": "https://registry.npmjs.org/get-intrinsic/-/get-intrinsic-1.3.0.tgz",
      "integrity": "sha512-9fSjSaos/fRIVIp+xSJlE6lfwhES7LNtKaCBIamHsjr2na1BiABJPo0mOjjz8GJDURarmCPGqaiVg5mfjb98CQ==",
      "license": "MIT",
      "dependencies": {
        "call-bind-apply-helpers": "^1.0.2",
        "es-define-property": "^1.0.1",
        "es-errors": "^1.3.0",
        "es-object-atoms": "^1.1.1",
        "function-bind": "^1.1.2",
        "get-proto": "^1.0.1",
        "gopd": "^1.2.0",
        "has-symbols": "^1.1.0",
        "hasown": "^2.0.2",
        "math-intrinsics": "^1.1.0"
      },
      "engines": {
        "node": ">= 0.4"
      },
      "funding": {
        "url": "https://github.com/sponsors/ljharb"
      }
    },
    "node_modules/get-proto": {
      "version": "1.0.1",
      "resolved": "https://registry.npmjs.org/get-proto/-/get-proto-1.0.1.tgz",
      "integrity": "sha512-sTSfBjoXBp89JvIKIefqw7U2CCebsc74kiY6awiGogKtoSGbgjYE/G/+l9sF3MWFPNc9IcoOC4ODfKHfxFmp0g==",
      "license": "MIT",
      "dependencies": {
        "dunder-proto": "^1.0.1",
        "es-object-atoms": "^1.0.0"
      },
      "engines": {
        "node": ">= 0.4"
      }
    },
    "node_modules/gopd": {
      "version": "1.2.0",
      "resolved": "https://registry.npmjs.org/gopd/-/gopd-1.2.0.tgz",
      "integrity": "sha512-ZUKRh6/kUFoAiTAtTYPZJ3hw9wNxx+BIBOijnlG9PnrJsCcSjs1wyyD6vJpaYtgnzDrKYRSqf3OO6Rfa93xsRg==",
      "license": "MIT",
      "engines": {
        "node": ">= 0.4"
      },
      "funding": {
        "url": "https://github.com/sponsors/ljharb"
      }
    },
    "node_modules/has-flag": {
      "version": "4.0.0",
      "resolved": "https://registry.npmjs.org/has-flag/-/has-flag-4.0.0.tgz",
      "integrity": "sha512-EykJT/Q1KjTWctppgIAgfSO0tKVuZUjhgMr17kqTumMl6Afv3EISleU7qZUzoXDFTAHTDC4NOoG/ZxU3EvlMPQ==",
      "license": "MIT",
      "engines": {
        "node": ">=8"
      }
    },
    "node_modules/has-symbols": {
      "version": "1.1.0",
      "resolved": "https://registry.npmjs.org/has-symbols/-/has-symbols-1.1.0.tgz",
      "integrity": "sha512-1cDNdwJ2Jaohmb3sg4OmKaMBwuC48sYni5HUw2DvsC8LjGTLK9h+eb1X6RyuOHe4hT0ULCW68iomhjUoKUqlPQ==",
      "license": "MIT",
      "engines": {
        "node": ">= 0.4"
      },
      "funding": {
        "url": "https://github.com/sponsors/ljharb"
      }
    },
    "node_modules/hasown": {
      "version": "2.0.2",
      "resolved": "https://registry.npmjs.org/hasown/-/hasown-2.0.2.tgz",
      "integrity": "sha512-0hJU9SCPvmMzIBdZFqNPXWa6dqh7WdH0cII9y+CyS8rG3nL48Bclra9HmKhVVUHyPWNH5Y7xDwAB7bfgSjkUMQ==",
      "license": "MIT",
      "dependencies": {
        "function-bind": "^1.1.2"
      },
      "engines": {
        "node": ">= 0.4"
      }
    },
    "node_modules/http-errors": {
      "version": "2.0.0",
      "resolved": "https://registry.npmjs.org/http-errors/-/http-errors-2.0.0.tgz",
      "integrity": "sha512-FtwrG/euBzaEjYeRqOgly7G0qviiXoJWnvEH2Z1plBdXgbyjv34pHTSb9zoeHMyDy33+DWy5Wt9Wo+TURtOYSQ==",
      "license": "MIT",
      "dependencies": {
        "depd": "2.0.0",
        "inherits": "2.0.4",
        "setprototypeof": "1.2.0",
        "statuses": "2.0.1",
        "toidentifier": "1.0.1"
      },
      "engines": {
        "node": ">= 0.8"
      }
    },
    "node_modules/http-errors/node_modules/statuses": {
      "version": "2.0.1",
      "resolved": "https://registry.npmjs.org/statuses/-/statuses-2.0.1.tgz",
      "integrity": "sha512-RwNA9Z/7PrK06rYLIzFMlaF+l73iwpzsqRIFgbMLbTcLD6cOao82TaWefPXQvB2fOC4AjuYSEndS7N/mTCbkdQ==",
      "license": "MIT",
      "engines": {
        "node": ">= 0.8"
      }
    },
    "node_modules/iconv-lite": {
      "version": "0.6.3",
      "resolved": "https://registry.npmjs.org/iconv-lite/-/iconv-lite-0.6.3.tgz",
      "integrity": "sha512-4fCk79wshMdzMp2rH06qWrJE4iolqLhCUH+OiuIgU++RB0+94NlDL81atO7GX55uUKueo0txHNtvEyI6D7WdMw==",
      "license": "MIT",
      "dependencies": {
        "safer-buffer": ">= 2.1.2 < 3.0.0"
      },
      "engines": {
        "node": ">=0.10.0"
      }
    },
    "node_modules/inherits": {
      "version": "2.0.4",
      "resolved": "https://registry.npmjs.org/inherits/-/inherits-2.0.4.tgz",
      "integrity": "sha512-k/vGaX4/Yla3WzyMCvTQOXYeIHvqOKtnqBduzTHpzpQZzAskKMhZ2K+EnBiSM9zGSoIFeMpXKxa4dYeZIQqewQ==",
      "license": "ISC"
    },
    "node_modules/ipaddr.js": {
      "version": "1.9.1",
      "resolved": "https://registry.npmjs.org/ipaddr.js/-/ipaddr.js-1.9.1.tgz",
      "integrity": "sha512-0KI/607xoxSToH7GjN1FfSbLoU0+btTicjsQSWQlh/hZykN8KpmMf7uYwPW3R+akZ6R/w18ZlXSHBYXiYUPO3g==",
      "license": "MIT",
      "engines": {
        "node": ">= 0.10"
      }
    },
    "node_modules/is-fullwidth-code-point": {
      "version": "3.0.0",
      "resolved": "https://registry.npmjs.org/is-fullwidth-code-point/-/is-fullwidth-code-point-3.0.0.tgz",
      "integrity": "sha512-zymm5+u+sCsSWyD9qNaejV3DFvhCKclKdizYaJUuHA83RLjb7nSuGnddCHGv0hk+KY7BMAlsWeK4Ueg6EV6XQg==",
      "license": "MIT",
      "engines": {
        "node": ">=8"
      }
    },
    "node_modules/is-promise": {
      "version": "4.0.0",
      "resolved": "https://registry.npmjs.org/is-promise/-/is-promise-4.0.0.tgz",
      "integrity": "sha512-hvpoI6korhJMnej285dSg6nu1+e6uxs7zG3BYAm5byqDsgJNWwxzM6z6iZiAgQR4TJ30JmBTOwqZUw3WlyH3AQ==",
      "license": "MIT"
    },
    "node_modules/js-tokens": {
      "version": "4.0.0",
      "resolved": "https://registry.npmjs.org/js-tokens/-/js-tokens-4.0.0.tgz",
      "integrity": "sha512-RdJUflcE3cUzKiMqQgsCu06FPu9UdIJO0beYbPhHN4k6apgJtifcoCtT9bcxOpYBtpD2kCM6Sbzg4CausW/PKQ==",
      "license": "MIT"
    },
    "node_modules/jsesc": {
      "version": "3.1.0",
      "resolved": "https://registry.npmjs.org/jsesc/-/jsesc-3.1.0.tgz",
      "integrity": "sha512-/sM3dO2FOzXjKQhJuo0Q173wf2KOo8t4I8vHy6lF9poUp7bKT0/NHE8fPX23PwfhnykfqnC2xRxOnVw5XuGIaA==",
      "license": "MIT",
      "bin": {
        "jsesc": "bin/jsesc"
      },
      "engines": {
        "node": ">=6"
      }
    },
    "node_modules/json5": {
      "version": "2.2.3",
      "resolved": "https://registry.npmjs.org/json5/-/json5-2.2.3.tgz",
      "integrity": "sha512-XmOWe7eyHYH14cLdVPoyg+GOH3rYX++KpzrylJwSW98t3Nk+U8XOl8FWKOgwtzdb8lXGf6zYwDUzeHMWfxasyg==",
      "license": "MIT",
      "bin": {
        "json5": "lib/cli.js"
      },
      "engines": {
        "node": ">=6"
      }
    },
    "node_modules/lru-cache": {
      "version": "5.1.1",
      "resolved": "https://registry.npmjs.org/lru-cache/-/lru-cache-5.1.1.tgz",
      "integrity": "sha512-KpNARQA3Iwv+jTA0utUVVbrh+Jlrr1Fv0e56GGzAFOXN7dk/FviaDW8LHmK52DlcH4WP2n6gI8vN1aesBFgo9w==",
      "license": "ISC",
      "dependencies": {
        "yallist": "^3.0.2"
      }
    },
    "node_modules/lucide-react": {
      "version": "0.552.0",
      "resolved": "https://registry.npmjs.org/lucide-react/-/lucide-react-0.552.0.tgz",
      "integrity": "sha512-g9WCjmfwqbexSnZE+2cl21PCfXOcqnGeWeMTNAOGEfpPbm/ZF4YIq77Z8qWrxbu660EKuLB4nSLggoKnCb+isw==",
      "license": "ISC",
      "peerDependencies": {
        "react": "^16.5.1 || ^17.0.0 || ^18.0.0 || ^19.0.0"
      }
    },
    "node_modules/math-intrinsics": {
      "version": "1.1.0",
      "resolved": "https://registry.npmjs.org/math-intrinsics/-/math-intrinsics-1.1.0.tgz",
      "integrity": "sha512-/IXtbwEk5HTPyEwyKX6hGkYXxM9nbj64B+ilVJnC/R6B0pH5G4V3b0pVbL7DBj4tkhBAppbQUlf6F6Xl9LHu1g==",
      "license": "MIT",
      "engines": {
        "node": ">= 0.4"
      }
    },
    "node_modules/media-typer": {
      "version": "1.1.0",
      "resolved": "https://registry.npmjs.org/media-typer/-/media-typer-1.1.0.tgz",
      "integrity": "sha512-aisnrDP4GNe06UcKFnV5bfMNPBUw4jsLGaWwWfnH3v02GnBuXX2MCVn5RbrWo0j3pczUilYblq7fQ7Nw2t5XKw==",
      "license": "MIT",
      "engines": {
        "node": ">= 0.8"
      }
    },
    "node_modules/merge-descriptors": {
      "version": "2.0.0",
      "resolved": "https://registry.npmjs.org/merge-descriptors/-/merge-descriptors-2.0.0.tgz",
      "integrity": "sha512-Snk314V5ayFLhp3fkUREub6WtjBfPdCPY1Ln8/8munuLuiYhsABgBVWsozAG+MWMbVEvcdcpbi9R7ww22l9Q3g==",
      "license": "MIT",
      "engines": {
        "node": ">=18"
      },
      "funding": {
        "url": "https://github.com/sponsors/sindresorhus"
      }
    },
    "node_modules/mime-db": {
      "version": "1.54.0",
      "resolved": "https://registry.npmjs.org/mime-db/-/mime-db-1.54.0.tgz",
      "integrity": "sha512-aU5EJuIN2WDemCcAp2vFBfp/m4EAhWJnUNSSw0ixs7/kXbd6Pg64EmwJkNdFhB8aWt1sH2CTXrLxo/iAGV3oPQ==",
      "license": "MIT",
      "engines": {
        "node": ">= 0.6"
      }
    },
    "node_modules/mime-types": {
      "version": "3.0.1",
      "resolved": "https://registry.npmjs.org/mime-types/-/mime-types-3.0.1.tgz",
      "integrity": "sha512-xRc4oEhT6eaBpU1XF7AjpOFD+xQmXNB5OVKwp4tqCuBpHLS/ZbBDrc07mYTDqVMg6PfxUjjNp85O6Cd2Z/5HWA==",
      "license": "MIT",
      "dependencies": {
        "mime-db": "^1.54.0"
      },
      "engines": {
        "node": ">= 0.6"
      }
    },
    "node_modules/ms": {
      "version": "2.1.3",
      "resolved": "https://registry.npmjs.org/ms/-/ms-2.1.3.tgz",
      "integrity": "sha512-6FlzubTLZG3J2a/NVCAleEhjzq5oxgHyaCU9yYXvcLsvoVaHJq/s5xXI6/XXP6tz7R9xAOtHnSO/tXtF3WRTlA==",
      "license": "MIT"
    },
    "node_modules/nanoid": {
      "version": "3.3.11",
      "resolved": "https://registry.npmjs.org/nanoid/-/nanoid-3.3.11.tgz",
      "integrity": "sha512-N8SpfPUnUp1bK+PMYW8qSWdl9U+wwNWI4QKxOYDy9JAro3WMX7p2OeVRF9v+347pnakNevPmiHhNmZ2HbFA76w==",
      "funding": [
        {
          "type": "github",
          "url": "https://github.com/sponsors/ai"
        }
      ],
      "license": "MIT",
      "bin": {
        "nanoid": "bin/nanoid.cjs"
      },
      "engines": {
        "node": "^10 || ^12 || ^13.7 || ^14 || >=15.0.1"
      }
    },
    "node_modules/negotiator": {
      "version": "1.0.0",
      "resolved": "https://registry.npmjs.org/negotiator/-/negotiator-1.0.0.tgz",
      "integrity": "sha512-8Ofs/AUQh8MaEcrlq5xOX0CQ9ypTF5dl78mjlMNfOK08fzpgTHQRQPBxcPlEtIw0yRpws+Zo/3r+5WRby7u3Gg==",
      "license": "MIT",
      "engines": {
        "node": ">= 0.6"
      }
    },
    "node_modules/node-releases": {
      "version": "2.0.27",
      "resolved": "https://registry.npmjs.org/node-releases/-/node-releases-2.0.27.tgz",
      "integrity": "sha512-nmh3lCkYZ3grZvqcCH+fjmQ7X+H0OeZgP40OierEaAptX4XofMh5kwNbWh7lBduUzCcV/8kZ+NDLCwm2iorIlA==",
      "license": "MIT"
    },
    "node_modules/object-assign": {
      "version": "4.1.1",
      "resolved": "https://registry.npmjs.org/object-assign/-/object-assign-4.1.1.tgz",
      "integrity": "sha512-rJgTQnkUnH1sFw8yT6VSU3zD3sWmu6sZhIseY8VX+GRu3P6F7Fu+JNDoXfklElbLJSnc3FUQHVe4cU5hj+BcUg==",
      "license": "MIT",
      "engines": {
        "node": ">=0.10.0"
      }
    },
    "node_modules/object-inspect": {
      "version": "1.13.4",
      "resolved": "https://registry.npmjs.org/object-inspect/-/object-inspect-1.13.4.tgz",
      "integrity": "sha512-W67iLl4J2EXEGTbfeHCffrjDfitvLANg0UlX3wFUUSTx92KXRFegMHUVgSqE+wvhAbi4WqjGg9czysTV2Epbew==",
      "license": "MIT",
      "engines": {
        "node": ">= 0.4"
      },
      "funding": {
        "url": "https://github.com/sponsors/ljharb"
      }
    },
    "node_modules/on-finished": {
      "version": "2.4.1",
      "resolved": "https://registry.npmjs.org/on-finished/-/on-finished-2.4.1.tgz",
      "integrity": "sha512-oVlzkg3ENAhCk2zdv7IJwd/QUD4z2RxRwpkcGY8psCVcCYZNq4wYnVWALHM+brtuJjePWiYF/ClmuDr8Ch5+kg==",
      "license": "MIT",
      "dependencies": {
        "ee-first": "1.1.1"
      },
      "engines": {
        "node": ">= 0.8"
      }
    },
    "node_modules/once": {
      "version": "1.4.0",
      "resolved": "https://registry.npmjs.org/once/-/once-1.4.0.tgz",
      "integrity": "sha512-lNaJgI+2Q5URQBkccEKHTQOPaXdUxnZZElQTZY0MFUAuaEqe1E+Nyvgdz/aIyNi6Z9MzO5dv1H8n58/GELp3+w==",
      "license": "ISC",
      "dependencies": {
        "wrappy": "1"
      }
    },
    "node_modules/parseurl": {
      "version": "1.3.3",
      "resolved": "https://registry.npmjs.org/parseurl/-/parseurl-1.3.3.tgz",
      "integrity": "sha512-CiyeOxFT/JZyN5m0z9PfXw4SCBJ6Sygz1Dpl0wqjlhDEGGBP1GnsUVEL0p63hoG1fcj3fHynXi9NYO4nWOL+qQ==",
      "license": "MIT",
      "engines": {
        "node": ">= 0.8"
      }
    },
    "node_modules/path-to-regexp": {
      "version": "8.3.0",
      "resolved": "https://registry.npmjs.org/path-to-regexp/-/path-to-regexp-8.3.0.tgz",
      "integrity": "sha512-7jdwVIRtsP8MYpdXSwOS0YdD0Du+qOoF/AEPIt88PcCFrZCzx41oxku1jD88hZBwbNUIEfpqvuhjFaMAqMTWnA==",
      "license": "MIT",
      "funding": {
        "type": "opencollective",
        "url": "https://opencollective.com/express"
      }
    },
    "node_modules/picocolors": {
      "version": "1.1.1",
      "resolved": "https://registry.npmjs.org/picocolors/-/picocolors-1.1.1.tgz",
      "integrity": "sha512-xceH2snhtb5M9liqDsmEw56le376mTZkEX/jEb/RxNFyegNul7eNslCXP9FDj/Lcu0X8KEyMceP2ntpaHrDEVA==",
      "license": "ISC"
    },
    "node_modules/picomatch": {
      "version": "4.0.3",
      "resolved": "https://registry.npmjs.org/picomatch/-/picomatch-4.0.3.tgz",
      "integrity": "sha512-5gTmgEY/sqK6gFXLIsQNH19lWb4ebPDLA4SdLP7dsWkIXHWlG66oPuVvXSGFPppYZz8ZDZq0dYYrbHfBCVUb1Q==",
      "license": "MIT",
      "engines": {
        "node": ">=12"
      },
      "funding": {
        "url": "https://github.com/sponsors/jonschlinkert"
      }
    },
    "node_modules/postcss": {
      "version": "8.5.6",
      "resolved": "https://registry.npmjs.org/postcss/-/postcss-8.5.6.tgz",
      "integrity": "sha512-3Ybi1tAuwAP9s0r1UQ2J4n5Y0G05bJkpUIO0/bI9MhwmD70S5aTWbXGBwxHrelT+XM1k6dM0pk+SwNkpTRN7Pg==",
      "funding": [
        {
          "type": "opencollective",
          "url": "https://opencollective.com/postcss/"
        },
        {
          "type": "tidelift",
          "url": "https://tidelift.com/funding/github/npm/postcss"
        },
        {
          "type": "github",
          "url": "https://github.com/sponsors/ai"
        }
      ],
      "license": "MIT",
      "dependencies": {
        "nanoid": "^3.3.11",
        "picocolors": "^1.1.1",
        "source-map-js": "^1.2.1"
      },
      "engines": {
        "node": "^10 || ^12 || >=14"
      }
    },
    "node_modules/proxy-addr": {
      "version": "2.0.7",
      "resolved": "https://registry.npmjs.org/proxy-addr/-/proxy-addr-2.0.7.tgz",
      "integrity": "sha512-llQsMLSUDUPT44jdrU/O37qlnifitDP+ZwrmmZcoSKyLKvtZxpyV0n2/bD/N4tBAAZ/gJEdZU7KMraoK1+XYAg==",
      "license": "MIT",
      "dependencies": {
        "forwarded": "0.2.0",
        "ipaddr.js": "1.9.1"
      },
      "engines": {
        "node": ">= 0.10"
      }
    },
    "node_modules/qs": {
      "version": "6.14.0",
      "resolved": "https://registry.npmjs.org/qs/-/qs-6.14.0.tgz",
      "integrity": "sha512-YWWTjgABSKcvs/nWBi9PycY/JiPJqOD4JA6o9Sej2AtvSGarXxKC3OQSk4pAarbdQlKAh5D4FCQkJNkW+GAn3w==",
      "license": "BSD-3-Clause",
      "dependencies": {
        "side-channel": "^1.1.0"
      },
      "engines": {
        "node": ">=0.6"
      },
      "funding": {
        "url": "https://github.com/sponsors/ljharb"
      }
    },
    "node_modules/range-parser": {
      "version": "1.2.1",
      "resolved": "https://registry.npmjs.org/range-parser/-/range-parser-1.2.1.tgz",
      "integrity": "sha512-Hrgsx+orqoygnmhFbKaHE6c296J+HTAQXoxEF6gNupROmmGJRoyzfG3ccAveqCBrwr/2yxQ5BVd/GTl5agOwSg==",
      "license": "MIT",
      "engines": {
        "node": ">= 0.6"
      }
    },
    "node_modules/raw-body": {
      "version": "3.0.1",
      "resolved": "https://registry.npmjs.org/raw-body/-/raw-body-3.0.1.tgz",
      "integrity": "sha512-9G8cA+tuMS75+6G/TzW8OtLzmBDMo8p1JRxN5AZ+LAp8uxGA8V8GZm4GQ4/N5QNQEnLmg6SS7wyuSmbKepiKqA==",
      "license": "MIT",
      "dependencies": {
        "bytes": "3.1.2",
        "http-errors": "2.0.0",
        "iconv-lite": "0.7.0",
        "unpipe": "1.0.0"
      },
      "engines": {
        "node": ">= 0.10"
      }
    },
    "node_modules/raw-body/node_modules/iconv-lite": {
      "version": "0.7.0",
      "resolved": "https://registry.npmjs.org/iconv-lite/-/iconv-lite-0.7.0.tgz",
      "integrity": "sha512-cf6L2Ds3h57VVmkZe+Pn+5APsT7FpqJtEhhieDCvrE2MK5Qk9MyffgQyuxQTm6BChfeZNtcOLHp9IcWRVcIcBQ==",
      "license": "MIT",
      "dependencies": {
        "safer-buffer": ">= 2.1.2 < 3.0.0"
      },
      "engines": {
        "node": ">=0.10.0"
      },
      "funding": {
        "type": "opencollective",
        "url": "https://opencollective.com/express"
      }
    },
    "node_modules/react": {
      "version": "19.2.0",
      "resolved": "https://registry.npmjs.org/react/-/react-19.2.0.tgz",
      "integrity": "sha512-tmbWg6W31tQLeB5cdIBOicJDJRR2KzXsV7uSK9iNfLWQ5bIZfxuPEHp7M8wiHyHnn0DD1i7w3Zmin0FtkrwoCQ==",
      "license": "MIT",
      "engines": {
        "node": ">=0.10.0"
      }
    },
    "node_modules/react-dom": {
      "version": "19.2.0",
      "resolved": "https://registry.npmjs.org/react-dom/-/react-dom-19.2.0.tgz",
      "integrity": "sha512-UlbRu4cAiGaIewkPyiRGJk0imDN2T3JjieT6spoL2UeSf5od4n5LB/mQ4ejmxhCFT1tYe8IvaFulzynWovsEFQ==",
      "license": "MIT",
      "dependencies": {
        "scheduler": "^0.27.0"
      },
      "peerDependencies": {
        "react": "^19.2.0"
      }
    },
    "node_modules/react-refresh": {
      "version": "0.18.0",
      "resolved": "https://registry.npmjs.org/react-refresh/-/react-refresh-0.18.0.tgz",
      "integrity": "sha512-QgT5//D3jfjJb6Gsjxv0Slpj23ip+HtOpnNgnb2S5zU3CB26G/IDPGoy4RJB42wzFE46DRsstbW6tKHoKbhAxw==",
      "license": "MIT",
      "engines": {
        "node": ">=0.10.0"
      }
    },
    "node_modules/require-directory": {
      "version": "2.1.1",
      "resolved": "https://registry.npmjs.org/require-directory/-/require-directory-2.1.1.tgz",
      "integrity": "sha512-fGxEI7+wsG9xrvdjsrlmL22OMTTiHRwAMroiEeMgq8gzoLC/PQr7RsRDSTLUg/bZAZtF+TVIkHc6/4RIKrui+Q==",
      "license": "MIT",
      "engines": {
        "node": ">=0.10.0"
      }
    },
    "node_modules/rollup": {
      "version": "4.52.5",
      "resolved": "https://registry.npmjs.org/rollup/-/rollup-4.52.5.tgz",
      "integrity": "sha512-3GuObel8h7Kqdjt0gxkEzaifHTqLVW56Y/bjN7PSQtkKr0w3V/QYSdt6QWYtd7A1xUtYQigtdUfgj1RvWVtorw==",
      "license": "MIT",
      "dependencies": {
        "@types/estree": "1.0.8"
      },
      "bin": {
        "rollup": "dist/bin/rollup"
      },
      "engines": {
        "node": ">=18.0.0",
        "npm": ">=8.0.0"
      },
      "optionalDependencies": {
        "@rollup/rollup-android-arm-eabi": "4.52.5",
        "@rollup/rollup-android-arm64": "4.52.5",
        "@rollup/rollup-darwin-arm64": "4.52.5",
        "@rollup/rollup-darwin-x64": "4.52.5",
        "@rollup/rollup-freebsd-arm64": "4.52.5",
        "@rollup/rollup-freebsd-x64": "4.52.5",
        "@rollup/rollup-linux-arm-gnueabihf": "4.52.5",
        "@rollup/rollup-linux-arm-musleabihf": "4.52.5",
        "@rollup/rollup-linux-arm64-gnu": "4.52.5",
        "@rollup/rollup-linux-arm64-musl": "4.52.5",
        "@rollup/rollup-linux-loong64-gnu": "4.52.5",
        "@rollup/rollup-linux-ppc64-gnu": "4.52.5",
        "@rollup/rollup-linux-riscv64-gnu": "4.52.5",
        "@rollup/rollup-linux-riscv64-musl": "4.52.5",
        "@rollup/rollup-linux-s390x-gnu": "4.52.5",
        "@rollup/rollup-linux-x64-gnu": "4.52.5",
        "@rollup/rollup-linux-x64-musl": "4.52.5",
        "@rollup/rollup-openharmony-arm64": "4.52.5",
        "@rollup/rollup-win32-arm64-msvc": "4.52.5",
        "@rollup/rollup-win32-ia32-msvc": "4.52.5",
        "@rollup/rollup-win32-x64-gnu": "4.52.5",
        "@rollup/rollup-win32-x64-msvc": "4.52.5",
        "fsevents": "~2.3.2"
      }
    },
    "node_modules/router": {
      "version": "2.2.0",
      "resolved": "https://registry.npmjs.org/router/-/router-2.2.0.tgz",
      "integrity": "sha512-nLTrUKm2UyiL7rlhapu/Zl45FwNgkZGaCpZbIHajDYgwlJCOzLSk+cIPAnsEqV955GjILJnKbdQC1nVPz+gAYQ==",
      "license": "MIT",
      "dependencies": {
        "debug": "^4.4.0",
        "depd": "^2.0.0",
        "is-promise": "^4.0.0",
        "parseurl": "^1.3.3",
        "path-to-regexp": "^8.0.0"
      },
      "engines": {
        "node": ">= 18"
      }
    },
    "node_modules/rxjs": {
      "version": "7.8.2",
      "resolved": "https://registry.npmjs.org/rxjs/-/rxjs-7.8.2.tgz",
      "integrity": "sha512-dhKf903U/PQZY6boNNtAGdWbG85WAbjT/1xYoZIC7FAY0yWapOBQVsVrDl58W86//e1VpMNBtRV4MaXfdMySFA==",
      "license": "Apache-2.0",
      "dependencies": {
        "tslib": "^2.1.0"
      }
    },
    "node_modules/safe-buffer": {
      "version": "5.2.1",
      "resolved": "https://registry.npmjs.org/safe-buffer/-/safe-buffer-5.2.1.tgz",
      "integrity": "sha512-rp3So07KcdmmKbGvgaNxQSJr7bGVSVk5S9Eq1F+ppbRo70+YeaDxkw5Dd8NPN+GD6bjnYm2VuPuCXmpuYvmCXQ==",
      "funding": [
        {
          "type": "github",
          "url": "https://github.com/sponsors/feross"
        },
        {
          "type": "patreon",
          "url": "https://www.patreon.com/feross"
        },
        {
          "type": "consulting",
          "url": "https://feross.org/support"
        }
      ],
      "license": "MIT"
    },
    "node_modules/safer-buffer": {
      "version": "2.1.2",
      "resolved": "https://registry.npmjs.org/safer-buffer/-/safer-buffer-2.1.2.tgz",
      "integrity": "sha512-YZo3K82SD7Riyi0E1EQPojLz7kpepnSQI9IyPbHHg1XXXevb5dJI7tpyN2ADxGcQbHG7vcyRHk0cbwqcQriUtg==",
      "license": "MIT"
    },
    "node_modules/scheduler": {
      "version": "0.27.0",
      "resolved": "https://registry.npmjs.org/scheduler/-/scheduler-0.27.0.tgz",
      "integrity": "sha512-eNv+WrVbKu1f3vbYJT/xtiF5syA5HPIMtf9IgY/nKg0sWqzAUEvqY/xm7OcZc/qafLx/iO9FgOmeSAp4v5ti/Q==",
      "license": "MIT"
    },
    "node_modules/semver": {
      "version": "6.3.1",
      "resolved": "https://registry.npmjs.org/semver/-/semver-6.3.1.tgz",
      "integrity": "sha512-BR7VvDCVHO+q2xBEWskxS6DJE1qRnb7DxzUrogb71CWoSficBxYsiAGd+Kl0mmq/MprG9yArRkyrQxTO6XjMzA==",
      "license": "ISC",
      "bin": {
        "semver": "bin/semver.js"
      }
    },
    "node_modules/send": {
      "version": "1.2.0",
      "resolved": "https://registry.npmjs.org/send/-/send-1.2.0.tgz",
      "integrity": "sha512-uaW0WwXKpL9blXE2o0bRhoL2EGXIrZxQ2ZQ4mgcfoBxdFmQold+qWsD2jLrfZ0trjKL6vOw0j//eAwcALFjKSw==",
      "license": "MIT",
      "dependencies": {
        "debug": "^4.3.5",
        "encodeurl": "^2.0.0",
        "escape-html": "^1.0.3",
        "etag": "^1.8.1",
        "fresh": "^2.0.0",
        "http-errors": "^2.0.0",
        "mime-types": "^3.0.1",
        "ms": "^2.1.3",
        "on-finished": "^2.4.1",
        "range-parser": "^1.2.1",
        "statuses": "^2.0.1"
      },
      "engines": {
        "node": ">= 18"
      }
    },
    "node_modules/serve-static": {
      "version": "2.2.0",
      "resolved": "https://registry.npmjs.org/serve-static/-/serve-static-2.2.0.tgz",
      "integrity": "sha512-61g9pCh0Vnh7IutZjtLGGpTA355+OPn2TyDv/6ivP2h/AdAVX9azsoxmg2/M6nZeQZNYBEwIcsne1mJd9oQItQ==",
      "license": "MIT",
      "dependencies": {
        "encodeurl": "^2.0.0",
        "escape-html": "^1.0.3",
        "parseurl": "^1.3.3",
        "send": "^1.2.0"
      },
      "engines": {
        "node": ">= 18"
      }
    },
    "node_modules/setprototypeof": {
      "version": "1.2.0",
      "resolved": "https://registry.npmjs.org/setprototypeof/-/setprototypeof-1.2.0.tgz",
      "integrity": "sha512-E5LDX7Wrp85Kil5bhZv46j8jOeboKq5JMmYM3gVGdGH8xFpPWXUMsNrlODCrkoxMEeNi/XZIwuRvY4XNwYMJpw==",
      "license": "ISC"
    },
    "node_modules/shell-quote": {
      "version": "1.8.3",
      "resolved": "https://registry.npmjs.org/shell-quote/-/shell-quote-1.8.3.tgz",
      "integrity": "sha512-ObmnIF4hXNg1BqhnHmgbDETF8dLPCggZWBjkQfhZpbszZnYur5DUljTcCHii5LC3J5E0yeO/1LIMyH+UvHQgyw==",
      "license": "MIT",
      "engines": {
        "node": ">= 0.4"
      },
      "funding": {
        "url": "https://github.com/sponsors/ljharb"
      }
    },
    "node_modules/side-channel": {
      "version": "1.1.0",
      "resolved": "https://registry.npmjs.org/side-channel/-/side-channel-1.1.0.tgz",
      "integrity": "sha512-ZX99e6tRweoUXqR+VBrslhda51Nh5MTQwou5tnUDgbtyM0dBgmhEDtWGP/xbKn6hqfPRHujUNwz5fy/wbbhnpw==",
      "license": "MIT",
      "dependencies": {
        "es-errors": "^1.3.0",
        "object-inspect": "^1.13.3",
        "side-channel-list": "^1.0.0",
        "side-channel-map": "^1.0.1",
        "side-channel-weakmap": "^1.0.2"
      },
      "engines": {
        "node": ">= 0.4"
      },
      "funding": {
        "url": "https://github.com/sponsors/ljharb"
      }
    },
    "node_modules/side-channel-list": {
      "version": "1.0.0",
      "resolved": "https://registry.npmjs.org/side-channel-list/-/side-channel-list-1.0.0.tgz",
      "integrity": "sha512-FCLHtRD/gnpCiCHEiJLOwdmFP+wzCmDEkc9y7NsYxeF4u7Btsn1ZuwgwJGxImImHicJArLP4R0yX4c2KCrMrTA==",
      "license": "MIT",
      "dependencies": {
        "es-errors": "^1.3.0",
        "object-inspect": "^1.13.3"
      },
      "engines": {
        "node": ">= 0.4"
      },
      "funding": {
        "url": "https://github.com/sponsors/ljharb"
      }
    },
    "node_modules/side-channel-map": {
      "version": "1.0.1",
      "resolved": "https://registry.npmjs.org/side-channel-map/-/side-channel-map-1.0.1.tgz",
      "integrity": "sha512-VCjCNfgMsby3tTdo02nbjtM/ewra6jPHmpThenkTYh8pG9ucZ/1P8So4u4FGBek/BjpOVsDCMoLA/iuBKIFXRA==",
      "license": "MIT",
      "dependencies": {
        "call-bound": "^1.0.2",
        "es-errors": "^1.3.0",
        "get-intrinsic": "^1.2.5",
        "object-inspect": "^1.13.3"
      },
      "engines": {
        "node": ">= 0.4"
      },
      "funding": {
        "url": "https://github.com/sponsors/ljharb"
      }
    },
    "node_modules/side-channel-weakmap": {
      "version": "1.0.2",
      "resolved": "https://registry.npmjs.org/side-channel-weakmap/-/side-channel-weakmap-1.0.2.tgz",
      "integrity": "sha512-WPS/HvHQTYnHisLo9McqBHOJk2FkHO/tlpvldyrnem4aeQp4hai3gythswg6p01oSoTl58rcpiFAjF2br2Ak2A==",
      "license": "MIT",
      "dependencies": {
        "call-bound": "^1.0.2",
        "es-errors": "^1.3.0",
        "get-intrinsic": "^1.2.5",
        "object-inspect": "^1.13.3",
        "side-channel-map": "^1.0.1"
      },
      "engines": {
        "node": ">= 0.4"
      },
      "funding": {
        "url": "https://github.com/sponsors/ljharb"
      }
    },
    "node_modules/source-map-js": {
      "version": "1.2.1",
      "resolved": "https://registry.npmjs.org/source-map-js/-/source-map-js-1.2.1.tgz",
      "integrity": "sha512-UXWMKhLOwVKb728IUtQPXxfYU+usdybtUrK/8uGE8CQMvrhOpwvzDBwj0QhSL7MQc7vIsISBG8VQ8+IDQxpfQA==",
      "license": "BSD-3-Clause",
      "engines": {
        "node": ">=0.10.0"
      }
    },
    "node_modules/statuses": {
      "version": "2.0.2",
      "resolved": "https://registry.npmjs.org/statuses/-/statuses-2.0.2.tgz",
      "integrity": "sha512-DvEy55V3DB7uknRo+4iOGT5fP1slR8wQohVdknigZPMpMstaKJQWhwiYBACJE3Ul2pTnATihhBYnRhZQHGBiRw==",
      "license": "MIT",
      "engines": {
        "node": ">= 0.8"
      }
    },
    "node_modules/string-width": {
      "version": "4.2.3",
      "resolved": "https://registry.npmjs.org/string-width/-/string-width-4.2.3.tgz",
      "integrity": "sha512-wKyQRQpjJ0sIp62ErSZdGsjMJWsap5oRNihHhu6G7JVO/9jIB6UyevL+tXuOqrng8j/cxKTWyWUwvSTriiZz/g==",
      "license": "MIT",
      "dependencies": {
        "emoji-regex": "^8.0.0",
        "is-fullwidth-code-point": "^3.0.0",
        "strip-ansi": "^6.0.1"
      },
      "engines": {
        "node": ">=8"
      }
    },
    "node_modules/strip-ansi": {
      "version": "6.0.1",
      "resolved": "https://registry.npmjs.org/strip-ansi/-/strip-ansi-6.0.1.tgz",
      "integrity": "sha512-Y38VPSHcqkFrCpFnQ9vuSXmquuv5oXOKpGeT6aGrr3o3Gc9AlVa6JBfUSOCnbxGGZF+/0ooI7KrPuUSztUdU5A==",
      "license": "MIT",
      "dependencies": {
        "ansi-regex": "^5.0.1"
      },
      "engines": {
        "node": ">=8"
      }
    },
    "node_modules/supports-color": {
      "version": "8.1.1",
      "resolved": "https://registry.npmjs.org/supports-color/-/supports-color-8.1.1.tgz",
      "integrity": "sha512-MpUEN2OodtUzxvKQl72cUF7RQ5EiHsGvSsVG0ia9c5RbWGL2CI4C7EpPS8UTBIplnlzZiNuV56w+FuNxy3ty2Q==",
      "license": "MIT",
      "dependencies": {
        "has-flag": "^4.0.0"
      },
      "engines": {
        "node": ">=10"
      },
      "funding": {
        "url": "https://github.com/chalk/supports-color?sponsor=1"
      }
    },
    "node_modules/tinyglobby": {
      "version": "0.2.15",
      "resolved": "https://registry.npmjs.org/tinyglobby/-/tinyglobby-0.2.15.tgz",
      "integrity": "sha512-j2Zq4NyQYG5XMST4cbs02Ak8iJUdxRM0XI5QyxXuZOzKOINmWurp3smXu3y5wDcJrptwpSjgXHzIQxR0omXljQ==",
      "license": "MIT",
      "dependencies": {
        "fdir": "^6.5.0",
        "picomatch": "^4.0.3"
      },
      "engines": {
        "node": ">=12.0.0"
      },
      "funding": {
        "url": "https://github.com/sponsors/SuperchupuDev"
      }
    },
    "node_modules/toidentifier": {
      "version": "1.0.1",
      "resolved": "https://registry.npmjs.org/toidentifier/-/toidentifier-1.0.1.tgz",
      "integrity": "sha512-o5sSPKEkg/DIQNmH43V0/uerLrpzVedkUh8tGNvaeXpfpuwjKenlSox/2O/BTlZUtEe+JG7s5YhEz608PlAHRA==",
      "license": "MIT",
      "engines": {
        "node": ">=0.6"
      }
    },
    "node_modules/tree-kill": {
      "version": "1.2.2",
      "resolved": "https://registry.npmjs.org/tree-kill/-/tree-kill-1.2.2.tgz",
      "integrity": "sha512-L0Orpi8qGpRG//Nd+H90vFB+3iHnue1zSSGmNOOCh1GLJ7rUKVwV2HvijphGQS2UmhUZewS9VgvxYIdgr+fG1A==",
      "license": "MIT",
      "bin": {
        "tree-kill": "cli.js"
      }
    },
    "node_modules/tslib": {
      "version": "2.8.1",
      "resolved": "https://registry.npmjs.org/tslib/-/tslib-2.8.1.tgz",
      "integrity": "sha512-oJFu94HQb+KVduSUQL7wnpmqnfmLsOA/nAh6b6EH0wCEoK0/mPeXU6c3wKDV83MkOuHPRHtSXKKU99IBazS/2w==",
      "license": "0BSD"
    },
    "node_modules/type-is": {
      "version": "2.0.1",
      "resolved": "https://registry.npmjs.org/type-is/-/type-is-2.0.1.tgz",
      "integrity": "sha512-OZs6gsjF4vMp32qrCbiVSkrFmXtG/AZhY3t0iAMrMBiAZyV9oALtXO8hsrHbMXF9x6L3grlFuwW2oAz7cav+Gw==",
      "license": "MIT",
      "dependencies": {
        "content-type": "^1.0.5",
        "media-typer": "^1.1.0",
        "mime-types": "^3.0.0"
      },
      "engines": {
        "node": ">= 0.6"
      }
    },
    "node_modules/unpipe": {
      "version": "1.0.0",
      "resolved": "https://registry.npmjs.org/unpipe/-/unpipe-1.0.0.tgz",
      "integrity": "sha512-pjy2bYhSsufwWlKwPc+l3cN7+wuJlK6uz0YdJEOlQDbl6jo/YlPi4mb8agUkVC8BF7V8NuzeyPNqRksA3hztKQ==",
      "license": "MIT",
      "engines": {
        "node": ">= 0.8"
      }
    },
    "node_modules/update-browserslist-db": {
      "version": "1.1.4",
      "resolved": "https://registry.npmjs.org/update-browserslist-db/-/update-browserslist-db-1.1.4.tgz",
      "integrity": "sha512-q0SPT4xyU84saUX+tomz1WLkxUbuaJnR1xWt17M7fJtEJigJeWUNGUqrauFXsHnqev9y9JTRGwk13tFBuKby4A==",
      "funding": [
        {
          "type": "opencollective",
          "url": "https://opencollective.com/browserslist"
        },
        {
          "type": "tidelift",
          "url": "https://tidelift.com/funding/github/npm/browserslist"
        },
        {
          "type": "github",
          "url": "https://github.com/sponsors/ai"
        }
      ],
      "license": "MIT",
      "dependencies": {
        "escalade": "^3.2.0",
        "picocolors": "^1.1.1"
      },
      "bin": {
        "update-browserslist-db": "cli.js"
      },
      "peerDependencies": {
        "browserslist": ">= 4.21.0"
      }
    },
    "node_modules/vary": {
      "version": "1.1.2",
      "resolved": "https://registry.npmjs.org/vary/-/vary-1.1.2.tgz",
      "integrity": "sha512-BNGbWLfd0eUPabhkXUVm0j8uuvREyTh5ovRa/dyow/BqAbZJyC+5fU+IzQOzmAKzYqYRAISoRhdQr3eIZ/PXqg==",
      "license": "MIT",
      "engines": {
        "node": ">= 0.8"
      }
    },
    "node_modules/vite": {
      "version": "7.1.12",
      "resolved": "https://registry.npmjs.org/vite/-/vite-7.1.12.tgz",
      "integrity": "sha512-ZWyE8YXEXqJrrSLvYgrRP7p62OziLW7xI5HYGWFzOvupfAlrLvURSzv/FyGyy0eidogEM3ujU+kUG1zuHgb6Ug==",
      "license": "MIT",
      "dependencies": {
        "esbuild": "^0.25.0",
        "fdir": "^6.5.0",
        "picomatch": "^4.0.3",
        "postcss": "^8.5.6",
        "rollup": "^4.43.0",
        "tinyglobby": "^0.2.15"
      },
      "bin": {
        "vite": "bin/vite.js"
      },
      "engines": {
        "node": "^20.19.0 || >=22.12.0"
      },
      "funding": {
        "url": "https://github.com/vitejs/vite?sponsor=1"
      },
      "optionalDependencies": {
        "fsevents": "~2.3.3"
      },
      "peerDependencies": {
        "@types/node": "^20.19.0 || >=22.12.0",
        "jiti": ">=1.21.0",
        "less": "^4.0.0",
        "lightningcss": "^1.21.0",
        "sass": "^1.70.0",
        "sass-embedded": "^1.70.0",
        "stylus": ">=0.54.8",
        "sugarss": "^5.0.0",
        "terser": "^5.16.0",
        "tsx": "^4.8.1",
        "yaml": "^2.4.2"
      },
      "peerDependenciesMeta": {
        "@types/node": {
          "optional": true
        },
        "jiti": {
          "optional": true
        },
        "less": {
          "optional": true
        },
        "lightningcss": {
          "optional": true
        },
        "sass": {
          "optional": true
        },
        "sass-embedded": {
          "optional": true
        },
        "stylus": {
          "optional": true
        },
        "sugarss": {
          "optional": true
        },
        "terser": {
          "optional": true
        },
        "tsx": {
          "optional": true
        },
        "yaml": {
          "optional": true
        }
      }
    },
    "node_modules/wrap-ansi": {
      "version": "7.0.0",
      "resolved": "https://registry.npmjs.org/wrap-ansi/-/wrap-ansi-7.0.0.tgz",
      "integrity": "sha512-YVGIj2kamLSTxw6NsZjoBxfSwsn0ycdesmc4p+Q21c5zPuZ1pl+NfxVdxPtdHvmNVOQ6XSYG4AUtyt/Fi7D16Q==",
      "license": "MIT",
      "dependencies": {
        "ansi-styles": "^4.0.0",
        "string-width": "^4.1.0",
        "strip-ansi": "^6.0.0"
      },
      "engines": {
        "node": ">=10"
      },
      "funding": {
        "url": "https://github.com/chalk/wrap-ansi?sponsor=1"
      }
    },
    "node_modules/wrappy": {
      "version": "1.0.2",
      "resolved": "https://registry.npmjs.org/wrappy/-/wrappy-1.0.2.tgz",
      "integrity": "sha512-l4Sp/DRseor9wL6EvV2+TuQn63dMkPjZ/sp9XkghTEbV9KlPS1xUsZ3u7/IQO4wxtcFB4bgpQPRcR3QCvezPcQ==",
      "license": "ISC"
    },
    "node_modules/y18n": {
      "version": "5.0.8",
      "resolved": "https://registry.npmjs.org/y18n/-/y18n-5.0.8.tgz",
      "integrity": "sha512-0pfFzegeDWJHJIAmTLRP2DwHjdF5s7jo9tuztdQxAhINCdvS+3nGINqPd00AphqJR/0LhANUS6/+7SCb98YOfA==",
      "license": "ISC",
      "engines": {
        "node": ">=10"
      }
    },
    "node_modules/yallist": {
      "version": "3.1.1",
      "resolved": "https://registry.npmjs.org/yallist/-/yallist-3.1.1.tgz",
      "integrity": "sha512-a4UGQaWPH59mOXUYnAG2ewncQS4i4F43Tv3JoAM+s2VDAmS9NsK8GpDMLrCHPksFT7h3K6TOoUNn2pb7RoXx4g==",
      "license": "ISC"
    },
    "node_modules/yargs": {
      "version": "17.7.2",
      "resolved": "https://registry.npmjs.org/yargs/-/yargs-17.7.2.tgz",
      "integrity": "sha512-7dSzzRQ++CKnNI/krKnYRV7JKKPUXMEh61soaHKg9mrWEhzFWhFnxPxGl+69cD1Ou63C13NUPCnmIcrvqCuM6w==",
      "license": "MIT",
      "dependencies": {
        "cliui": "^8.0.1",
        "escalade": "^3.1.1",
        "get-caller-file": "^2.0.5",
        "require-directory": "^2.1.1",
        "string-width": "^4.2.3",
        "y18n": "^5.0.5",
        "yargs-parser": "^21.1.1"
      },
      "engines": {
        "node": ">=12"
      }
    },
    "node_modules/yargs-parser": {
      "version": "21.1.1",
      "resolved": "https://registry.npmjs.org/yargs-parser/-/yargs-parser-21.1.1.tgz",
      "integrity": "sha512-tVpsJW7DdjecAiFpbIB1e3qxIQsE6NoPc5/eTdrbbIC4h0LVsWhnoa3g+m2HclBIujHzsxZ4VJVA+GUuc2/LBw==",
      "license": "ISC",
      "engines": {
        "node": ">=12"
      }
    }
  }
}

\end{minted}
\newpage
\section{package.json}
\begin{minted}[breaklines, linenos, fontsize=\small, frame=single]{json}
{
  "name": "umedcta",
  "version": "1.0.0",
  "type": "module",
  "description": "",
  "main": "index.js",
  "scripts": {
    "dev": "concurrently \"npm run server\" \"npm run client\"",
    "server": "node server.js",
    "client": "vite",
    "build": "vite build",
    "preview": "vite preview"
  },
  "repository": {
    "type": "git",
    "url": "git+https://github.com/TioSavich/UMEDCTA.git"
  },
  "keywords": [],
  "author": "",
  "license": "ISC",
  "bugs": {
    "url": "https://github.com/TioSavich/UMEDCTA/issues"
  },
  "homepage": "https://github.com/TioSavich/UMEDCTA#readme",
  "dependencies": {
    "@vitejs/plugin-react": "^5.1.0",
    "concurrently": "^9.2.1",
    "cors": "^2.8.5",
    "dotenv": "^17.2.3",
    "express": "^5.1.0",
    "lucide-react": "^0.552.0",
    "react": "^19.2.0",
    "react-dom": "^19.2.0",
    "vite": "^7.1.12"
  }
}

\end{minted}
\newpage
\section{server.js}
\begin{minted}[breaklines, linenos, fontsize=\small, frame=single]{javascript}
import express from 'express';
import cors from 'cors';
import dotenv from 'dotenv';

dotenv.config();

const app = express();
const PORT = 3001;

// Middleware
app.use(cors());
app.use(express.json({ limit: '10mb' }));

// Proxy endpoint for Anthropic API
app.post('/api/anthropic', async (req, res) => {
  try {
    const apiKey = process.env.VITE_ANTHROPIC_API_KEY;

    if (!apiKey) {
      return res.status(500).json({
        error: 'API key not configured. Please create a .env file with VITE_ANTHROPIC_API_KEY'
      });
    }

    const response = await fetch('https://api.anthropic.com/v1/messages', {
      method: 'POST',
      headers: {
        'Content-Type': 'application/json',
        'x-api-key': apiKey,
        'anthropic-version': '2023-06-01'
      },
      body: JSON.stringify(req.body)
    });

    const data = await response.json();

    if (!response.ok) {
      return res.status(response.status).json(data);
    }

    res.json(data);
  } catch (error) {
    console.error('Server error:', error);
    res.status(500).json({
      error: 'Failed to process request',
      message: error.message
    });
  }
});

app.listen(PORT, () => {
  console.log(`🚀 Backend server running on http://localhost:${PORT}`);
  console.log(`📡 Ready to proxy requests to Anthropic API`);
});

\end{minted}
\newpage
\section{strategy\_game.py}
\begin{minted}[breaklines, linenos, fontsize=\small, frame=single]{python}
import sys
import os
import time
import random
import contextlib
import io
import pandas as pd

# Add the Python_Tests directory to sys.path so we can import the modules
current_dir = os.path.dirname(os.path.abspath(__file__))
strategies_dir = os.path.join(current_dir, 'Calculator', 'Python_Tests')
sys.path.append(strategies_dir)

# Context manager to suppress stdout during imports of scripts that run code on import
@contextlib.contextmanager
def suppress_stdout():
    s = io.StringIO()
    old_stdout = sys.stdout
    sys.stdout = s
    try:
        yield
    finally:
        sys.stdout = old_stdout

# Import the strategy modules safely
print("Loading Educational Modules...")
with suppress_stdout():
    try:
        import SAR_ADD_RMB
        import SAR_SUB_Sliding
        # We might need to copy the DPDA logic if counting_on_back is hard to import
        # But let's try importing it.
        import counting_on_back
    except ImportError as e:
        print(f"\nError loading modules: {e}")
        print("Make sure you are running this from the UMEDCTA root directory.")
        sys.exit(1)
    except Exception as e:
        # Some other error during execution of the scripts
        pass

print("Modules Loaded Successfully.")

class PedagogyQuest:
    def __init__(self):
        self.score = 0
        self.name = ""

    def clear_screen(self):
        os.system('cls' if os.name == 'nt' else 'clear')

    def type_text(self, text, speed=0.02, newline=True):
        for char in text:
            sys.stdout.write(char)
            sys.stdout.flush()
            time.sleep(speed)
        if newline:
            print()

    def get_input(self, prompt):
        print(f"\n{prompt}")
        return input("> ").strip()

    def start(self):
        self.clear_screen()
        self.type_text("Welcome to the UMEDCTA Pedagogical Simulator.")
        self.type_text("You are a Master Teacher training to diagnose and guide student thinking.")
        self.name = self.get_input("Enter your name, Professor:")
        
        while True:
            self.clear_screen()
            print(f"Professor {self.name} | Score: {self.score}")
            print("="*40)
            print("SELECT A MODULE:")
            print("1. The Robot Counter (Algorithmic Thinking)")
            print("2. Sarah's Addition (Rearranging to Make Bases)")
            print("3. Sam's Subtraction (Sliding/Constant Difference)")
            print("Q. Quit")
            
            choice = self.get_input("Choose a module:")
            
            if choice == '1':
                self.run_counting_level()
            elif choice == '2':
                self.run_rmb_level()
            elif choice == '3':
                self.run_sliding_level()
            elif choice.lower() == 'q':
                print("Class dismissed.")
                break
            else:
                print("Invalid selection.")
                time.sleep(1)

    # --- LEVEL 1: COUNTING (DPDA) ---
    def run_counting_level(self):
        self.clear_screen()
        self.type_text("MODULE 1: THE ROBOT COUNTER")
        self.type_text("A robot uses a stack of plates to count. H=Hundreds, T=Tens, U=Units.")
        self.type_text("It processes 'ticks' (count up) and 'tocks' (count down).")
        
        # Generate a problem
        start_val = random.randint(0, 20)
        ticks = random.randint(5, 15)
        direction = random.choice(['up', 'down'])
        
        # If down, make sure we don't go below zero for this simple level
        if direction == 'down' and ticks > start_val:
            start_val = ticks + random.randint(1, 10)
            
        self.type_text(f"\nScenario: The robot starts with {start_val}.")
        self.type_text(f"It receives {ticks} '{'tick' if direction == 'up' else 'tock'}' signals.")
        
        # Use the imported logic to get the real answer
        try:
            # The count_dpda function in counting_on_back.py takes (N, k, direction)
            # N is initial ticks, k is additional operations
            # So we simulate N=start_val, k=ticks
            correct_val = counting_on_back.count_dpda(start_val, ticks, direction)
            
            ans = self.get_input(f"What number will the robot display?")
            
            if ans.isdigit() and int(ans) == correct_val:
                self.type_text("Correct! The automaton state matches your prediction.")
                self.score += 10
            else:
                self.type_text(f"Incorrect. The robot displays {correct_val}.")
                self.type_text("Remember: The robot handles carries and borrows automatically via stack rules.")
                
        except Exception as e:
            self.type_text(f"Simulation Error: {e}")
            
        input("\nPress Enter to continue...")

    # --- LEVEL 2: ADDITION (RMB) ---
    def run_rmb_level(self):
        self.clear_screen()
        self.type_text("MODULE 2: REARRANGING TO MAKE BASES (RMB)")
        self.type_text("Student: Sarah. Strategy: Make 10 (or Base B).")
        
        base = 10
        A = random.randint(6, 9)
        B = random.randint(4, 9)
        # Ensure we actually cross a ten
        if A + B < 10: B = 10 - A + random.randint(1, 5)
        
        self.type_text(f"\nProblem: {A} + {B}")
        self.type_text(f"Sarah wants to keep the {A} and make it a {base}.")
        
        # Run simulation to get the "truth"
        rmb = SAR_ADD_RMB.RMBAutomatonIterative(A, B, Base=base)
        # We run it to populate history, but we want to step through it conceptually
        rmb.run()
        history = rmb.history
        
        # Step 1: Gap
        target_base = ((A // base) + 1) * base
        k_needed = target_base - A
        
        ans = self.get_input(f"How many does {A} need to become {target_base}?")
        if ans == str(k_needed):
            self.type_text("Correct. That is the Gap (K).")
            self.score += 5
        else:
            self.type_text(f"Not quite. {A} needs {k_needed} to reach {target_base}.")
            
        # Step 2: Decompose B
        b_rem = B - k_needed
        ans = self.get_input(f"If she takes {k_needed} from {B}, what is left?")
        if ans == str(b_rem):
            self.type_text("Correct. That is the Remainder.")
            self.score += 5
        else:
            self.type_text(f"No. {B} - {k_needed} = {b_rem}.")
            
        # Step 3: Result
        result = A + B
        self.type_text(f"So she has {target_base} + {b_rem}.")
        ans = self.get_input("Final Answer?")
        if ans == str(result):
            self.type_text("Excellent. You have successfully guided the student.")
            self.score += 10
        else:
            self.type_text(f"The answer is {result}.")
            
        input("\nPress Enter to continue...")

    # --- LEVEL 3: SUBTRACTION (SLIDING) ---
    def run_sliding_level(self):
        self.clear_screen()
        self.type_text("MODULE 3: SLIDING (CONSTANT DIFFERENCE)")
        self.type_text("Student: Sam. Strategy: Adjust both numbers to make subtraction easy.")
        
        M = random.randint(30, 90)
        S = random.randint(11, M - 10)
        # Ensure S is not a multiple of 10, to make it interesting
        if S % 10 == 0: S += random.randint(1, 9)
        if S > M: S = M - random.randint(1, 10) # Safety check
        
        self.type_text(f"\nProblem: {M} - {S}")
        self.type_text("Sam wants to slide the numbers so the subtrahend (bottom number) becomes a friendly base.")
        
        # Run simulation
        slider = SAR_SUB_Sliding.SlidingAutomaton(M, S)
        slider.run()
        
        # Step 1: Target
        target_s = ((S // 10) + 1) * 10
        k = target_s - S
        
        ans = self.get_input(f"What is the nearest higher multiple of 10 for {S}?")
        if ans == str(target_s):
            self.type_text("Correct.")
            self.score += 5
        else:
            self.type_text(f"Target is {target_s}.")
            
        # Step 2: Adjustment
        ans = self.get_input(f"How much do we add to both numbers (the slide)?")
        if ans == str(k):
            self.type_text("Correct. We slide up by {k}.")
            self.score += 5
        else:
            self.type_text(f"We need to add {k}.")
            
        # Step 3: New Problem
        new_m = M + k
        new_s = S + k
        self.type_text(f"New Problem: {new_m} - {new_s}")
        
        ans = self.get_input("Final Result?")
        if ans == str(new_m - new_s):
            self.type_text("Perfect. The distance remains constant.")
            self.score += 10
        else:
            self.type_text(f"Result is {new_m - new_s}.")
            
        input("\nPress Enter to continue...")

if __name__ == "__main__":
    game = PedagogyQuest()
    game.start()

\end{minted}
\newpage
\section{vite.config.js}
\begin{minted}[breaklines, linenos, fontsize=\small, frame=single]{javascript}
import { defineConfig } from 'vite';
import react from '@vitejs/plugin-react';

export default defineConfig({
  plugins: [react()],
  server: {
    port: 3000,
  },
});

\end{minted}
\newpage
\end{document}