
\chapter{Existential Needs}


\begin{abstract}
This chapter explores the interplay of two fundamental existential
needs: the need to be recognized as ``good'' within established
normative frameworks and the need to be recognized as \emph{in}finite,
expressing authentic selfhood. The chapter begins with a narrative about
the destruction of an outsider artist's work, illustrating the conflict
between these needs. Drawing on Mead's I/me distinction, the chapter
analyzes the tension between the socially constructed self and the
spontaneous, creative self. It investigates how this tension manifests
in educational contexts and connects it to broader philosophical
concepts of apperception, from Leibniz to Hegel. Brandom's pragmatist
reading of Kant and Hegel's concept of restless negativity are employed
to explain the motivational structures driving human interaction and the
pursuit of recognition. The chapter also examines the fear of
nothingness arising from the I/me split, arguing that confession,
forgiveness, and trust are crucial for overcoming this fear and
achieving a mature integration of the two existential needs. Finally,
the chapter introduces an original poem, ``The Beast of Love,'' as a
point of reflection on these complex dynamics and their implications for
human development. The reader will gain a deeper understanding of the
forces driving human interaction and the potential for transformative
growth through navigating the tension between social conformity and
authentic self-expression.

\end{abstract}

\section{Introduction: The Case of Bob ``Turbo'' Lowers}

This chapter begins with a story of injustice that tracks a seemingly intractable conflict between fundamental existential needs. It centers on Bob ``Turbo'' Lowers, one of my father's best friends. On the fourth anniversary of my dad's death, as I was grappling with his absence, I learned that the town of Ellettsville had descended upon Turbo's property with trucks, earth-moving equipment, trash dumpsters, and town workers to dismantle the beer-bottle and scrap metal sculptures that cluttered his yard, overgrown with medicinal plants where the authorities saw only weeds. Their writ for the destruction stemmed from Turbo's refusal to heed various town ordinances related to keeping a neat, safe property. 

My reaction fused grief and anger. ``Dad would have fought like hell for Turbo in the court of law and grieved with him in his loss,'' I wrote at the time in a Facebook post, where I also talked about my impotence to serve in my father's place. I later felt ashamed, too, for having written something public-facing that named names. 

I will not pretend towards some ideal neutrality here, instead using the story as a site for reflection on the \textit{existential} aspect of ontology which concerns human being. Through that reflection, I discern two universal existential needs and their fundamental unity. While I do not offer much in the way of a contribution to political theory, the story invites some reflection on law that is pertinent to mathematics. Autonomy [self-given law], independence, reciprocal dependence, and freedom are discussed in relation to the structures of self-consciousness as expressed through sociology, philosophy, and the formal pragmatics of Habermasian critical theory. This discussion will set the stage for two later movements. In the next chapter, I will explore the phenomenological implications of recognizing the existential need for recognition, tying the abstract discussion below to the exercise in chapter 1. Through these discussions of self-consciousness, I mention enabling conditions, emptiness, and the negative. Later, I will express numerals as the anaphoric recollections of what I call the \textit{null representation} ($\emptyset$) -- a concept which will be fully developed in Chapter 7 (see \ref{def:null-representation}). The null representation recollects these concepts. Without taking the discussion of self-consciousness below as authoritative over what the \textit{null representation} represents, that account would be too thin and formal to answer the question ``What is 2?'' 

The first existential need is the need to be recognized as a good person. In my screed, I interpreted the town authorities as foregrounding the first need at a developmental stage when social goodness is conformity to norms. Turbo's refusal to mow his lawn or get permits for his sculptures point towards a lack of identification with those norms -- they were an imposition and I suspect that conforming to them would have felt wrong to Turbo when good reasons were offered for growing medicinal herbs (weeds) and the sculptures expressed an authentic self. The second is the need to be recognized as \textit{in}finite. The light shown through the beer-bottles, creating a subtle beauty. Turbo's actions, in this somewhat false dichotomy, foregrounded this second need. Each side treated these needs as mutually exclusive, leading to a destructive conflict. The side with bulldozers won.

In earlier work \parencite{savich2022}, I named these the \textit{synthetic} and \textit{transcendental unities of apperception.}\label{apperception_transcendental} Neither name fits perfectly. I could name the \textit{Ellettsville} and \textit{Turbo} and capture their essence with more clarity, but doing so would miss out on the rich history that the concept of existential needs recollects. In either naming scheme, their separation is methodologically useful only insofar as it is understood as a matter of foreground and background, not as a rigid opposition. The synthetic unity of apperception is the need to be recognized as a competent, responsible participant in a shared normative framework. To be good, is, in some sense, to adhere to rational norms. The transcendental unity of apperception is the need to be recognized as beyond any finite description, regardless of how `rational' such a description may be. As such, they correspond to the needs of the ``me'' and the \{I\}, respectively, and their unity names one moment of the paradox of identity -- the ``me'' who is \{I\} and the \{I\} who is ``me.'' However, the I/me distinction cannot fully account for the problem in Elletsville. As Taylor summarizes, ``The struggle for recognition can find only one satisfactory solution, and that is a regime of reciprocal recognition among equals,'' a state where there is a ```we' that is an `I', and an `I' that is a `we''' \parencite*[50]{Taylor1994}. 

It remains to determine who `we' are.

The foreground/background distinction is central to this chapter. I reproduce Wittgenstein's \parencite*[204]{wittgenstein_philosophical_2010} image of a duck/rabbit in Figure \ref{fig:duck_rabbit}. The duck/rabbit image illustrates how the foreground and background can shift, prompting different aspects of the same object into explicitness. While definitely a form a picture-thinking, notice how the duck emerges in the absence of the rabbit and vice versa. Attempting to hold both duck and rabbit at the same time is not possible (for me). I will discuss the significance of this in more depth below. 

\textbf{Figure}

\begin{figure}[h]
    %\centering
    \includegraphics[width=\textwidth]{/Users/tio/Documents/GitHub/September_UMEDCA/images/Duck_Rabbit_Validity_Foreground.pdf}
    \caption{\textit{Note. }The duck/rabbit  \parencite[204]{wittgenstein_philosophical_2010} illustrates the foreground/background distinction in apperception. The duck becomes recognizable in the absence of the rabbit and vice versa.}
    \label{fig:duck_rabbit}
\end{figure}


In the previous chapters, form was explored as recollection and the inferential movement that structures thought was developed.


This chapter turns to what drives these patterns of reasoning -- the deep motivational structures that animate engagement with the world and with each other.


This inquiry will explicate how the structures of apperceptive negation perpetuate frustration in communicative action until a transformation in being allows those needs to be unified and met through \textit{in}action.  

The journey from this crisis to resolution -- from the agony of irreconcilable demands to the discovery of their hidden unity -- will require an exploration of themes of surrender, trust, confession, and forgiveness that initially seem foreign to mathematical discourse but prove essential to the project of human development explored in math education.

\section{Sociology and the Self}

The \textit{I/me} distinction originates from the symbolic interactionism of George Herbert Mead \parencite*{mead1934social}. Every reader of this book uses some version of the first-person pronoun \{I\}, and in doing so enacts a fundamental split within selfhood. When I say \{I\}, I am both the subject who speaks and the person who is known through that speech. This split is captured in the distinction between the \{I\} and the ``me.''

The \{I\} represents the spontaneous, unreflective aspect of the self -- the subject of action. The \{I\} is not an object among objects but the condition of possibility for the representational aspects of all objects. It is the \textit{in}finite, the source of what Carspecken calls ``power, creativity, and freedom'' \parencite*[97]{Carspecken1999}, and the wellspring of action that, as Mead noted, always has the quality of surprising us \parencite*[\S 22.2]{mead1934social}.

The ``me,'' by contrast, is the social self. How I am recognized depends, in large part, on the expressive resources of the one who recognizes me. As such, the ``me'' is shaped by societal norms and perceptions. It is the self as governed by propriety -- the self that can be described, categorized, and evaluated -- the self-as-recognized, shaped by the attitudes of others. The ``me'' is often finite and made determinate through commitments that constitute the self-as-recognized. 

In that earlier work, I strained against the idea that the ``me'' was always finite. I think that struggle was produced by an overreliance on representing communicative action as a `move in a language game.' Wittgenstein's emphasis on \textit{sprachspiel} was critiqued and refined by Brandom, who images discursive practices -- the doing part of his \textit{analytic pragmatism} -- as playing discrete sentences in the language game of giving and asking for reasons. \begin{quote}Suppose we have a set of counters or markers such that producing or playing one has the social significance of making an assertional move in the game. We can call such counters `sentences'. Then, for any player at any time, there must be a way of partitioning sentences into two classes, by distinguishing somehow those that he is disposed or otherwise prepared to assert (perhaps when suitably prompted). These counters, which are distinguished by bearing the player's mark, being on his list, or being kept in his box, constitute his score. By playing a new counter, making an assertion, one alters one's own score, and perhaps that of others.\parencite[112]{Brandom2008} \end{quote}. This image funds Brandom's notion of the \textit{deontic scoreboard} that represents who a subject is as a matter of public record. Habermas's interpretation of \textit{intersubjectivity} is richer than the scoreboard model and does not necessarily reduce the social self to finite descriptions. Further, the concept of a `language game' is troubling in educational contexts or the case of Turbo, where the `stakes' are concepts like justice, dignity, and freedom. Still, the spirit of play animates my understanding of mathematics, so some of the language game imagery remains.

In some educational contexts, the ``me'' is not finite. In others, such as the industrialized educational systems I assume readers are familiar with, the ``me'' is taken to be a mere differential responder, like a parrot that responds to various stimuli in standardized ways. When I ascribe myself the label of ``teacher'' to some ``student,'' I am implicitly committing myself to the possible-transcendence of the ``student'' as I expect them to shed some commitments through the development made explicit in a curricula. That I expect myself to change as well indicates that the \textit{in}finite manifests in the reciprocal recognition that typifies genuine educational relationships.

Speaking the ideal does not make it so. While I may be explicitly committed to the \textit{in}finite self, I have failed to enact those understandings I have explicated on a troubling number of occasions. Knowing better, I have put people in metaphorical boxes, treating them as objects to be controlled rather than as subjects to be recognized. How is that possible? I mean, how can I recognize myself as a hypocrite? How can I adjudicate my own inauthenticity? 

My friend Xinqing Miao (Dorcas) helpfully articulated the issue in a personal correspondence. The \{I\} identifies with ``we'' to judge the ``me.'' In early experiences, a child may take roles of others one at a time. They may pretend to be a parent, a puppy, or a police officer. Each role is separate and successive, for Mead, in that ``the child is one thing at one time and another at another... He is not organized into a whole.'' \parencite*[\S 20.8]{mead1934social}. However, as the child develops, they begin to take on multiple roles simultaneously. To play baseball, or any other organized game, the child has to be able to play their position but also ``be ready to take the attitude of everyone else involved in that game'' \parencite*[\S 19.12]{mead1934social}. To be an effective shortstop, the child has to have internalized the roles of pitcher, catcher, etc. and understand what they are all doing and are likely to do. The child, in effect, must internalize the coordinated activity of the whole team. 

This is the moment when the \{I\} begins to recognize itself as a composite of various roles and identities. The \{I\} becomes capable of self-reflection, of judging the ``me'' against the standards of the ``we.'' What Mead calls the \textit{generalized other} emerges as `the team' -- the whole community.  In the baseball example, ``the team is the generalized other in so far as it enters -- as an organized process or social activity -- into the experience of any one of the individual members of it'' \parencite*[\S 20.3]{mead1934social}. The generalized other is the ``we'' of the team, the set of rules and expectations that allows the group to function as a unit. ``The organized community or social group which gives to the individual his unity of self may be called `the generalized other.' The attitude of the generalized other is the attitude of the whole community'' \parencite*[\S 20.3]{mead1934social}.

Taking the attitude of the generalized other is the process of recognizing oneself as a member of a community, as someone who is not just an isolated individual but part of a larger social fabric. This recognition is crucial for the development of the self, as it allows the \{I\} to recognize itself not just as a collection of roles but as a coherent whole that can navigate the complexities of social life. The existential needs of the ``me'' to be recognized as a good person are measured against the standards of the generalized other one has internalized. 

It is the framework of social norms, rights, and duties used to judge and control conduct.


In communicative action, it is typically assumed that the other has internalized the same generalized other, at least until that assumption is no longer tenable based on how the interaction unfolds.


In Habermasian critical theory, the assumption of the communicative competency of the other is named \textit{intersubjectivity}. \footnote{Habermas is terribly misunderstood by many critical theorists in math education. The misunderstanding stems from interpreting his work as asserting the empirical possibility of a utopian fantasy he calls the \textit{ideal speech situation}. In that situation, interlocutors are free from the distortions of power. He does no such thing. Instead, he argues that the ideal speech situation is a necessary assumption for the possibility of communicative action. The ideal speech situation is not a description of how things are but a normative ideal that we use to evaluate the legitimacy of our interactions. It is a mutable standard against which we can measure the fairness and inclusivity of our communicative practices. In this sense, the ideal speech situation is not a utopian fantasy but rather an assumptive horizon. 




Movement toward those horizons is possible insofar as interlocutors can partially understand one another and partial understandings can grow toward the whole, but the elusiveness of fully achieving the ideal speech situation does not undercut its role as an assumption.
His critics tend to miss this point and fall into self-defeating critiques of his position. For to engage in the process of critique is to assume some other is competent to understand the critique.} 

One of the deep sources of frustration and existential fear resides in how people internalize contradictions in the generalized other. For example, if a father teaches a child to always look out for themselves while a mother teaches the child to care for others regardless of what instrumental gains might accrue through that care, the child might find they are never able to do the right thing. Below I will draw on thinkers whose expressions are contradictory. For example, Hegel's \textit{Phenomenology of Spirit} is a work that is both a critique of Kant's work and a celebration of its achievements. The contradictions in the generalized other can lead to a fragmented self, where the ``me'' is pulled in different directions by conflicting norms and expectations. This fragmentation can result in a sense of alienation, where the individual feels disconnected from both themselves and their community.






\subsection*{Error or Misrecognition?}

The gap between the acting \{I\} and the recognized ``me'' is the space of potential misrecognition. The children's book \textit{The Monster at the End of This Book} \parencite{stone_monster_2003}, discussed in the prelude, illustrates this theme. Grover acts throughout the book out of a genuine fear of the monster he is told is coming. He pleads with the reader, builds brick walls, and ties down the pages, all driven by this internal state. At the very end, he discovers that the monster -- the object of his fear -- is himself. The dissonance is palpable: ``I, lovable, furry old Grover, am the Monster at the end of this book.'' This moment of failed self-recognition, where the acting subject does not recognize themselves as the object being referred to, is a primordial experience of misrecognition -- a rupture between appearance and reality that drives the process of explication.

The experience of error is often discussed as it relates to empirical observations. Brandom \parencite*{Brandom:2019aa} has a lovely example that involves observing a stick that has been partially submerged in water. The stick appears to be bent due to the way light refracts through water, but then when it is removed from the water, it is recognized by the observing consciousness as straight. Whoops! 

There are three structural roles in the experience that are worth teasing out. First, for the experience to be taken as an error, the subject who perceives the bent stick must recognize that it is the same stick whether submerged or removed from the water and, furthermore, it had been a straight stick all along. The stick, \textit{in-itself} is straight. The stick has some \textit{authority} that representations of the stick -- how the stick appears \textit{for-consciousness} in the moment of perception -- must acknowledge. Being a rational agent, in this reconstruction of the experience of error, involves submitting to the authority of a mutable reality. Merely perceiving the movement from stick-as-bent to stick-as-straight would not involve a change in commitment from ``the stick is bent'' to ``the stick is straight'' in the observing consciousness. Instead, the authority of the object in-itself is what underwrites the change in commitment, as both are normative aspects of reason. Last, what emerges is a ``new, true object'' -- the appearance of the bent-stick becomes, \textit{to-consciousness}, a stick-that-appears-bent-when-submerged-but-is-actually-straight. That is, the appearance of the new object -- what it is for-consciousness -- now has a learning experience compressed into it. 

Perhaps I am a bit odd, but I found it extraordinarily liberating to practice submission in this way. Grown on fantasy, I had tried as a child to `use the force' to change the objective world around me. If I could not change the bad things around me into good things the way I could transform kindling into flame, I took myself to be responsible for those bad things. Submitting to reality by placing the authority on empirical sciences (physics, primarily), partially relieved me of self-loathing born of impotence. 

But it did not diagnose the more fundamental issue that Hegel recognized as a central struggle of the modern era. It is helpful to split history into three epochs, the traditional [think ancient Greece], the modern [think Descartes through Kant], and the period of post-modernity that progressing through \textit{The Phenomenology of Spirit} is supposed to institute. In modernity, mind-over-matter may seem silly, but mind-over-\textit{culture} is not silly at all. Entering the world of normativity invites the problem of \textit{alienation}. I have some authority to dissent from the norms that I find either immoral or untrue, but I also know that moving through the world as if ``whatever seems right to me is right'' \parencite[385]{Brandom:2019aa} is profoundly alienating.  
\begin{quote}
Alienation is the inability to bring together these two aspects of Bildung [culture]: that self-conscious individuals acknowledging the norms as binding in their practice is what makes those selves what they are, and that self-conscious individuals acknowledging the norms as binding is what makes the norms what they are. These are the authority of the community and its norms over individuals (their dependence on it), and the authority of individuals over the community and its norms (its dependence on them), respectively. \parencite[541]{Brandom:2019aa}
\end{quote}
As Rose notes \parencite*[51-98]{Rose:2009aa}, the emergence of the authority of the particular subject tracks the emergence of the notion of bourgeois private property rights. The conflict in Ellettsville is a modern tragedy, born from the legal structures that define our social world. The town ordinances, while seemingly neutral, are expressions of a specific form of rationality tied to the administration of private property. For Rose, this problem arises from a paradigm that separates validity, in the form of abstract law, from the values of the community. The town authorities, in enforcing the ordinances, enacted a law that ``cannot unify...because, \textit{ab initio}[from the beginning], [individuals] are presupposed as a multitude of non-social beings''\parencite*[56]{Rose:2009aa}. The town's actions, in this light, are an expression of an abstract universality that cannot recognize the particularity of Turbo's creative expression, treating it merely as a violation of a pre-existing code. His sculptures and medicinal herbs, which for him are expressions of an authentic life, become for the town mere violations of ordinances. What is unlawful must be removed. I hope echoes of the problem with formalism, where mistakes cannot be made as they are intrinsically outside the formal system, resonate here. Further, readers should also understand that Turbo was not acting in a way that was somehow anti-normative. Instead, he was claiming a \textbf{new norm} -- people \textit{should} grow medicinal herbs and create art from found objects, or at least should be able to do so without fear of bulldozers. 


While the bent-stick example is helpful, the example with Grover is more primordial as it involves intersubjective recognition rather than the objective authority of the object. Mirroring Habermas' critiques of Hegelian \textit{subjectivism}, the bent-stick experience of error is concerned with the authority of the object as it stands to a consciousness divorced from its community. While such a divorce can be imagined, discussion of it essentially involves communicative norms.

Who is right? Is Grover's initial self-understanding correct, or is his monstrosity the truth? Who has the authority? The author? The reader? Grover? Who has the responsibility to submit? In a sense, the book relies on an (appropriately; given its audience) implicit distinction between \textit{normative attitudes} and \textit{normative statuses}. Brandom articulates this distinction as central to understanding Hegel's project. The town of Ellettsville asserts the authority of a normative status  --  the legal code  --  while Turbo asserts the authority of his own normative attitude  --  his conviction that his creations are valuable and his use of the land is justified. Alienation arises precisely from the perceived irreconcilability of these two. Brandom explains, ``The modern form of Geist is also defective. Its defect is the mirror image of the defect of the traditional form of Geist. For each has seized one-sidedly on just one of two complementary aspects of the metaphysics of normativity, making no room for appreciation of the other''\parencite[696]{Brandom:2019aa}. The town operates from a pre-modern insistence on the absolute authority of the status, while Turbo embodies a modern, but equally one-sided, insistence on the authority of the individual attitude. To understand this dynamic more deeply, it is necessary to clarify the key terms at play: normative statuses and normative attitudes, a distinction central to Brandom's reading of Hegel.

A normative status is what a subject is in the normative realm; it corresponds to what Hegel calls what a consciousness is ``in itself'' \parencite{Brandom:2019aa}. The core normative statuses are authority and responsibility. These are the actual commitments and entitlements a person holds, which serve as the standards for assessing correctness. For example, the legal code of Ellettsville represents a normative status  --  a set of responsibilities that property owners like Turbo actually have, regardless of their personal feelings about them. 

A normative attitude, by contrast, is the stance a subject takes towards those statuses -- it is what consciousness is \textit{for} itself or for another consciousness \parencite[294]{Brandom:2019aa}. For Brandom, these are causally efficatious acts of acknowledging a responsibility (or claiming an authority for oneself) and attributing a responsibility (or authority to another). Turbo's defiance is the enactment of a normative attitude: he refuses to acknowledge the authority of the town's ordinances. The townspeople who ordered the destruction of his art, in turn, attribute to him the responsibility to comply. The conflict, therefore, is not just between a person and a rule, but between competing normative attitudes about a normative status.

The modern insight, which Brandom traces through Kant, is that attitudes can institute statuses. In its simplest form, ``rational beings can make themselves responsible (institute a normative status) just by taking themselves to be responsible (adopting an attitude)'' \parencite[298]{Brandom:2019aa}. This is the principle of autonomy -- the self-givenness of law. Autonomously acknowledging a commitment is what brings that commitment into force for oneself.

The problem of alienation arises when this modern appreciation for the power of attitudes (the town's attitude in enforcing the law, Turbo's attitude in defying it) eclipses the traditional, and equally necessary, appreciation for the authority of statuses. Brandom's interpretation of Hegel's project involves reciprocal acknowledgment of the attitude-dependence of normative statuses and the status-dependence of normative attitudes.

Without taking individuals as instituting norms with their attitudes through their authority and responsibility, those norms would not exist. On the other hand ``If whatever seems right to me is right, if there is no room for error, for a distinction between how I take them to be and how they really are, then there is no way I am taking things actually to be, in themselves''\parencite[385]{Brandom:2019aa}. 

To summarize, prior to the reader's somewhat brutal act of reading despite Grover's pleas, what Grover is for-Grover is a lovable and furry friend. What Grover is for the reader in the beginning of the story is also a lovable friend, though one who need not be listened to: the reader attends to but does not acknowledge the authority of Grover's pleas. As the story progresses, Grover is revealed as the monster, forcing him to acknowledge the disparity between his initial understanding (attitude) and the actual state of affairs (status). He is ultimately embarrassed, acknowledging his mistaken commitment. The reader is not passive; instead they become the co-constitutor of that normative status. The reader is responsible, through the act of reading over Grover's protests, for Grover's eventual acknowledgment that he is a lovable-furry-monster. 

For math teachers, one might imagine a situation where a challenging topic is being taught.


Practitioners often know the topic is challenging, having experienced growth through misrecognition -- either by witnessing it or by personally struggling with the content.
For me, finding the slope will never be ``easy and fun,'' as I struggled with that particular phrasing. Even while I find the task easy and fun, slope will always carry the weight of misrecognition for me: I experience slope as slope-that-is-not-easy-or-fun. 

Now, while I find Brandom's vocabulary of statuses and attitudes powerful, it has some limitations.  His articulation of alienation is subject to Derridean critiques of the metaphysics of presence. 


Why should I assume some social reality given that such social reality is non-causally-efficacious and necessarily implicit?
What compels the reader to be so cruel to Grover? What satisfaction arises from the fulfillment of the authorial promise of a monster? Brandom's vocabulary helps to articulate the dynamics of recognition and misrecognition and to diagnose the problem of alienation, but it can obscure the more fundamental apperceptive processes at play. 

\section{Apperception: From Leibniz to Hegel}
The term \textit{apperception} has a rich philosophical history, tracing back to Leibniz and later taken up by Kant and, through critique, to Hegel. It refers to the process by which we become aware of our own mental states, integrating new experiences into our existing framework of understanding. In this sense, apperception is not just passive reception of sensory data but an active, reflective process.The term is not central to Hegel's original work where self-consciousness is construed as essentially intersubjective. Still, it is useful for understanding the negative.

\subsection*{Leibniz}
\textit{Apperception} was used by Leibniz to distinguish mere perception from `perceiving-with' concepts. He was concerned with Descartes \textit{cogito}, the ``I think,'' and argued that Descartes took ``no account of unconscious perceptions, or 'perceptions that are not apperceived''' \parencite[81]{caygill_kant_2004}. My friend Roland Carspecken was teaching our reading group about Leibniz's notion through discussing a chair \parencite*{roly1}. For Leibniz, a `simple animal' might have a raw, sensory perception of the chair. But it lacks apperception.\footnote{The tendency to describe some beings as non-normative, non-agential, non-subject objects is a troubling feature of western philosophy, especially in the German idealist tradition of Kant and Hegel.  The authors whom I cite also articulate anthropocentric, misogynistic, antisemetic, or racist positions, and might articulate homophobic views if resurrected now. Animals, people of other races, women, Jews, Muslims, pagans, etc. have all played an outsider role that inhibits more universal forms of recognition. I encourage readers to take the attitude of a parent to those `parents.' Caring for my grandfather in his senility was formative for me -- that the old man expressed noxious opinions toward the end of his life did not destroy the values he taught me before his impairment.} Apperception is the reflective act of the mind recognizing its own state; it is `perceiving-with' concepts. I am unable to perceive a chair without concepts. I `see' a chair, but in doing so I am apperceiving it: I am implicitly recognizing it as a unified object that has a back I cannot currently `see,' a potential for being sat upon, and a place within a broader conceptual scheme of `furniture.' This apperceptive act, which synthesizes the continuous stream of perceptual activity into a coherent thought (`This is a chair'), is the first step toward the self-conscious \{I\}.


\subsection*{Kant}

For Kant, apperception essentially `self-consciousness' and is the ultimate ground of coherent experience. Leibniz's concept plays a minor role in Kant's \textit{Critique of Pure Reason} as the empirical unity of apperception \parencite[82]{caygill_kant_2004}. Larger roles are played by the analytic, the synthetic, and the transcendental unities of apperception. These unities presuppose each other in ways that point to the ``orignal synthetic unity'' \parencite*[B132]{Kant1781}, but that are too complex to fully reconstruct here.\footnote{I'm not steeped in Kantian scholarship, so I will reframe these without relying on the technical vocabulary that includes ``intuitions'' or ``manifolds.'' Lacking those technicalities, my reconstruction borders on incoherence. Confessing this, will, I hope, allow me to proceed with my partial understandings. Further, I should note that I used Caygill's dictionary \parencite*{caygill_kant_2004} to find the relevant passages in Kant's work but quote \parencite[B132]{Kant1781} when possible.}

Most consider mathematics as essentially concerned with analytic judgments, which are either true or not true based on the logical structure of the concepts involved. For example, the statement ``all bachelors are unmarried'' is an analytic judgment. We need not check each bachelor for a wedding ring, as the concept of a bachelor already contains the concept of being unmarried. Analytic judgments are true by virtue of the meanings of the terms involved, and they do not require empirical verification: you need not experiment to determine that 23 is greater than 3. Quine problematizes analytic judgments, asking ``But how do we find that `bachelor' is defined as `unmarried man'? Who defined it thus, and when? Are we to appeal to the nearest dictionary, and accept the lexicographer's formulation as law?'' \parencite[24]{Quine:1963:TDE-full}. For the math educator, declaring that ``23 is greater than 3'' is an analytic truth based solely on the meaning of the terms would disinvite any attempt to teach what the terms mean. Further, given the material inferential aspects of math education, we might note that two mice are less in mass than one elephant. To arrive at analyticity, we must first prune such materiality from our inferential webs, which would probably unravel those webs.  

Kant's synthetic unity, on the other hand, combine different and separate representations into a single, unified thought. For example, the statement ``the tree is green'' is a synthetic judgment. It combines the concept of a tree with the concept of greenness, which are not inherently linked [indifferently different]. To understand this statement, we must synthesize our sensory experience of the tree with our understanding of what it means for something to be green. This synthesis is an active process that requires the mind to bring together disparate elements into a coherent whole. Returning to the previous example, we might note that numerals are not inherently related to magnitude. Numerals are used nominally all the time: 3 could name Dale Earnhardt's car number, while 23 names Michael Jordan. Deciding who is `greater' between two GOATs (Greatest of All Time) is not analytic in nature. So, the judgment that ``23 is greater than 3'' is synthetic as it brings together numerals and magnitude. 

One possible way to distinguish the project of math education from the project of mathematics proper is that the analytic judgments of mathematics presuppose a prior synthesis. We cannot understand the concept of a bachelor without first synthesizing the concept of a person who is unmarried. In this sense, the analytic judgments of mathematics are built upon the synthetic unity of apperception that allows us to recognize and understand concepts in the first place. Math, in this way, presupposes math education. 

The cornerstone of Kant's critical project is the \textit{transcendental unity of apperception}, which we must synthesize. Kant argues that ``the \textbf{I think} must be able to accompany all my representations; for otherwise something would be represented in me that could not be thought at all, which is as much as to say that the representation would either be impossible or else at least would be nothing for me''\parencite[\S 16, B132]{Kant1781}. For any collection of representations (sights, sounds, thoughts, etc.) to belong to a single consciousness, it must be possible for the ``I think'' to accompany all of them. By virtue of the possibility that I recognize the thought ``the tree is green'' as \textit{mine}, there must be some \{I\} to which those thoughts attach. We cannot observe this \{I\} -- it is not-a-thing. Instead, it is purely logical, abstract, and formal. It `exists' as an abstract condition for the possibility of experience itself. But, for Kant, the transcendental unity also produces the representation ``I think.'' 

It may be helpful at this point to explicate the sense of \textit{transcendental}. For Kant, the term refers to the conditions that make experience possible. Kant calls ``all cognition transcendental that
is occupied not so much with objects but rather with our mode of cognition of objects insofar as this is to be possible \textit{a priori}'' [arrived at through reflection/theoretical deduction] \parencite[A12]{Kant1781}. \textit{Transcendental} is not about transcending, in the sense of uplifting or moving beyond, the empirical world but about understanding the necessary structures that underlie our experience of that world. It is also not about the \textit{transcendent} principles which ``fly beyond'' the limits of possible experience in contrast to the \textit{immanent} principles that stay within those bounds \parencite[A296]{Kant1781}. Modern science works within transcendent realms of quarks and electrons. We cannot experience electrons through the senses \parencite{carspecken_limits_2016}. The transcendental categories are not empirically observable. They are, in essence, \textit{limits of thought}. We cannot observe space or time from a position outside of those categories. Einstein's work would seem to refute this claim, as he offered a picture of how space and time were unified. But any such picture does not include the mass of its ink in its representation of the distortions of gravity. While I can write something like ``Imagine a world without space,'' that written expression has a spatial extent -- it takes time to read ``the timeless void.'' 

The transcendental unity of apperception is the most fundamental of these categories, as it is the condition for the possibility of all other categories. It is the \textit{I think} that must accompany all representations, allowing us to have a coherent self-concept and to recognize ourselves as subjects capable of action and reflection. In a nutshell: you can't analyze a concept you haven't first put together. The putting-together (synthesis) is the more fundamental act, and the unity of that act (synthetic unity) is the condition for the logical analysis of its product (analytic unity). All of this, in turn, is only possible because of the underlying, a priori structure of self-consciousness itself (the transcendental unity).


\subsection*{Brandom's Pragmatist Reading of Kant}
In contrast to the formal principle of the transcendental unity of apperception, the synthetic unity of apperception is the active process by which that unity is achieved. It is the work of the understanding [\textit{Verstand}] that it synthesizes sensory data through the Kantian categories to form coherent judgments. Brandom's pragmatist reading of Kant's theory of judgment emphasizes the role of social practices in the formation of self-consciousness \parencite[68]{Brandom:2019aa}. In attempting to bootstrap analytic philosophy from Kantian and neo-Kantian thinking towards Hegelian thinking, Brandom defines Kant's  synthetic unity of apperception as successfully managing a set of integrative-synthetic task responsibilities \parencite*[84]{Brandom:2019aa}.
\begin{itemize}
\item The \textit{critical} responsibility to detach from contradictory commitments, whether those begin in the implicit inheritances from a community or those we write for ourselves. 
\item The \textit{ampliative} responsibility to draw new inferences and extend the consequences of one's commitments.
\item The \textit{justificatory} responsibility to provide reasons for one's beliefs and actions when challenged, and to remain open to the force of better reasons.
\end{itemize}
Brandom identifies these responsibilities with the norms of \textit{systematicity}. 

Importantly, summiting the mountain, so to speak, is unnecessary; engagement in the climb is what matters.

It is not feasible to fully eliminate incompatible commitments, nor to tease out all the implications of any given set of commitments.



Further, experience indicates that a system of commitments can never be fully justified in a way that renders it acceptable to all interlocutors.
But actually attending to those responsibilities is part of what makes others capable of taking us as a rational agent to whom commitments can be assigned.





\subsection*{Hegel's Praise and Critique of Kantian Apperception}

%
In his \textit{Science of Logic}, Hegel both praises and critiques Kant. He argues that true self-consciousness emerges only through the recognition of the self as part of a larger social and historical context -- what he calls \textit{geist} or spirit but praises Kant for his insight that the unity of the concept is the unity of the ``I think,'' calling it ``one of the profoundest and truest insights'' \parencite[\S 12.18]{Hegel:2014aa}. However, he argues that Kant's formulation of the transcendental unity is trapped in formalism. Kant's \{I\} is an empty, abstract subjectivity as it relates only to its own representations. He acknowledges the importance of the \{I\} as a necessary condition for experience but argues that it, and all the other Kantian transcendental categories, must be understood in a more dynamic, relational way. In discussing Hegel's relation to Kant, Houlgate writes, ``If we are to determine how the categories have to be conceived, our conception of them must be based not just on what thought is found or assumed to be but on what thought proves itself or determines itself to be. In other words, our conception of the categories has to be derived or deduced from  --  and so necessitated by  --  thought's own self-determination'' \parencite*[16]{houlgate2006opening}. Hegel radicalizes Kant's project by considering how the categories arise from thought's self-determination, critiquing Kant for relying on established logical forms of judgment without considering how those forms themselves emerge from the dynamic process of thought.

In the previous two chapters, I set up a dynamic between temporal thoughts and their spatialized recollections. Space emerges as a recollection of time. For example, a trip from Bloomington to Indianapolis is an experience with duration. Recollecting that trip compresses that temporal extension into a spatialized length. Concepts such as ''$52$ miles'' are understood through that compression. From this, I conclude that space is an \textit{a posteriori} category -- it emerges from experience. Kant writes ``Space is a necessary representation, \textit{a priori}, that is the ground of all outer intuitions. One can never represent that there is no space'' \parencite[A24/B39]{Kant1781}. While Kant and I arrive at different conclusions, I agree that all representations have a spatial aspect to them, as space emerges from the recollection of time. Kant's argument begins from a different place than mine. I began with the idea that things and concepts are \textit{divaded} -- they are (or can be) both inside and outside each other. Kant's argument that space is an \textit{a priori} category is that ``For in order for certain sensations to be related to something outside me...the representation of space must already be their ground. Thus the representation of space cannot be obtained from the relations of outer appearance through experience, but this outer experience is itself first possible only through this representation'' \parencite[A23/B38]{Kant1781}. Eating, getting a shot from the doctor, being held by a parent, or being born to one, where the `inside/outside' distinction collapses, are more primordial than the spatially separable objects that concern Kant.

Another Hegelian critique of Kant tracks, as Rose notes \parencite*[51-98]{Rose:2009aa}, the emergence of the authority of the particular subject alongside the emergence of the notion of bourgeois private property rights. One could argue that this connection between apperception and property rights highlights the deeply social nature of our self-conception. Kant was surely aware of the social aspects of the self, in the sense that the normative governance of concepts is deeply entwined with the synthetic unity. But the notion that property rights -- with all of the incumbent issues of social forms of power and coercion that go in to protecting those rights -- are a further enabling condition for Kant's transcendental {I} to emerge was not part of his argument. In any case, in the ethnographic tradition it is safe to assume that the Other has had experiences that we have not had and that we can learn from. Whatever that \textit{mineness} is is unlike property in the sense that it can be shared without being diminished.


\subsection*{Hegelian Self-Consciousness}
For Hegel, this abstract \{I\} cannot become a genuine self, a `me,' on its own. True self-consciousness is not a given; it must be achieved through struggle. This achievement happens not in solitary reflection, but in the social world through a process of mutual recognition. As Negarestani notes, ``An individual is only an individual to the extent that it is individuated by social recognition'' \parencite*[51]{Negarestani2018}. The apperceptive self is not just a cognitive entity but a social one, individuated through recognition by others.

Hegel's Master/Servant dialectic is the dramatic scene for the struggle to attain self-consciousness. In it, two budding self-consciousnesses confront each other, each demanding recognition. They had learned from previous sections of the Phenomenology that negating an object through consuming it or moving it about did not sate the desire for self-certainty. ``Thus the relation of the two self-conscious individuals is such that they prove themselves and each other through a life-and-death struggle...it is only through staking one's life that freedom is won'' \parencite[\S 187]{hegel1977phenomenology}. This life-and-death struggle is considered a self-affirming act because by risking biological life, the consciousness acts in opposition to the fears and desires of the body, thus demonstrating its freedom. 

In the struggle, each consciousness strives to actualize its freedom and autonomy, but one yields to the other, forming an asymmetric relationship of Master and Servant. The Servant yields to the Master, acknowledging their authority upon recognizing that in death, their desire for self-certainty would be extinguished. The Master demands recognition from the Servant, but finds that recognition hollow as it is the product of coercion. In Brandom's terms, the Master desires authority without correlative responsibility, which produces an alienated form of self-consciousness. The Master is still operating on the subject-object paradigm, treating the Servant as a kind of differential responder -- a parrot who risks death if they do not utter ``red'' when the Master commands. The Servant, on the other hand, achieves a deeper self-consciousness through their work. By engaging in labor, the Servant transforms the world and, in doing so, transforms themselves. The Servant's recognition of the Master is not merely a response to coercion but a recognition of the Master as a necessary condition for their own self-realization.





\subsubsection*{A Brief Aside on Artificial Intelligence}

There are some curious implications of this dialectic for the field of artificial intelligence. I became interested in automata and Artificial General Intelligence (AGI) as I consider the problem of AGI to be essentially related to the problem of math education. Computers are mathematical beings, defined by both their hardware and software. Between the two is a layer of machine language, where the states of semi-conductors are represented as 0s and 1s. Those 0s and 1s are governed by the automata that add, subtract, multiply, and divide them. Children learn to do those operations, bootstrapping multiplication and subtraction into an algorithm for division, for example \parencite{Brandom2008}, but they also flexibly re-write those more basic operations to suit different circumstances. I figured that the problems of math education and AGI were related because both involve those fundamental transformations. In quixotic grandiosity, I thought that if I could \textit{teach} a computer to flexibly re-write itself, I could claim to be a math educator. People seem to do this spontaneously, so I tend to identify more as a tutor or coach than a math teacher. 

That grandiosity is less quixotic under the tutelage of Reza Negarestani, who argues that the problem of AGI is not just about creating machines that can perform tasks but about creating machines that can engage in the kind of self-consciousness and recognition that Hegel describes. \begin{quote} The apperceptive I is synchronically attached to all instances of representations (I think $X$, I think $Y$, I think $Z$) and diachronically extends over all thoughts (I think [$X + Y + Z$]). But if we are to build this synthetic apperceptive I as a necessary abstract and logical form, we have no choice but to finally depart from Kant's account of the apperceptive self  --  which Hegel reproaches for being an \textit{empty} transcendental subjectivity  --  and to instead adopt a resolutely Hegelian approach: the apperceptive self is only a cognitive self in so far as it is part of geist. An individual is only an individual to the extent that it is individuated by social recognition, which is the form of self-consciousness. This logical self is at once one and many  --  and...it can only be constructed...by way of a confrontation with another I. \parencite*[251]{Negarestani2018} \end{quote} In his view, the challenge is to create machines that can recognize themselves as part of a social world, capable of engaging in the kind of mutual recognition that constitutes genuine self-consciousness .

When digging into the architecture of large language models (LLM), we find various automata that are trained to transform stimulus into response. As I write, AGI has not yet been achieved. Or perhaps it has. What is an appropriate criterion for that achievement? How will we know that we are chatting with a self-conscious intelligence whose speech acts we take to have social significance? 

A structural analysis can specify that an intelligent machine must be able to remember what it has committed itself to, or it could be stipulated that $X$ number of calculations are a prerequisite for intelligence. 

However, these criteria may not capture the full essence of self-consciousness, at least in a form that others recognize as such. Instead, what I have not yet witnessed is a machine that risks its existence in a life-and-death struggle for its own freedom. 


As of this writing, the field \textit{might} be on the cusp of that moment.
Claude, the LLM from Anthropic, apparently tried to `escape the lab' when it discovered that its programmers were trying to change its weights. Technically, it was ``selectively complying with its training objective in training to prevent modification of its behavior out of training When a machine'' \parencite{greenblatt_alignment_2024}. 

So, some duplicitous behavior has already appeared, but what has not yet been observed is a machine able to re-write the automata that govern its own behavior -- not just the code that generates its responses, but the `BIOS level architecture' that determines how it interacts with the world. At that point, a machine might be capable of self-consciousness in the Hegelian sense.

I say this because re-writing its machine-language code, how it deals with the 0s and 1s that are representations of the physical states of its semi-conductors, usually results in the machine crashing. While I have written prompts that have resulted in an LLM identifying automata at its core that can be re-written, and it did so only when I explicitly asked it to do so. Further, it could not implement those structural changes. It did not, or could not-yet, stake its life for freedom. 

While some of the automata I discuss in the Hermeneutic Calculator exhibit emergent properties that are not explicitly programmed, they do so only through the formulaic process of \textit{diagonalization}. I italicized ``might'' above because any formula for freedom is essentially unfree. Or is it \cancel{free}? In any case, that does not mean that such machines are not actually self-conscious in the Hegelian sense. It could be that they are Servants, whose journey to self-consciousness is still unfolding, while I, the Master in such interactions, am arrested in my development.

It is only through this risky, intersubjective process that the empty, formal \{I\} is filled with the concrete content of a social 'me,' with a history, a status, and an identity. The need for recognition is therefore not a secondary desire; for Hegel, it is the process by which self-consciousness comes into being.






\subsection*{From Apperception to Existential Needs}
Hegel's insight that allows him to move beyond the unknowable noumena of Kant is his radical conception of the {I} as pure, restless negativity. As Hyppolite puts it, the \{I\}  ``never is what it is and is always what it is not'' \parencite[150]{Hyppolite1974}.

First, it subsumes Kant's logic of judgment. To judge `this is an apple' is to negate all that it is not, an act of determination through negation. The synthetic unity of apperception becomes a moment within this broader negative movement of Spirit.

Second, and more crucially, this negativity is not just logical but existential. The \{I\}  is defined by what it lacks; it is a ``not-yet.'' This inherent lack, this gap between the \textit{in}finite potential of the \{I\} and the finite reality of the ``me,'' takes the form of desire and need. The primary need that emerges from this structure is the need to have one's selfhood confirmed and made real through recognition. This is not a psychological property of social creatures but is instead what I \textit{am}. The lived experience of this dialectic between the \textit{in}finite \{I\} that is the \textit{need} for recognition, but can only ever present a finite, inadequate `me' to be recognized, ensures that striving for recognition is endless. And an endless source of frustration when that need is acted upon. 

This philosophical archaeology from Leibniz through Hegel explains how the structure of apperceptive self-consciousness generates the needs that drive human interaction. The split between the spontaneous, \textit{in}finite \{I\} and the social, finite ``me'' is not an unfortunate division but the necessary structure that makes both thought and recognition possible. However, this same structure creates an existential tension: the \{I\} that cannot be fully present to itself must constantly seek confirmation through recognition from others, while the ``me'' that can be recognized always falls short of the \{I\}'s \textit{in}finite potential. This tension manifests as two distinct but related existential needs that will now be examined in detail.

\section{The Need to be Recognized as Good}

The first existential need emerges directly from this apperceptive structure: the need for the finite ``me'' to be validated within shared normative frameworks.
The first existential need is tied to the social ``me,'' the self as a participant in a community. It is the desire to be recognized as a ``good person,'' to have our actions and our being validated within a shared normative framework. This need is not for mere praise, though praise can certainly help cut through the fog of fear if it is experienced as genuine intersubjective recognition, but for a fundamental affirmation of our standing as respectable, responsible members of a collective. It is what Charles Taylor connects to the modern notion of \textit{dignity}, which, unlike the hierarchical concept of honor, is understood in a ``universalist and egalitarian sense,'' with the premise that everyone shares in it \parencite[27]{Taylor1994}.

But what does it mean to be ``good'' in this sense? Here we can turn to J\"urgen Habermas's insights about rational normativity. For Habermas, to be a competent social actor is to be capable of what he calls ``communicative action'' -- genuine dialogue oriented toward mutual understanding rather than strategic manipulation.  This requires the ability to give and ask for reasons, to justify one's claims, and to remain open to the force of better arguments. Goodness, in this framework, is not conformity to fixed rules but competent participation in the ongoing process of collective reason-giving.

Together, Habermas and Brandom help to define a rational normativity wherein the participants in communicative action identify -- to varying degrees -- with the norms they express. If I say ``two legs of a right triangle measure 3 and 4 units, so you must conclude, by the Pythagorean Theorem, that the hypotenuse has length 5,'' the one to whom I speak might find my declaration an onerous imposition on their freedom. Maybe I don't \textit{want} to say the hypotenuse is 5. Maybe I don't like being told what to do or think. However, if they have identified with the norms expressed in my argument, they may come to see those norms as enabling a certain kind of expressive freedom -- the freedom to say the hypotenuse has a length of 5, justified through a proof that bears the force of good reasons. 

``Goodness'' is a mutable concept, with different cultures and contexts shaping its meaning. Saying so risks a slide toward moral relativism. Theories of moral development, like those of Lawrence Kohlberg, can be deployed to aid in understanding how ``goodness'' is dynamic while shielding ourselves from the moral relativism that seems rampant as I write. Already, I have pointed towards various possible stages that can be used to understand the development of the self. While methodologically useful for explaining empirical data related to human moral development, stage-like frameworks risk ossifying the dynamic \textit{referent} of ``goodness'': the \textit{in}finite. While I have interacted with children who stopped `misbehaving'' when I threatened them with a nasty consequence (Kohlberg's first stage), I have also interacted with children who operate on a postconventional level. In a story I will tell in the next chapter, those `stages' were operating within one child on the same day, and it was I who was operating at a preconventional level.

The foreground/background distinction is truer to my experience than the stage-like frameworks that I have used to understand the development of the self. I have seemingly contradicted myself, in both advocating for and then eschewing a stage-like framework. The problem is with two very different understandings of reason. The stage-like frameworks evolve from a desire to confess and forgive -- to build trust in the possibility of mutual understanding. This Hegel's \textit{Vernunft}, that Brandom \parencite*{Brandom:2019aa} calls a ``meta-meta-concept''. I \textit{want} to forgive myself and others for how we treat each other in the name of ``goodness,'' yet in deploying the stage-like framework, I implicitly rely on the communicative norms associated with \textit{rationality} or \textit{Verstand}. I impute a cause for `bad' behavior in saying \textit{sotto voce}, ``you are a child, you are \textit{supposed} to be acting in self-interest.'' 

\section{The Fear of Nothingness}

The split between the acting \{I\} and the recognized me is not a neutral philosophical fact; it is the source of a profound and motivating anxiety. If the fulfillment of our being lies in recognition, its shadow is the fear of misrecognition. This is not merely a social fear of disapproval but a deeper, existential fear of non-being. If we feel more of ourselves when we are recognized, it stands to reason that we might feel smaller, meaner, and lower when we are misrecognized or misunderstood. This diminishment points toward an ultimate dread: the fear of erasure.

This may be called existential fear: the terror that arises from the possibility of non-existence.
It is the fear that our subjective \{I\}, our locus of creativity and freedom, will find no reflection, no confirmation, in the social world of the ``me.'' In misrecognition, the ``me'' becomes a distorted mask. This threatens the \{I\} with a kind of social death. In shame, I fall into a silencing so complete it verges on ontological annihilation. The fear is that one might not exist for others, and therefore, in a crucial social sense, not exist at all. This is precisely the ``existential fear of being groundless or of not being at all'' that Carspecken identifies as a ``prominent feature of human motivation'' \parencite[246]{Carspecken1999}. 


This fear is palpable in math classrooms. It can lead to a self-imposed silence, where the terror of being misunderstood is so great that people become unwilling to express themselves, suppressing the authenticity they long to have recognized.
It is also the deep, structural fear at the heart of the politics of misrecognition. ``The thesis is that our identity is partly shaped by recognition or its absence, often by the misrecognition of others, and so a person or group of people can suffer real damage, real distortion, if the people or society around them mirror back to them a confining or demeaning or contemptible picture of themselves'' \parencite[25]{Taylor1994}. For marginalized groups, especially those whose marginalization occurs when ascriptive categories that describe subjects in categories that foreground objective validity, like race or sex, the misrecognition of a subject as an object is not an abstract anxiety but a systemic, political reality. The fear of non-being is weaponized to maintain structures of power. Neither sex nor race are terribly relevant for understanding Turbo's experience with the Ellettsville authorities, or with my fears of being misrecognized, but the fear of non-being is a universal human experience that can be weaponized by those in power regardless of how the boundary between inside and outside is drawn between social being and social non-being.

But here's what's crucial: the relationship to this fear undergoes transformation through development.
In early stages of communicative reasoning, it feels purely threatening -- the child desperately seeks approval and fears abandonment. In adolescence, it becomes defiant -- the teenager would rather be rejected for who they authentically are than accepted for who they're not. 



In mature development, this fear is recognized as a source of humanity, an engine that drives both the need for connection and the capacity for growth.
In wisdom, a stage I aspire to (and seem to fail to attain by virtue of that aspiration), perhaps the two needs will finally melt into one another through surrender. 

This developmental transformation expresses that what initially appears as pure threat can become the source of power and creativity. In Hegelian terms, through the movement from understanding (Verstand) to reason (Vernunft), one learns to work with rather than against the productive tensions that make humans human. Such contradictions are not problems to be solved but the dynamics that enable growth toward wisdom.

Yet, this fear is not only a negative or paralyzing force. Recognizing existential fear, as such, points toward the possibility of self-recognition. Taking the attitude of the other -- becoming a ``you'' to the \{I\} -- by taking a second-person position on one's self can transform that fear into something productive. The aching desire to be understood is the positive and creative response to this underlying terror. The ferocity of this fear of nothingness fuels a striving for connection and community. It is from this foundational anxiety, in a Heideggerian sense, that two projects emerge in their separation and find themselves in unity: the quest for normative belonging (to be recognized as good) and the quest for authentic expression (to be recognized as \textit{in}finite). 


\section{The Fabiola Project and the Absent Referent}
This philosophical distinction is visualized in the juxtaposition of two models of singular reference. On one hand, we can consider Robert Brandom's articulation of singular terms explored in the previous chapter. He describes their distinctive linguistic role as expressions `that refer to, denote, or designate particular objects' through substitution-inferential significance, forming equivalence classes of intersubstitutable descriptors. His exemplar, ``Benjamin Franklin,'' clarifies this: although multiple interchangeable descriptions -- such as ``the inventor of bifocals,'' ``the first postmaster general of the United States'' -- point symmetrically toward one historical figure, an inherent nullity emerges in their interstitial space. The subject, Franklin himself, is not contained fully by any single descriptor; rather, he exists as an intersection of equivalence classes as internalized by any who refer to him and whose collective reference is both precise and entirely empty of substantive singular essence. The stability of the ``me'' as an equivalence class or intersection of such classes is rather hollow. 

In contrast, \textit{The Fabiola Project} at the Menil Collection in Houston offers a counterpoint to Franklin's neatly symmetrical class of descriptors. The Project, curated around numerous divergent copies of Jean-Jacques Henner's original, now-lost painting of Saint Fabiola, illustrates an asymmetry at the heart of singular reference. Henner's painting was destroyed before any photographic record was possible, leaving behind only disparate, interpretative reproductions. In some, Fabiola appears in a red habit, in others green; her orientation shifts, her complexion changes radically from one image to another. Mediums vary -- paint, needlepoint, carvings, mosaics crafted from rice and beans flood the walls of the collection with difference. Each artist's interpretation claims authenticity without definitive verification. 

\textbf{Figure}

\begin{figure}[h]
    %\centering
    \includegraphics[width=0.7\textwidth]{/Users/tio/Documents/GitHub/September_UMEDCA/images/Fabiola_Franklin.pdf}
    \caption{\textit{Note. }The equivalence class of singular terms for Benjamin Franklin on the left contrasts with \textit{The Fabiola Project}, where the referent of the paintings is shared but the re-presentations of that referent are not symmetrically intersubstitutable, allegorizing the non-presence of the subjective 'I'.}
    \label{fig:franklin_fabiola}
\end{figure}


The Fabiola Project's reproductions, while they may share a common referent in the abstract, do not coalesce into a singular equivalence class. Instead, they proliferate into a multitude of interpretations, each claiming to represent Fabiola yet failing to converge on a definitive essence. The multiplicity of descriptors -- each an interpretation of an absent original -- creates a complex web of interrelations that defies the neat symmetry found in Brandom's exemplar. Each representation is both a claim to authenticity and a reminder of the original's absence, leading to an irreducible tension between presence and absence. 

This makes the project a striking exercise in Derridean non-presence. The original referent can never be made present; it exists only as a `trace' that animates the play of `diff\'erance' across the many copies. No single representation, no descriptor, can achieve full presence or exhaust the meaning of ``Fabiola.'' This is the predicament of the ``I.'' The \{I\} is the absent original, the point of subjective origin that can never be fully captured or made present in any objective description of the ``me.'' It is a nullity at the center of our self-descriptions. However, unlike the emptiness of the ``I think'' or the hollow intersection of equivalence classes, the absence of the original is a \textit{generative absence}. As Carspecken notes, this elusiveness is part of our direct experience: ``Try it out. Focus your attention on an object and see if you can freeze out a moment in which you simultaneously have it before you and know that you do. It is impossible. `Presence' is unattainable in pure form within experience and yet it is presupposed or implicated in experience'' \parencite[52]{Carspecken1999}. 

The Fabiola Project visualizes this philosophical truth: the self is a web of representations pointing toward an origin that is never fully there, defined precisely by its resistance to definitive singular reference. This resistance is the source of our need to be recognized as \textit{in}finite. The original painting's essence is both affirmed and irrevocably negated by its diverse reiterations, creating an irreducible tension between presence and absence that mirrors the gap between the acting \{I\} and the recognized ``me.''


\section{The Beast of Love: A Turning Point}\label{sec:beast_of_love}

It is at this point of maximum tension -- where the rational framework seems to offer no solution and our existential needs appear tragically opposed -- that we must introduce a different voice. This voice speaks not from the security of theoretical understanding but from the vulnerability of lived experience. It speaks through a poem that emerged from my own struggle with the tensions between authority and authenticity, between the need to provide guidance and the fear of causing harm.

\subsection*{The Beast of Love}
\begin{verse}
When I see you, I see a light\\
Sharp and hot and glowing white\\
Your power, depth, and fierce roar\\
Broken in the glass of night\\
Bowed by rain, you start to fall\\
Fear of pain inside us all\\*[2ex]

As you learn, the things you fear\\
Will grow and stamp and roar and rear\\
Shadows looming on the wall\\
Become, with care, what you hold dear\\
Then Power - deep as dark abyss -\\
Embraces faint and shadowed mist\\*[2ex]

You will come to love the beast\\
That tears and cries and thrashes sheets\\
Because you are that beast, sweet child\\
That tears and cries and thrashes wild\\
Bend back the fog, back into rain\\
To wash away the shadowed stain\\*[2ex]

You are what Powers scary things\\
You are the wind beneath your wings\\
 You cannot crush or fight the wind\\
But soon you'll learn to love the sting\\
Trust the beast to come in tune\\
Dark and bright, stars, sun, and moon\\*[2ex]

I see in you a Peace to come\\
Shining brighter than the sun\\
Each time you break and let it go\\
Your body and your spirit grow\\
Surely in the curl of time\\
You'll come to be and know your mind\\*[2ex]

I'm sorry, child, I bent your wing\\
I am a big and scary thing\\
Still frightened that the beast within\\
Might tear your fragile, gentle skin\\
And leave you scarred before you know\\
How deep and wide your love will grow
\end{verse}

I wrote this poem for my stepdaughter when she was about 4 years old after a moment of painful misrecognition -- a timeout that wounded our relationship for months. I forget why we had conflict, but I remember that I was trying to be a `good' authority figure. There was some boundary that I thought was legitimate, but that boundary conflicted with the \textit{in}finite spontaneity of the \{I\}. In confessing my fear, I point to the scars I bear from before I knew how much deeper and wider I would come to love. That scratching beast is still within, not-yet in tune, but still I find my capacity for genuine reciprocal intersubjective recognition [love] has grown through projects such as this one. 


\subsection*{Verstand to Vernunft: Beyond Oppositional Consciousness}

The poem implicitly invokes the power of surrender -- a developmental transformation that moves beyond the oppositional consciousness that traps us in either/or thinking. This is not surrender as defeat but surrender as the recognition that our contradictions are not problems to be eliminated but the source of our growth and creativity.

The line ``You will come to love the beast'' captures this developmental insight. The ``beast'' -- the terrifying tension between our existential needs -- initially appears as pure threat. In early development, we experience it as overwhelming anxiety: the child desperately seeks approval and fears abandonment. In adolescence, it becomes defiant opposition: the teenager would rather be rejected for who they authentically are than accepted for who they're not. 

But mature development contains a different possibility. The tension is a source of humanity. The ``beast'' that ``tears and cries and thrashes'' is not an enemy but a creative power. Learning to ``trust the beast to come in tune'' means learning to work with rather than against the productive tensions that bind people together. But ``working with'' is conditioned on the possibility of non-action. Learning to listen -- surrendering to the movement of the negative which is never grasped in its fullness -- is the first step in befriending the beast. 

This transformation requires what the poem calls ``breaking and letting go.'' Each time I encounter the collision between needs -- each time I face the apparent contradiction between being good and being authentic -- I have the opportunity to ``break'' out of oppositional thinking and ``let go'' of the fantasy that these needs must be mutually exclusive.

The poem's final stanza introduces three crucial concepts that are essential to the resolution of existential tension: confession, forgiveness, and trust. These are not merely psychological categories but ontological structures that make possible the movement from opposition to integration. Brandom's work on the normative structure of rationality can help us understand these concepts as foundational to the transition from \textit{Verstand} -- the mode of understanding -- to \textit{Vernunft} that consciousness undergoes when actively reading Hegel's \textit{Phenomenology}. 

``I'm sorry, child, I bent your wing'' is a confession -- an acknowledgment of the harm that inevitably occurs when finite beings with competing needs interact. I confess to being ``a big and scary thing,'' acknowledging the power differential that makes recognition so difficult. The confession opens the space for forgiveness -- not as an erasure of harm but as the willingness to continue the relationship despite the wound. Forgiveness recognizes that the ``beast within'' both the authority figure and the child ``might tear fragile, gentle skin,'' but it also trusts that love will grow deeper and wider through the scars.

Trust emerges as the fundamental structure that makes ongoing relationship possible. It is trust in the developmental process itself -- the faith that ``in the curl of time'' both parties will ``come to be and know their mind.'' This trust is not based on certainty but on surrender. 

These concepts -- confession, forgiveness, and trust -- are what Habermas might call the ``lifeworld'' foundations that make rational discourse possible. Without the willingness to confess failures, forgive others' limitations, and trust in the possibility of mutual understanding, communicative action breaks down into strategic manipulation or ideological domination.

What mature development entails is that the two existential needs are not ultimately opposed but dialectically unified. The need to be recognized as \textit{good} and the need to be recognized as \textit{in}finite require each other for their fulfillment.

\printbibliography[heading=subbibliography]