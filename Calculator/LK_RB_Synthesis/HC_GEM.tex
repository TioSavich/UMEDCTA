% Options for packages loaded elsewhere
\PassOptionsToPackage{unicode}{hyperref}
\PassOptionsToPackage{hyphens}{url}
%
\documentclass[
]{article}
\usepackage{amsmath,amssymb}
\usepackage{iftex}
\ifPDFTeX
  \usepackage[T1]{fontenc}
  \usepackage[utf8]{inputenc}
  \usepackage{textcomp} % provide euro and other symbols
\else % if luatex or xetex
  \usepackage{unicode-math} % this also loads fontspec
  \defaultfontfeatures{Scale=MatchLowercase}
  \defaultfontfeatures[\rmfamily]{Ligatures=TeX,Scale=1}
\fi
\usepackage{lmodern}
\ifPDFTeX\else
  % xetex/luatex font selection
\fi
% Use upquote if available, for straight quotes in verbatim environments
\IfFileExists{upquote.sty}{\usepackage{upquote}}{}
\IfFileExists{microtype.sty}{% use microtype if available
  \usepackage[]{microtype}
  \UseMicrotypeSet[protrusion]{basicmath} % disable protrusion for tt fonts
}{}
\makeatletter
\@ifundefined{KOMAClassName}{% if non-KOMA class
  \IfFileExists{parskip.sty}{%
    \usepackage{parskip}
  }{% else
    \setlength{\parindent}{0pt}
    \setlength{\parskip}{6pt plus 2pt minus 1pt}}
}{% if KOMA class
  \KOMAoptions{parskip=half}}
\makeatother
\usepackage{xcolor}
\usepackage{longtable,booktabs,array}
\usepackage{calc} % for calculating minipage widths
% Correct order of tables after \paragraph or \subparagraph
\usepackage{etoolbox}
\makeatletter
\patchcmd\longtable{\par}{\if@noskipsec\mbox{}\fi\par}{}{}
\makeatother
% Allow footnotes in longtable head/foot
\IfFileExists{footnotehyper.sty}{\usepackage{footnotehyper}}{\usepackage{footnote}}
\makesavenoteenv{longtable}
\setlength{\emergencystretch}{3em} % prevent overfull lines
\providecommand{\tightlist}{%
  \setlength{\itemsep}{0pt}\setlength{\parskip}{0pt}}
\setcounter{secnumdepth}{-\maxdimen} % remove section numbering
\usepackage{bookmark}
\IfFileExists{xurl.sty}{\usepackage{xurl}}{} % add URL line breaks if available
\urlstyle{same}
\hypersetup{
  pdftitle={Hermeneutic Calculator},
  hidelinks,
  pdfcreator={LaTeX via pandoc}}

\title{Hermeneutic Calculator}
\author{}
\date{}

\begin{document}
\maketitle

I am building a unified, testable theory of how students \emph{develop}
arithmetic understanding---not just a catalog of strategy names, but a
developmental map of how those strategies \emph{emerge, elaborate,
invert, and nest}. My target is to formalize roughly 25 student-invented
strategies across addition, subtraction, multiplication, and division,
showing how each one is algorithmically constructed from prior embodied
practices.

\subsection{Choreography as
Computation}\label{choreography-as-computation}

I treat each strategy as \textbf{written choreography for embodied
cognition}. A formal automaton (register machine, bounded DPDA, or
related model) becomes a script for the temporal unfolding of thought:
initialize, transform, check, recurse, terminate. The power of this
framing is that it preserves \emph{how} a student actually moves through
a calculation---counting up, pausing at a boundary, decomposing a
number---rather than replacing those moves with opaque symbolic
shortcuts.

\subsection{Two Fundamental Movements}\label{two-fundamental-movements}

I analyze student action through a dialectic of temporal structure: 1.
\textbf{Temporal Compression (Sublation / Recollection):} Unitizing many
micro-acts into a larger cognitive unit (ten ones \(\to\) one ten; 3
base jumps \(\to\) a single composite stride). Compression accelerates
flow. 2. \textbf{Temporal Decompression (Determinate Negation):}
Strategically undoing or expanding a composite to restore fine control
(borrowing a ten; splitting \(5\) into \(2+3\) in RMB; decomposing a
factor for distributive reasoning). Fluency grows as students coordinate
these movements, learning \emph{when} to expand and \emph{when} to
re-compress.

\subsection{Fractal Architecture: Iterative Core + Strategic
Shell}\label{fractal-architecture-iterative-core-strategic-shell}

Across strategies I repeatedly recover the same \textbf{fractal
pattern}: * \textbf{Iterative Core:} A minimal loop (initialize \(\to\)
step (\(+1\), \(-1\), \(+Base\), \(+Chunk\)) \(\to\) condition check).
Counting by ones, skip counting, and accumulation loops in division all
instantiate this engine. * \textbf{Strategic Shell:} A supervisory layer
that \emph{prepares}, \emph{optimizes}, or \emph{transforms} the problem
so that the core runs fewer or cognitively lighter iterations. RMB,
Rounding \& Adjusting, Chunking, Sliding, Distributive and Inverse
Distributive Reasoning all wrap the core with analysis (e.g., ``find gap
\(K\)'', ``split factor'', ``slide both numbers''). Because the shell
often \emph{invokes} the core as a subroutine (e.g., CountUpToBase,
CountBackK), the global structure becomes self-similar: strategies
\emph{contain} (and sometimes nest) earlier strategies. This produces a
genuine computational fractal---not metaphorical flourish, but
recurrence of the same control schema at different conceptual scales.

\subsection{Mechanisms of Elaboration}\label{mechanisms-of-elaboration}

I observe three progressive forms of algorithmic elaboration: 1.
\textbf{Compression of Action:} Replacing many \(+1\) steps with
\(+Base\) or \(+StructuredChunk\) (COBO, Chunking). 2.
\textbf{Optimization of Iteration:} Dynamically computing the
\emph{size} of a future stride (RMB gap \(K\); Chunking's bridging
chunk; Sliding's constant difference) before acting---analyze then
accelerate. 3. \textbf{Structural Transformation:} Rewriting the problem
space (Rounding detour + compensation; Distributive split; Sliding
invariance; Inverse Distributive decomposition of dividend). Advanced
strategies chain these moves (e.g., Rounding = transformation \(\to\)
compressed addition \(\to\) compensatory inverse steps).

\subsection{Inversion of Practice}\label{inversion-of-practice}

Subtraction fluency emerges not by inventing alien procedures but by
\textbf{inverting or repurposing} addition shells: Missing Addend
reframes subtraction as forward accumulation; Counting Back mirrors
Counting On; Sliding preserves difference across a translation;
Borrowing reverses carry (decompression of a prior sublation). Division
analogues (Dealing by Ones vs.~Coordinating Two Counts; Inverse
Distributive Reasoning) continue the same inversion logic.

\subsection{Collaboration and Iterative
Refinement}\label{collaboration-and-iterative-refinement}

This project is explicitly \emph{collaborative} with an AI assistant. I
bring student transcripts, pedagogical insight, and theoretical intent;
the assistant supplies relentless formal scrutiny---flagging
mis-specified state sets, hidden non-determinism, premature algebraic
assumptions, or missing termination guarantees. A typical refinement
cycle: 1. Draft informal description from transcript. 2. Specify
automaton (states, registers, transitions) in a first-pass formalism. 3.
Implement executable prototype (Python) to test determinism,
termination, and behavioral alignment with the transcript (sequence
reconstruction like ``\(46, 56, 66, \dots\)''). 4. Trace failures (e.g.,
an early Rounding model produced an infinite loop after overshoot; an
initial Chunking diagram hid the cognitive search for \(K\)) and revise.
5. Re-abstract the corrected machine into concise LaTeX-friendly
specification. This loop ensures every claimed cognitive choreography
\emph{runs}---a falsifiability and reproducibility standard often
missing in purely diagrammatic accounts.

\subsection{Why Executable Formal Models
Matter}\label{why-executable-formal-models-matter}

An executable automaton does four things for me: * \textbf{Validity
Check:} Catches hidden cycles or unreachable states. *
\textbf{Phenomenological Fidelity:} Lets me align generated action
traces with verbatim student utterances. * \textbf{Comparative Anatomy:}
Normalizes different strategies into a shared tuple structure so I can
map elaboration edges precisely. * \textbf{Pedagogical Insight:}
Identifies which internal subroutines (e.g., ``CountBackK'') must be
instructionally stabilized before a composite strategy will consolidate.

\subsection{Algorithmic Elaboration (Brandom
Frame)}\label{algorithmic-elaboration-brandom-frame}

Following Robert Brandom, I treat these developments as
\textbf{algorithmic elaborations}: later practices are
\emph{PP-sufficient} expansions of earlier ones---achieved by
reorganizing, nesting, or inverting existing abilities rather than
importing foreign primitives. Some strategies become \textbf{LX}
relative to prior practice: they both derive from and make explicit what
was implicit (RMB makes base boundaries explicit; Borrowing renders the
reversibility of carry explicit; Distributive Reasoning makes latent
additivity across factors explicit).

\subsection{Scope of This Document}\label{scope-of-this-document}

Below I give each strategy a uniform template: description, formal
specification, choreography (compression/decompression dynamics), and
genealogical lineage. I remove historical critique and raw code to
foreground the structural logic while preserving testability through the
already verified prototypes. The introduction you are reading
consolidates the nuance of origin (embodiment), evolution (fractal
elaboration), collaboration (human + AI), and rigor (execution +
revision) from the longer source manuscript without duplicating
passages.

\begin{center}\rule{0.5\linewidth}{0.5pt}\end{center}

\section{Hermeneutic Calculator: Strategy
Formalizations}\label{hermeneutic-calculator-strategy-formalizations}

This draft reorganizes the strategies as primary sections. Each section
supplies:

\begin{enumerate}
\def\labelenumi{\arabic{enumi}.}
\tightlist
\item
  Phenomenological description (student-facing practice).
\item
  Formal automaton / register-machine specification in LaTeX-friendly
  notation.
\item
  Core choreography (temporal compression/decompression dynamics).
\item
  Algorithmic elaboration lineage (what primitives it builds upon).
\end{enumerate}

All implementation details (Python prototypes) and historical critiques
have been removed. Mathematical symbols are formatted for Pandoc \(\to\)
LaTeX conversion.

Notation (uniform across strategies):

\begin{itemize}
\tightlist
\item
  \(M = (Q, V, \delta, q_0, F)\): machine with states \(Q\), registers
  (or variables) \(V\), transition function \(\delta\), start state
  \(q_0\), accepting states \(F\).
\item
  When convenient, we use auxiliary internal variables; these are
  included in \(V\) implicitly.
\item
  Counting primitives: Count Up (\(+1\)), Count Back (\(-1\)) regarded
  as atomic embodied actions.
\item
  Temporal Compression: synthesizing many unit actions into a
  higher-order unit (e.g., a ``ten'').
\item
  Temporal Decompression: strategic expansion of a unit into constituent
  parts.
\end{itemize}

\begin{center}\rule{0.5\linewidth}{0.5pt}\end{center}

\section{Counting and Counting On}\label{counting-and-counting-on}

\textbf{Description.} Sequential unit counting within a bounded base-10
place-value structure (\(0-999\)). Embodied iterations (``ticks'')
increment units, propagate carries (sublation) into tens and hundreds.

\textbf{Formal Model (Sketch).} Deterministic PDA (bounded) or
3-register counter. For LaTeX exposition we specify a DPDA tuple:

\[M_{count} = (Q, \Sigma, \Gamma, \delta, q_{start}, Z_0, F)\]

with place-value stack symbols \(U_i, T_j, H_k\). The transition
function \(\delta\) is defined as follows:

\begin{longtable}[]{@{}
  >{\raggedright\arraybackslash}p{(\linewidth - 10\tabcolsep) * \real{0.1667}}
  >{\raggedright\arraybackslash}p{(\linewidth - 10\tabcolsep) * \real{0.1667}}
  >{\raggedright\arraybackslash}p{(\linewidth - 10\tabcolsep) * \real{0.1667}}
  >{\raggedright\arraybackslash}p{(\linewidth - 10\tabcolsep) * \real{0.1667}}
  >{\raggedright\arraybackslash}p{(\linewidth - 10\tabcolsep) * \real{0.1667}}
  >{\raggedright\arraybackslash}p{(\linewidth - 10\tabcolsep) * \real{0.1667}}@{}}
\toprule\noalign{}
\begin{minipage}[b]{\linewidth}\raggedright
Current State
\end{minipage} & \begin{minipage}[b]{\linewidth}\raggedright
Input
\end{minipage} & \begin{minipage}[b]{\linewidth}\raggedright
Top of Stack
\end{minipage} & \begin{minipage}[b]{\linewidth}\raggedright
Next State
\end{minipage} & \begin{minipage}[b]{\linewidth}\raggedright
Action (Stack)
\end{minipage} & \begin{minipage}[b]{\linewidth}\raggedright
Interpretation
\end{minipage} \\
\midrule\noalign{}
\endhead
\bottomrule\noalign{}
\endlastfoot
\(q_{start}\) & \(\varepsilon\) & \(Z_0\) & \(q_{idle}\) &
Push(\(U_0, T_0, H_0\)) & Initialize count to 0. \\
\(q_{idle}\) & \texttt{tick} & \(U_n\) (\(n<9\)) & \(q_{idle}\) & Pop;
Push(\(U_{n+1}\)) & Increment units. \\
\(q_{idle}\) & \texttt{tick} & \(U_9\) & \(q_{inc\_tens}\) & Pop & Unit
overflow, carry to tens. \\
\(q_{inc\_tens}\) & \(\varepsilon\) & \(T_m\) (\(m<9\)) & \(q_{idle}\) &
Pop; Push(\(T_{m+1}, U_0\)) & Increment tens, reset units. \\
\(q_{inc\_tens}\) & \(\varepsilon\) & \(T_9\) & \(q_{inc\_hundreds}\) &
Pop & Ten overflow, carry to hundreds. \\
\(q_{inc\_hundreds}\) & \(\varepsilon\) & \(H_k\) (\(k<9\)) &
\(q_{idle}\) & Pop; Push(\(H_{k+1}, T_0, U_0\)) & Increment hundreds,
reset lower places. \\
\(q_{inc\_hundreds}\) & \(\varepsilon\) & \(H_9\) & \(q_{halt}\) & Pop;
Push(\(H_0, T_0, U_0\)) & Counter overflow. \\
\end{longtable}

\textbf{Choreography.} Carry = temporal compression: ten unit steps
recollected as one higher unit. Borrow (in inverse counting) is temporal
decompression.

\textbf{Elaboration Lineage.} Primitive for all subsequent additive,
subtractive, multiplicative, and divisional strategies.

\begin{center}\rule{0.5\linewidth}{0.5pt}\end{center}

\section{Rearranging to Make Bases
(RMB)}\label{rearranging-to-make-bases-rmb}

\textbf{Description.} For \(A + B\), identify gap \(K\) from \(A\) to
next base (e.g., 10, 100), decompose \(B = K + R\), form \(A' = A + K\)
(a base), then compute \(A' + R\).

\textbf{Machine.}

\[M_{RMB} = (Q, V, \delta, q_0, F)\]

with
\[Q = \{q_{start}, q_{calcK}, q_{decompose}, q_{recombine}, q_{accept}\}\]
\[V = \{A, B, K, A', R\}\]

\textbf{Transition Function (\(\delta\)):}

\begin{longtable}[]{@{}
  >{\raggedright\arraybackslash}p{(\linewidth - 8\tabcolsep) * \real{0.2000}}
  >{\raggedright\arraybackslash}p{(\linewidth - 8\tabcolsep) * \real{0.2000}}
  >{\raggedright\arraybackslash}p{(\linewidth - 8\tabcolsep) * \real{0.2000}}
  >{\raggedright\arraybackslash}p{(\linewidth - 8\tabcolsep) * \real{0.2000}}
  >{\raggedright\arraybackslash}p{(\linewidth - 8\tabcolsep) * \real{0.2000}}@{}}
\toprule\noalign{}
\begin{minipage}[b]{\linewidth}\raggedright
Current State
\end{minipage} & \begin{minipage}[b]{\linewidth}\raggedright
Condition
\end{minipage} & \begin{minipage}[b]{\linewidth}\raggedright
Next State
\end{minipage} & \begin{minipage}[b]{\linewidth}\raggedright
Action
\end{minipage} & \begin{minipage}[b]{\linewidth}\raggedright
Interpretation
\end{minipage} \\
\midrule\noalign{}
\endhead
\bottomrule\noalign{}
\endlastfoot
\(q_{start}\) & - & \(q_{calcK}\) & \(K \leftarrow 0\);
\(A_{temp} \leftarrow A\) & Initialize. \\
\(q_{calcK}\) & \(A_{temp}\) \textless{} NextBase(\(A\)) & \(q_{calcK}\)
& \(A_{temp} \leftarrow A_{temp} + 1\); \(K \leftarrow K + 1\) & Count
up to find gap \(K\). \\
\(q_{calcK}\) & \(A_{temp}\) == NextBase(\(A\)) & \(q_{decompose}\) &
\(A' \leftarrow A_{temp}\) & Gap found. Store new base \(A'\). \\
\(q_{decompose}\) & \(K > 0\) & \(q_{decompose}\) &
\(B \leftarrow B - 1\); \(K \leftarrow K - 1\) & Decompose \(B\) by
transferring \(K\). \\
\(q_{decompose}\) & \(K == 0\) & \(q_{recombine}\) & \(R \leftarrow B\)
& Remainder \(R\) is what's left of \(B\). \\
\(q_{recombine}\) & - & \(q_{accept}\) & Output \(A' + R\) & Combine new
base and remainder. \\
\end{longtable}

\textbf{Choreography.} Decompression (splitting \(B\)) enables immediate
compression (forming base \(A'\)).

\textbf{Lineage.} Elaborates Counting Up + Counting Down primitives;
anticipates strategic boundary manipulation used later in Rounding,
Chunking, Sliding.

\begin{center}\rule{0.5\linewidth}{0.5pt}\end{center}

\section{COBO (Counting On by Bases then
Ones)}\label{cobo-counting-on-by-bases-then-ones}

\textbf{Description.} For \(A + B\), decompose \(B = b \cdot Base + r\);
iterate base jumps (\(+Base\)) then unit steps (\(+1\)).

\textbf{Machine.} \(M_{COBO}\) with states
\(\{q_{start}, q_{bases}, q_{ones}, q_{accept}\}\) and registers
\(\{Sum, BaseCounter, OneCounter\}\).

\textbf{Transition Function (\(\delta\)):}

\begin{longtable}[]{@{}
  >{\raggedright\arraybackslash}p{(\linewidth - 8\tabcolsep) * \real{0.2000}}
  >{\raggedright\arraybackslash}p{(\linewidth - 8\tabcolsep) * \real{0.2000}}
  >{\raggedright\arraybackslash}p{(\linewidth - 8\tabcolsep) * \real{0.2000}}
  >{\raggedright\arraybackslash}p{(\linewidth - 8\tabcolsep) * \real{0.2000}}
  >{\raggedright\arraybackslash}p{(\linewidth - 8\tabcolsep) * \real{0.2000}}@{}}
\toprule\noalign{}
\begin{minipage}[b]{\linewidth}\raggedright
Current State
\end{minipage} & \begin{minipage}[b]{\linewidth}\raggedright
Condition
\end{minipage} & \begin{minipage}[b]{\linewidth}\raggedright
Next State
\end{minipage} & \begin{minipage}[b]{\linewidth}\raggedright
Action
\end{minipage} & \begin{minipage}[b]{\linewidth}\raggedright
Interpretation
\end{minipage} \\
\midrule\noalign{}
\endhead
\bottomrule\noalign{}
\endlastfoot
\(q_{start}\) & - & \(q_{initialize}\) & Read \(A, B\) & Start. \\
\(q_{initialize}\) & - & \(q_{add\_bases}\) & \(Sum \leftarrow A\);
\(BaseCounter \leftarrow B // Base\);
\(OneCounter \leftarrow B \pmod{Base}\) & Initialize Sum. Decompose
\(B\). \\
\(q_{add\_bases}\) & \(BaseCounter > 0\) & \(q_{add\_bases}\) &
\(Sum \leftarrow Sum + Base\);
\(BaseCounter \leftarrow BaseCounter - 1\) & Add one Base unit
(Loop). \\
\(q_{add\_bases}\) & \(BaseCounter == 0\) & \(q_{add\_ones}\) & - & All
bases added. Transition. \\
\(q_{add\_ones}\) & \(OneCounter > 0\) & \(q_{add\_ones}\) &
\(Sum \leftarrow Sum + 1\); \(OneCounter \leftarrow OneCounter - 1\) &
Add one unit (Loop). \\
\(q_{add\_ones}\) & \(OneCounter == 0\) & \(q_{accept}\) & Output
\(Sum\) & All ones added. Accept. \\
\end{longtable}

\textbf{Choreography.} Two-phase rhythm: compressed temporal blocks
(bases) followed by decompressed fine resolution (ones).

\textbf{Lineage.} Builds on counting; prepares for Chunking and Rounding
by habitualizing base jumps.

\begin{center}\rule{0.5\linewidth}{0.5pt}\end{center}

\section{Rounding and Adjusting
(Addition)}\label{rounding-and-adjusting-addition}

\textbf{Description.} Select addend closer to next base: round up
\(A \to A' = A + K\), compute \(A' + B\), then adjust back:
\((A' + B) - K\).

\textbf{Machine.} States
\(\{q_{start}, q_{calcK}, q_{add}, q_{adjust}, q_{accept}\}\); registers
\(\{A,B,K,A',Temp,Result\}\).

\textbf{Transition Function (\(\delta\)):}

\begin{longtable}[]{@{}
  >{\raggedright\arraybackslash}p{(\linewidth - 6\tabcolsep) * \real{0.2500}}
  >{\raggedright\arraybackslash}p{(\linewidth - 6\tabcolsep) * \real{0.2500}}
  >{\raggedright\arraybackslash}p{(\linewidth - 6\tabcolsep) * \real{0.2500}}
  >{\raggedright\arraybackslash}p{(\linewidth - 6\tabcolsep) * \real{0.2500}}@{}}
\toprule\noalign{}
\begin{minipage}[b]{\linewidth}\raggedright
Current State
\end{minipage} & \begin{minipage}[b]{\linewidth}\raggedright
Subroutine / Action
\end{minipage} & \begin{minipage}[b]{\linewidth}\raggedright
Next State
\end{minipage} & \begin{minipage}[b]{\linewidth}\raggedright
Interpretation
\end{minipage} \\
\midrule\noalign{}
\endhead
\bottomrule\noalign{}
\endlastfoot
\(q_{start}\) & Read \(A, B\); Heuristic select \(Target\) &
\(q_{calcK}\) & Start. Select number closer to the next base. \\
\(q_{calcK}\) & \textbf{Count Up To Base(\(Target\))}
\(\to K, A_{rounded}\) & \(q_{add}\) & Determine \(K\) by counting up
from \(Target\). \\
\(q_{add}\) & \textbf{COBO(\(A_{rounded}\), Other)} \(\to TempSum\) &
\(q_{adjust}\) & Add Other to the rounded \(A\). \\
\(q_{adjust}\) & \textbf{Count Back(\(TempSum, K\))} \(\to Result\) &
\(q_{accept}\) & Adjust by counting back \(K\). \\
\end{longtable}

\textbf{Choreography.} Strategic temporal detour: initial decompression
(deriving \(K\)) enables major compression (base addition), followed by
inverse correction.

\textbf{Lineage.} Elaborates RMB (boundary anticipation) and COBO (base
efficiency); introduces explicit compensation schema.

\begin{center}\rule{0.5\linewidth}{0.5pt}\end{center}

\section{Chunking (Addition)}\label{chunking-addition}

\textbf{Description.} Decompose \(B\) into large base chunk + strategic
residual chunks to force successive bases: \(B = B_{base} + K + R\)
where \(K\) bridges current sum to next base.

\textbf{Transition Function (\(\delta\)):}

\begin{longtable}[]{@{}
  >{\raggedright\arraybackslash}p{(\linewidth - 8\tabcolsep) * \real{0.2000}}
  >{\raggedright\arraybackslash}p{(\linewidth - 8\tabcolsep) * \real{0.2000}}
  >{\raggedright\arraybackslash}p{(\linewidth - 8\tabcolsep) * \real{0.2000}}
  >{\raggedright\arraybackslash}p{(\linewidth - 8\tabcolsep) * \real{0.2000}}
  >{\raggedright\arraybackslash}p{(\linewidth - 8\tabcolsep) * \real{0.2000}}@{}}
\toprule\noalign{}
\begin{minipage}[b]{\linewidth}\raggedright
Current State
\end{minipage} & \begin{minipage}[b]{\linewidth}\raggedright
Condition
\end{minipage} & \begin{minipage}[b]{\linewidth}\raggedright
Next State
\end{minipage} & \begin{minipage}[b]{\linewidth}\raggedright
Action
\end{minipage} & \begin{minipage}[b]{\linewidth}\raggedright
Interpretation
\end{minipage} \\
\midrule\noalign{}
\endhead
\bottomrule\noalign{}
\endlastfoot
\(q_{init}\) & - & \(q_{addBase}\) & \(Sum \leftarrow A\); Decompose
\(B\) into \(B_{base}, B_{ones}\) & Initialize Sum. Decompose \(B\). \\
\(q_{addBase}\) & - & \(q_{calcK}\) & \(Sum \leftarrow Sum + B_{base}\)
& Add the entire base chunk at once. \\
\(q_{calcK}\) & \(Sum <\) NextBase(\(Sum\)) & \(q_{calcK}\) &
\(Sum \leftarrow Sum + 1\); \(K \leftarrow K + 1\) & Iteratively find
gap \(K\) to next base. \\
\(q_{calcK}\) & \(Sum ==\) NextBase(\(Sum\)) & \(q_{applyK}\) & - & Gap
found. \\
\(q_{applyK}\) & \(B_{ones} \ge K\) & \(q_{calcK}\) &
\(Sum \leftarrow Sum + K\); \(B_{ones} \leftarrow B_{ones} - K\) & Add
strategic chunk \(K\). Loop back. \\
\(q_{applyK}\) & \(B_{ones} < K\) & \(q_{finishR}\) & - & Not enough
ones for full chunk. \\
\(q_{finishR}\) & - & \(q_{accept}\) & \(Sum \leftarrow Sum + B_{ones}\)
& Add remaining residue. \\
\end{longtable}

\textbf{Choreography.} Iterative cycle: (1) large compression via
aggregated base, (2) micro decompression to find \(K\), (3)
re-compression to new base, (4) terminal residue.

\textbf{Lineage.} Synthesizes COBO (bulk bases) + RMB (strategic gap
finding).

\begin{center}\rule{0.5\linewidth}{0.5pt}\end{center}

\section{Subtraction Chunking (Three
Orientations)}\label{subtraction-chunking-three-orientations}

Given \(M - S = D\).

\textbf{A. Backwards by Part (Take-Away).} Sequentially subtract
decomposed parts of \(S\) (place value or strategic chunks) from \(M\).

\textbf{B. Forwards from Part (Missing Addend).} Treat as \(S + D = M\);
Count Up (RMB logic) accumulating \(D\).

\textbf{C. Backwards to Part (Distance Down To).} Count Back from \(M\)
toward \(S\) using strategic base landings; accumulate distance.

Each orientation is a register machine. Below are the key transition
schemas.

\textbf{A. Backwards by Part (Take-Away):}
\(V = \{CurrentValue, S_{rem}\}\)

\begin{longtable}[]{@{}
  >{\raggedright\arraybackslash}p{(\linewidth - 4\tabcolsep) * \real{0.3333}}
  >{\raggedright\arraybackslash}p{(\linewidth - 4\tabcolsep) * \real{0.3333}}
  >{\raggedright\arraybackslash}p{(\linewidth - 4\tabcolsep) * \real{0.3333}}@{}}
\toprule\noalign{}
\begin{minipage}[b]{\linewidth}\raggedright
State
\end{minipage} & \begin{minipage}[b]{\linewidth}\raggedright
Condition
\end{minipage} & \begin{minipage}[b]{\linewidth}\raggedright
Action
\end{minipage} \\
\midrule\noalign{}
\endhead
\bottomrule\noalign{}
\endlastfoot
\(q_{init}\) & - & \(CurrentValue \leftarrow M\);
\(S_{rem} \leftarrow S\) \\
\(q_{chunk}\) & \(S_{rem} > 0\) & \(Chunk \leftarrow\)
Decompose(\(S_{rem}\));
\(CurrentValue \leftarrow CurrentValue - Chunk\);
\(S_{rem} \leftarrow S_{rem} - Chunk\) \\
\(q_{chunk}\) & \(S_{rem} == 0\) & Accept \(CurrentValue\) \\
\end{longtable}

\textbf{B. Forwards from Part (Missing Addend):}
\(V = \{CurrentValue, Distance\}\)

\begin{longtable}[]{@{}
  >{\raggedright\arraybackslash}p{(\linewidth - 4\tabcolsep) * \real{0.3333}}
  >{\raggedright\arraybackslash}p{(\linewidth - 4\tabcolsep) * \real{0.3333}}
  >{\raggedright\arraybackslash}p{(\linewidth - 4\tabcolsep) * \real{0.3333}}@{}}
\toprule\noalign{}
\begin{minipage}[b]{\linewidth}\raggedright
State
\end{minipage} & \begin{minipage}[b]{\linewidth}\raggedright
Condition
\end{minipage} & \begin{minipage}[b]{\linewidth}\raggedright
Action
\end{minipage} \\
\midrule\noalign{}
\endhead
\bottomrule\noalign{}
\endlastfoot
\(q_{init}\) & - & \(CurrentValue \leftarrow S\);
\(Distance \leftarrow 0\) \\
\(q_{chunk}\) & \(CurrentValue < M\) & \(Chunk \leftarrow\)
CalcStrategicChunk(\(CurrentValue, M\));
\(CurrentValue \leftarrow CurrentValue + Chunk\);
\(Distance \leftarrow Distance + Chunk\) \\
\(q_{chunk}\) & \(CurrentValue == M\) & Accept \(Distance\) \\
\end{longtable}

\textbf{C. Backwards to Part (Distance Down To):}
\(V = \{CurrentValue, Distance\}\)

\begin{longtable}[]{@{}
  >{\raggedright\arraybackslash}p{(\linewidth - 4\tabcolsep) * \real{0.3333}}
  >{\raggedright\arraybackslash}p{(\linewidth - 4\tabcolsep) * \real{0.3333}}
  >{\raggedright\arraybackslash}p{(\linewidth - 4\tabcolsep) * \real{0.3333}}@{}}
\toprule\noalign{}
\begin{minipage}[b]{\linewidth}\raggedright
State
\end{minipage} & \begin{minipage}[b]{\linewidth}\raggedright
Condition
\end{minipage} & \begin{minipage}[b]{\linewidth}\raggedright
Action
\end{minipage} \\
\midrule\noalign{}
\endhead
\bottomrule\noalign{}
\endlastfoot
\(q_{init}\) & - & \(CurrentValue \leftarrow M\);
\(Distance \leftarrow 0\) \\
\(q_{chunk}\) & \(CurrentValue > S\) & \(Chunk \leftarrow\)
CalcStrategicChunk(\(CurrentValue, S\));
\(CurrentValue \leftarrow CurrentValue - Chunk\);
\(Distance \leftarrow Distance + Chunk\) \\
\(q_{chunk}\) & \(CurrentValue == S\) & Accept \(Distance\) \\
\end{longtable}

\textbf{Choreography.} Orientation selects temporal direction;
strategies B and C exploit boundary compression via RMB subroutines.

\begin{center}\rule{0.5\linewidth}{0.5pt}\end{center}

\section{Subtraction COBO / CBBO}\label{subtraction-cobo-cbbo}

\textbf{COBO (Missing Addend).} Start at \(S\), perform base jumps
toward \(M\) (without overshoot), then ones; distance accumulated is
\(D\).

\textbf{CBBO (Counting Back).} Start at \(M\), subtract base units (from
decomposed \(S\)) then ones; final position is \(D\).

\textbf{Machines.} Two dual register machines are defined.

\textbf{COBO (Missing Addend):}
\(V = \{CurrentValue, Distance, Target\}\)

\begin{longtable}[]{@{}
  >{\raggedright\arraybackslash}p{(\linewidth - 4\tabcolsep) * \real{0.3333}}
  >{\raggedright\arraybackslash}p{(\linewidth - 4\tabcolsep) * \real{0.3333}}
  >{\raggedright\arraybackslash}p{(\linewidth - 4\tabcolsep) * \real{0.3333}}@{}}
\toprule\noalign{}
\begin{minipage}[b]{\linewidth}\raggedright
State
\end{minipage} & \begin{minipage}[b]{\linewidth}\raggedright
Condition
\end{minipage} & \begin{minipage}[b]{\linewidth}\raggedright
Action
\end{minipage} \\
\midrule\noalign{}
\endhead
\bottomrule\noalign{}
\endlastfoot
\(q_{init}\) & - & \(CurrentValue \leftarrow S\);
\(Distance \leftarrow 0\); \(Target \leftarrow M\) \\
\(q_{add\_bases}\) & \(CurrentValue + Base \le Target\) &
\(CurrentValue \leftarrow CurrentValue + Base\);
\(Distance \leftarrow Distance + Base\) \\
\(q_{add\_bases}\) & \(CurrentValue + Base > Target\) & transition to
\(q_{add\_ones}\) \\
\(q_{add\_ones}\) & \(CurrentValue < Target\) &
\(CurrentValue \leftarrow CurrentValue + 1\);
\(Distance \leftarrow Distance + 1\) \\
\(q_{add\_ones}\) & \(CurrentValue == Target\) & Accept \(Distance\) \\
\end{longtable}

\textbf{CBBO (Counting Back):}
\(V = \{CurrentValue, BaseCounter, OneCounter\}\)

\begin{longtable}[]{@{}
  >{\raggedright\arraybackslash}p{(\linewidth - 4\tabcolsep) * \real{0.3333}}
  >{\raggedright\arraybackslash}p{(\linewidth - 4\tabcolsep) * \real{0.3333}}
  >{\raggedright\arraybackslash}p{(\linewidth - 4\tabcolsep) * \real{0.3333}}@{}}
\toprule\noalign{}
\begin{minipage}[b]{\linewidth}\raggedright
State
\end{minipage} & \begin{minipage}[b]{\linewidth}\raggedright
Condition
\end{minipage} & \begin{minipage}[b]{\linewidth}\raggedright
Action
\end{minipage} \\
\midrule\noalign{}
\endhead
\bottomrule\noalign{}
\endlastfoot
\(q_{init}\) & - & \(CurrentValue \leftarrow M\); Decompose \(S\) into
\(BaseCounter, OneCounter\) \\
\(q_{sub\_bases}\) & \(BaseCounter > 0\) &
\(CurrentValue \leftarrow CurrentValue - Base\);
\(BaseCounter \leftarrow BaseCounter - 1\) \\
\(q_{sub\_bases}\) & \(BaseCounter == 0\) & transition to
\(q_{sub\_ones}\) \\
\(q_{sub\_ones}\) & \(OneCounter > 0\) &
\(CurrentValue \leftarrow CurrentValue - 1\);
\(OneCounter \leftarrow OneCounter - 1\) \\
\(q_{sub\_ones}\) & \(OneCounter == 0\) & Accept \(CurrentValue\) \\
\end{longtable}

\textbf{Choreography.} Directional inversion of the same two-phase
rhythm (bases \(\to\) ones). Overshoot detection acts as control
boundary in COBO.

\begin{center}\rule{0.5\linewidth}{0.5pt}\end{center}

\section{Subtraction Decomposition
(Borrowing)}\label{subtraction-decomposition-borrowing}

\textbf{Description.} Left-to-right: subtract higher place (tens),
detect insufficiency in lower place, decompose (borrow) one higher unit
into base smaller units, then subtract ones.

\textbf{Transition Function (\(\delta\)):}

\begin{longtable}[]{@{}
  >{\raggedright\arraybackslash}p{(\linewidth - 8\tabcolsep) * \real{0.2000}}
  >{\raggedright\arraybackslash}p{(\linewidth - 8\tabcolsep) * \real{0.2000}}
  >{\raggedright\arraybackslash}p{(\linewidth - 8\tabcolsep) * \real{0.2000}}
  >{\raggedright\arraybackslash}p{(\linewidth - 8\tabcolsep) * \real{0.2000}}
  >{\raggedright\arraybackslash}p{(\linewidth - 8\tabcolsep) * \real{0.2000}}@{}}
\toprule\noalign{}
\begin{minipage}[b]{\linewidth}\raggedright
Current State
\end{minipage} & \begin{minipage}[b]{\linewidth}\raggedright
Condition
\end{minipage} & \begin{minipage}[b]{\linewidth}\raggedright
Next State
\end{minipage} & \begin{minipage}[b]{\linewidth}\raggedright
Action
\end{minipage} & \begin{minipage}[b]{\linewidth}\raggedright
Interpretation
\end{minipage} \\
\midrule\noalign{}
\endhead
\bottomrule\noalign{}
\endlastfoot
\(q_{init}\) & - & \(q_{sub\_bases}\) & Decompose \(M, S\) into place
values \(R_T, R_O, S_T, S_O\). & Initialize registers. \\
\(q_{sub\_bases}\) & - & \(q_{check\_ones}\) &
\(R_T \leftarrow R_T - S_T\) & Subtract the bases (Tens). \\
\(q_{check\_ones}\) & \(R_O \ge S_O\) & \(q_{sub\_ones}\) & - &
Sufficient ones. No borrow needed. \\
\(q_{check\_ones}\) & \(R_O < S_O\) & \(q_{decompose}\) & - &
Insufficient ones. Borrow. \\
\(q_{decompose}\) & \(R_T > 0\) & \(q_{sub\_ones}\) &
\(R_T \leftarrow R_T - 1\); \(R_O \leftarrow R_O + Base\) & Decompose
(borrow) one ten. \\
\(q_{sub\_ones}\) & - & \(q_{accept}\) & \(R_O \leftarrow R_O - S_O\);
Result \(\leftarrow R_T \cdot Base + R_O\) & Subtract ones and combine
result. \\
\end{longtable}

\textbf{Choreography.} Inversion of sublation: temporal decompression of
a ten into ten ones to restore operability.

\textbf{Lineage.} Builds on internalized carry (from counting) now
executed in reverse.

\begin{center}\rule{0.5\linewidth}{0.5pt}\end{center}

\section{Subtraction Rounding and
Adjusting}\label{subtraction-rounding-and-adjusting}

\textbf{Description.} Dual rounding (e.g., \(M \to M'\) down,
\(S \to S'\) down) yields simplified \(M' - S'\), then contrasting
compensations: add \(K_M\), subtract \(K_S\).

\textbf{Transition Function (\(\delta\)):}

\begin{longtable}[]{@{}
  >{\raggedright\arraybackslash}p{(\linewidth - 6\tabcolsep) * \real{0.2500}}
  >{\raggedright\arraybackslash}p{(\linewidth - 6\tabcolsep) * \real{0.2500}}
  >{\raggedright\arraybackslash}p{(\linewidth - 6\tabcolsep) * \real{0.2500}}
  >{\raggedright\arraybackslash}p{(\linewidth - 6\tabcolsep) * \real{0.2500}}@{}}
\toprule\noalign{}
\begin{minipage}[b]{\linewidth}\raggedright
Current State
\end{minipage} & \begin{minipage}[b]{\linewidth}\raggedright
Action
\end{minipage} & \begin{minipage}[b]{\linewidth}\raggedright
Next State
\end{minipage} & \begin{minipage}[b]{\linewidth}\raggedright
Interpretation
\end{minipage} \\
\midrule\noalign{}
\endhead
\bottomrule\noalign{}
\endlastfoot
\(q_{start}\) & Read \(M, S\) & \(q_{roundM}\) & Start. \\
\(q_{roundM}\) & \(M' \leftarrow\) RoundDown(\(M\));
\(K_M \leftarrow M - M'\) & \(q_{roundS}\) & Round \(M\) down. Store
adjustment \(K_M\). \\
\(q_{roundS}\) & \(S' \leftarrow\) RoundDown(\(S\));
\(K_S \leftarrow S - S'\) & \(q_{subtract}\) & Round \(S\) down. Store
adjustment \(K_S\). \\
\(q_{subtract}\) & \(Temp \leftarrow M' - S'\) & \(q_{adjustM}\) &
Calculate intermediate result. \\
\(q_{adjustM}\) & \(Temp \leftarrow Temp + K_M\) & \(q_{adjustS}\) &
Compensate for \(M\) (Add back). \\
\(q_{adjustS}\) & \(Result \leftarrow Temp - K_S\) (via chunking) &
\(q_{accept}\) & Compensate for \(S\) (Subtract). \\
\end{longtable}

\textbf{Choreography.} Opposed adjustments highlight subtraction
asymmetry: modification of minuend vs.~subtrahend impacts result in
inverse directions.

\textbf{Lineage.} Integrates rounding (addition strategy) and inverse
compensation sequencing.

\begin{center}\rule{0.5\linewidth}{0.5pt}\end{center}

\section{Subtraction Sliding (Constant
Difference)}\label{subtraction-sliding-constant-difference}

\textbf{Description.} Find \(K\) so that \(S + K\) is a base (or
friendly) number; compute \((M + K) - (S + K)\) exploiting invariance:
\(M - S = (M+K) - (S+K)\).

\textbf{Transition Function (\(\delta\)):}

\begin{longtable}[]{@{}
  >{\raggedright\arraybackslash}p{(\linewidth - 6\tabcolsep) * \real{0.2500}}
  >{\raggedright\arraybackslash}p{(\linewidth - 6\tabcolsep) * \real{0.2500}}
  >{\raggedright\arraybackslash}p{(\linewidth - 6\tabcolsep) * \real{0.2500}}
  >{\raggedright\arraybackslash}p{(\linewidth - 6\tabcolsep) * \real{0.2500}}@{}}
\toprule\noalign{}
\begin{minipage}[b]{\linewidth}\raggedright
Current State
\end{minipage} & \begin{minipage}[b]{\linewidth}\raggedright
Action
\end{minipage} & \begin{minipage}[b]{\linewidth}\raggedright
Next State
\end{minipage} & \begin{minipage}[b]{\linewidth}\raggedright
Interpretation
\end{minipage} \\
\midrule\noalign{}
\endhead
\bottomrule\noalign{}
\endlastfoot
\(q_{start}\) & Read \(M, S\) & \(q_{calcK}\) & Start. Target \(S\) for
adjustment. \\
\(q_{calcK}\) & \(K \leftarrow\) CountUpToBase(\(S\)) & \(q_{slide}\) &
Iteratively find the gap \(K\). \\
\(q_{slide}\) & \(M' \leftarrow M+K\); \(S' \leftarrow S+K\) &
\(q_{subtract}\) & Apply the slide \(K\) to both \(M\) and \(S\). \\
\(q_{subtract}\) & \(Result \leftarrow M' - S'\) & \(q_{accept}\) &
Perform the simplified subtraction. \\
\end{longtable}

\textbf{Choreography.} Up-front decompression (deriving \(K\)) enables
single compressed subtraction against a base-aligned subtrahend.

\textbf{Lineage.} Extends RMB gap-finding; anticipates relational
``distance'' framing central to subtraction fluency.

\begin{center}\rule{0.5\linewidth}{0.5pt}\end{center}

\section{Commutative Reasoning (Multiplication
Optimization)}\label{commutative-reasoning-multiplication-optimization}

\textbf{Description.} For \(A \times B\), evaluate heuristic difficulty
of \((A,B)\) vs \((B,A)\); select orientation minimizing cognitive load
(iteration count \& skip difficulty), then perform iterative addition
(skip counting).

\textbf{Transition Function (\(\delta\)):}

\begin{longtable}[]{@{}
  >{\raggedright\arraybackslash}p{(\linewidth - 6\tabcolsep) * \real{0.2500}}
  >{\raggedright\arraybackslash}p{(\linewidth - 6\tabcolsep) * \real{0.2500}}
  >{\raggedright\arraybackslash}p{(\linewidth - 6\tabcolsep) * \real{0.2500}}
  >{\raggedright\arraybackslash}p{(\linewidth - 6\tabcolsep) * \real{0.2500}}@{}}
\toprule\noalign{}
\begin{minipage}[b]{\linewidth}\raggedright
Current State
\end{minipage} & \begin{minipage}[b]{\linewidth}\raggedright
Condition / Heuristic
\end{minipage} & \begin{minipage}[b]{\linewidth}\raggedright
Next State
\end{minipage} & \begin{minipage}[b]{\linewidth}\raggedright
Action
\end{minipage} \\
\midrule\noalign{}
\endhead
\bottomrule\noalign{}
\endlastfoot
\(q_{evaluate}\) & \(H(B, A) < H(A, B)\) & \(q_{repackage}\) & - \\
\(q_{evaluate}\) & (Otherwise) & \(q_{calc}\) & \(Groups \leftarrow A\);
\(Items \leftarrow B\) \\
\(q_{repackage}\) & - & \(q_{calc}\) & \(Groups \leftarrow B\);
\(Items \leftarrow A\) \\
\(q_{calc}\) & - & \(q_{accept}\) & \(Total \leftarrow\)
IterativeAdd(\(Groups, Items\)) \\
\end{longtable}

\textbf{Choreography.} Meta-level selection precedes execution;
commutative symmetry exploited for temporal compression.

\textbf{Lineage.} Builds on C2C / Skip Counting; introduces optimization
layer.

\begin{center}\rule{0.5\linewidth}{0.5pt}\end{center}

\section{Coordinating Two Counts
(C2C)}\label{coordinating-two-counts-c2c}

\textbf{Description.} Foundational multiplication: nested
counting---items within group, groups within total; total
\(T = N \cdot S\) emerges from exhaustive unit enumeration.

\textbf{Transition Function (\(\delta\)):}

\begin{longtable}[]{@{}
  >{\raggedright\arraybackslash}p{(\linewidth - 6\tabcolsep) * \real{0.2500}}
  >{\raggedright\arraybackslash}p{(\linewidth - 6\tabcolsep) * \real{0.2500}}
  >{\raggedright\arraybackslash}p{(\linewidth - 6\tabcolsep) * \real{0.2500}}
  >{\raggedright\arraybackslash}p{(\linewidth - 6\tabcolsep) * \real{0.2500}}@{}}
\toprule\noalign{}
\begin{minipage}[b]{\linewidth}\raggedright
Current State
\end{minipage} & \begin{minipage}[b]{\linewidth}\raggedright
Condition
\end{minipage} & \begin{minipage}[b]{\linewidth}\raggedright
Next State
\end{minipage} & \begin{minipage}[b]{\linewidth}\raggedright
Action
\end{minipage} \\
\midrule\noalign{}
\endhead
\bottomrule\noalign{}
\endlastfoot
\(q_{init}\) & - & \(q_{checkG}\) &
\(G \leftarrow 0, I \leftarrow 0, T \leftarrow 0\) \\
\(q_{checkG}\) & \(G < N\) & \(q_{countItems}\) & - \\
\(q_{checkG}\) & \(G == N\) & \(q_{accept}\) & Output \(T\) \\
\(q_{countItems}\) & \(I < S\) & \(q_{countItems}\) &
\(I \leftarrow I+1, T \leftarrow T+1\) \\
\(q_{countItems}\) & \(I == S\) & \(q_{nextGroup}\) & - \\
\(q_{nextGroup}\) & - & \(q_{checkG}\) &
\(G \leftarrow G+1, I \leftarrow 0\) \\
\end{longtable}

\textbf{Choreography.} Maximal temporal decompression (no compression
yet); establishes structural scaffold for later compression (skip
counting, distributive reasoning).

\textbf{Lineage.} Direct elaboration of counting primitives into nested
loops.

\begin{center}\rule{0.5\linewidth}{0.5pt}\end{center}

\section{Conversion to Bases and Ones (CBO
Multiplication)}\label{conversion-to-bases-and-ones-cbo-multiplication}

\textbf{Description.} Redistribute units among groups so that many
groups become exact base multiples, leaving a compact residual:
\((k \cdot Base) + r\).

\textbf{Transition Function (\(\delta\)):}

\begin{longtable}[]{@{}
  >{\raggedright\arraybackslash}p{(\linewidth - 6\tabcolsep) * \real{0.2500}}
  >{\raggedright\arraybackslash}p{(\linewidth - 6\tabcolsep) * \real{0.2500}}
  >{\raggedright\arraybackslash}p{(\linewidth - 6\tabcolsep) * \real{0.2500}}
  >{\raggedright\arraybackslash}p{(\linewidth - 6\tabcolsep) * \real{0.2500}}@{}}
\toprule\noalign{}
\begin{minipage}[b]{\linewidth}\raggedright
Current State
\end{minipage} & \begin{minipage}[b]{\linewidth}\raggedright
Condition
\end{minipage} & \begin{minipage}[b]{\linewidth}\raggedright
Next State
\end{minipage} & \begin{minipage}[b]{\linewidth}\raggedright
Action
\end{minipage} \\
\midrule\noalign{}
\endhead
\bottomrule\noalign{}
\endlastfoot
\(q_{init}\) & - & \(q_{select\_source}\) & Initialize \(Groups\) array
with value \(S\). \\
\(q_{select\_source}\) & \(N>0\) & \(q_{transfer}\) & Select a
\(SourceIdx\). \\
\(q_{transfer}\) & \(Groups[Source] > 0\) AND not all targets full &
\(q_{transfer}\) & Transfer 1 unit from \(Source\) to next available
\(Target\). \\
\(q_{transfer}\) & (Source empty OR all targets full) & \(q_{finalize}\)
& - \\
\(q_{finalize}\) & - & \(q_{accept}\) & Total
\(\leftarrow \sum Groups\). \\
\end{longtable}

\textbf{Choreography.} Proactive sublation: simultaneous decompression
(source group) and compression (targets) to manufacture base units
early.

\textbf{Lineage.} Multiplicative analogue of RMB and addition Chunking
with explicit inter-group transfers.

\begin{center}\rule{0.5\linewidth}{0.5pt}\end{center}

\section{Distributive Reasoning
(Multiplication)}\label{distributive-reasoning-multiplication}

\textbf{Description.} Decompose \(S = S_1 + S_2\) (heuristically
``easy'' numbers), compute \(N S_1\) and \(N S_2\) (skip counting or
compressed methods), then sum.

\textbf{Transition Function (\(\delta\)):}

\begin{longtable}[]{@{}lll@{}}
\toprule\noalign{}
Current State & Action & Next State \\
\midrule\noalign{}
\endhead
\bottomrule\noalign{}
\endlastfoot
\(q_{split}\) & \(S_1, S_2 \leftarrow\) HeuristicSplit(\(S\)) &
\(q_{P1}\) \\
\(q_{P1}\) & \(P_1 \leftarrow\) IterativeAdd(\(N, S_1\)) & \(q_{P2}\) \\
\(q_{P2}\) & \(P_2 \leftarrow\) IterativeAdd(\(N, S_2\)) &
\(q_{sum}\) \\
\(q_{sum}\) & \(Total \leftarrow P_1 + P_2\) & \(q_{accept}\) \\
\end{longtable}

\textbf{Choreography.} Temporal decompression (factor split) followed by
parallelizable compressed sub-calculations and final recombination.

\textbf{Lineage.} Extends skip counting with heuristic structural
decomposition; precursor to algebraic distributivity recognition.

\begin{center}\rule{0.5\linewidth}{0.5pt}\end{center}

\section{Dealing by Ones (Division --
Sharing)}\label{dealing-by-ones-division-sharing}

\textbf{Description.} Partitive division: distribute single units
round-robin into \(N\) groups until total \(T\) exhausted; per-group
size \(S\) emerges.

\textbf{Transition Function (\(\delta\)):}

\begin{longtable}[]{@{}
  >{\raggedright\arraybackslash}p{(\linewidth - 6\tabcolsep) * \real{0.2500}}
  >{\raggedright\arraybackslash}p{(\linewidth - 6\tabcolsep) * \real{0.2500}}
  >{\raggedright\arraybackslash}p{(\linewidth - 6\tabcolsep) * \real{0.2500}}
  >{\raggedright\arraybackslash}p{(\linewidth - 6\tabcolsep) * \real{0.2500}}@{}}
\toprule\noalign{}
\begin{minipage}[b]{\linewidth}\raggedright
Current State
\end{minipage} & \begin{minipage}[b]{\linewidth}\raggedright
Condition
\end{minipage} & \begin{minipage}[b]{\linewidth}\raggedright
Next State
\end{minipage} & \begin{minipage}[b]{\linewidth}\raggedright
Action
\end{minipage} \\
\midrule\noalign{}
\endhead
\bottomrule\noalign{}
\endlastfoot
\(q_{init}\) & - & \(q_{deal}\) & \(Remaining \leftarrow T\); Initialize
\(Groups\) array to 0s. \\
\(q_{deal}\) & \(Remaining > 0\) & \(q_{deal}\) &
\(Groups[idx] \leftarrow Groups[idx]+1\);
\(Remaining \leftarrow Remaining-1\);
\(idx \leftarrow (idx+1) \pmod N\) \\
\(q_{deal}\) & \(Remaining == 0\) & \(q_{accept}\) & Output
\(Groups[0]\) \\
\end{longtable}

\textbf{Choreography.} Maximal temporal decompression; rhythmic rounds
establish invariant increase pattern (foundation for later compression
insights).

\textbf{Lineage.} Inversion of C2C perspective (constructing equal
groups from total rather than composing total from groups).

\begin{center}\rule{0.5\linewidth}{0.5pt}\end{center}

\section{Inverse Distributive Reasoning
(Division)}\label{inverse-distributive-reasoning-division}

\textbf{Description.} Measurement division \(T / S\): decompose \(T\)
into known multiples of \(S\): \(T = \sum_i (m_i S)\); quotient
\(= \sum_i m_i\).

\textbf{Transition Function (\(\delta\)):}

\begin{longtable}[]{@{}
  >{\raggedright\arraybackslash}p{(\linewidth - 6\tabcolsep) * \real{0.2500}}
  >{\raggedright\arraybackslash}p{(\linewidth - 6\tabcolsep) * \real{0.2500}}
  >{\raggedright\arraybackslash}p{(\linewidth - 6\tabcolsep) * \real{0.2500}}
  >{\raggedright\arraybackslash}p{(\linewidth - 6\tabcolsep) * \real{0.2500}}@{}}
\toprule\noalign{}
\begin{minipage}[b]{\linewidth}\raggedright
Current State
\end{minipage} & \begin{minipage}[b]{\linewidth}\raggedright
Condition
\end{minipage} & \begin{minipage}[b]{\linewidth}\raggedright
Next State
\end{minipage} & \begin{minipage}[b]{\linewidth}\raggedright
Action
\end{minipage} \\
\midrule\noalign{}
\endhead
\bottomrule\noalign{}
\endlastfoot
\(q_{init}\) & - & \(q_{search}\) & \(Remaining \leftarrow T\);
\(TotalQ \leftarrow 0\); Load KB for \(S\). \\
\(q_{search}\) & Found \((P_T, P_Q)\) in KB where \(P_T \le Remaining\)
& \(q_{apply}\) & Select largest such \((P_T, P_Q)\). \\
\(q_{search}\) & No suitable fact found & \(q_{accept}\) & Output
\(TotalQ\). \\
\(q_{apply}\) & - & \(q_{search}\) &
\(Remaining \leftarrow Remaining - P_T\);
\(TotalQ \leftarrow TotalQ + P_Q\). \\
\end{longtable}

\textbf{Choreography.} Temporal compression via retrieval of
pre-compressed multiplication facts; loop greedily subtracts largest
available chunk.

\textbf{Lineage.} Inversion of Distributive Reasoning in multiplication
(switch from constructing product to decomposing dividend).

\begin{center}\rule{0.5\linewidth}{0.5pt}\end{center}

\section{Using Commutative Reasoning (Division via Iterated
Accumulation)}\label{using-commutative-reasoning-division-via-iterated-accumulation}

\textbf{Description.} For \(E / G\) (sharing reframed as measurement):
iteratively accumulate \(G\) until total \(E\) reached; iteration count
is quotient.

\textbf{Transition Function (\(\delta\)):}

\begin{longtable}[]{@{}
  >{\raggedright\arraybackslash}p{(\linewidth - 6\tabcolsep) * \real{0.2500}}
  >{\raggedright\arraybackslash}p{(\linewidth - 6\tabcolsep) * \real{0.2500}}
  >{\raggedright\arraybackslash}p{(\linewidth - 6\tabcolsep) * \real{0.2500}}
  >{\raggedright\arraybackslash}p{(\linewidth - 6\tabcolsep) * \real{0.2500}}@{}}
\toprule\noalign{}
\begin{minipage}[b]{\linewidth}\raggedright
Current State
\end{minipage} & \begin{minipage}[b]{\linewidth}\raggedright
Condition
\end{minipage} & \begin{minipage}[b]{\linewidth}\raggedright
Next State
\end{minipage} & \begin{minipage}[b]{\linewidth}\raggedright
Action
\end{minipage} \\
\midrule\noalign{}
\endhead
\bottomrule\noalign{}
\endlastfoot
\(q_{init}\) & - & \(q_{iterate}\) & \(Acc \leftarrow 0\);
\(Q \leftarrow 0\). \\
\(q_{iterate}\) & - & \(q_{check}\) & \(Acc \leftarrow Acc + G\);
\(Q \leftarrow Q + 1\). \\
\(q_{check}\) & \(Acc < E\) & \(q_{iterate}\) & - \\
\(q_{check}\) & \(Acc == E\) & \(q_{accept}\) & Output \(Q\). \\
\end{longtable}

\textbf{Choreography.} Symmetric inversion of repeated addition
(multiplication) focusing on completion criterion instead of fixed loop
count.

\textbf{Lineage.} Bridges between Dealing by Ones and chunk-based
division (fact retrieval).

\begin{center}\rule{0.5\linewidth}{0.5pt}\end{center}

\section{Conversion to Groups Other than Bases (CGOB
Division)}\label{conversion-to-groups-other-than-bases-cgob-division}

\textbf{Description.} Leverage base decomposition of dividend \(T\)
(e.g., tens \& ones) plus analysis of base/divisor relation:
\(Base = q_1 S + r_1\); process all base units in bulk, aggregate
remainders, finalize.

\textbf{Transition Function (\(\delta\)):}

\begin{longtable}[]{@{}
  >{\raggedright\arraybackslash}p{(\linewidth - 4\tabcolsep) * \real{0.3333}}
  >{\raggedright\arraybackslash}p{(\linewidth - 4\tabcolsep) * \real{0.3333}}
  >{\raggedright\arraybackslash}p{(\linewidth - 4\tabcolsep) * \real{0.3333}}@{}}
\toprule\noalign{}
\begin{minipage}[b]{\linewidth}\raggedright
Current State
\end{minipage} & \begin{minipage}[b]{\linewidth}\raggedright
Action
\end{minipage} & \begin{minipage}[b]{\linewidth}\raggedright
Next State
\end{minipage} \\
\midrule\noalign{}
\endhead
\bottomrule\noalign{}
\endlastfoot
\(q_{init}\) & Decompose \(T\) into \(T_B, T_O\);
\(Q \leftarrow 0, R \leftarrow 0\). & \(q_{analyze}\) \\
\(q_{analyze}\) & \(S_{inB} \leftarrow B // S\);
\(R_{inB} \leftarrow B \pmod S\). & \(q_{processBases}\) \\
\(q_{processBases}\) & \(Q \leftarrow Q + T_B \cdot S_{inB}\);
\(R \leftarrow R + T_B \cdot R_{inB}\). & \(q_{combineR}\) \\
\(q_{combineR}\) & \(R \leftarrow R + T_O\). & \(q_{processR}\) \\
\(q_{processR}\) & \(Q \leftarrow Q + R // S\);
\(R \leftarrow R \pmod S\). & \(q_{accept}\) \\
\end{longtable}

\textbf{Choreography.} Dual decompression (dividend by base, base by
divisor) \(\to\) large compression (bulk quotient) \(\to\) residual
resolution.

\textbf{Lineage.} Division analogue of CBO (multiplication) and
Distributive Reasoning; integrates multi-level structural analysis.

\begin{center}\rule{0.5\linewidth}{0.5pt}\end{center}

\section{Conceptual Dependency Graph
(Narrative)}\label{conceptual-dependency-graph-narrative}

Counting \(\to\) (RMB, COBO) \(\to\) (Chunking, Rounding \& Adjusting,
Sliding) \(\to\) (Subtraction Inversions: COBO/CBBO, Chunking
orientations, Decomposition, Rounding, Sliding) \(\to\) (C2C) \(\to\)
(Skip Counting / implicit in COBO Multiplication) \(\to\) (Commutative
\& Distributive Reasoning, CBO Multiplication) \(\to\) (Division
primitives: Dealing by Ones, Iterated Accumulation) \(\to\) (Inverse
Distributive Reasoning, Fact-Based Decomposition) \(\to\) (CGOB
Division).

Each arrow denotes an algorithmic elaboration where prior compressed
units or reversible decompositions become callable subroutines.

\begin{center}\rule{0.5\linewidth}{0.5pt}\end{center}

\section{Temporal Dynamics Summary}\label{temporal-dynamics-summary}

\begin{itemize}
\tightlist
\item
  \textbf{Primitive Decompression:} Counting by ones; Dealing by Ones;
  C2C (inner loop).
\item
  \textbf{First Compression Layer:} COBO (bases as units); subtraction
  COBO/CBBO; iterative accumulation for division.
\item
  \textbf{Strategic Boundary Forcing:} RMB, Chunking, Sliding, Rounding
  (anticipatory manipulation of base thresholds).
\item
  \textbf{Structural Decomposition / Synthesis:} Distributive Reasoning,
  Inverse Distributive Reasoning, CBO (Multiplication \& Division),
  CGOB.
\end{itemize}

\begin{center}\rule{0.5\linewidth}{0.5pt}\end{center}

\section{Glossary of Symbols}\label{glossary-of-symbols}

\begin{itemize}
\tightlist
\item
  \(Base\): Typically 10 (extendable to other positional bases).
\item
  \(K\): Gap to next base (RMB, Rounding, Chunking, Sliding).
\item
  \(R\): Remainder after decomposition or partial processing.
\item
  \(S_1, S_2\): Split components of a factor (Distributive Reasoning).
\item
  \(Groups, Items\): Multiplicative roles after commutative
  optimization.
\item
  \(Remaining\): Unprocessed portion of a dividend in division
  strategies.
\item
  \(Q\): Quotient / accumulated result in division; also generic state
  set symbol context-dependent.
\item
  \(Acc\): Accumulated total during iterative division.
\end{itemize}

\begin{center}\rule{0.5\linewidth}{0.5pt}\end{center}

End of Clean Draft.

\end{document}
