\chapter{6: Limits of Thought, Deconstruction, and the Voice}

\begin{abstract}
This chapter explores the limits of thought and language through the author's experience of personal loss, philosophical inquiry, and musical expression. Building on the previous chapters' development of determinate negation and intersubjective recognition, the analysis examines the relationship between absence and meaning-making. The chapter draws on Giorgio Agamben's concept of Voice---the pure event of language that subtends all particular utterances---to theorize how meaning emerges at the boundary between saying and unsaying. Several autoethnographic artifacts structure the investigation: a eulogy delivered at the author's father's funeral, a letter written to the father after his death, and original songs that attempt to articulate grief, memory, and the complexities of recognition. These artifacts serve as sites for philosophical reflection on how absence functions as an enabling condition for presence, and how the work of mourning parallels the work of mathematical understanding. The chapter also investigates how meaning emerges through shared practices and interpretations, even when individual expressions seem inadequate to their task. Theoretical discussions interweave with personal narrative, exploring concepts such as différance, reciprocal sense-dependence, and Habermas's knowledge-constitutive interests. The chapter connects these themes to the concept of null representation in mathematics, arguing that accepting absence as a necessary condition of understanding enables a richer conception of mathematical thought.
\end{abstract}

\section{The Eulogy and the Problem of Presence}

I was in the middle of writing my dissertation when my father died suddenly. My mom found him and ran to get me. He was downstairs in the living room. He had been laughing at a baseball game, and then he was not. We delivered CPR until the paramedics came, but he was already gone. I found myself singing \enquote{Brokedown Palace} by Jerry Garcia and Robert Hunter as they took him away, a goodbye song about a river rolling and a mama rocking her baby.

At the funeral, I spoke about my father and tried to articulate what his death meant for those of us left behind. I used the metaphor of light passing through a prism: \enquote{The bright, white light of who you were has been broken by the prism of death into a brilliant rainbow, diffuse, with each of us radiating our own memories of you as discrete shards of your spectrum.} The image captured something true about grief—how the wholeness of a person fragments into the scattered memories held by different people. No single memory can reconstitute the whole.

I also used the metaphor of weaving. I said: \enquote{I am standing here like a woven cloth. The warp is the thread that stays fixed to the loom. It's what holds the structure together. The weft is the thread that moves, that creates the design. Dad runs through my life as the solid threads on which the strange designs of my life are drawn.} He was the constant, the enabling condition, even when I did not recognize it.

I told two memories. The first was recent: a trip to a Sox game where Dad told me the story of Kent Todd, a banjo player who said he didn't play \textit{Muleskinner Blues} anymore because his father had died that year. Later, Dad cried in the car when my sister told him about her partner Terry's cancer diagnosis. He cried so hard the car shook. After the storm passed, there was a rainbow.

The second memory was distant: the Bryan Park pool, where I rode on Dad's back as he plunged down into the water. I remember clinging to him, breaking through the surface together. I remember when he removed my training wheels with that firm hand and swift push. I remember chasing his draft on bike rides, trying to keep up.

The eulogy wrestles with what Agamben identifies as the fundamental problem of Western metaphysics: the relationship between language and death \parencite{agamben_language_2006}. How do I speak to someone who is no longer here? How do I address an absent referent? The diffuse rainbow metaphor attempts to make the absent fully present through the collected shards of memory, but the attempt fails. The whole—the \enquote{white light of a person}—once gone, cannot be reconstituted.
\section{The Negative Foundation: Language and Death}

Agamben draws a crucial distinction between \textit{Voice} and \textit{voice}. The \textit{Voice} (capitalized) is the silent, unsayable ground that makes language possible. It is not a sound. It is the \textit{removal} of sound, the negation of the animal cry, that creates the space for meaningful speech. The \textit{voice} (lowercase), by contrast, is the physical sound made by the body—the vibrations of vocal cords, the breath moving through the larynx.

The Voice is a double negation \parencite{agamben_language_2006}. It is the removal of the animalistic cry that allows for the articulation of meaningful speech. But the Voice itself is not speech. It is the silent ground that speech presupposes. The Voice is \textit{not} that which it enables. It is the \textit{absent} foundation, the negative condition that makes the spoken word possible.

Agamben connects this to Heidegger's concept of \textit{dasein}, which is characterized by being-towards-death. For Heidegger, the \enquote{animal} ceases to live, while the person truly \textit{dies}. Death is not merely biological cessation but the horizon against which meaning becomes possible. The Voice operates in a similar way—it is the absent ground that enables language, just as death is the absent horizon that enables the meaning of a life.

When I sang the songs at my father's funeral and later in the empty basement, I was deeply immersed in the \textit{voice}—the physical act of producing sound. But what I was reaching for was the \textit{Voice}—the unsayable, silent ground that my father's absence had made palpable. The text of the lyrics is, in Derrida's terms, a trace—the absence of music provides a hollowing-out, a space where the Voice should be but cannot be captured.

The Voice is not a presence that can be articulated. It is the \textit{enabling condition} of articulation itself. The diffuse rainbow metaphor in the eulogy inadvertently captures this: the attempt to gather the shards of memory into a unified presence fails because the ground of that presence—the Voice, the silent foundation—cannot itself be made present.
\section{Analyzing the Limits of Knowledge and Reference}

Habermas \parencite{Habermas:1971aa} identifies three \textit{knowledge-constitutive interests} that correspond to different ways humans relate to being and different forms of knowledge. These interests are not merely academic categories but existential orientations that shape how we engage with the world. Carspecken \parencite{Carspecken1999} provides a helpful table that I adapt here:

\begin{table}[ht]
\centering
\small
\begin{tabular}{p{2.8cm}p{3.5cm}p{3.5cm}p{3.5cm}}
\hline
\textbf{Dimension} & \textbf{Empirical-Analytic} & \textbf{Historical-Hermeneutic} & \textbf{Critical-Emancipatory} \\
\hline
\textbf{Motivations} & Tangible states; explaining ``what is'' & Reaching agreements; being understood; recognition needs & Self-understanding; freedom; autonomy \\
\textbf{Actions} & Instrumental & Communicative & Reflection; desire for desire; nonaction \\
\textbf{Knowledge Forms} & Copy knowledge; formalized languages & Hermeneutically explicated; ordinary language & Internal recognition; noncommunicable states \\
\textbf{Relation to Being} & Knowledge/being separated & Knowledge self-transcending; changes reality & Knowledge fused with being \\
\textbf{Knower} & Anonymous universal observer & Individuated self with autobiography & Pure reflection; loved Other or God; or no self \\
\textbf{Reflection} & Restricted & Internal to hermeneutics & Method and object simultaneously \\
\textbf{Validity} & Successful predictions & Insider recognition & Self-formative process continuation; enlightenment \\
\hline
\end{tabular}
\caption{\textit{Knowledge-Constitutive Interests} (adapted from Carspecken, 1999)}
\label{tab:knowledge-interests}
\end{table}

The \textbf{empirical-analytic} interest is motivated by the need to predict and control tangible states. When I performed CPR on my father, I was operating within this interest. I was following an algorithm, applying pressure, counting compressions, attempting to restart his heart. The knowledge involved is \enquote{copy knowledge}—a procedure learned and executed. The knower is an anonymous, universal observer. The validity criterion is success: Did it work?

The \textbf{historical-hermeneutic} interest is motivated by the need to reach agreements, to be understood, to be recognized. The eulogy itself operates within this interest. I was trying to articulate what my father's life meant, how his death should be understood by the community gathered at the funeral. The knowledge involved is hermeneutically explicated through ordinary language. The knower is an individuated self with an autobiography—I spoke as \textit{Ted}, the son, not as an anonymous observer. The validity criterion is insider recognition: Did the community recognize the truth of what I said?

The \textbf{critical-emancipatory} interest is motivated by the need for self-understanding, freedom, and autonomy. This is the interest that emerges in the aftermath of the eulogy, in the solitary moments of grief when I must come to terms with my father's absence. The knowledge involved is internal recognition, often noncommunicable. The knower is either pure reflection or the loved Other (or God), or there is no self at all—the boundaries dissolve. The validity criterion is the continuation of the self-formative process, what Habermas calls \enquote{enlightenment.}

The eulogy attempts to bridge these three interests, but ultimately reveals their limits. The empirical-analytic interest cannot revive the dead. The historical-hermeneutic interest cannot reconstitute the whole from the scattered shards. The critical-emancipatory interest confronts the irreducible absence—the Voice—that cannot be captured in language.
\section{My Masterpiece: The Paradox of Recognition}

Five years before my father died, I wrote him a letter expressing frustration with my life. I do not remember what I wrote, but I remember his response. He sent me the lyrics to Bob Dylan's \enquote{When I Paint My Masterpiece}, a song about feeling unfulfilled and waiting to create something worthy. Then he wrote:

\begin{quote}
Your letter brought this song to mind, like you are feeling unfulfilled and needing, waiting to paint your masterpiece. But you, Theodore Michael Dougherty Savich, you are my masterpiece. Sure, I kept on always trying to please Grandpa after you were born and I still do. But you have made my life a success.
\end{quote}

He then listed predicates: \enquote{smart\ldots creative\ldots good looking\ldots handsome\ldots funny\ldots great guitar player and singer\ldots thinker and an innovator\ldots great teacher because you care.} He was attempting to \textit{recognize} me, to articulate what made me valuable, to give me a sense of my own worth.

But here is the paradox: I did not \enquote{hear} his words until after he was dead. I could not identify as a value for $x$ in the predicates \enquote{$x$ is smart,} \enquote{$x$ is creative,} or \enquote{$x$ is handsome.} The letter failed to land not because its predicates were false, but because the living subject it addresses can never be fully present as an object of description. The \{I\} exceeds any finite list of predicates.

There is also a reciprocal sense-dependence at work here \parencite{brandom1994making}. My father wrote, \enquote{You have made my life a success.} For him, to talk about himself was to talk about me. To talk about me was to talk about him. The essentiality flows in both directions. He could not be who he was without me, and I could not be who I am without him. But this mutual dependence cannot be captured by a simple list of attributes.

The letter reveals the structure of recognition as inherently paradoxical. Recognition requires treating the other as an object that can be described, categorized, and understood. But the other—the \textit{in}finite \{I\}—resists being fully captured by any description. The predicates my father used were not wrong. They were \textit{insufficient}. The gap between the predicates and the person they are meant to describe is the trace of the absent referent, the \textit{in}finite that cannot be made present.
\section{Reflection on Deferral: Différance and the Trace}

Quine \parencite{quine_what_1948} famously addressed the problem of how to talk about non-being without attributing being to that which is named. His solution was to shift the burden of existence from names to \textit{ontological commitments} and linguistic self-reference. The name \enquote{Pegasus} exists as a linguistic entity, but we can deny that the variable $x$ in the sentence \enquote{$x$ is a Pegasus} attains any value: $\neg \exists x$ such that $x$ is a Pegasus.

Quine's dictum—\enquote{To be is to be the value of a variable}—takes on new significance in this context \parencite{quine_what_1948}. The variable $x$ is what Agamben calls a \textit{shifter} \parencite{agamben_language_2006}. A shifter is a bit of language (like \{I\} or a variable) that does not simply refer to an object. Instead, it refers to the \textit{event} of language. The variable $x$ is the pure pronoun that does not refer to any particular object but to the event of language that the Voice makes possible.

Derrida's concept of \textit{différance} \parencite{derrida1982margins} captures the dual movement at work here. Différance (with an \textit{a}) is the simultaneous \textit{deferral} and \textit{differing} of meaning. Meaning is always deferred because it depends on absent contexts, on traces of what is not present. Meaning is always differing because each new context alters what the words signify.

My father's letter failed to land immediately because the meaning it attempted to convey was \textit{deferred}. The recognition he offered required a context I did not yet have—the context of his death, the context of my own maturation, the context of understanding what it means to be someone's masterpiece. The letter carried the \textit{trace} of the absent referent—the \{I\} that cannot be fully captured by predicates, the \textit{in}finite that resists being made finite.

The gap between when my father wrote the letter and when I \enquote{heard} it (five years later, after his death) is not a failure of communication. It is the structure of meaning itself. Meaning is never fully present. It is always deferred, always differing, always carrying the trace of what is absent. The Voice—the silent, unsayable ground—is the condition of possibility for this deferral and differing. It is what allows language to mean anything at all, precisely because it is not-speech, not-presence, not-being.
\section{Integration: The Song of Home}

After my father died, I wrote songs. The first, \textit{Love's Memory}, began while he was still alive, but I only understood what I had written after his death. The song moves through images of autumn, dead oceans, polar winter, and the delayed recognition of his letter. Between verses, I yodel and whistle—inarticulate vowel sounds that attempt to express what words cannot.

The second song, \textit{Still Feels Like Home}, was written after I stripped the basement bare following his death. The concrete floors made a resonant chamber. I was trying to make the sound of the Voice—the unsayable—as big as possible. The song describes cutting my hand on a carpet knife, tearing out mold-dark night, breaking my crown soaking up storm light. The chorus speaks of letting summer wind blow rain in, of curtains dripping on the floor, of things breaking with every storm.

But then comes the bridge: \enquote{But other words blow in on the same warm wind / Filling cracks with honey dripping from the comb / It still feels like home.} Musically, the song begins ploddingly in the key of D\#. When it reaches \enquote{But other words blow in}—the Voice—it modulates down to the key of C\# as the melody ascends. The harmonic descent creates space for the melodic ascent, mirroring the structure of the Voice as the negative foundation that enables positive articulation.

Agamben writes that the Voice does not will any proposition or event; it wills \textit{that language exist}, it wills the \textit{originary event} that contains the possibility of every event \parencite{agamben_language_2006}. Philosophy, for Agamben, is the human word's \textit{nostos}—its return from itself to itself. After becoming meaningful discourse, it returns in the end, as absolute wisdom, to the Voice.

The home is not about real estate or biological embodiment. It is the feeling of returning home to the I-feeling. Recognizing the event of language as home lends Descartes' \textit{sum}—the linguistically non-falsifiable \enquote{I am}—the feeling of self-certainty. When I sang these songs in the empty basement, filling the space with sound, I was attempting a \textit{nostos}, a return to the ground that my father's absence had made palpable.

If Agamben's Voice is the unsayable, pre-linguistic ground that allows language to exist, then Quine's variable is the fundamental tool of reference within that existing language. I read Quine's variable $x$ as a formal shifter. It is the pure pronoun that does not refer to any particular object but to the event of language that the Voice makes possible. \enquote{To be is to be the value of a variable} thus signifies that to be is to be something that can be spoken of, something that can enter the linguistic field opened by the Voice.
\section{Conclusion: To the Bridge}

This chapter has explored the limits of thought and language through the experience of loss. The eulogy attempted to make an absent referent present through the metaphor of a diffuse rainbow—scattered shards of memory that cannot reconstitute the whole. Agamben's distinction between Voice and voice revealed that language depends on a silent, negative foundation—a removal of the animal cry that creates the space for meaningful speech. Habermas's three knowledge-constitutive interests showed how different forms of knowledge relate to being, each with its own limits and validity criteria.

My father's letter revealed the paradox of recognition: the attempt to capture the \textit{in}finite \{I\} through finite predicates necessarily fails, not because the predicates are false but because they are insufficient. Derrida's différance and Quine's variables as shifters illuminate the structure of meaning as deferral and differing, as a trace of the absent referent that can never be made fully present. The songs I wrote—with their inarticulate vowel sounds and harmonic descents creating space for melodic ascents—attempt to articulate the Voice, the \textit{nostos} or homecoming to the unsayable ground.

What emerges from this analysis is a recognition that \textit{absence is not a deficiency}. The Voice is not a lack that needs to be filled. It is the enabling condition for all articulation. The gap between my father's letter and my hearing it is not a failure of communication but the structure of meaning itself. The diffuse rainbow is not a failed attempt at wholeness but an accurate representation of how meaning is distributed across contexts and deferred across time.

This understanding prepares us for the next chapter, where we turn to a mathematical formalization of these ideas through Dr. Seuss's \textit{The Sneetches} and Georg Cantor's diagonal proof. Just as the Voice is the absent foundation that enables language, the empty set is the absent foundation that enables mathematical structure. Just as shifters refer to the event of language rather than to objects, variables refer to the event of mathematical discourse rather than to numbers. The pattern of sublation—preservation, negation, and elevation—that we have traced through autoethnographic artifacts will now be traced through the history of mathematical proof.

The bridge we are building connects the existential need to be recognized as both finite and \textit{in}finite with the formal structures that mathematics uses to express similar tensions. Recognition requires treating persons as objects with predicates, yet persons exceed any finite list of predicates. Mathematical systems require treating infinities as objects with properties, yet infinities exceed any finite representation. Both domains reveal the necessity of the negative, the absent, the null—not as failures but as enabling conditions.

\printbibliography[heading=subbibliography]