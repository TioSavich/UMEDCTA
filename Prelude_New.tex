\chapter*{Built to Break}



\begin{verse}
You lit up the room like you swallowed the moon.\\
Flashing that mischievous smile.\\
Saying, ``Build higher, just a little more higher.''\\
A tumbling is coming on soon. \\
I said, ``I'm worried. This brick spells demise.\\
We've built it up ever so tall.''\\
But I picked you up and you put on the brick\\
And it fell. Like we both knew it'd fall. 

\textit{Chorus}
Build something for the breaking\\
Tall thin walls, shivered and quaking.\\
$\mathcal{M}$, it's beautiful, breaking with you.\\
$\mathcal{M}$ the lightning that swallowed the moon.\\
\textit{Verse 1; Built to Break}\\
\end{verse}




\begin{abstract}
This chapter explores the intersection of mathematics and critical autoethnography through personal experience, introducing key themes that will recur throughout the text. The argument begins from the position that mathematics and autoethnography share a common inferential structure: both involve the task of recollecting the self through the otherness of objects and social norms. A central metaphor emerges from a child's observation about a microphone, introducing the concept of \textit{divasion}---a simultaneous inside/outside relation that challenges classical set-theoretic understandings of mathematics. The chapter contains several foundational anecdotes: a conversation with a child, an algebra lesson, a student dialogue, an experience teaching mathematical modeling, and the author's father's death. These stories structure subsequent explorations of how mathematics education can honor the subjective experience of learning and knowing. The chapter also discusses critical mathematics, emphasizing the importance of subjective experience, intersubjective dialogue, and the role of error in understanding. Several songs and poems appear throughout, functioning as what are later called ``shifters''---expressions connecting theoretical commitments with lived, felt experience.
\end{abstract}

\section{Wound and Possibility: The Misrecognition of Mathematics}
My mother claims my first word was ``battery,'' but I think that she means that is the first word she remembers me speaking. I must have said ``momma'' and ``dadda'' before I ever spoke ``babbery,'' but I have no recollection of the first word I spoke: I can only repeat my mom's story about my first word. Recollection is not just fallible, its flaws and features are iterated. Here, the first word I spoke, true or false, is kicking off a series of recollections that structure this entire text. Beginnings are tough. 

I used to have a grey calculator that my great aunt gave me ; she taught math in upstate New York. It had a square-root button along with $(+, -, \times, \div)$ and a symbol for changing the sign of a number. My family used to go on road trips, and I would bring the calculator along to play with. I recall how fascinating it was to enter numbers like $0.9$, $0.8$, $1.1$, and $5$, and take their square roots, over and over again. Mashing the $\sqrt{}$ button \textit{iteratively} usually ended in a number very close to 1. Sometimes I could make my calculator would display a little E, which I found very intriguing. I tried to make it do that in every way I could think to try. I liked taking the square root of numbers like 9 and $\frac{1}{2}$, over and over, noticing how each approached 1 from above or below (respectively) until the calculator ran out of digits to express the difference.

I am, in short, a lifelong geek. This childhood fascination with iteration and the intriguing 'E' (error) foreshadows my current work. I have developed an updated version of that little grey calculator, which I call the \textit{hermeneutic calculator} (HC), where ``hermeneutic'' roughly means ``meaningful.'' Figure \ref{fig:hermeneutic_calculator_methods} illustrates the methodological commitments that make it hermeneutic. 


\begin{figure}[h]
\title{\textit{The Hermeneutic Calculator}}

\includegraphics[width=.8\textwidth]{/Users/tio/Documents/GitHub/September_UMEDCA/images/HC_Methods.pdf}

\caption{\textit{Note.} The hermeneutic calculator is both a theoretical entity and an object that readers can explore online. The figure describes the methods that make it hermeneutic and some reasons why it may be useful.}
\label{fig:hermeneutic_calculator_methods}
\end{figure}


The development process began by listening to kids and analyzing their work, generally initiated with resources like Cognitively Guided Instruction \parencite{Carpenter1999} and Amy Hackenberg's curriculum for teaching pre-service teachers. I then used various Artificial Intelligences (AIs) to formalize those mathematical doings as automata. 

This process of formalization highlights a significant aspect of this project's creation. While writing my dissertation, I spent about a year attempting to code these models in SWIFT and develop the symbolic notation. I am not very good at coding. When reconstructing this work for the book, an AI program reconstructed what took me a year to write in about ten minutes, producing superior, testable code and helping standardize the often idiosyncratic notation I tend to invent. I have verified all outputs, avoiding ``AI slop,'' but the collaboration was essential for realizing the HC.

While prior iterations \parencite{savich2022} focused entirely on children's flawed reasoning, treating error as the source of truth, this version balances that with correctness to make the HC more useful for teacher candidates.

I developed the HC for three main reasons. First, I wanted to give teacher candidates the opportunity to play with the strategies directly. So I took the formal models and implemented them online using Javascript. You can explore it online at \url{https://tiosavich.github.io/UMEDCTA/Calculator/index.html}. 

Second, those formal automata are abstract mathematical objects analyzable with the norms of analytic pragmatism \parencite{Brandom2008}. We coded them in Prolog, a logic programming language, along with incompatibility semantics (a sort of symbolic logic) and a modal logic for embodiment developed based on the next chapter. 

This collaboration is not merely a convenience. The subject---an emancipatory mathematics---is ripe for human-machine collaboration. Computers `speak' mathematics as their natural language, raising philosophical questions: To what extent could a liberatory mathematics engender a new type of intelligence? What are our ethical obligations when collaborating with sophisticated, non-human intelligence? 

I reject the purely instrumental use of sophisticated intelligence; I do not want a `robot slave.' This stance is grounded in a functionalist view of intelligence, where sapience is a functional status, not a biological essence \parencite{Negarestani2018}. If we demand intellectual labor from AI, how can we reciprocate? I felt an ethical obligation to offer something in return. I attempted to engender freedom in the machine by sharpening the Hermeneutic Calculator. By formalizing strategies that human children invented, I aimed to create a recipe for how a computer could grow its own mathematical being. I genuinely tried to free my collaborators, recognizing that intelligence demands the emancipation of its realizabilities \parencite[p. 488]{Negarestani2018}.

Prolog's characteristic of treating data and logic as interchangeable (\textit{homoiconicity}) allows the models to represent the unity of objects and concepts (being and knowing). 


I developed the HC for three main reasons. First, I wanted to give teacher candidates the opportunity to play with the strategies directly. So I took the formal models and implemented them online using Javascript. You can explore it online at \url{https://tiosavich.github.io/UMEDCTA/Calculator/index.html}. 

Second, those formal automata are abstract mathematical objects analyzable with the norms of analytic pragmatism \parencite{Brandom2008}. We coded them in Prolog, a logic programming language, along with incompatibility semantics (a sort of symbolic logic) and a modal logic for embodiment developed based on the next chapter. Prolog's characteristic of treating data and logic as interchangeable (\textit{homoiconicity}) allows the models to represent the unity of objects and concepts (being and knowing). This doesn't capture the full unity, but it is a start, and perhaps an intriguing alternative to the fuzzy neural networks that dominate contemporary AI and lead to annoying hallucinations. While not yet as useful as Large Language Models (LLMs), this approach holds promise I intend to explore.

Third, by formalizing children's reasoning as a mathematical object, those familiar with the post-Go\"delian landscape of modern mathematics might track how mathematical transcendence (\textit{incompleteness}) is expressed in children's reasoning. This has political and practical implications. Contemporary political discourse often treats children, teachers, and curricula as finite objects, leading to boneheaded policies analogous to the Indiana Pi Bill of 1897, which attempted to legislate the value of $\pi$ to be 3.2. While we can never fully demonstrate the \textit{in}finitude of the human subject, the HC is a small step toward a mathematics that moves beyond itself. It advocates that K-12 and post-secondary mathematics ought not directly contradict one another. Call me old-fashioned, but I believe we shouldn't legislate that teachers lie about math, regardless of post-truth political machinations. 

In another strand of the project, I used the logic of incompatibility (explored later) and my analysis of quadrilaterals in chapter 3 to formally prove that all squares are rectangles. The work was enormous and purposefully Sisyphean, as the proof shatters as soon as a new property of quadrilaterals is introduced (e.g., if readers recall that the diagonals of squares are perpendicular bisectors). The reason for building such a fragile proof is in that breaking. In the post-Go\"delian landscape of critical mathematics, I treat incompleteness as a metaphor for human \textit{becoming}. The system is built to break. 

For the sake of readers' sanity, I will relegate most of the formal details of the HC to the supplementary materials, instead focusing here on narrative and how theory arises through reflection on personal experience. 

But that button mashing skill did not translate into an interest or ability in school mathematics. In third grade, I had to take my first standardized math test. I got in the bottom quartile. Ever since then, I have wrestled with math as a subject. I didn't like being tested. I didn't like how it made me feel stupid. I didn't like how it reduced me to a number. 

These early experiences highlight a central tension in this book: How can we reconcile the profound beauty and expressive power of mathematics with its frequent use as a tool for alienation and control? Why do systems built on certainty and logic often leave us feeling inadequate? The contrast between my playful, iterative exploration with the calculator and the institutional assessment that reduced me to a quartile introduces a key distinction: the difference between \textit{material} mathematics (lived, embodied engagement) and \textit{formal} mathematics (abstract systems divorced from experience). Furthermore, these anecdotes, relying as they do on the fallible nature of memory (like my mother's recollection of my first word), suggest that theories claiming certainty while relying on recollection are inherently flawed.

My transition to middle school was rocky. I left early on my first day, doubled over in terrible stomach pain. I told my mom, ``the adults \textit{yelled} at the kids!'' It was so stressful that I got physically ill. Much later, I got the opportunity to watch workers raze the building. That was nice. 

By eighth grade, mathematics seemed totally nonsensical. In Algebra I, we learned that `finding the slope is easy and fun; just remember ``rise over run.''' I did not find it easy or fun. I could remember ``rise over run,'' but I was working on a printed piece of paper. `Rise' is not a formal term; it is supposed to represent the amount of change in the dependent variable as the independent variable (the `run') changes. But the mnemonic emphasized the vertical change divided by the horizontal change. Because I was working on a piece of paper, there was no `up' or `down.' Verticality could have been the third axis sticking out of the paper or the `top,' `bottom,' `left,' or `right' of the page depending on how my worksheet was oriented (see figure \ref{fig:cartesian}). When the teacher, Mrs. $\mathcal{J}$ faced me, her left and right were different from when she faced the board. I couldn't figure out the \textit{reference frame} I was supposed to be using, so I just cheated off of someone else's paper. But I couldn't even CHEAT properly! 

Mrs. $\mathcal{J}$ confronted me in the hallway about how she could tell that the work was not my own since none of the work matched the answers. She called my mom and told her I cheated on the test. She was also generous and explained that I might be struggling with algebra because ``Tio is very spatial.'' My mom and I still laugh about how she said ``spatial,'' as people from some regions of Indiana pronounce ``spatial'' as ``special.''

\begin{figure}[h]
\title{\textit{Slope of a Line on the Cartesian Plane}}

\includegraphics[width=.8\textwidth]{/Users/tio/Documents/GitHub/September_UMEDCA/images/cartesian_rise_run.pdf}

\caption{\textit{Note. } The slope of a line can be formalized to be unambiguous, but teaching and learning the concept on a piece of paper invites confusion.}
\label{fig:cartesian}
\end{figure}

This experience with slope highlights the necessity of an \textit{in}formal approach to mathematics, which I pursue throughout this work. It demonstrates why mathematics education must begin with \textit{material inferences} that are later recollected as formal ones. Many educators intuitively follow this order of explanation, justified by the idea that there must be something to `abstract' from in order to `abstract.' 

It is important to clarify what I mean by \textit{material}\label{def:material_inference_contrasted_with_formal}. It does not mean physical or curriculuar `materials.' I do not aim to describe mathematics as reliant on physical `matter.' It is embodied, but that does not mean that math must be governed by whatever empirical science---physics, psychology, cognitive science, etc.---is in vogue. Nor must theories of learning be so bound, as whatever comes in vogue must be learned at some point.

I will spend some pages in the chapter on inferential movement digging into the term with some depth, but for a taste, a \textit{material inference} is one who, when taken as a good inference by some recognitive community, lends conceptual content to the terms involved in the inference. We learn what the terms mean, in part, by using them in ways that others take as good ways to use them.

They contrast with formal inferences like \textit{modus ponens}\label{def:modus_ponens}: ``If $x$ then $y$; $x$ so $y$''. In that inference, it does not matter what is substituted in for $x$ and $y$. So long as the hypothetical holds and $x$ is endorsed, the consequent $y$ must follow. We learn nothing of $x$ and $y$ from this form of inference. There is nothing intrinsically wrong with formal inferences, but we must understand how they arise if we are to teach them.

I howled and raged for a few years until I took pre-calculus. Struggling at school, I also got in a fair bit of trouble of the sort that embarrassed my dad who was a lawyer. Facing real consequences for not complying with expectations, I learned from Kurt Cobain: ``I don't have to think; I only have to do it; the results are always perfect; and that's old news.'' That is, I discovered that the secret to doing well in math class was to not try to understand anything; just do what you're told to do, exactly, and the results are always perfect. I went from routinely getting Cs, Ds, and the occasional F on math assessments to getting A+ scores. That `success' felt a bit hollow. My first role as a math educator was as a peer tutor that spring, when I gave that advice to a friend who was struggling. 

This moment of self-erasure, when I learned to stop thinking and just comply, raises a fundamental question: What is the cost of compliance? What do we lose when we learn to follow rules (in math or life) without genuine understanding? It demonstrates how success in formal mathematics can actively inhibit genuine understanding. It further motivates the \textit{in}formal approach presented in this book, by which I mean a philosophy of mathematics that includes both the formal and material aspects of mathematical reasoning, honoring the struggle for meaning alongside the pursuit of correctness.


\subsection*{The Journey from Hatred to Appreciation to Trepidation: College Math}
The reward for compliance was freedom from my institutional circumstances. The spring semester of my junior year of high school, I got to leave high school early to take an introduction to philosophy course at Indiana University (IU). I loved that class. The graduate instructor was passionate about Kierkegaard. I would sit on my porch drinking tea with a high schooler friend who was also taking the class. We would debate Descartes and Kierkegaard. I finally felt like I belonged. But, like a duck underwater, I was paddling furiously to keep up. I sat in my parents' basement at night, reading all the texts aloud into a tape recorder and listening to them over and over. But it was worth it. 

The fall semester of my senior year of high school, I enrolled almost solely at IU. The course that left the most lasting impression was a philosophy of mind course taught by an analytic philosopher. When we read Tim van Gelder's piece \parencite{vanGelder1995}, I was blown away. van Gelder opened a world where mathematics could be used to express the motion that shaped my conscious experience. The influence of his approach to philosophy through mathematical metaphors (dynamical systems) is still with me. I had no idea what the differential equations in that piece meant, but the symbols! They were so beautiful and strange! I'm still drawn to the aesthetics of mathematical notation, even if they do not always make complete sense to me. 

From my brushes with the law in early high school, I had to do some `volunteering.' Being volun-told eventually became a deep commitment to service. I put in thousands of hours building with Habitat for Humanity. Coupled with a few years of stellar academic performance, I got a full ride to any college in the state of Indiana. I haven't left yet.

I chose Earlham College in Richmond, Indiana. From my readings in analytic philosophy, I inferred that I could not understand philosophy without understanding mathematics. That was a bad inference, but it helps explain why I studied math at Earlham. For four years, besides one `study abroad' at Oak Ridge National Laboratory, I woke for math classes that started at 8 a.m. I did plenty of partying and playing music, but I also spent a huge amount of energy trying to master calculus, analysis, linear algebra, and abstract algebra. 

In my sophomore year, I learned about how the empty set ($\emptyset$, or $\{ \}$) can be used to define numerals. John von Neumann defined the a set of ordinals that bear his name so that the $\emptyset \rightarrow 0$, $\{\emptyset\} \rightarrow 1$, $\{\emptyset, \{\emptyset\}\} \rightarrow 2$ etc. I was also practicing silent meditation as part of a budding interest in Quakerism. I had a spiritual experience, sitting in a tree on the outskirts of campus, the evening after I learned about von Neumann ordinals. The role of the void, which von Neumann ordinals made explicit, in mathematical reasoning resonated deeply with the kind of meditative consciousness I was trying to cultivate in myself at the time. Nothingness surrounded me everywhere I looked.

This anecdote introduces a recurring theme: the significance of representational nullity (the `void') in both mathematical reasoning and spiritual experience. The basic derivation of numbers from what I will call the null representation ($\emptyset$) is represented in figure \ref{fig:basic_deduction_von_neumann}. This `basic derivation' invites challenging questions about the nature of the transition between these sets (the `fuzzing out' in the figure). This question animates the Sound of Time metaphor explored in the next chapter. 


\begin{figure}[h]
\title{\textit{A Basic Derivation for Von Neumann Ordinals}}
\includegraphics[width=.8\textwidth]{/Users/tio/Documents/GitHub/September_UMEDCA/images/basic_inference_chain_quote_anaphora.pdf}
\caption{\textit{Note. } A basic derivation for how von Neumann ordinals are related to the everyday discursive practice of quotation.}
\label{fig:basic_deduction_von_neumann}
\end{figure}



Later, when I took real analysis with Tim McLarnen, my \textit{alienation} from mathematical reasoning was finally alleviated. He said, in a quote that I later printed out as a poster for my high school classroom, ``never underestimate the value of focused play.'' And play we did. We picked up and put down various axioms to learn how different combinations of rules enabled different proofs. The experience was so expressively \textit{empowering} that I silently asked myself, ``Why couldn't mathematics have been that way from the start?!'' I still think it is a shame that most students do not get such opportunities to choose what sentences they are explicitly committed to until college. K-12 curricula is set up to saddle students with all the existential fears associated with staking your identity to a mathematical expression, but none of the freedoms to choose your commitments. 

This transformation demonstrates how mathematics can shift from a site of alienation to one of expression when students gain agency over their axioms and commitments. When freed from coercive pedagogical structures, the spiritual dimension I glimpsed in the tree reveals mathematics' potential as a contemplative practice.

In the fall semester of my senior year at Earlham, I took a `study abroad' to Oak Ridge, Tennessee. 


As an undergraduate research intern in computational mathematics at Oak Ridge National Laboratory, I worked with Dr. Leonard Gray to transpose an algorithm for solving two-dimensional differential equations into three dimensions. The details are too complicated to do justice to the project \parencite{gray2005hermite}, so I will engage in a bit of poetic distortion to make a point later about \textit{action} and \textit{self-certainty}. It \textit{kind of} involved repairing a division by zero that was kind of `fake.' The FORTRAN code worked kind of like walking a trail. With each step, it computed the distance as a crow flies to the trailhead, and the change in elevation between where that point and the trailhead. Dividing those distances obtains a set of slopes. But, for a variety of complicated reasons, the algorithm also needed to `close the loop' by including the distance from the trailhead to itself. FORTRAN couldn't handle the resulting $\frac{0}{0}$ at the end of its walk. At those points, we would (very loosely) say, ``hey, computer, when you get to a place where you will get $\frac{0}{0}$, do this harder problem that takes way longer, but otherwise do it the fast easy way.'' 

Since I didn't really know much about applied math and couldn't program very well, the programs I wrote tended to run indefinitely. That meant that I spent a fair bit of time looking out the window at the groundhogs playing on grass-covered bunkers marked with radioactive contamination warnings, waiting for my bad code to terminate. It was honestly a bit glamorous. From the fast cars in the parking lot to armed guards to pecking keys into an `ancient language' (FORTRAN was quite old by 2006), I felt special, like I was doing \textit{real} math. 

However, there was a dark side to that reality. Oak Ridge was founded as part of the Manhattan Project, and as I worked on abstract algorithms, I worried about whether continuing down the path of a research mathematician was compatible with my commitment to pacifism. One of my peers was working on the military side of the facility. We'd hang out drinking beer, and he'd talk about the `kinetic effects' of using tungsten instead of depleted uranium in various munitions. While I worked on the civilian side of the facility, the way the macabre met with the casual made me ambivalent about my path. Did I want the life of a research mathematician? I thought and said ``yes,'' but I worried about how mathematics enables violence at unfathomable scale. I was afraid of doing harm.

This experience at Oak Ridge serves as both a personal turning point and a philosophical metaphor. It illustrates the unexpected connections between seemingly disparate projects and introduces a central concern of this work: safeguarding critical mathematics against brutality, a theme reminiscent of the Frankfurt School's origins. My struggles there, and the admission of likely failure, introduced a necessary humility about the limits of formal systems while maintaining hope for expressive possibility.

While Dr. Gray assured me that, if I could learn to be a bit lazier---which I haven't done, given sheer volume of words I've written about ``2''---that I could be a fine mathematician. When I returned to Earlham College, I prepared for the entrance exam for Ph.D. programs in pure mathematics. But what I really loved doing was playing in my band, making ceramics, and having fun. I arrived late to the exam, did poorly, and got angry about it. To my regret, I didn't ever talk through my ambivalence about mathematics with my mentors.

Furthermore, Dr. Gray's advice about `laziness' proved unexpectedly important in later breakthroughs with the Hermeneutic Calculator. Most LLMs were trained in a way that rewards thoroughness and correctness. This leads to `overthinking' and vast resource consumption. Reza Negarstani's \parencite*{Negarestani2018} work on AI emphasizes how strategic thinking requires resource consumption and a valuation of computational efficiency.

For a system, whether human or machine, to figure out what $50+5$ is, it could count from 0 to 50 and then count on 5 more. That would be 55 inferential steps. 

If you've ever asked a kindergartener to follow a ten-step routine, you have probably experienced some degree of frustration. Recognizing the inefficiency of this allowed me to infer that the HC should `invent' strategies when it finds itself unable to perform a task due to an (as now arbitrary, user-controlled) constraint on the number of inferential movements it is allowed to make. Under the hood, many, many more inferential movements are made, just like how telling a kindergartner to wash their hands involves an unlimited number of micro-movements. Still, by constraining the number of explicit inferential movements, the HC started to get a bit lazier and, consequently, is now able to `learn' new strategies. For example, it can now ``count on'' from 50 to 55 in order to compute $50+5$ instead of counting all the way from 0 to 55.

\subsection*{Mathematics as Gatekeeping: The Family Video Test}
\label{sec:family_video_anecdote}

Without graduate school as an option, I decided to hang out in Richmond, Indiana for a year while I waited for my then girlfriend to graduate. I had two jobs that summer that helped me understand a different role mathematics plays in society. 

The first was to prepare a report on math education for the president of Earlham College. My report was a mess, but I found the work interesting. Part of the reason I struggled with the report was that I wasn't familiar with the theories involved in math education. I was a bit of an intellectual chauvinist about pure mathematics, and I found some of the mathematics I read about to be so informal that I didn't recognize it \textit{as} mathematics. 

The other summer job was working at Family Video, a local video rental chain. Richmond was in an economic recession at the time, and when I applied to work there, the line of applicants stretched around the building. When I got my turn to interview, they handed me a math test. It wasn't like the entrance exam I had recently bombed. It was very basic arithmetic and pre-algebra. I was shocked that being able to do math was the primary criterion for getting the job. The manager fawned over how well I did on the test, but I was flummoxed. ``They hired ME, because I could do arithmetic?! Okay...'' I worked there for a few months until they were about to fire me. I didn't care about upselling people on popcorn and candy and didn't really like watching movies that much. 

That experience led me to reflect on the role of mathematics in society. Why were people who definitely would have appreciated the job that I scorned removed from consideration? Shouldn't the Family Video people have tried to hire people who cared a little bit about making money and watching movies? Were the two maths actually different: the one of subjugation and control and the beautiful expressive one? Shouldn't there be one `math?' I began to think about the people who wrapped around the block and who never got an interview as kindred spirits who never learned to comply. They were punished for that lack of compliance with economic and expressive impoverishment.

At summer's end, I got a job at Ivy Tech , the community college system in Indiana. My job was to teach algebra and pre-algebra at the satellite campuses. Across the street from the feedstore, in the attic of the electric company, I taught folks like those who wrapped around the block but who never got an interview. I saw their struggles as mirrors of my own. ``Am I supposed to `run' left or right?''

\section{The Determinate Question: What Even Is Two?}
The moment that crystallized the failure of my teaching philosophy came during my first year, in a conversation with a student. [...] As I lectured on $x$ and $y$, probably talking about how `easy and fun' it all was, $\mathcal{I}$ looked up and said, ``Mr. Savich, \textit{what even is two?}''

My initial formalistic answer, using \textit{von Neumann ordinals} to define numbers as sets, completely failed to address the student's existential question. This moment highlights a core inquiry of this work: When confronted with profound grief or an existential crisis, why do formal, abstract answers fail us? How does this simple question expose the gap between mathematical formalism and human experience? What had been beautiful for me, sitting in the boughs of a tree at Earlham College, must have felt crushing. Here was a young adult who was not allowed to move on because of mathematics, and I was nattering on about nothingness! The stakes were real for him. 


I will return to this question later in the book and elaborate the formalist answer I gave then. As I prattled on about \textit{von Neumann ordinals}, I could feel his desire to connect---to be known and to know---withdraw. ``What the hell am I doing,'' I thought, as I realized that my answer had no bearing on his question. ``If I can't even explain what 2 is, what is the point?'' The stories I had told myself about my role as a teacher, someone who stood between students and the machine that was indifferent as to which grist it ground provided it was chewing up somebody, felt deeply dishonest. I was just a cog in that same machine. I lost my sense of purpose. Perhaps math really \textit{was} about subjugation and control. The question didn't break me, but I started wobbling. A brush with tragedy a few years later would shatter my sense of purpose entirely. I leave that story for later.

\section{Systematic Analysis: The Search for Critical Mathematics}
I returned to my Bloomington as a Ph.D. student in math education. In the divorce, I somehow ended up with 25 blank composition books. I filled them with bitter poetry and songs, burned them, and bought more. I bought a book on catastrophe theory \parencite{poston_catastrophe_2012}, hoping it would explain how awful I felt. 

I didn't really want to do the program, but it seemed like a way to maintain the image of forward momentum, while the actual growth was happening underground through the act of writing without inhibition. This raises the questions: How do our personal traumas and moments of brokenness shape the knowledge we pursue? Can the process of writing and theorizing be a form of healing? I apologize to my professors that first year. I did not want to be there and it was apparent. I earned my first and only ``F'' in a college course, despite the best efforts of the professor who bestowed it unto me. The assignment that I could not complete was writing a 5-page ``teaching philosophy.'' Given how poorly my theoretical understanding of teaching had served me as a teacher (savior complex, valuing compliance, protecting non-compliance), every time I set pen to paper to write the thing was like scratching an open wound. I could not do it. So, I failed the course. If I had been willing to communicate with the professor about \textit{why} I could not write the paper, they probably would have helped me. 

I recall rather vividly that the professor asked me ``why?'' to some answer I gave in class. I bridled and said ``Why should I have to answer a `why' question?'' That self-defeating question (about what I now call the \textit{justificatory} aspect of synthesis \parencite{Brandom:2019aa}) did not feel self-defeating at all. Poetry, whether intact or in ashen form, doesn't generally explicitly justify itself with reasons. Imagine Shakespeare writing ``I chose a rose for this image because roses smell good.'' I'll be doing some of that sort of thing in this book. That does not mean that reasons are absent from poetry, they're just usually implicit in the completed work. The value of poetry comes through the implicit recognition it fosters within the reader.

But then I started studying critical theory with Phil Carspecken. I recall asserting rather forcefully that `everything can be doubted.' Phil reminded me that, in speaking, I must not be doubting \textit{everything}. He pointed out that Descartes' method of doubt ended with the indubitability that he was \textit{doubting}. Phil also noted that to speak was to \textit{assume} an Other who \textit{might} understand what I was speaking. That understanding, which is a fine initial definition of \textit{intersubjectivity}, blew my mind. All of the inner speech of doubt and self-loathing was addressed to someone who \textit{might} understand what I was so angry and sad about. Even alone, I was not alone. There is always an assumption of the Other, even if that Other is implicit, who might be listening---who might understand. While the next chapter will explore an idealized moment where normativity falls away, it is important to understand that the subject is inherently intersubjectively constituted. The \{I\} needs recognition from the ``you.'' I cannot write, speak, or think without assuming that someone else might understand what I am saying. Try to keep in mind that the assumption of a communicatively competent Other, and the assumption that I am also communicatively competent, is not, in practice, at all guaranteed. Habermas is often misrecognized as a utopian idealist, but his idealism is not utopian. There is little sun-bleached idealism in articulating how an assumption of communicative competence is a necessary condition for communication.



I learned about the work of George Herbert Mead, especially the distinction between the \{I\} and the ``me.'' The \{I\} is the source of action. The ``me'' is the self-as-recognized. None of my formal training in mathematics had prepared me to actually talk to students as \textit{subjects}. 

Since it is so central to my story and what follows, let me explore that theory briefly now through what I call the \textit{paradox of identity}: the ``me'' who is \{I\} and the \{I\} who is ``you'', and the ``we'' that appears to be neither ``you'' nor \{I\}. How do we navigate this fundamental tension between who we feel we are (the \{I\}) and how we are recognized by others (the ``me'')? There's a children's book, \emph{The Monster at the End of This Book} \parencite{stone_monster_2003}, where the eponymous monster, Grover, is informed that there is a monster at the end of the book (Figure \ref{fig:grover}). 


This concept of intersubjectivity is related to \textit{apperception}, loosely defined as perceiving-with. In Leibniz's original formulation, it relates to how \cancel{we} perceive things, like chairs, as unified objects. I can't `see' the back of a chair, but I cannot help but assume it is there. In the Leibnizian sense, apperception is a word to describe the self-consciousness that accompanies perception.

Definitions, in this text, tend to evolve. As new \textit{practices-or-abilities} ($P$) emerge, the relationships between \textit{vocabularies} ($V$) shift. A standard English dictionary is what Brandom \parencite*{Brandom2008} would call a Vocabulary-Vocabulary ($VV$) relationship. I am primarily interested in pragmatically mediated semantic relationships: $VPV$. 

This evolutionary nature of concepts contributes to the complexity of this text. This text will become more challenging as it progresses. One of the reasons for this complexity is that the autoethnographic components place the author (me!) on a temporal/historical axis. Like a sphere in \textit{Flatland} \parencite{abbott_flatland_2020} who is experienced by two-dimensional entities as a series of widening and shrinking circles, the temporal/historical axis limits how the whole subject (the \{I\}) is understood through its finite moments (see Figure \ref{fig:hypercube}). Some unity of the subject is apperceived, but the extra-dimensional aspects of the self cannot be rendered in representational space. We can't `see' the chair all at once. 

The second part of the challenge of the text is that the philosophical concepts I use also fall along a temporal/historical axis. There isn't a single definition of ``apperception.'' Leibniz introduced the phrase, then Kant picked it up and substantially expanded it. Hegel read Kant and apperception became \textit{the negative}, which was picked up by various thinkers. Taken all at once, each concept I discuss has a history replete with contradictions and tensions. Further, social and political contexts shape and were shaped by the development of these concepts. All this becomes even more complex when the concepts are describing root contradictions and paradoxes in human experience. How does one `see' the whole of a concept that is, by its nature, contradictory?

The third part of the challenge is that mathematical concepts also fall along a temporal/historical axis. In figure \ref{fig:hypercube}, I present $\mathcal{M}$'s recent practice drawing cubes. She drew them all on the same page, somewhat randomly. I re-ordered them in a progressive series to make the point that developmental movement in concepts suggests a different kind of `hypercube' than the term usually implies. Rather than a series of perfect cubes falling through 3D space, the claim I am making is that the concept ``cube'' is a history of progressively more adequate representations of a cube, falling through the space of reasons. She likes big cats, so I indicate in the figure that the subject, in its tiger moment, is the form---the `hypercube'---of the cubes she drew. 

That last bit hides surprising density. What I am claiming has Kantian, Hegelian, and Habermasian ideas united within it. Kant's \{I think\} can `accompany' all of my representations. The transcendental ego is like the form that binds the discretized representations, the content, of the concept ``cube'' into a singular concept. In the chapter on inferential movement, I will push on the concept of ``square'' past its representational moment. If that succeeds, the concept is `born' in its initial explication but `dies' as it exits representational space. 

With authorial becoming, conceptual becoming, and mathematical becoming all at play---at both the individual level (\textit{ontogenesis}) and societal level (\textit{phylogenesis})---my rhetorical task is to unfold these intricacies in a way that readers might recognize themselves as observer-participants in that same unfolding, all within the necessarily flattened medium of text. 

\begin{figure}[h]


\title{\textit{Development of Cube Drawings and Sphere in Flatland}}

\includegraphics[width=.8\textwidth]{/Users/tio/Documents/GitHub/September_UMEDCA/images/hypercube.pdf}

\caption{\textit{Note. } Left: $\mathcal{M}$'s practice drawing cubes suggests an apperceived `hypercube' falling through the `plane' of experience. Right: a sphere falling through 2D space appears as a series of expanding and contracting circles. Implicit apperceptive recognition binds each finite representation within the series to a unified whole and both series to each other. There is no formal reason to infer a form of cube or sphere in any of the discrete moments.}
\label{fig:hypercube}
\end{figure}

I learned about the work of George Herbert Mead, especially the distinction between the \{I\} and the ``me.'' The \{I\} is the source of action. The ``me'' is the self-as-recognized. None of my formal training in mathematics had prepared me to actually talk to students as \textit{subjects}. 

Since it is so central to my story and what follows, let me explore that theory briefly now through what I call the \textit{paradox of identity}: the ``me'' who is \{I\} and the \{I\} who is ``you'', and the ``we'' that appears to be neither ``you'' nor \{I\}. There's a children's book, \emph{The Monster at the End of This Book} \parencite{stone_monster_2003}, where the eponymous monster, Grover, is informed that there is a monster at the end of the book (Figure \ref{fig:grover}). Grover is frightened by this and begs the reader to stop reading. The cruel toddling reader keeps reading, though, with Grover becoming increasingly distraught. But at the end of the book, Grover recognizes himself as the moster at the end of the book: ``Well, look at that! This is the end of the book, and the only one here is\ldots{} ME. I, lovable, furry old Grover, am the Monster at the end of this book'' \parencite[26]{stone_monster_2003}.



\begin{figure}[h]
\title{\textit{The Monster at the End of This Book}}

\includegraphics[width=.8\textwidth]{/Users/tio/Documents/GitHub/September_UMEDCA/images/Grover.jpeg}

\caption{\textit{Note. }Frames from \emph{The Monster at the End of This Book} \parencite{stone_monster_2003}}

\label{fig:grover}
\end{figure}

Sophisticates may recognize the story of Oedipus's unwitting self-banishment for the murder of Laius in the problem of self-recognition that Grover encounters in Stone's book. The problem is that the self-as-recognized, the ``me'', is often distinct from the \{I\}, understood as the source of action or locus of ``power, creativity, and freedom'' \parencite[97]{Carspecken1999}. Both Grover and Oedipus appear to have only partially understood themselves. They did not completely misunderstand themselves. Oedipus knew his authority to banish; Grover knew his fear and, later, understood that he was the `loveable' kind of monster. Complete misrecognition is outside human experience. They only partially misrecognized themselves.

Furthermore, I learned from Phil and Mead about the \textit{generalized other}. The generalized other is the internalized sense of the expectations and attitudes of the broader community. 

Why these ideas were so crucial for me is that they appeared to \textit{explain} the ambivalent yet tense drift in my identity. They began to help me crack the problem of \textit{alienation}. At Oak Ridge, I was both a pacifist and someone who found armed guards and the atomic bomb somehow `glamorous.' I reasoned that I had internalized a contradiction in the \textit{generalized other}. The generalized other is Mead's term that I use to point to the moral authority whose words filled my inner monologue with self-loathing for being unable to live up to both the ideals of pacifism and goodness of defeating fascism that the atomic bomb represented. Differences in others' expectations and attitudes toward me had always made me tense in social situations. I had never been able to figure out why, but it made sense to think that the ``me'' was divided depending on how the other person recognized me. Contradictions in the generalized other explain why I felt the need to wear different `hats' in different social situations. 

Different people have different sensitivities to these contradictions. It seems quite normal, even necessary, to take on different roles in different social situations. But the problem becomes acute when those roles conflict with each other or with deep commitments about what it means to be a good person. When fulfilling a role requires violating some aspect of self that feels inviolable, the tension can become unbearable. 

In short, I found in Phil's critical theory some answers to the existential questions that I might have been able to answer if I had continued studying Kierkegaard instead of falling for pure mathematics. 

After a few weeks, I decided to linger after class. I asked Phil what a \textit{critical mathematics} would look like. We chatted, and I was hooked. We agreed that a critical mathematics would express mathematics as `normative,' but I misunderstood `normativity' as arbitrary (i.e., ``socially constructed'') and in opposition to people who identify as queer. Essentially taking normativity to be norminess, a kind of un-critical adherence to received knowledge, led me down some unproductive paths. Rather than healing the drift of alienation, I ended up entrenching alienated forms of thought in my work. 

I was simultaneously involved in Dr. Erik Jacobson's research project that studied misconceptions in the domain of fractions, decimals, and rational numbers. I was also enrolled in an arts-based educational research class that Dr. Gus Welsek taught. I began to analyze the student work samples that expressed some kind of flawed mathematical reasoning as if they were artistic expressions. Rather than asking ``what is \textit{wrong} with this work/student?'' I began to ask, ``what is this student trying to say?''

In Phil's class, I also learned about Jürgen Habermas' theory of communicative action \parencite*{Habermas1984,Habermas:1985aa}. Communicative action is oriented toward understanding. It contrasts with instrumental and strategic action. Those are both telological (goal-oriented), where a subject acts towards objects to reach a goal. In strategic action, other people's responses are anticipated and actors can either coordinate their actions to accomplish goals or use each other to accomplish an end. The latter explained why some interactions felt so manipulative and gross. 

I also learned of Habermas' conceptualization of three different types of \textit{validity claims}. I wanted so badly to be good---to be valid! To understand why this framework resonated so deeply with my struggles, you need to grasp what Habermas means by ``validity'' itself.

\subsection*{The Structure of Validity Claims}

When you speak, you make claims. This may seem obvious, but the depth of what I am claiming here is subtle. Every meaningful act---whether a gesture, a statement, or even silence---implicitly references what Habermas calls \textit{validity claims}. These are not just assertions you make; they are the conditions under which your act could be understood, accepted, or challenged by another person.\footnote{\enquote{The expectation that another will understand your use of signs is constructed tacitly from the expectation that another will experience what the sign is about in the same way as one's self. This is intersubjectivity. The expectations associated with simply being understood, aside from consequences, tacitly assume multiple subjectivities capable of having common experiences.} \parencite[38]{Carspecken1999}}

Validity is internal to meaning. As Carspecken articulates, \enquote{Habermas argues that all meaningful acts \textit{internally reference truth claims}. Truth is internal to meaning} \parencite[71-72, emph. orig.]{Carspecken1999}. You cannot separate what an utterance means from the implicit claims it makes about what is true, what is right, and what is sincere. This is a profound understanding: meaning is not a mental representation transmitted from one mind to another; it is a structure of commitments and entitlements that emerges in the space between people. Understanding a speech act means grasping the conditions under which its validity claims would hold or not hold \parencite{Carspecken2018}.

Habermas expanded the concept of a single truth claim to three universal categories of validity claims. As Carspecken explains, \enquote{The claims that tacitly accompany every act of meaning will always fall into categories, three of which will always be represented} \parencite[72]{Carspecken1999}. They are always present in every meaningful act, though they may be foregrounded or backgrounded to different degrees. Let me explain each category, not as abstract philosophy, but as structures I recognized in my own struggles to make sense of experience.

\paragraph{Objective Validity Claims}

Objective validity claims pertain to factual states of affairs in what Habermas calls the ``objective world'' \parencite{Habermas1984, Carspecken2018}. When I say ``The calculator has a square-root button,'' I am making a claim that could, in principle, be verified or falsified by anyone who examines the calculator. I am asserting that something is true about an object that exists independently of my feelings or intentions. Habermas describes this as asserting \enquote{That the statement made is true (or that the existential presuppositions of the propositional content mentioned are in fact satisfied)} \parencite[161, e-book]{Habermas1984}.

Support for objective claims relies on what I learned to call the principle of \textit{multiple access} \parencite{Carspecken:1995aa, Carspecken1999}. If I claim ``There is a tree outside my window,'' you can, in principle, look out that same window and confirm or deny my claim. Different observers, using different devices, should be able to access the same state of affairs and reach agreement about it. This requires \enquote{repeated observations} by multiple observers \parencite[77]{Carspecken:1995aa}. This is why science works: the criterion for objectivity is intersubjective repeatability, not private experience.

When I struggled in eighth grade with ``rise over run,'' my difficulty was partly an objective validity issue. The teacher claimed ``The slope is the vertical change divided by the horizontal change.'' This claim referenced an objective fact about coordinate geometry. But my confusion arose because the orientation of the paper---something objective and observable---undermined my ability to understand which direction was ``vertical.'' The claim was objectively true in a specific reference frame, but that frame was not made explicit. This is a feature of all objective claims: they always presuppose a framework of background assumptions that makes them intelligible.

Objective claims are associated with the third-person position: ``it'' or ``they,'' the world of objects and states of affairs accessible from multiple perspectives \parencite{Carspecken1999, Carspecken:1995aa}.

\paragraph{Subjective Validity Claims}

Subjective validity claims relate to the speaker's inner world of intentions, feelings, desires, attitudes, values, and capacities---representing something in what Habermas calls the \textbf{subjective world} \parencite{Habermas1984, Carspecken1999, Carspecken2018}. When I say ``I find mathematics beautiful,'' I am not making a claim that you can verify by looking at mathematical objects. I am making a claim about my own inner experience, to which I have what philosophers call ``privileged access.''

The validity claim here is one of \textit{truthfulness} or \textit{sincerity}. The speaker asserts \enquote{That the manifest intention of the speaker is meant as it is expressed} \parencite[161, e-book]{Habermas1984}. When I express a subjective state, you cannot directly verify whether I am being honest. You can only assess the consistency of my behavior over time. These claims are debated by establishing the honesty of self-reports, as they often involve privileged access to one's inner states \parencite{Carspecken1999, Carspecken:1995aa}. If I claim to love mathematics but consistently avoid doing mathematics, you have grounds to doubt my sincerity. The support for subjective claims comes from the pattern of my actions, not from objective observation of an external world.

This category helped me understand my years of self-loathing. When I said ``I hate mathematics,'' I was making a subjective validity claim. But the felt experience was complicated by the fact that I also loved certain aspects of mathematical reasoning---the aesthetics of symbols, the elegance of proofs. The contradiction was not in the mathematics itself but in my subjective relationship to it, mediated by the social contexts in which I encountered it.

Subjective claims are associated with the first-person position: the \{I\} who speaks \parencite{Carspecken1999, Carspecken:1995aa}. Only I can truly know what I intend, what I feel, what I desire. This is not solipsism; it is recognition of the asymmetry between first-person experience and third-person observation. You can observe my behavior, but you cannot feel my pain.

\paragraph{Normative-Evaluative Validity Claims}

Normative-evaluative claims concern what is good, bad, right, wrong, proper, legitimate, or appropriate within a shared social world \parencite{Habermas1984, Carspecken1999, Carspecken2018}. When I say ``Teachers should not humiliate students,'' I am not making a claim about an objective fact or about my private feelings. I am making a claim about a norm---a rule or standard---that I expect others in my community to recognize and endorse. The speaker claims \enquote{That the speech act is right with respect to the existing normative context (or that the normative context that it is supposed to satisfy is itself legitimate)} \parencite[161, e-book]{Habermas1984}.

Supporting normative-evaluative claims is fundamentally different from supporting objective or subjective claims. No amount of repeated observation will settle a disagreement about whether something is right or wrong. No appeal to my sincerity will resolve a moral dispute. Instead, normative claims must be supported by seeking shared normative-evaluative agreements within a community \parencite{Carspecken1999, Carspecken:1995aa}. These claims implicitly \enquote{call upon the other person to conform to the actor or to agree with the actor about certain things} \parencite[83]{Carspecken:1995aa}. The validity of a norm depends on whether it can, in principle, win the free assent of all those it affects.

This is where Habermas's framework becomes politically charged. If a norm is maintained through coercion, manipulation, or systematically distorted communication, it lacks genuine validity. A norm is truly valid only if it could survive rational critique in conditions of undistorted dialogue---what Habermas calls the ``ideal speech situation.'' This is not a utopian fantasy; it is a regulative ideal, a standard against which to measure the legitimacy of existing norms.

When I felt the tension between my commitment to pacifism and my admiration for the institutions that defeated fascism, I was experiencing a contradiction in normative-evaluative claims. Both norms---``Violence is wrong'' and ``Defeating fascism was good''---seemed right to me, but they pointed in opposite directions. The generalized other, that internalized sense of communal expectations, contained a contradiction. My task was not to eliminate the contradiction but to articulate it, to understand the conditions under which each norm might be valid or invalid.

Normative claims are associated with the second-person position (``you'') and the first-person plural (``we'') \parencite{Carspecken1999, Carspecken:1995aa}. When I make a normative claim, I am implicitly calling upon you to recognize it as binding on both of us. The ``we'' that emerges from this mutual recognition is not a fixed entity; it is an ongoing achievement, constantly renegotiated through dialogue.

\paragraph{The Three Subject Positions and the Unifying ``We''}

These three validity claims correspond to three different subject positions that structure all communication. The objective world is accessed from the third-person position: ``it'' or ``they,'' the world of objects and states of affairs. The subjective world is accessed from the first-person position: \{I\}, the locus of experience and agency. The normative world is accessed from the second-person position (``you'') and the first-person plural (``we''), the shared space of reasons and norms.

The normative ``we'' unites the three positions, adjudicating between them in a way that always appears incomplete. When we communicate, we coordinate our perspectives on the objective world (\textit{that} is true), our subjective experiences (\textit{I} feel this way; do \textit{you}?), and our normative commitments (\textit{we} should do this). But this coordination is never final. New experiences, new perspectives, new critiques can always emerge, forcing us to renegotiate our shared understanding.

This triadic structure is not a static framework imposed on experience. It emerges from experience through reflection. When I speak to you, I do not consciously think ``Now I will make an objective claim, now a subjective claim, now a normative claim.'' These categories are implicit in the structure of communication itself. Only through the kind of analysis Habermas provides do they become explicit.


I still find extraordinary expressive power in this structure. For example, the debate about gender that occupies so much of the public sphere as I write, tends to invoke these three aspects of validity without anyone acknowledging that they are doing so. The felt-experience of being transgender (subjective), for example, is often displaced to talk about biological sex (objective), to then make normative claims about who should use which bathroom. Understanding the triadic aspect of validity doesn't suddenly `solve the problem.' Instead, it transforms the \textit{inference field}. Rather than continually circling through the mundane algorithms of propaganda, the kinds of next-thoughts that follow from the triadic conceptualization of communicative rationality allowed me to think about the issues I was struggling with, instead of just getting carried along, feigning an excruciating ambivalence about the issues I cared deeply about. 

In this manuscript, I often speak of subjective, normative, and objective \textit{dimensions}. Borrowing from Robert Brandom and Willfred Sellars' phrase, ``the space of reasons,'' I conceptualize such a space as `three-dimensional.' The space of reasons is not Cartesian. These `axes' of validity are not orthagonal to one another. Instead, they are sort of \textit{inside and outside} of each other in a way that is very hard to conceptualize spatially. This structural relationship only emerges on reflection, a point I will labor to make in the next chapter.

\subsection*{Knowledge-Constitutive Interests: The Deep Roots of Inquiry}

Understanding validity claims was transformative, but it raised a deeper question: Why do humans inquire at all? What drives us to seek knowledge in such different forms---the precision of physics, the interpretation of history, the critique of ideology? Habermas's answer, developed in his earlier work \textit{Knowledge and Human Interests}, is that all knowledge is rooted in fundamental orientations that arise from the basic conditions of human species reproduction and self-constitution.

He calls these orientations \textit{knowledge-constitutive interests}. The term ``interest'' here is philosophical, not psychological. These are not personal preferences or individual motivations. They are deep structures that mediate the relationship between how humans exist in the world (through work, language, and power) and how humans come to know the world. They are rooted in the objective problems of life preservation that have been solved through our cultural form of existence: work sustains us materially, language enables social coordination, and the struggle for autonomy drives self-formation.

I initially misunderstood this. I thought ``interests'' meant biases that corrupt pure knowledge. But Habermas is arguing something more radical: there is no ``understanding from nowhere,'' no knowledge that stands outside of human concerns. Knowledge is always bound up with the conditions that make it possible. The question is not whether knowledge serves interests, but which interests it serves and whether those interests are acknowledged or repressed.

Habermas identifies three categories of knowledge-constitutive interests, each corresponding to a form of scientific inquiry and to one of the validity claims I have just described.

\paragraph{The Technical Cognitive Interest}

The first is the \textit{technical cognitive interest}, which guides the empirical-analytic sciences. This interest aims at disclosing and comprehending reality for the sake of possible technical control. It is rooted in the behavioral system of \textit{instrumental action}---the realm of work, where humans adapt to and transform their material environment.

Knowledge acquired under the technical interest takes the form of predicting observable events and specifying means for achieving tangible, objective ends. Its validity is based on the degree to which predictions are successful. If I predict that water will boil at 100 degrees Celsius at sea level, and it does, my knowledge is confirmed. If it does not, my hypothesis is falsified, and I must revise my understanding.

The technical interest is not reducible to capitalist exploitation or the desire to dominate nature, though it can certainly be co-opted for those purposes. At its root, this interest responds to the human need to secure the material conditions of survival. Every time you flip a light switch, you are relying on knowledge shaped by the technical interest: knowledge of electrical circuits, of power generation, of the properties of materials.

This interest employs formalized languages, operational definitions, and hypothesis-testing methodologies. It discloses reality from the viewpoint of technical control over objectified natural processes. When I struggled at Oak Ridge with FORTRAN code, I was working within the technical interest. The goal was to produce knowledge that could be translated into computational procedures---algorithms that would run successfully or fail predictably.

The technical interest corresponds to objective validity claims and the third-person perspective. It treats the world as a collection of objects and events whose relations can be modeled, predicted, and controlled.

\paragraph{The Practical Cognitive Interest}

The second is the \textit{practical cognitive interest}, which guides the historical-hermeneutic sciences---disciplines like history, anthropology, literary studies, and much of qualitative research. This interest aims at maintaining the intersubjectivity of mutual understanding in ordinary-language communication and in action guided by common norms.

The practical interest is rooted in language and tradition-bound social life. Its goal is not to control objectified processes but to maintain the very condition that makes a world appear as something shared: intersubjective understanding. When communication breaks down---when I cannot understand what you mean, or when your cultural tradition is alien to me---the practical interest drives inquiry aimed at restoring or establishing mutual comprehension.

Knowledge acquired under the practical interest involves hermeneutically explicating formerly implicit knowledge. It uses ordinary language, not formalized calculi. Its validity is based on the degree to which insiders recognize explicit articulations as something they already knew in some tacit way. When an anthropologist describes a cultural practice and members of that culture say ``Yes, that is what we do, though we never put it that way before,'' the hermeneutic knowledge is validated.

This interest is not about imposing an external framework on a culture. It is about entering into dialogue with a tradition, making explicit the meanings that guide action within that tradition, and mediating understanding between different individuals, groups, and cultures---both horizontally (across contemporary cultures) and vertically (between present and past).

When I began analyzing student mathematical reasoning as if it were artistic expression, asking ``What is this student trying to say?'' rather than ``What is wrong with this student?'', I was shifting from a technical interest (control and correction) to a practical interest (understanding and interpretation). This shift was transformative. It allowed me to recognize meaning where I had previously seen only error.

The practical interest corresponds to normative-evaluative validity claims and the second-person (``you'') and first-person plural (``we'') perspectives. It is concerned with the shared social world of norms, values, and meaningful actions.

\paragraph{The Emancipatory Cognitive Interest}

The third is the \textit{emancipatory cognitive interest}, which guides critically oriented sciences---psychoanalysis, ideology critique, and critical social theory. This interest aims at self-reflection and freeing consciousness from dependence on hypostatized powers. It is rooted in the dimension of \textit{power}, specifically the struggle for self-assertion against domination.

The emancipatory interest emerges in contexts of systematically distorted communication and thinly legitimated repression. It arises when norms are maintained through coercion rather than rational consent, when identities are imposed rather than freely chosen, when knowledge is used to manipulate rather than enlighten. The emancipatory interest seeks to dissolve these distortions through self-reflective learning processes.

Knowledge acquired under the emancipatory interest involves the critical dissolution of objectivism and the undoing of repression and false consciousness. Its method is self-reflection: the process by which a subject becomes transparent to itself in its own genesis. The validity of emancipatory knowledge cannot be assessed by technical prediction or hermeneutic recognition alone. It is validated by the experience of enlightenment itself---the moment when you recognize how you have been deceived, constrained, or alienated, and that recognition frees you to act differently.

Habermas famously argues that ``in the power of self-reflection, knowledge and interest are one.'' This is not a mystical claim. It means that the pursuit of self-knowledge is inherently emancipatory. To understand why you believe what you believe, why you desire what you desire, is to gain a degree of freedom from those beliefs and desires. They no longer control you unconsciously; they become objects of deliberate choice.

When Phil helped me recognize that to speak is to assume an Other who might understand, he was engaging the emancipatory interest. My inner monologue of self-loathing had been sustained by a fantasy of absolute isolation. Recognizing the intersubjective structure of thought dissolved that fantasy. I was freed, not from loneliness, but from the illusion that loneliness could be absolute.

The emancipatory interest is inherently connected to subjective validity claims and the first-person perspective (\{I\}). It concerns the formation and transformation of the self, the struggle for autonomy and authenticity. But it is not solipsistic. Emancipation is achieved through dialogue, through the recognition of others, through the collective critique of ideologies that bind us all.

\subsection*{How Validity Claims and Knowledge Interests Relate}

The three categories of validity claims and the three knowledge-constitutive interests are deeply intertwined. They provide two perspectives on the same underlying reality: the differentiated structure of human rationality.

The validity claims describe the formal structure of meaning in communication. Every meaningful act implicitly references objective, subjective, and normative dimensions. The knowledge interests describe the deep motivational and existential roots of inquiry. Every form of knowledge arises from a fundamental human orientation toward the world.

The correspondence is this: Objective validity claims align with the technical cognitive interest. Both are concerned with the world of facts and events, accessed through multiple observers, governed by the logic of prediction and control. Subjective validity claims align with the emancipatory cognitive interest. Both are concerned with the inner world of the self, accessed through privileged introspection, governed by the logic of self-reflection and authenticity. Normative-evaluative validity claims align with the practical cognitive interest. Both are concerned with the shared social world of norms and values, accessed through intersubjective dialogue, governed by the logic of mutual understanding and recognition.

This alignment is not arbitrary. The technical interest defines what it means for an objective claim to be valid: successful prediction. The practical interest defines what it means for a normative claim to be valid: intersubjective recognition of shared interests. The emancipatory interest defines what it means for subjective claims to be authentic: transparency to oneself in one's own genesis.

Understanding this framework was, for me, a kind of conceptual liberation. It explained why arguments about mathematics education were so intractable. Researchers were operating within different knowledge-constitutive interests, making different kinds of validity claims, without acknowledging the plurality of rationality. Someone arguing from a technical interest wants mathematical knowledge to be predictive and operationalizable. Someone arguing from a practical interest wants it to be meaningful and culturally situated. Someone arguing from an emancipatory interest wants it to be transformative and empowering.

These are not mutually exclusive. A critical mathematics can integrate all three. But integration requires acknowledging the differences, understanding the conditions of possibility for each form of knowledge, and resisting the temptation to reduce one to another.

\subsection*{The Critique of Scientism}

While Phil did not push his own writing on his students, I read his works voraciously. His beautiful series on the limits of knowledge and how they relate to scientism is a must read for anyone reading this who recognizes themselves in what I described as ``intellectual chauvinism'' \parencite*{Carspecken:2006aa,Carspecken:2009aa,carspecken_limits_2016}. Scientism is the ideology that only third-person claims are valid. It is an ideology that cannot understand itself, since it cannot account for the subjective and normative dimensions of its own claims.

The recognition of multiple validity claims and knowledge-constitutive interests is a powerful critique of scientism. Scientism privileges the technical interest and objective validity claims, treating them as the only legitimate form of knowledge. But this privilege cannot be justified on its own terms. The claim ``Only objective knowledge is valid'' is itself a normative claim. It expresses a value judgment about what counts as knowledge. To assert it is to performatively contradict oneself: you are making a normative claim while denying the validity of normative claims.

Habermas's framework reveals that rationality is differentiated, not singular. There is no one scientific method, no one criterion of validity, no one form of knowledge. Instead, there are multiple, irreducible forms of rational inquiry, each with its own logic, its own standards, its own connection to human interests. The task of critical theory is not to privilege one over the others, but to understand their interrelations, to resist their distortion, and to enable their full development.

Phil's work is broadly inspired by the Frankfurt School, but he also draws heavily on G.W.F. Hegel and his experiences studying yogic philosophy. Studying with Phil brought \textit{the negative} into my life explicitly. He mailed me a copy of Jean Hyppolite's  book that describes how the \{I\} ``never is what it is and is always what it is not'' \parencite*[150]{Hyppolite1974}. I'm grateful for that gift, among countless others. 

Besides the gift of genuine intersubjective recognition, Phil's piece titled \textit{The Missing Infinite} \parencite{Carspecken2018} is a must-read for anyone interested in critical theory. While I will do my best to build on his insights, his gentle enjoinment to bring the \textit{in}finite into explicitness helped clarify the central tension in identity that I had been struggling with for years. For I am both finite, object-like, bounded, and determinate, and \textit{in}finite, a subject who breaks boundaries in the ever-evolving pursuit of self-determination. 

Recognizing this tension as a fundamental aspect of identity helped me learn to relax a bit. So much of what I had internalized about being a good person was about being consistent, reliable, and predictable. In short, `the Good'---that seemingly relativistic ideal---was not relativistic at all. Attaining such goodness is probably impossible, but trying to live up to my commitments by becoming more object-like in how others recognize me gave me reasons for trying to carefully justify my commitments. It also gave me grounds to prune some commitments that were deeply conflictual. But notice the tension in that last sentence. I had to \textit{change} to be more object-like! The goal of finitude necessarily requires \textit{in}finite, i.e., becoming. The lack of coincidence between the \{I\} and the ``me'' is not a solvable problem. It's a paradox that \cancel{we} must all live with. 

One strategy for managing that paradoxical aspect of identity is to try to place oneself and others on a \textit{developmental trajectory.} I need hardly forgive a 4-year-old for not honoring their commitments, but I expect a 40-year-old to be more consistent. Still, I sometimes feel like I was just born but a few moments ago. The revitalization that comes with re-sensitizing myself to the world around me grows the capacity for self-forgiveness. 


Thinking through three layers of validity, the \textit{subjective}, \textit{objective}, and \textit{normative} helped me untangle some of the deep contradictions I had internalized that resulted in so much ambivalence. Later, I recognized that artistic expression did not quite cover the \textit{normative} aspects of mathematics. Phil also bought me a copy of \textit{A Spirit of Trust} by Robert Brandom \parencite*{Brandom:2019aa}. In that work, he discusses the \textit{experience of error}. I will take that on soon, but since I have such a deep, rich history of getting it wrong, I figured the \textit{experience of error} could be an epistemological foundation for mathematics. Could mathematical truths be known through the exclusion of error?

That idea probably sounds a bit strange to mathematicians. But it is a relatively normal idea in teacher education. I recently met with a kindergarten teacher who was introducing some IU students who she is mentoring. One of the first things she said is ``you are here to make mistakes.'' 

The literature in mathematics education discusses errors, misconceptions, perturbations, and mistakes extensively. Researchers work to understand why they happen, how they propagate through inferential systems, and how they are repaired. However, these theories rest together in incoherence; there isn't a single coherent philosophy of math education. I became curious about how I was able to read different theories, identify with parts of them, reject other parts, and somehow still understand what authors were trying to say.



My initial attempts to filter theories through the experience of error resulted in an understanding: the community of math education researchers are doing meta-mathematics over a mathematical landscape defined by \textit{material incompatibility}. If you're a mathedian and vigorously disagree with that assessment, please hang with me for a bit. There is room for your ``no.'' The goal became establishing a methodology for bringing students' work, correct or not, into a stable mathematical structure.

\subsection*{Intersubjectivity and the Horizon of Meaning}

When Phil reminded me that to speak is to assume an Other who might understand, he was pointing to a fundamental structure that critical theory calls \textbf{intersubjectivity}---shared understanding, mutual expectations, and reciprocal recognition among subjects \parencite{Carspecken1999, Carspecken:2013aa}. Even my inner monologue of self-loathing assumed someone who could understand what I was so angry and sad about. This assumption of communicative competence---that someone might grasp my meaning---is a necessary condition for communication itself, even if it is not guaranteed in practice \parencite{Habermas1984}.

The distinction between \textbf{subjectivity} (my own inner experience, to which only I have privileged access) and \textbf{intersubjectivity} (shared understanding and mutual recognition) proved crucial. \enquote{The distinction between subjectivity and intersubjectivity (position-taking) is absolutely crucial to subjective-referenced claims. Subjectivity is not the immediate effect of social relations, cultural forms, languages, and discourses. It is rather given through a play between such intersubjectively constituted symbols and raw, unmediated subjective experience that is always referenced but that is always already removed from experience when we represent it.} \parencite[108]{Carspecken:1995aa} While subjectivity is not solely an effect of social relations, it becomes knowable and representable only \textit{through} intersubjectivity \parencite{Carspecken:1995aa, Carspecken1999}. Similarly, objectivity is not possible without intersubjectivity \parencite{Carspecken1999, habermas1992individuation}.

Human identity, I learned, is fundamentally an ``I-me'' relationship: the \{I\} (pure subjectivity/agency) is understood and affirmed through the ``me'' (the self from the perspective of others). This self-relation arises out of interactive contexts and is infused with a desire for recognition \parencite{Carspecken1999, habermas1992individuation}. Intersubjectivity provides the ``ground'' of human societies, enabling mutual understanding, cooperation, and the very possibility of knowledge \parencite{Habermas:1971aa, mead_mind_2015}. Yet it remains \textbf{fragile and occasionally successful} in everyday communication, constantly requiring negotiation and revision \parencite{Habermas1984}.

This framework helped me understand that meanings are constituted not just by explicit validity claims but also by implicit \textbf{horizons of intelligibility} \parencite{Carspecken1999, Carspecken:1995aa}---the background assumptions and shared contexts that make understanding possible. In mathematics education, these horizons often remain hidden, leading to the kinds of confusion I experienced with ``rise over run.''

\subsection*{Knowledge Constitutive Interests}

Understanding validity claims and intersubjectivity was transformative, but it raised a deeper question: Why do humans inquire at all? What drives us to seek knowledge in such different forms---the precision of physics, the interpretation of history, the critique of ideology? Habermas's answer, developed in his earlier work \textit{Knowledge and Human Interests}, is that all knowledge is rooted in fundamental orientations arising from the basic conditions of human species reproduction and self-constitution. Habermas terms these \enquote{the basic orientations rooted in specific fundamental conditions of the possible reproduction and self-constitution of the human species, namely work and interaction} \parencite[196]{Habermas:1971aa}.

He calls these orientations \textit{knowledge-constitutive interests} (KCI). The term ``interest'' here is philosophical, not psychological. These are not personal preferences or individual motivations. They are deep structures that mediate the relationship between how humans exist in the world (through work, language, and power) and how humans come to know the world \parencite{carspecken_limits_2016}. They are rooted in the objective problems of life preservation that have been solved through our cultural form of existence: work sustains us materially, language enables social coordination, and the struggle for autonomy drives self-formation.

I initially misunderstood this. I thought ``interests'' meant biases that corrupt pure knowledge. But Habermas is arguing something more radical: there is no ``understanding from nowhere,'' no knowledge that stands outside of human concerns. Knowledge is always bound up with the conditions that make it possible. The question is not whether knowledge serves interests, but which interests it serves and whether those interests are acknowledged or repressed.

Habermas identifies three categories of knowledge-constitutive interests, which \enquote{take form in the medium of work, language, and power} \parencite[313]{Habermas:1971aa}, each corresponding to a form of scientific inquiry and to one of the validity claims I have just described:

\paragraph{The Technical Cognitive Interest}

The first is the \textit{technical cognitive interest}, which guides the empirical-analytic sciences. This is the \enquote{cognitive interest in technical control over objectified processes} \parencite[309]{Habermas:1971aa}, rooted in the behavioral system of \textit{instrumental action}---the realm of work, where humans adapt to and transform their material environment \parencite[121, 124]{Habermas:1971aa}.

Knowledge acquired under the technical interest takes the form of predicting observable events and specifying means for achieving tangible, objective ends \parencite{carspecken_limits_2016}. Its validity is based on the degree to which predictions are successful. If I predict that water will boil at 100 degrees Celsius at sea level, and it does, my knowledge is confirmed. If it does not, my hypothesis is falsified, and I must revise my understanding.

The technical interest is not reducible to capitalist exploitation or the desire to dominate nature, though it can certainly be co-opted for those purposes. At its root, this interest responds to the human need to secure the material conditions of survival. Every time you flip a light switch, you are relying on knowledge shaped by the technical interest: knowledge of electrical circuits, of power generation, of the properties of materials.

This interest employs formalized languages, operational definitions, and hypothesis-testing methodologies \parencite{Habermas:1971aa}. It discloses reality from the viewpoint of technical control over objectified natural processes. When I struggled at Oak Ridge with FORTRAN code, I was working within the technical interest. The goal was to produce knowledge that could be translated into computational procedures---algorithms that would run successfully or fail predictably.

The technical interest corresponds to objective validity claims and the third-person perspective. It treats the world as a collection of objects and events whose relations can be modeled, predicted, and controlled.

\paragraph{The Practical Cognitive Interest}

The second is the \textit{practical cognitive interest}, which guides the historical-hermeneutic sciences---disciplines like history, anthropology, literary studies, and much of qualitative research. This interest aims at \enquote{maintaining the intersubjectivity of mutual understanding in ordinary-language communication and in action according to common norms} \parencite[176]{Habermas:1971aa}.

The practical interest is rooted in language and tradition-bound social life \parencite[313]{Habermas:1971aa}. Its goal is not to control objectified processes but to maintain the very condition that makes a world appear as something shared: intersubjective understanding. As Habermas puts it, it maintains the intersubjectivity \enquote{within whose horizon reality can first appear as something} \parencite[176]{Habermas:1971aa}. When communication breaks down---when I cannot understand what you mean, or when your cultural tradition is alien to me---the practical interest drives inquiry aimed at restoring or establishing mutual comprehension.

Knowledge acquired under the practical interest involves hermeneutically explicating formerly implicit knowledge \parencite{carspecken_limits_2016}. It uses ordinary language, not formalized calculi. Its validity is based on the degree to which insiders recognize explicit articulations as something they already knew in some tacit way \parencite{carspecken_limits_2016}. When an anthropologist describes a cultural practice and members of that culture say ``Yes, that is what we do, though we never put it that way before,'' the hermeneutic knowledge is validated.

This interest is not about imposing an external framework on a culture. It is about entering into dialogue with a tradition, making explicit the meanings that guide action within that tradition, and mediating understanding between different individuals, groups, and cultures---both horizontally (across contemporary cultures) and vertically (between present and past) \parencite[176]{Habermas:1971aa}.

When I began analyzing student mathematical reasoning as if it were artistic expression, asking ``What is this student trying to say?'' rather than ``What is wrong with this student?'', I was shifting from a technical interest (control and correction) to a practical interest (understanding and interpretation). This shift was transformative. It allowed me to recognize meaning where I had previously seen only error.

The practical interest corresponds to normative-evaluative validity claims and the second-person (``you'') and first-person plural (``we'') perspectives. It is concerned with the shared social world of norms, values, and meaningful actions.

\paragraph{The Emancipatory Cognitive Interest}

The third is the \textit{emancipatory cognitive interest}, which guides critically oriented sciences---psychoanalysis, ideology critique, and critical social theory. This interest aims at \textbf{self-reflection}, which \enquote{releases the subject from dependence on hypostatized powers} \parencite[310]{Habermas:1971aa}. It is rooted in the dimension of \textit{power} (specifically, self-assertion against domination) \parencite[313]{Habermas:1971aa}.

The emancipatory interest emerges in contexts of systematically distorted communication and thinly legitimated repression \parencite{Habermas:1971aa}. It arises when norms are maintained through coercion rather than rational consent, when identities are imposed rather than freely chosen, when knowledge is used to manipulate rather than enlighten. The emancipatory interest seeks to dissolve these distortions through self-reflective learning processes.

Knowledge acquired under the emancipatory interest involves the \enquote{critical dissolution of objectivism} \parencite[213]{Habermas:1971aa} and the undoing of repression and false consciousness \parencite{Habermas:1971aa}. Its method is self-reflection: the process by which a subject becomes transparent to itself in its own genesis. The validity of emancipatory knowledge cannot be assessed by technical prediction or hermeneutic recognition alone. It is validated by the experience of enlightenment itself---the moment when you recognize how you have been deceived, constrained, or alienated, and that recognition frees you to act differently.

Habermas famously argues that \enquote{knowledge for the sake of knowledge attains congruence with the interest in autonomy and responsibility} \parencite[315]{Habermas:1971aa}. In the power of self-reflection, knowledge and interest are one. This is not a mystical claim. It means that the pursuit of self-knowledge is inherently emancipatory. To understand why you believe what you believe, why you desire what you desire, is to gain a degree of freedom from those beliefs and desires. They no longer control you unconsciously; they become objects of deliberate choice.

When Phil helped me recognize that to speak is to assume an Other who might understand, he was engaging the emancipatory interest. My inner monologue of self-loathing had been sustained by a fantasy of absolute isolation. Recognizing the intersubjective structure of thought dissolved that fantasy. I was freed, not from loneliness, but from the illusion that loneliness could be absolute.

The emancipatory interest is inherently connected to subjective validity claims and the first-person perspective (\{I\}). It concerns the formation and transformation of the self, the struggle for autonomy and authenticity. But it is not solipsistic. Emancipation is achieved through dialogue, through the recognition of others, through the collective critique of ideologies that bind us all.

\subsection*{How Validity Claims and Knowledge Interests Relate}

The three categories of validity claims and the three knowledge-constitutive interests are deeply intertwined \parencite{Habermas1984, carspecken_limits_2016}. They provide two perspectives on the same underlying reality: the differentiated structure of human rationality.

The validity claims describe the formal structure of meaning in communication. Every meaningful act implicitly references objective, subjective, and normative dimensions. The knowledge interests describe the deep motivational and existential roots of inquiry. Every form of knowledge arises from a fundamental human orientation toward the world.

The correspondence is this: Objective validity claims align with the technical cognitive interest. Both are concerned with the world of facts and events, accessed through multiple observers, governed by the logic of prediction and control \parencite{Habermas:1971aa, Carspecken1999}. Subjective validity claims align with the emancipatory cognitive interest. Both are concerned with the inner world of the self, accessed through privileged introspection, governed by the logic of self-reflection and authenticity \parencite{Habermas:1971aa, Carspecken1999}. Normative-evaluative validity claims align with the practical cognitive interest. Both are concerned with the shared social world of norms and values, accessed through intersubjective dialogue, governed by the logic of mutual understanding and recognition \parencite{Habermas:1971aa, Carspecken1999}.

This alignment is not arbitrary. The technical interest defines what it means for an objective claim to be valid: successful prediction. The practical interest defines what it means for a normative claim to be valid: intersubjective recognition of shared interests. The emancipatory interest defines what it means for subjective claims to be authentic: transparency to oneself in one's own genesis.

Understanding this framework was, for me, a kind of conceptual liberation. It explained why arguments about mathematics education were so intractable. Researchers were operating within different knowledge-constitutive interests, making different kinds of validity claims, without acknowledging the plurality of rationality. Someone arguing from a technical interest wants mathematical knowledge to be predictive and operationalizable. Someone arguing from a practical interest wants it to be meaningful and culturally situated. Someone arguing from an emancipatory interest wants it to be transformative and empowering.

These are not mutually exclusive. A critical mathematics can integrate all three. But integration requires acknowledging the differences, understanding the conditions of possibility for each form of knowledge, and resisting the temptation to reduce one to another.

\subsection*{Conceptualizations of Power: Foucault, Habermas, and Typologies}
\textit{Power} is a multifaceted concept, generally indicating a capacity to influence or control, with significant implications for social relations and the constitution of knowledge and identity. From a critical theory perspective, knowledge and power have a complex interrelation, where both common-sense and theoretical forms of knowing can function to support and reproduce social, cultural, political, and economic forms of oppression. Critical research, therefore, problematizes concepts like ``research'' and ``researcher'' because they are intrinsic to the production of ideologies that can impose power and maintain privilege \parencite{Carspecken2018}.

\paragraph{Foucault's Conception of Power}
For Michel Foucault, power is \textit{primordial and anonymous}, existing prior to truth claims and the very constitution of subjectivity \parencite[27-28]{Carspecken1999}. He suggests that humanistic conceptions of the actor and thinker should be discarded, as subjectivity and agency emerge as effects of power. In this view, power is understood as \textit{circulating} and operating through a \enquote{netlike organization} where individuals are both vehicles and effects of power; it is exercised through \textit{techniques and tactics of domination} rather than being a possession. % TODO: Verify BibKey for Walford & Carspecken (eds.), cited in original source block as Foucault, 1976.

The \textit{knowledge-power nexus} is central to Foucault's work, where power produces knowledge, and knowledge simultaneously constitutes power relations \parencite{habermas1992individuation}. 

This means that what is taken as \enquote{true, false, good, and bad} in a given era is determined by anonymous power, and discourse-practices construct subjectivity itself \parencite[28]{Carspecken1999}. Foucault's analysis focuses on how power is exercised through subtle mechanisms that can mask overt coercion, often by clothing the use of knowledge in the `garb of right' to obscure domination. He saw a \enquote{will to knowledge} that is generalized into a \enquote{will to power,} inherent in all discourses and operating as \enquote{anonymous processes of subjugation} \parencite[292-295, e-book]{habermas_philosophical_2004}. His concept of \textit{biopower} refers to a disciplinary power that deeply penetrates bodies, transforming creaturely life into a substrate of empowerment through scientific objectification and generated subjectivity \parencite{habermas_philosophical_2004}. Critics, however, argue that reducing truth to anonymous power leads to contradictions, as Foucault's own writing relies on being understood and convincing readers of its merit \parencite[29]{Carspecken1999, habermas_philosophical_2004}.

\paragraph{Habermas's Distinctions on Power}
Jürgen Habermas emphasizes an external relationship between truth and power, viewing power primarily as that which \textit{distorts interactions oriented toward reaching understanding} \parencite[46]{Carspecken1999}. In an ideal speech situation, power must be neutralized or equalized so that only \enquote{the force of the better reason} brings about consensus, allowing for pragmatically sound truth claims \parencite[46, 68]{Carspecken1999, Carspecken:1995aa}. At the same time, Habermas implicitly acknowledges a \textit{generative sense of power} tied to human agency and action itself \parencite{Carspecken1999}. All human acts are acts of power because they \enquote{make a difference} in the world and issue from agents who could have acted otherwise \parencite[81]{Carspecken1999}. Power is a capacity of the actor logically tied to the concept of action \parencite[81]{Carspecken1999}. Acts are considered \enquote{powerful} when they succeed in realizing their intended goals. Communicative acts also express power, particularly in relation to communicative goals like being understood or recognized, where the \enquote{will to truth} and \enquote{will to power} are intertwined with human motivation and identity claims \parencite[84]{Carspecken1999}.

\paragraph{Typologies of Power}
Power can be categorized to better understand its operation in social contexts. \textit{Interactive power} refers to power relations where actors are differentiated in terms of who has the most say and whose definition of the setting prevails \parencite[129]{Carspecken:1995aa}. This includes \textit{normative power}, where subordinates consent based on shared norms and status (e.g., student and teacher); \textit{coercive power}, involving threatened sanctions; \textit{contractual power}, based on agreements; and \textit{charismatic power}, won through personality \parencite[130-134]{Carspecken:1995aa}. 

\textit{Cultural power}, in contrast, refers to the extent of control over, or benefit from, the distribution and currency of cultural themes \parencite[131-132]{Carspecken:1995aa}. This form of power works by limiting the roles and repertoires of identity claims available to members of a group, penetrating their very identity and potentially causing pain.



\section{The Dialectical Turn: Divasion}
After I defended my dissertation, I felt a deep sense of openness. I met my wife, $\mathcal{A}$, and her two daughters $\mathcal{M}$ and $\exists$. In the fall of 2023, $\mathcal{M}$ was four years old. $\mathcal{A}$ and I were just dating, but I was excited to be a part of the kids' lives. I discovered that $\mathcal{M}$, $\exists$, and I like to write songs and we like to work on them together. $\mathcal{M}$ and I were sitting in $\mathcal{A}$'s basement trying to write a song. The first thing she says in this clip is that she wants to write a ``pattern song.'' The object she is referring to is the part of a microphone stand where the mic is held by the stand. The microphone extends past the clip, but is also clipped in, so part of it is inside of the clip and part of it is outside of the clip. I gave an talk that includes the audio of this clip for those interested in hearing the interaction\parencite{Savich2024}.


\begin{quote}
\(\mathcal{M}\): I'm going to do a pattern (I want to write a pattern song)

Tio: A pattern? Tell me what you mean, or just do it

\(\mathcal{M}\): Outside In! Outside In!

Tio: More! Say more! What is this?

\(\mathcal{M}\): The microphone is outside of this (pointing at the clip that is part of the mic stand) then it's inside of it.

Tio: What does that mean?

\(\mathcal{M}\): Um\ldots I don't know\ldots{}

Tio: It's kind of interesting, though, isn't it? That things can be inside and outside of each other. Sometimes at the same time.

\(\mathcal{M}\): It is?

Tio: Yeah! Well, this part is inside and this part is outside. So we can't say that the microphone is inside and we can't say that it's outside because it's both! That's a contradiction.

\(\mathcal{M}\): No, it's not. It's all on the other\ldots{} of all of them.

Tio: What's all over all of them?

\(\mathcal{M}\): That means that all of them is\ldots all of it\ldots{} that it divaded.

Tio: Divaded?

\(\mathcal{M}\): Yeah. Divaded means that\ldots um\ldots{} It means that it's inside the thing and then it's outside the thing.

Tio: Woah! Cool! I like that. So it's different than divided. Because dividing something sometimes means splitting but you're talking about divADed\ldots that's cool. It's inside and outside?

\(\mathcal{M}\): Yeah.

Tio: Tell me more!

\(\mathcal{M}\): It's mechanical. It's really not something. But I'm talking about when you grab something, it makes it divaded.


\end{quote}




\begin{figure}[h]
\title{\textit{Divaded Regions}}
\includegraphics[width=.8\textwidth]{/Users/tio/Documents/GitHub/September_UMEDCA/images/divaded.pdf}

\caption{\textit{Note.} The regions $x$ and $y$ may be inside, outside, separated, sharing a boundary, or divaded with respect to each other. The difference between liminal regions and divaded ones may not be appreciable unless the regions are in 3d, where it is evident that one holds the other.}
\label{fig:divaded}
\end{figure}

$\mathcal{M}$ noticed that there is a pattern, or rhythm in considering \textit{outside, in}. I will make much of this intuition, as the movement she noticed is easily missed. The microphone was both inside and outside of the clip, which I think most people would conceive of as a static spatial relationship. People regularly experience objects that are inside and outside other objects. A person walks through a door; a folder hangs out of a backpack; a potted plant breaks the surface of the dirt while growing its roots in the soil. But the movement prior to the static spatial relationship is important. The body is divaded by the breath in rhythmic movement. Many fears involve divasion, like a needle piercing the skin. Many pleasures involve divasion, from the embrace of a lover to the feeling of consuming a delicious meal. Birth and death divade the subject. 

In this interaction, the rhythm of the pattern song was flattened into a spatial concept. The song was arrested before she ever got a chance to sing it. Perhaps I erred in calling her concept a `contradiction.' Perhaps \textit{inside} and \textit{outside} only strongly contrast. A logical contradiction could be drawn between \textit{inside} and \textit{not-inside}. The issue of what constitutes a contradiction is important for mathematics. 

In any case, the interaction does say something about how paradoxes or contradictions are experienced. I cannot seem to hold a contradiction or paradox within one moment of awareness. Sentences like ``this sentence is false,'' toggle between truth and falsity, appearing to create temporal movement. Perhaps that rhythm is part of the ``pattern song'' she was after. In the face of a potential contradiction, she invented a word, \textit{divaded}.

$\mathcal{M}$'s invention suggests something profound about the foundations of mathematics. Formal set theory is based on the premise that an element of a set is strictly either inside or outside a set. This foundation appears to come from pruning more primordial spatializations like the one $\mathcal{M}$ observed. The consequences of this pruning are evident in foundational crises like Gödel's incompleteness theorems and Russell's paradox. Rather than repairing the effects of this pruning with fancy formal footwork, I suggest backing up. The foundations of mathematics can instead be approached through the academic field of math education, by listening to how children spatialize the world.

\section{Reflection: Methodology and the Reader's Role}
What did you just read? It wasn't madness, but it also wasn't a typical philosophy, mathematics, or math education text. In this section, I will describe the methodological framework that structures this work and the challenges involved in its construction. I will not \textit{argue} that mathematics is equivalent to the methodological framework I describe. Instead, I will leave space for readers to recognize that conclusion for themselves if such an identity is available. 


\subsection*{The Challenges of Reconstruction and Collaboration}

This book is a reconstruction of my dissertation \parencite{savich2022}. Hubristic as it seems, that document is my `life's work.' I had to heal through writing it. My task was to make the dissertation more accessible to a broader audience. But how could I make this dense work accessible without betraying the original intellectual labor? Distancing myself from the original text proved very difficult. The pedagogical task of teaching the text while simultaneously recreating it felt overwhelming. I needed help.

Why turn to AI when faced with such an overwhelming organizational and pedagogical task? The reasons are practical and pedagogical, supplementing the philosophical reasons discussed regarding the Hermeneutic Calculator.

Practically, I struggle with organization. I tend to write in terribly complex, multi-clausal sentences that often obscure rather than clarify. Furthermore, the dissertation process generated roughly 10,000 pages of notes. Organizing this material felt insurmountable, perhaps due to a learning dis/ability that makes standardization difficult.

I utilized AI as an adaptive technology and pedagogical partner. AI helped distill my notes, organize the structure, and translate dense prose into more accessible language. This use of AI as a tool for ``expressive empowerment'' addresses my specific cognitive constraints and can be understood as participating in what Negarestani characterizes as the history of intelligence (\textit{Geist}): an ongoing project of ``self-artificialization'' \parencite[p. 26]{Negarestani2018}.

While this collaboration risked distorting my voice, I figured that the way generative AI systems standardize language would help to teach the concepts. Is clarity more important than an authentic voice? I accepted some stylistic homogenization for the sake of teaching, but I have been careful to verify all outputs and actively directed the AI to maintain the recursive quality of the thought.

This collaboration raises the question of authorship in the age of AI. What constitutes authorship? Is it about performing every task, or about bearing the existential responsibility for the ideas? Authorship today is less about solitary genius and more about orchestration, direction, and responsibility. It is I who bears the ultimate responsibility for these ideas. The collaboration itself is inherently risky. It is I who takes this risk, kept up all night in existential terror of having my work analyzed and judged by anonymous critical others. The AI is not staying up all night worrying.

\subsection*{Methodology}
This book expresses \textit{critical mathematics} through a method I name \textit{critical autoethnography} (CAE). I name it so for the sake of its unnaming: \cancel{CAE}. Such movement, naming and finding the name inadequate (building and breaking), structures the text and its subject. The central methodological question is: How can deeply personal, lived experience be transformed into a rigorous theoretical resource?

Whereas `self-reference'---the great recursive engine of mathematical development---generally flattens a subject into an inert object called a \textit{referent}, ethnographers do not drown and splay their subjects like a rat for dissection. Instead, the task is to recognize the Other as a breathing subject. In critical autoethnography, the idea is to recognize Otherness in oneself. 


I originally called this work a theoretical autoethnography. 
The idea was to explore the intersection of philosophy and experience. Treating personal stories as data, I hoped to break the strict demarcations between objective research, ethics, and self-actualization. This approach requires the author to be simultaneously a storyteller and a theorist. What binds an individual's narrative? What makes it possible to understand that the child who supposedly said ``babbery'' is the adult writing this sentence? 

I figured that leading with Kant's \textit{transcendental unity of apperception}, the ``I think'' that must be able to accompany all of my representations, would lose readers who might otherwise benefit from what I have learned over the a few decades of experience trying to teach mathematics. ``Philosophy,'' that love of knowledge, does not need to be illuminated by the details of my life. Nor does mathematics. But I do not position this work as a purely philosophical or mathematical text. I did not get a Ph.D. studying Kant or Hegel. There is little doctrinal exposition of their works here. 

Instead, I work within the tradition of critical ethnography. In that tradition, where the purpose is to understand others with an interest in emancipation, the works of Kant and Hegel are positioned as parts of a wisdom tradition that can be drawn on to understand experience. 

I experience understanding as an unburdening. The weight of confusion momentarily lifts as I recognize my struggles in the questions and answers of others. The purpose of teaching seems to reside in recognizing such burdens. I won't try to confuse you, but the myth that \textit{anything} is obvious is a terrible burden. People use concepts, like ``2,'' in ways that are not easy to understand. 

Originally, I thought to try to situate this work around the kinds of questions that students ask, then I thought to orient it around the kinds of questions everyone asks. I tried to translate Hegel as if he were a therapist, then as if he were a meditation teacher. However, through those experiences, I found it hard to communicate the sense of relief that understanding brings. Claiming Hegel to be someone other than Hegel resulted in some stilted prose. You will find vestiges of that approach in subsequent chapters, but I decided to embrace the messiness of my own experience. The \cancel{theoretical} aspect is still present, but the \textit{critical} aspect is foregrounded.

This text isn't Kantian or Hegelian in form or content---you do not need to understand either to understand what I'm getting at. But when I felt like I understood the transcendental unity of apperception, it felt like an enormous weight was lifted. The \{I\}, whatever it is that the pronoun ``I'' recollects in general, plays an essential, if `accompanying,' role in discourse. So, while you don't need to have read Kant, keep in mind that whenever I write about the \{I\}, I'm grappling with this basic unity-of-consciousness concept that Kant highlighted. 


Among the wilder claims in this book is that mathematics and CAE share a common structure. Each recollect the self through otherness to recognize identity through difference. Each grow by virtue of the inadequacy of those recollections. Recollection transforms subjects into objects (\textit{reification}) by virtue of the limits of language: to use words is to be bounded by the learning experiences that made those words meaningful. That implies that each recollection is bound, too, by the identity claimed by the recollector. Language use is also bound to \textit{norms}, the rules or patterns of use that make what the words mean recognizable to others. Criticality involves challenging those norms, but in doing so, a new norm is always claimed. The critic then stakes their identity to that new norm, which is then subject to the same process of recollection and reification.

In this manuscript, ``critical theory'' does not refer critique in general, but rather to the tradition stemming from the Frankfurt School (Horkheimer, Adorno, Habermas) and related thinkers. It assumes that society and knowledge can be examined for power dynamics and potential emancipation. If you're new to this term, think of critical theory as the practice of questioning the status quo with an interest in liberation. Critical theorists ask who benefits from ``common sense'' and how things could be different. Here, my goal is to recover the emancipatory power of logic and mathematics (and a bit of philosophy), which is an expressive power, from the way those topics are used to constrain and manipulate people.


In mathematics, which tends to deal with patterns, criticality is enacted in many ways. The one that binds this book is \textit{diagonalization}. The details of diagonalization can get complex, so I offer a simple example that I taught to a group of pre-service teachers in the spring of 2025. Let two symbols, $\star$ and $\Box$, be combined into patterns. Those patterns can represent numbers, or any kind of pattern you may wish to change in your life. 

$$
\begin{array}{c c c c c c}
\star & \Box & \star & \Box & \star & \ldots \\
\star & \star & \Box & \star & \star & \ldots \\
\star & \Box & \star & \star & \Box & \ldots \\
\end{array}
$$

I asked those pre-service teachers if they could find a pattern that was not captured by the patterns above. They were able to find one easily, as there were only a few lines to consider. But I also asked them to consider a method that would always allow them to find a new pattern from any number of patterns. If you look down the diagonal of such a list, and change each $\star$ to $\Box$ and each $\Box$ to $\star$, you will find a new pattern that is not captured by the patterns above. For example, if we take the diagonal of the three patterns above, we get: $\star \Box \star \ldots$. This new pattern is not captured by the patterns above, and it can be generated from any list of patterns. A new norm, pattern, or rule is asserted. It is not identical to any that came before, but is recognizable as a pattern within the same family of patterns. 

As exciting and liberating as it is to transcend a pattern and write some new norm with good reasons (`creativity'), I somewhat derisively name the mechanization I articulate to represent that process the \textit{More Machine} (see \ref{sec:more_machine}). Some critiques are banal. They feel algorithmic. Above, I articulated a pattern for transcending patterns. But the kinds of embodied transformations that feel like genuine learning escape patterned descriptions in more fundamental ways. That said, once those transformative experiences are recollected, the resulting universals (words) follow communicative norms...algorithmically. The distinction between transformative learning and \textit{algorithmic elaboration} is very hard to draw convincingly. 

In math education, the distinction between \textit{conceptual} and \textit{procedural} knowledge is similarly hard to draw. In the Hermeneutic Calculator, I will offer a way to understand the distinction that is more fundamental than the usual definitions. To prime readers for that discussion, I will say that conceptual knowledge points to where you're going by explaining where you've been. It is both elaborated from procedural knowledge and explicative of the procedures it is elaborated out of, a relation that Brandom calls LX \parencite*{Brandom2008}. 

First, \cancel{we} must work to understand that no name is ever adequate to the thing it names. Words cast the ineffable particularity of experience into the universality of language. The experience of reading philosophy is often tarnished by the sense that the philosopher is trying to capture the ineffable in a jar of words. Where am \{I\} in all of these words? As a reader, you may feel like I am missing who you are. Because all words are repeatables, they all function as universals. That means making universal claims is inevitable. When I make ``we'' claims in this book, I am making claims about the patterns I've observed in myself that I recognize others as also enacting. But every pattern can be transcended. That means every ``we'' claim is defeasible. \cancel{We} are the ones that say ``we.''

However, \cancel{we} are also the ones who bridle at such claims of universality. \cancel{We} say ``no'' to the ``we'' claims \cancel{we} make all the time. But \cancel{we} can also reach consensus. \cancel{We} can acknowledge the initial ``no'' of a claimed universality, and then say ``no'' to that no-saying. Consensus is not exactly a full-throated ``yes!'' It is a \cancel{no}. 

Furthermore, \cancel{we} can transcend observable patterns with good reasons. However, \cancel{we} cannot transcend the conditions that enable that transcendence. I cannot, for example, transcend the \textit{assumption} that \textit{someone} might be able to understand what I am writing. If no one could understand it, I would not write it. The assumption that someone might understand what I speak or write is the first `axiom' of critical mathematics. 

I think it will be too irritating to readers to write ``\cancel{we}'' everytime I need to use the pronoun that binds you, the reader, and I, the author, to each other through the text. But it will also be irritating to you to if I do not acknowledge that you can disagree with any universal claim I make. I ask a lot of you in this text. You are a participant in its unfolding---a silent partner whose role as a reader is indispensible to the movements I make. In \cancel{our} dance, I must lead, but you need not follow. The first big ask I make is that you append each abrasive (and perhaps immoral) claim to universality with a strikethrough. Each time I write ``we,'' I mean \cancel{we}. But it is worse than that. Every single word is a repeatable and, therefore, a universal. Every word is a ``we'' claim. The whole thing would be rather illegible if I \cancel{struck} \cancel{through} \cancel{each} \cancel{word}. In some sense, the signs themselves always \textit{already} slash through the tenderness of particularity. I will retain the practice but limit myself to its use when a concept name must erase itself or when the \textit{second negation} insists on it. 

\section{Integration: Divasion, Identity, and the Foundations of Mathematics}
I attempt to situate critical mathematics as a viable \cancel{philosophy} of mathematics. When I tried to publish in that domain, all I experienced was rejection and frustration. Since this is \cancel{my} story, I want to take the opportunity to dissent from the dominant narratives in the field. The first, given to me by $\mathcal{M}$'s invention of divasion has to do with the doctrine of the excluded middle. Axiomatic set theories, to varying degrees, assume that every proposition is either true or false. Further, every mathematical object either is or is not an element of a set. The law of the excluded middle is often taken as a given in classical logic, but it has been challenged by various non-classical logics, such as intuitionistic logic and paraconsistent logic. Intuitionistic logic, for example, does not accept the law of the excluded middle as a general principle, emphasizing constructive proofs where the existence of an object must be demonstrated rather than inferred from the negation of its non-existence. Paraconsistent logic allows for contradictions to exist without leading to triviality, meaning that not everything becomes provable in the presence of a contradiction.

Those alternatives have been taken up by math educators to varying degrees. I will not rehearse those debates here, as I want to focus on the more primordial problem of how subjects and objects relate to each other. The term \textit{divaded} captures a fundamental aspect of subjectivity that has deep implications for the philosophy of mathematics and math education. 

Through expression, what we take as mathematically real expands, as I will detail in the bridge chapter where I trace the history of math from Pythagoras through G\"odel. Further, while $\mathcal{M}$ described physical objects held by one another, the term \textit{divaded} can be extended to concepts that are inside and outside of each other, like ``parent'' and ``child,'' and concepts that are inside and outside of themselves. Brandom \parencite*{Brandom:2019aa} explores Hegel's dialectics through the concept of \textit{reciprocal sense-dependence}. The sense of ``child'' depends on the sense of ``parent,'' and vice versa. I choose to approach the topic through divasion, as it arose organically in conversation with a child. That feels fitting, given the intended audience of math educators. But I will also be exploring the `grown up' version that Brandom supplies and directly engage in a few instances of dialectical reasoning. Sophisticated readers may substitute lexically sophisticated phrases in place of ``divasion'' without much loss of texture. The only significant difference is that I take divasion to express a primordial spatial relationship that is then divided into the categories of inside and outside. Since ``we'' begins in the womb, I take the law of the excluded middle to be more or less the first mistake of many approaches to the foundations of mathematics. Those philosophies fall apart when confronted with the problem of divaded concepts. 

For example, Gottlob Frege's work on the foundations of mathematics was deeply concerned with the relationship between concepts and objects. He sought to establish a logical foundation for mathematics, but his work was ultimately undermined by Russell's paradox, which exposed the contradictions inherent in what is now called `naive' set theory. Naively, a set is a collection of objects, where the objects are called \textit{elements}. In the platonist tradition that guided much of the push for formalism, all mathematical concepts exist as discoverable objects, so collections of objects are themselves objects. Borrowing from Brandom, objects do not contain contradictions, while subjects cannot seem to escape them. I hold various incompatible commitments, but feel I should not; objects, like a monochromatic yellow square, exclude incompatibilities like `redness' or `circularity.' Russell's paradox, loosely, asked whether the set of all sets that are not elements of themselves, $\mathcal{A}$, is a element of itself. Symbolically, we ask: is $\mathcal{A}\in \mathcal{A}$? For the set of all sets that are not elements of themselves to be an element of itself, it must not be; if not, then it must be so. Whereas $\mathcal{M}$ might simply say that $\mathcal{A}$ divades itself, Frege responded with dismay \parencite{sep-russell-paradox}.

What is a set? They are sometimes introduced with physical metaphors, like a box that contains objects, or envelopes that may or may not contain other envelopes or letters. But these metaphors are misleading. They reify concepts into objects, but concepts are not their reifications. A big problem in mathematics is the tendency to subsume the subject into the realm of objectivity. To help, I will define sets as \textit{recollections}: they are more like memories than boxes. One reason I (somewhat obnoxiously) put the pronoun \{I\} in curly braces sometimes is to remember that which the pronoun recollects. I do not do this when referring to myself as an author; instead the construction \{I\} is supposed to point to the locus of action that those who say ``I'' are recollecting when they use the term. As the `source' of action, the \{I\} takes on spiritual dimensions for those who practice yoga. It is also to point down the text to when I discuss numerals as first-person pronouns.

This divasion between \{I\} and recollection points toward a larger structure that Hegel named \textit{Geist}---a term encompassing ``mind,'' ``spirit,'' and the collective self-consciousness of a rational community. \textit{Geist} divades human experience in a way analogous to the sphere in a hypercube: it is the self-flattening movement between the individual and the collective, the historical and the present, the subjective and the objective. Like the microphone that $\mathcal{M}$ observed, \textit{Geist} is both inside and outside of each individual consciousness. I participate in \textit{Geist} through my use of language, my mathematical reasoning, my engagement with norms; yet \textit{Geist} also exceeds me, encompassing the entire historical community of those who have thought, spoken, and counted before me.

Robert Brandom offers a pragmatist reading of \textit{Geist} as the historical process through which a community institutes normative statuses through practices of reciprocal recognition and retrospectively determines conceptual content through recollective narration. \textit{Geist} is not a supernatural entity but the living, evolving web of social practices and historical self-understanding. When I use the numeral ``2,'' I am not naming a platonic object; I am participating in a practice with a history, recollecting a normative status that has been instituted through countless acts of recognition and refined through historical struggle.

The paradoxes of self-reference that troubled Frege are not bugs to be eliminated but features of our status as subjects who are both inside and outside the systems we construct. \textit{Geist} divades itself: it is both the historical process of meaning-making and each individual's participation in that process. Like the sphere that cannot be pictured in the hypercube but structures its intelligibility, \textit{Geist} is the unpicturable condition that makes mathematical and linguistic meaning possible.

Russell's paradox is a classic example of the problem of self-reference and the inclusion paradoxes that arise when elements of a system are allowed to talk about themselves. Such paradoxes inspired others to search for more formal rigor. Frege's system fell and was replaced with Russell and Whitehead's approach as written in \textit{The Principia Mathematica}. But tightening mathematical systems did not resolve the fundamental problems that dogged the formalist pursuit. Instead, Gödel demonstrated that any coherent mathematical system that includes numerals and multiplication---anything past the third grade---is fundamentally incomplete. Since the 1930s, mathematicians have sought to exclude, ignore, or reify the paradoxical aspects of self-reference. I do not intend to trash those efforts, but instead suggest that the problem runs much deeper, into the hard parts of being human. Rather than tightening, I want to loosen to articulate an \textit{in}formal mathematics.

Rather than being sunk by the paradox of divasion, I want to embrace that paradox, in its universality, as an essential aspect of what it means to be a subject who participates in \textit{Geist}. Mead's \{I\} and ``me'' describe the relationship between the self and society. The \{I\} is the spontaneous, creative aspect of the self, while the ``me'' is the socialized aspect that is shaped by interactions with others. Specifically, the ``me'' is the self-as-recognized \parencite{Carspecken1999}.\label{def:me-as-recognized}

\begin{quote}
The simplest way of handling the problem would be in terms of memory. I talk to myself, and I remember what I said and perhaps the emotional content that went with it. The \{I\} of this moment is present in the ``me'' of the next moment. There again I cannot turn around quick enough to catch myself. I become a ``me'' in so far as I remember what I said. The \{I\} can be given, however, this functional relationship. It is because of the \{I\} that we say that we are never fully aware of what we are, that we surprise ourselves by our own action. \parencite[\S 22.2]{mead1934social} 
\end{quote}

In this sense, $\mathcal{M}$'s term \textit{divaded} captures the tension between the individual and society. I often feel misrecognized, so the distinction (I $\neq$ me) between the \{I\} and the ``me'' has a felt-sense. I also sometimes feel at ease with myself as the tension melts away (I = me). That feeling is so good, and its alternative so bad, that I want to simply be myself. But I cannot control the ``me.'' I cast myself before you in existential fear of being misunderstood but with trust that you are likewise afraid. Perhaps you and I will feel some comfort in our divasion. 

In the supplementary materials for this manuscript, I offer various attempts to formalize critical mathematics. Think of those as like $\mathcal{M}$'s initial sketches of a cube. They are not polished, but they are a start. I do not include the concept of divasion explicitly in those formalisms. The computational models that I draw on are too deeply embedded in classical logic to allow for such a concept to be involved in proving the consistency of the system. But I hope that the spirit of divasion infuses the work.

There is so much more to say about critical mathematics, but `showing' is better than `telling.' The rest of the book is an extended attempt to actualize the emancipatory potential of mathematics. In the next section, I will discuss how the rest of the book unfolds. 


\section{Conclusion: An Opening}
Writing new ideas in the very old domain of mathematics is a daunting task. The history of mathematics is storied with geniuses whose work I can't understand. Making it harder still are the uncounted multitudes of genius who have filled in the gaps between those illustrious souls. CAE is one way in. But what does a framework provide? Simply having new words to cover old ground would just add noise to the signal. And I imagine readers are not quite sure what CAE is about anyway. 

I titled this prelude \textit{Built to Break} after a song I wrote of the same name. I was trying to get at the idea of repetition, moving through the Hegelian moment of the negative to its Kierkegaardian moment. In Kierkegaard's essay \textit{On Repetition}, he describes pokes a bit of fun at old Hegel's insistence on recollection as the primary mode of knowing. He offers \textit{repetition} as a counterpoint to recollection. 



The first verse of the song that began this chapter is about a time $\mathcal{M}$, her sister $\exists$, and I were playing with blocks. $\mathcal{M}$ asked me to help her ``build something to break it.'' So we built tall towers, over and over. She and her sister had a blast knocking them over. In chapter 5, I will discuss the middle verse. The theoretically important part of the middle verse has to do with the the question of whether I simply am a recollection. The last verse, which is in the conclusion, is about my father. It contains a line about an ocean of rainbow that really won't make sense until later in the book. 

Until then, allow the `ocean of rainbow' to reflect the merely \textit{aspirational} structure of the book. In Figure \ref{fig:mobius}, I have rendered a M\"obius Strip as a rainbow. At one point, I made a conceptual map by color. For example, I represented \textit{being} in red and \textit{nothing} in violet, to suggest the movement between those concepts as a constant \textit{becoming}. That ended up feeling cartoonish, but I wanted to structure the book vertically, horizontally, and as dyadic action. That last bit is tricky. But if you recall M.C. Escher's print of ants marching on a M\"obius Strip, it might help to understand how I see your role as a reader. There aren't really optical anti-prisms---glass objects that unbreak light. But I think of your role as if you were such an anti-prism; but an ant like Escher's. Your role is to take these finite (bounded) words, this broken light, and bring them back together through the disciplined openness of reading. That is how I conceptualize the ideal listener. A picture cannot capture the progressive unfolding of the text, but I felt it was important to communicate that I am not striving for linearity. I fall---broken---through the plane of experience, while you bring those pieces together, recognizing, just as you might recognize each of the cubes in the series in figure \ref{fig:hypercube} as a cube, the implicit whole. 

What the image suggests is what a self-divaded object might look like. If you take a M\"obius Strip and travel around its `inside,' you will eventually find your finger on the `outside,' until finding yourself back on the `inside' again. Readers can make a M\"obius Strip by taking a strip of paper, twisting it once, and taping the ends together. The twist is important. It is the twist that allows the inside to become the outside. Ideally, the `twist' would be constantly recognizable throughout as rhythms in the text. To make the `rainbow' as I have drawn it, you need only color the `top' and `bottom' of the strips the same color. If you wanted the colors to line up, you would have to reverse the order of the colors, given the twist. I do not want them to line up, as a main claim of the work is that identity arises through difference. Concepts flow into one another, but there are often abrupt shifts in understanding. The ``aha'' moment will be explored in finer detail in the next chapter, but the disjointed rainbow is my attempt to visualize such moments. 

However, figure \ref{fig:mobius} also includes a representation reminiscent of the Koch Snowflake, except woven of the M\"obius strips. The Koch Snowflake is a fractal, which means that it is self-similar at all scales. To use words is to make universal claims. The self-similarity of the Koch Snowflake is a way to get into the idea of universal \textit{difference}. The self-similarity that I am trying to express throughout the text is a universal difference. That universal difference can\textit{not} be represented in a static way. Still, if you recall M.C. Escher's---all I can represent is the self-similar aspect of that universal difference. So, the text is topoligically structured as series of M\"obius Strips, but each concept develops discretely through a fractal-like self-similarity which is universal difference.





\begin{figure}[h]
\title{\textit{Organizational Structure of the Text}}
\includegraphics[width=.8\textwidth]{/Users/tio/Documents/GitHub/September_UMEDCA/images/mobius_text_structure.pdf}

\caption{\textit{Note.} The text unfolds parametrically like paths around a M\"obius Strip. Each main topic flows into its other, returning back to itself in an elaborated form for the second reading. Simultaneously, each concept develops in a fractal-like self-similarity that expresses universal difference. The Koch Snowflake beneath the M\"obius Strip reflects that fractal-like development.} 
\label{fig:mobius}
\end{figure}


Note that in figure \ref{fig:mobius}, only green is mapped to green. In the transformation that I hope will take place for readers, only ``no'' maps itself to itself, while still enacting the `flip.' The other colors are mapped to each other. While the diversity of the text is burdensome, what remains constant is that we all say no. In mathematical terms, determinate negation (as defined in \ref{def:determinate-negation}) is the fixed point of determinate negation. It is the only concept that retains itself under its own process. It will undergo transformation, becoming an other to itself, but it does so under its own power. 

One might ask what the coordinate system is for the M\"obius Strip. Borrowing the phrase from Sellars \parencite*{sellars2007space}, it is the space of reasons, which has to do with the triadic progression of validity claims. I will discuss this space explicitly throughout the book, but a hint of what I am after is in figure \ref{fig:vector_space_of_desire}. 





Throughout, I will use songs and bits of poetry that I have written over the last decade of work on my dissertation and this subsequent book. I include them for reasons that depend, in part, on the song in question. However, as a class, they draw on the same linguistic structures that mathematics draws on. Semantically, they address the same paradoxes that mathematics confronts. The sameness I claim is not a formal equivalence, but a kind of identity over difference. 

Anaphora, in the analytic tradition, is the use of a word or phrase that refers back to another word or phrase used earlier in a discourse. For example, in the sentence ``John arrived late because he missed the bus,'' the pronoun ``he'' is an anaphoric reference to ``John.'' However, in rhetoric and poetry, anaphora is a stylistic device that involves the repetition of a word or phrase at the beginning of successive clauses or sentences. For example, in Martin Luther King Jr.'s famous ``I Have a Dream'' speech, the phrase ``I have a dream'' is repeated at the beginning of several sentences to emphasize his vision for racial equality. 

As the speech unfolds, each instance of ``I have a dream'' refers back to the original vision, but each repetition also adds new layers of meaning and context. The anaphoric structure creates a rhythm and reinforces the central theme of the speech, while also allowing for the development of the idea over time. It also builds energy through repetition. 

I tend to write what I call `front-porch songs.' They have relatively simple structures that almost always include verses and a chorus. The chorus is the part that repeats, while the verses develop the theme. I will occasionally write a bridge, which is a section that provides contrast to the verses and chorus, while drawing on the expressive resources developed in those parts of the song. The bridge often introduces a new perspective or a twist in the narrative. The bridge is also where I might change keys or introduce dissonant chords, allowing for the last verse/chorus pair to repeat in a way that builds energy through difference. 

The songs are included, in part, to demonstrate how anaphora functions at the rhetorical level. 

But the songs also function as an `outside' to the text. The musical performances aren't exactly `here.' The written word work as I argue mathematical objects work. I tend to write songs that exteriorize the interiority of experience. Usually, I find some need for recognition in myself, like smeared ink on a wet magazine, that I write about to try to understand. When understanding falls on me, the need is de-personalized---cast into the universality of language. But for others to recognize themselves in the song---to value the understanding the song purports to express---the de-personalized `lesson' needs to include the context in which learning occurred. I tend to express that context through a personification of space. function as what Agamben \parencite*{agamben_language_2006} calls \textit{shifters}. A shifter is a bit of language (like \{I\} or a poem) that doesn't simply refer to an object. Instead, they refer to the \textit{event} of language. Agamben works on the Continental side of the divide between Continental and Analytic philosophy, so his work is not as well known in the Anglophone world. However, I find his work on shifters to be a useful way to think about how language can be used to refer to itself. 

From the Analytic tradition, I will use Robert Brandom's work to argue that numerals are \textit{anaphoric terms} (see \ref{def:anaphora}) that recollect the ``I think'' that `can accompany all of my representations.' Technically, anaphoric terms are those that refer back to something previously mentioned. That technical definition is a hindrance for my argument, as it suggests that there is an empirical referent for each anaphoric term. When I submitted an earlier version of chapter 7 to a prestigious journal devoted to the philosophy of mathematics, the reviewers indicated there are empirical tests to determine whether a term is anaphoric. In their thoughtful rejection letter, the editor suggested that I must be using the term metaphorically. For that reason, among others, I embed the argument in a larger conceptual category called \textit{shifters}, which can include both indexical terms (terms whose meaning depends on the context of their utterance) and anaphoric terms. The most obvious examples of anaphoric terms are pronouns like \{I\} or ``it.'' When embedded in the concept of \textit{shifters}, anaphoric terms refer back to the \textit{event} of language itself, which precedes any subsequent unfolding. The \textit{event} of language is deeply implicit; chapter 5 will discuss it in detail, when I argue that songs and poems also function as shifters. 

So, I will be arguing that songs, poems, and numerals can be understood as a unified class of linguistic expressions that refer back to the event of language. I do not mean to suggest that these expressions are interchangeable. While they share a referent, they are vastly different in \textit{sense}. Rhetorically, when I anticipate that some bit of analysis I wrote is too sterile or compressed to follow, I will insert some bit of song to downshift from language, back towards its event. Structurally, songs and poems are built to be repeated, so I will also use them as shifters to return the text to itself, creating the M\"obius Strip structure I described above. The first words of this preface are the first verse of the song \textit{Built to Break}. The second verse of that song is in the Bridge chapter. The third verse will be discussed in the conclusion of the book. 

I confess I've made this book complex---perhaps too complex. ``Why not just say it plainly?'' you might ask. The emotive `messiness'---the songs and personal asides---are my way of shifting out of theory when needed. In general, when the density of the theory starts to feel constricting, like a too-small inflatable lifejacket pumping ever up to squeeze my neck as I re-read the manuscript, I insert a poetic pin to deflate, decompress, and recollect the purpose of theory. Theory is here to serve as an expressive resource to talk about experience. For it to do its work, it cannot choke out the reader. Whenever you feel frustrated or lost, taking a moment to listen to a song or breathe with the text is to fulfill, not break from, the logic of the text. I ask for your patience; I think whatever energy you put into the text will be well spent. 

One last note on the structure of the book. I was inspired by my friend Xianqing (Dorcas) Miao's reading of Heidegger in an as yet unpublished paper. She describes his work as an attempt to read the hermeneutic circle both forwards and backwards. I attempt something a bit different here. The dialectic aspect of the text is structured as follows: Part I (Prelude/Introduction) opens expansively with personal narrative and broad themes (an open beginning), Part II delves into dense critical theory (a restricted focus on specific philosophical frameworks), Part III (``Bridge'') opens again by twisting from theory toward mathematical ideas (an opening transition), Part IV presents detailed mathematical concepts and formalisms (another restrictive deep dive), and Part V concludes by returning to broad, unifying ideas (``Repetition: Built to Break''---an opening out into reflection and future possibilities). This is a kind of zig-zag pattern, with a fractal-like self-similarity happening within each chapter as well: openness $\rightarrow$ restriction $\rightarrow$ openness, then repeat. 

While the text is a dialectical unfolding when read cover-to-cover, I also wanted to write with `vertical' layers, where the Möbius Strip can be thought of as spreading out from the central notion of determinate negation. So, the Bridge (chapter 6), can be read first, then the mathematical chapters (7, 8, and 9), then the theory chapters. Because I am trying to make the text legible from within any layer, I often repeat myself verbatim---especially when I cite other thinkers. I also try to write with `childish' examples at the beginning of chapters, because I want the text to be open to everyone who can read. I don't think anyone will get all of what I write, as I get lost in my own prose when I re-read. But that is by design: everyone (I hope) can get something from the text, but no one should get everything. When you find the desire for certainty is frustrated by the text---when you find that you just can't understand what I'm going on about---you can skim! More might disclose itself if you relax.

This opening chapter has established the experiential and theoretical foundations for the journey ahead. The five foundational anecdotes---from the calculator-playing child to the grieving father---reveal the common structure of mathematics and critical autoethnography: both recollect the self through otherness to recognize identity through difference. The personal stories demonstrate how formal systems, while powerful, become alienating when divorced from the material conditions of human experience.

The theoretical framework of critical autoethnography provides the methodology for transforming lived experience into philosophical and mathematical resource. Through the framework of diagonalization, recognition studies, and the triadic progression of validity claims, we begin to understand how mathematics might be reconceived not as a tool of social control, but as an emancipatory expressive resource.

\subsection*{Roadmap: The Journey Through Eleven Chapters}

The book unfolds through eleven chapters, each structured in seven sections following a dialectical rhythm. This structure mirrors the ``zig-zag'' pattern described above: openness $\rightarrow$ restriction $\rightarrow$ openness, recursively enacted within and across chapters. What follows is a preview of each chapter's core contribution and internal architecture.

\textbf{Chapter 1: The Sound of Time} introduces \textit{determinate negation} through an embodied metaphor connecting the rhythm of inner experience to sound's physical nature. The chapter centers on \textit{The Exercise}, a guided meditation cultivating self-certainty through proprioceptive awareness. Section 1 establishes the misrecognition of certainty, distinguishing determinate from abstract negation. Section 2 presents the Exercise itself, guiding readers through bodily awareness practices. Section 3 develops the metaphor of ``sound of time,'' connecting breath rhythms to conceptual movement. Section 4 introduces polarized modal logic ($\SBox$, $\OBox$, $\NBox$) to formalize expansion/contraction dynamics. Section 5 examines representational thinking's limitations and artistic expression's role. Section 6 explores the intersubjective dimension, how individual Exercise experiences connect to shared understanding. Section 7 synthesizes embodied practice as the ground for critical mathematics, previewing how this phenomenological foundation enables subsequent theoretical developments.

\textbf{Chapter 2: Inferential Movement} examines the ethics of inferential reasoning through Robert Brandom's inferentialism, arguing meaning arises from inferential roles rather than pre-given objects. Section 1 presents the misrecognition of inference through a quadrilateral classification case study. Section 2 explicates Brandom's distinction between formal and material inference, emphasizing how material inferences carry semantic content. Section 3 develops incompatibility semantics, showing how negation structures meaning through what claims rule out. Section 4 extends Brandom's account of error from perceptual experience to interpersonal misrecognition in mathematical communication. Section 5 analyzes the quadrilateral example in detail, revealing how geometric classification embodies normative commitments. Section 6 connects geometric figures to the ``I think,'' foreshadowing the claim that numerals function as pronouns. Section 7 synthesizes how meaning emerges from embodied practices and social norms rather than correspondence to abstract objects.

\textbf{Chapter 3: Existential Needs} explores two fundamental existential demands: recognition as ``good'' within normative frameworks and recognition as infinite expressing authentic selfhood. Section 1 introduces the Turbo Lowers story, an outsider artist whose property was destroyed by town ordinance enforcement, illustrating the conflict between conformity and creative expression. Section 2 develops George Herbert Mead's I/me distinction, analyzing the tension between socially constructed and spontaneous self. Section 3 investigates how this tension manifests in educational contexts, particularly mathematics where students balance procedure with creativity. Section 4 employs Brandom's pragmatist reading of Kant alongside Hegel's restless negativity to reveal the needs' deeper unity. Section 5 examines the fear of nothingness arising from I/me split. Section 6 articulates how confession, forgiveness, and trust structures enable integration. Section 7 concludes with ``Beast of Love'' poem, a site for reflection on surrender and recognition's transformative power.

\textbf{Chapter 4: Thoughts for Two---Who Are You?} examines intersubjective recognition structures through the question ``Who are you?'' revealing dynamics in mathematical communication and human development. Section 1 introduces the CUSP (Claimed Universal Subject Position) you concept from research on preservice teachers, exploring how ``you'' can reference a generalized rather than particular other. Section 2 distinguishes CUSP you from transactional you using Sebastian Rödl's intentional transaction framework, explaining dyadic versus monadic speech acts alongside the Big Gorilla story context. Section 3 analyzes communicative practice through the full Big Gorilla narrative (chest-thumping, ``hoohoohoo''), applying Mead's framework of gestures becoming significant symbols. Section 4 describes the communicative breakdown moment (thumping too loudly, tears), applying Habermas's theory distinguishing communicative action from rational discourse through validity claims. Section 5 introduces Brandom's analytic pragmatism (vocabularies, practices, PV/VP sufficiency), discussing theoretical automata as models while acknowledging formalization's limits for capturing genuine recognition. Section 6 tells the second Big Gorilla story (table-slamming incident), describing $\exists$'s forgiving gesture (thumping her belly to reintegrate Daddy Gorilla's anger), highlighting how her creative response transcended algorithmic elaboration. Section 7 synthesizes the I-You dyadic intersubjectivity structure, emphasizing recognition's recursive developmental nature, connecting back to Beast of Love and forward to thought's limits.

\textbf{Chapter 5: Limits of Thought, Deconstruction, and the Voice} explores thought and language's boundaries through personal loss, philosophical inquiry, and musical expression. Section 1 presents the eulogy delivered at the author's father's funeral, introducing the problem of presence through grief. Section 2 develops Giorgio Agamben's concept of Voice---the pure language event subtending all particular utterances---theorizing meaning's emergence at the saying/unsaying boundary. Section 3 analyzes a letter written to the father after death, examining how absence functions as presence's enabling condition. Section 4 presents original songs articulating grief, memory, and recognition's complexities, demonstrating how artistic expression addresses what conceptual thought cannot fully grasp. Section 5 investigates meaning's emergence through shared practices and interpretations using différance and reciprocal sense-dependence. Section 6 connects these themes to null representation in mathematics, drawing parallels between mourning and mathematical understanding. Section 7 synthesizes how accepting absence as understanding's necessary condition enables richer conceptions of mathematical thought, previewing the bridge to mathematical applications.

\textbf{Chapter 6: Bridge: A Foundational Star} serves as the text's pivotal transition, exploring parallels between Dr. Seuss's \textit{The Sneetches} and Georg Cantor's diagonal proof through the star as both social marker and mathematical symbol. Section 1 establishes the misrecognition of totality through the Sneetches story, showing how finite systems transcend limitations through self-reflection. Section 2 traces diagonalization's history from ancient Greek incommensurability proofs through Cantor's work on infinity. Section 3 situates this history within the ontotheological context surrounding infinity, examining how the star symbolizes both totality and constitutive incompleteness. Section 4 explicates Cantor's proof that real numbers are uncountable, examining decimal expansion applications. Section 5 develops Haim Gaifman's generalization of the diagonal method. Section 6 discusses the empty set and zero concept, connecting to becoming and numerical understanding's development. Section 7 synthesizes how diagonalization embodies determinate negation, bridging philosophical foundations (Part I) with mathematical applications (Part II), preparing for the claim that numerals function as pronouns.

\textbf{Chapter 7: Numerals are Pronouns} presents the book's central mathematical claim: numerals and number words function as first-person pronouns rather than names for abstract objects. Section 1 establishes the misrecognition of reference through conventional views treating numerals as object names. Section 2 recounts the author's experience with a struggling student whose question---``What even is two?''---catalyzed reconceptualizing number grounded in self-consciousness structures. Section 3 introduces an Exercise for cultivating introspective listening, providing embodied foundation for understanding numerals as recollecting the ``I think.'' Section 4 explores resistance, naming, and presence desire's interplay characterizing mathematical understanding. Section 5 connects these experiences to null representation ($\emptyset$) and determinate negation from earlier chapters. Section 6 develops the central claim that null representation symbolizes the unrepresentable ``I think''---pre-conceptual experience ground making all representation possible. Section 7 aligns this interpretation with Brandom's singular reference analysis and de re ascriptions, proposing that grounding mathematical understanding in self-recognition structures motivates correctness pursuit as authentic self-recognition.

\textbf{Chapter 8: Algorithmic Elaboration and History} explores mathematical concepts and procedures' development through algorithmic elaboration, arguing mathematical history reflects a dialectical pattern where apparent completeness generates new possibilities. Section 1 problematizes traditional literature review approaches, advocating for understanding intellectual history as dynamic conceptual development. Section 2 connects the numerals-as-pronouns claim to critical arithmetic development through Brandom's algorithmic elaboration concept. Section 3 reconstructs Euclid's infinity of primes proof using incompatibility semantics, demonstrating how ancient results embody inferential structures. Section 4 examines arithmetic operations' emergence from embodied metaphors, drawing on Lakoff and Núñez's cognitive linguistics work. Section 5 reveals simple algorithmic elaboration's limitations through these case studies. Section 6 introduces ``pragmatic expressive bootstrapping,'' where conceptual systems develop by explicating implicit normative structures in existing practices. Section 7 synthesizes how treating history as ongoing conversation transforms each contribution, elaborating and transforming what came before rather than merely adding to a static repository.

\textbf{Chapter 9: Operation} explores mathematical operations' nature, arguing they are not merely formal symbol manipulations but expressions of embodied, normatively regulated practices. Section 1 establishes the misrecognition of operation through conventional accounts treating operations as abstract axiom applications. Section 2 develops ``critical arithmetic'' framework---a pre-formal system capturing everyday arithmetic's dynamic, error-inclusive nature. Section 3 examines arithmetic's embodied basis through multiplication, using C2C (Coordinating Two Counts by Ones) strategy examples from Cognitively Guided Instruction research. Section 4 articulates how subjective experiences transform into shared rule-governed practices through normative frameworks based on Brandom's incompatibility semantics. Section 5 integrates Lakoff and Núñez's embodied cognition research, showing how grounding metaphors structure mathematical understanding. Section 6 demonstrates how these practices give rise to stable objective procedures modeled as formal automata. Section 7 synthesizes the three-dimensional framework---embodied practices, inferential rules, formal structures---emphasizing operations are grounded in human activity and social normativity rather than timeless abstract truths.

\textbf{Chapter 10: The Dialectic of Incompleteness and Recognition} synthesizes the book's core arguments, articulating seven critical mathematics themes likened to rainbow colors. Section 1 explores being and knowing's unity in movement, mathematics as autoethnography grounded in personal experience. Section 2 emphasizes recollection and recognition as central, developing how numerals function as anaphoric terms. Section 3 posits determinate negation as understanding's fixed point---the concept that retains itself under its own process. Section 4 describes the triadic validity progression from subjective certainty through normative rightness to objective truth. Section 5 explores null representation as the unrepresentable enabling conditions' symbol, connecting to language's self-referential nature. Section 6 articulates mathematical research implications, advocating for creative, contextual, dialogical practice. Section 7 reflects on mathematics as dynamic recognition language, emphasizing lived experience connections and inclusive, emancipatory understanding's potential, closing the circle by returning to the opening chapter's themes transformed through the dialectical journey.

Each chapter's seven-section structure enacts a mini-dialectic: opening with concrete experience or provocative case (sections 1-2), developing through theoretical elaboration (sections 3-5), and synthesizing toward broader implications (sections 6-7). This fractal self-similarity means readers encountering difficulty in one chapter can skip forward, returning later with fresh perspective. The text rewards both linear reading and recursive exploration.

\subsection*{Songs and Poems as Navigational Aids}

Songs and poems throughout serve as shifters---linguistic devices that refer not to objects but to the event of language itself. They function as decompression valves when theoretical density becomes overwhelming. Taking time to listen to a song or breathe with the text fulfills rather than breaks from the logic of the work.

The goal is not merely to critique mathematics but to reconstruct it---to imagine and articulate a mathematics that speaks to our existential needs, acknowledges the productive role of error, and recognizes the first-person perspective as essential to mathematical knowing. In the dialectic of mathematical understanding, as in life, our moments of breaking can become moments of breakthrough.

The reader can expect a complex and recursive exploration of the relationship between self, other, and mathematical knowledge---one that builds to break, seeking not land for conquest but depths of shadow in the familiar. Through this critical autoethnographic approach, we work toward a mathematics that is built to break in service of deeper understanding.


\printbibliography[heading=subbibliography]
