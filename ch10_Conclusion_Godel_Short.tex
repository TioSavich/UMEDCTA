\chapter{11: The Dialectic of Incompleteness and Recognition}

\subsection*{Formalization as Revelation}

We have gone on quite an adventure. Let me summarize where we have been so far. In the prelude, I tried to `invent' a methodology, critical autoethnography, to structure my comments on mathematics. The concept of divasion required a history for its expression, and so I documented where the word came from: it came from the desire to write a pattern song---a rhythm of ``outside/in.'' From there, I articulated a series of exercises designed to explore this rhythm as the \textit{sound of time.} That gave a metaphor for how knowing is being; intersubjective space is where reasons are negotiated. It arises through the temporal compression of learning experiences as they are artificialized in language. From that rationale for how space functions as a recollection of temporally extended experiences, I articulated how geometric expressions rely on the negative, articulated in Brandom's favored vocabulary of material incompatibility. The act of classifying quadrilaterals allowed for a pseudo-formalization of inferential strength. By playing in the closed space of quadrilateral properties (which are likewise material inferential proprieties), I was able to `prove' that squares are rectangles because everything (that I listed) that is incompatible with a rectangle is also incompatible with a square. Through the chapter on inferential movement, quadrilateral properties, inferential strength, and what counts as good substitutions were made determinate. In fact, they were made determinate enough to program into Prolog. 

In the chapter on \textit{existential needs}, I articulated some of my ambivalence towards `the law,' whether that law was set to destroy a friend's artistic expression or to codify mathematical expressions as axioms. The law must come from somewhere. If it is simply imposed externally as the systemic crush of individuals, then ambivalence should transform into hostility. But the law cannot be entirely bad. I identify with some norms---they make sense. Other `laws' must be critiqued with the hope of enacting some change. 

In the chapter that questions ``Who are you?'' I began the work of articulating how intersubjectivity is transcendental-like. Mead's work provided a foundation for Habermas' later work on describing how intersubjectivity---the assumption of communicative competency---arises through social interaction. We worked out a finite-state automaton for describing Mead's dog fights, found it lacking, and then incorporated Zeeman's own analysis of fighting dogs using his Catastrophe Machine. I built one from a record player, using the elastic as a metaphor for the embodied tensions between existential needs. I imagined hooking two such machines together, and considered the challenge of navigating social relationships. When I imagined turning the record players on, representing the systemic forces that always seem to be pushing me towards objectifying and commodifying, instrumentalizing and strategizing---acting in ways that just seem to drain my body of its meaning---I pushed from Verstand to Vernunft. From the mechanical to the meaningful, I had to give up control. The beast of love, as hard as it is to admit, still prowls my nights. But the refrain---trust the beast to come in tune---served to introduce the concept of trust. 

From that, I began the work of reconstructing the Voice---that origin of language. As the source of action, the \{I\} is that Voice. And yet, stripped of music, words, text, and sound, the Voice requires the voice to actualize itself. Like Geist, or Spirit, the Voice must flatten itself into expression. I used the Eulogy for my father as an opportunity to explore how the dyadic negative---a unity---breaks into a rainbow of difference. 

In the Bridge, I used the Sneetches and their stars to teach Cantor's diagonalization argument. This allowed me to metaphorically equate Hegelian sublation with diagonalization. I argued that Euclid's proof recognized the incompleteness of finite lists of primes, Hippasus' proof recognized the incompleteness of the rationals, Cantor's proof recognized the incompleteness of the irrational numbers, and Gödel's proof recognized the incompleteness of any coherent mathematical system that can prove theorems, add, and multiply. None of these instances of `diagonalization' threw out the old system entirely. These were not abstract negations, but determinate negations. The history of mathematics is the history of its becoming. 

I then argued that numerals are anaphoric terms. In the mathematical system that we have been working to articulate, this means that proving something belongs in the system amounts to proving that it is reconstructable as a recollection of embodied practices. Various alternative understandings of number were explored as they related to the knowledge-constitutive interests that Habermas \parencite*{Habermas:1971aa} articulated. Treating numerals as pronouns allows for the underlying, embodied, implicit ordinality of counting to be cast into the space of reasons. But perhaps more interestingly, the domain of embodied ordinality seems to relate as equally to human subjects as it does to corvids. I do not speak crow, so I cannot tell, grammatically, how they might be using numerals. But I can recognize them as embodied thinkers, doing the same sort of thing that I do when I count. 

With an idea of what numerals are now explicit, I could move on to articulating the difference between algorithmic elaboration and pedagogical elaboration through training: \textit{becoming} split into two separable roles. Lakoff \& Núñez's \parencite*{Lakoff2000} embodied metaphors, like the measuring stick metaphor, require training. Students, like my daughters on their car-ride to Indianapolois, have to explore distance as a temporal compression of learning experiences. They can't merely be said to be taught. I gave an instance of algorithmic elaboration by reconstructing Euclid's proof in terms of the incompatibility semantics of Robert Brandom \parencite*{Brandom2008} and the embodied mathematics that Lakoff \& Núñez articulate. Perhaps it might read as a bit more nuanced than simply articulating how long division is elaborated from multiplication and subtraction, the archetypal example that Brandom gives for algorithmic elaboration, as it required the introduction of an \textit{incoherence frame}, but such a frame is no more complex than the sort of toy I played with at the doctor's office as a kid, where the triangular shape could not fit through the circular hole. 

I then moved to applying what I know of critical action theory to formal automata to express mathematical operations. While a surface level shift, I explored automata as represented with circles. I was trying to get at the idea of self-monitoring. When I act, the temporally extended movement is accompanied, to greater or lessor degree, by the ability to STOP acting when things feel like they are going wrong. Automata cannot fully model self-monitoring. Still, the idea that the end of an act must turn back to its beginning to assess whether the act was successful or not, was why I made the aesthetic decision to try and represent these little machines as circles. 

In that chapter, I articulated three automata in gross detail. I discussed counting, rearranging to make bases (RMB), and coordinating two counts by counting by ones (C2C). I include about 20 other strategies for arithmetic, and a few related to fractions in the supplementary materials. 

I did other things with words along the way. For example, I discussed how the song \textit{Breath and Kindling} uses rhetorical anaphora to build energy. I discussed how \textit{Goodbye Friends} uses structured ambiguity to express what I could never say directly. I discussed how the poetic side of my writing often uses substitution at a smaller grain size than Brandom articulates. Recall how in that song, I hide the rhyme \{seven, leaven, heaven\}, and made use of the frame $\Box$ven, to find rhymes that would work. Do such substitutions challenge the principle that every speech act can be reconstructed as an assertion? Not exactly. Word games at the subsentential level instead demonstrate the fractal-like nature of language and analysis. 

But I highlight the above aspects of the book for the purpose of making a telos of this work explicit. At every step, I have gleaned some assertions that are representable in computer language. The hermeneutic calculator (HC) can represent counting, addition, multiplication, and proof, all within a `coherent' logic. The coherence of the incompatibility semantics that I programmed into Prolog is debatable. I did a bit more than copy and paste Brandom's incompatibility semantics into Prolog, but I read about how other logicians took \parencite{Brandom2008} apart and found it had some inconsistencies. I have no doubt that if my code were subjected to the same analysis, it would also be found wanting. The sort of coherence tests I ran were minimal. The system doesn't immediately declare, for example, that $1=2$. I truly considered suppressing all the formal work my writing partners and I have done over the last 3 years, but then I decided that the negative attention my work might receive from logicians and mathematicians could make the work stronger. I find it genuinely exciting, but I also include it under the hope that it will be irritating enough to the professionals whom I admire but whose work I don't understand, to turn their attention on my system. Perhaps that attention will result in better reasons that serve to demonstrate the telos of my work. 

Every bit of it is designed to be incomplete, so it may be somewhat silly to express the telos: the HC is \textit{formally} incomplete. Proofs of incompletness involving diagonalization are species in the genus \textit{sublation.} I wanted to create a mathematical system that grows beyond itself. We have, in essence, developed the raw conceptual material necessary to enact Gödel's incompleteness theorem. 

While somewhat silly, given that the whole thing is \textit{supposed} to be incomplete, the exercise of instantiating Gödel's theorem, which we turn to next, speaks to the politically powerful who seek to control K-12 students, teachers, and curricula. If the politics surrounding math education aspire to \textit{mathematical} coherence, then the those subjects must be recognized as \textit{in}finite. \cancel{We} are non-finite. Treating kids or teachers as if they are buckets to be filled with mathematical knowledge, then buckets whose fullness can be measured with standardized tests, for the purposes of categorizing people as more-or-less able to fill some role within the system (e.g., become a videostore clerk or a nurse or a mathematician), probably isn't \textit{all bad.} But such moves fundamentally misrecognize the human and mathematical subject. 



that, I began the work of articulating a formal modal logic for embodiment. I began this inquiry with a question I could not yet articulate. Seven chapters have brought me here—through the embodied origins of counting, the emergence of operations from gesture and rhythm, the construction of number kinds, the recognition of incompatibility as productive crisis, the power of diagonalization as self-transcendence, and the formalization of these student-invented strategies into a computational system. The question I could not name was this: What does it mean that we learn mathematics at all?

The reductive answer—the answer implicit in calls for a ``science of math education'' that treats understanding as the accumulation of procedures—is that learning mathematics means acquiring a finite set of rules. The student is a vessel. The curriculum is the content poured in. Assessment measures how much remains. This view assumes closure: that mathematical understanding can be completely specified, that mastery is achievable, that the system is finite.

I formalized the strategies to prove this view wrong. Not through argument, but through demonstration. I built the Hermeneutic Calculator (HC)—a formal system grounding arithmetic in the very cognitive moves invented by children emerging from embodied practice. The system counts via rhythmic grouping. It adds through compression and elaboration (COBO). It multiplies by chaining (C2C). It recognizes boundaries and transcends them (from $\mathbb{N}$ to $\mathbb{Z}$ to $\mathbb{Q}$). The formalization captures not just calculation, but the logic of mathematical reasoning itself: axioms, rules of inference, proofs of theorems spanning arithmetic, geometry, and number theory.

And then I applied Gödel's First Incompleteness Theorem. The result is not a failure of the model. It is the revelation I needed. The HC—this rigorous formalization of elementary mathematics as invented by children—is \textit{necessarily incomplete}. There exist truths the system can articulate but cannot prove. The formalized ``me'' (the strategies-as-recognized) cannot exhaust the ``I'' (the source of mathematical action). The horizon is necessarily open. We are not vessels. We are boundary-recognizers. We are transcenders. We are \textit{in}finite.

\subsection*{The Hermeneutic Calculator Interprets Robinson Arithmetic}

The technical foundation begins with expressive power. Gödel's theorem applies to any consistent formal system capable of interpreting elementary arithmetic. The standard minimal system is Robinson Arithmetic (Q), which requires definitions for Zero, Successor, Addition, and Multiplication.

The HC robustly satisfies these requirements. Zero is grounded in the axiom \texttt{axiom(zero)}—the recognition that counting begins somewhere. Succession is implemented as the $+1$ operation, the rhythmic pulse underlying all numerical construction. Addition is captured by the COBO (Compression/Elaboration) strategy, verified across test cases like $7+5=12$ and $23+17=40$. Multiplication emerges through the C2C (Chaining) strategy, confirmed via computations like $3 \times 4 = 12$ and $5 \times 7 = 35$.

But the HC is more than a calculator. The file \texttt{incompatibility\_semantics.pl} reveals a complete axiomatic architecture. The system includes a sequent calculus prover (\texttt{proves/1}, \texttt{proves\_impl/2}) implementing standard logical rules. It contains explicit axioms: commutativity of addition, geometric entailments (squares $\rightarrow$ rectangles), and the axioms M4, M5, M6 formalizing Euclid's proof of the infinitude of primes. The logic is grounded in computation via the \texttt{is\_recollection/2} predicate, which verifies the constructive history of numbers based on the execution of student strategies.

The HC is not a sophisticated abacus. It is a formal system capable of proving theorems. Because it successfully implements the operations of Q, Gödel's theorem applies. The claim is stronger than ``student strategies are incomplete.'' The claim is: ``The entire formalized system of mathematical reasoning—spanning calculation, embodied modal logic, geometric proof, and number theory—is necessarily incomplete.''

\subsection*{Arithmetization: Encoding the System Within Itself}

For the incompleteness theorem to apply, the system must be capable of \textit{self-reference}. This requires encoding the syntax and mechanics of the HC as numbers the system itself can manipulate. The method is Gödel numbering via prime factorization.

Consider a snapshot of the C2C multiplication strategy computing $3 \times 4$. The automaton's state includes: the current control state ($q_{count}$), the goal register ($G=100$, encoding the problem), the index register ($I=100$, tracking progress), the total register ($T=100$, accumulating the result), and the parameters ($N=103$, $S=104$). Each component is assigned a number, and the entire configuration is encoded as:

\[
g(C) = 2^{1} \cdot 3^{100} \cdot 5^{100} \cdot 7^{100} \cdot 11^{103} \cdot 13^{104}
\]

The first prime encodes the state, the second encodes $G$, the third encodes $I$, and so forth. A single number captures the entire cognitive state. Transitions between states—the rules governing how the automaton updates its registers—become arithmetic predicates. The Counting Rule, for instance, states: ``If the state is $q_{count}$ and $I < S$, then increment both $I$ and $T$.'' This rule becomes a predicate $\text{Rule}_{\text{Count}}(X, Y)$ that checks whether number $X$ transitions to number $Y$ via this specific cognitive move.

The technical challenge is proving that these encoding and decoding operations are \textit{Primitive Recursive} (PR)—expressible using only the elementary arithmetic operations (addition, multiplication, exponentiation, bounded search) that the HC itself can perform. The key step involves extracting exponents from the prime factorization. The function $\text{exp}_p(N)$, which returns the exponent of prime $p$ in the factorization of $N$, is PR because it relies on \textit{bounded minimization}: we search for the largest exponent $e \leq N$ such that $p^e$ divides $N$ but $p^{e+1}$ does not. Since divisibility, exponentiation, and bounded search are all PR, the extraction function is PR. Therefore, the entire \texttt{Transition} predicate—the formalization of the HC's mechanics—is representable within the HC itself.\footnote{The rigorous proof of primitive recursion, including the detailed demonstration that $\text{exp}_p(N) = \mu e \leq N [\neg (p^{e+1} | N)]$ is PR via bounded minimization, is provided in Appendix A. The proof establishes that divisibility is PR via bounded quantification, that primality is PR, and that the entire \texttt{Transition} predicate is PR as a finite disjunction of PR rules.}

The system can talk about itself. The stage is set for the Gödelian construction.

\subsection*{The Gödel Sentence: $G$ (The Reflective Turn)}

Here is where the formalization breaks open. Using the Diagonal Lemma—a technique grounded in the same self-referential structure we encountered in Cantor's proof and Russell's paradox—we construct a sentence $G$ with a specific property: $G$ asserts, ``I am not provable in the HC.''

More precisely, $G$ is a statement about numbers. It says, ``There does not exist a number $n$ that encodes a valid proof of the sentence with Gödel number $\ulcorner G \urcorner$.'' The sentence is constructed so that it speaks about itself via the arithmetic encoding. This is not paradox; it is provable self-reference.

Now consider what $G$ implies. Suppose the HC is consistent (it does not prove contradictions). If the HC could prove $G$, then $G$ would be false (since $G$ asserts its own unprovability). But if the HC proves a false statement, it is inconsistent—contradicting our assumption. Therefore, the HC \textit{cannot} prove $G$.

But if the HC cannot prove $G$, then $G$ is \textit{true}. The sentence accurately describes the system's limitation. $G$ is a truth the system can articulate (it is a well-formed statement in the language of arithmetic) but cannot demonstrate.

This is the incompleteness. The HC, despite its expressive power, despite its capacity to prove commutativity, to verify Euclid's argument, to reason across domains, cannot reach $G$. The formalized strategies—no matter how sophisticated, no matter how grounded in embodied practice—cannot exhaust the mathematical reality they describe.

And this is not a deficiency. This is the structure of the \textit{in}finite.

\subsection*{Incompleteness as the \textit{In}finite: Self-Transcendence}

The Hegelian \textit{in}finite is not endlessness. It is not the ``bad infinite'' of endless iteration ($1, 2, 3, \ldots$). The \textit{in}finite is the capacity to relate to oneself \textit{as} finite—to recognize one's own boundary and, in that recognition, to transcend it.

The Gödel sentence $G$ formalizes this structure. $G$ is generated \textit{by} the system (it is expressible in the language of the HC). Yet $G$ points \textit{beyond} the system's deductive reach. The system contains within itself the mechanism to articulate its own limit. Recognizing the truth of $G$—seeing that it must be true because the system cannot prove it—requires stepping outside the current formalization. It requires the ``I'' to transcend the ``me.''

Consider the parallel in the HC's own architecture. The file \texttt{incompatibility\_semantics.pl} includes axioms designed to be transcended. The rule for subtraction in $\mathbb{N}$ states that for $a - b$, we must have $b < a$. This constraint is an axiom of the natural number domain. But when a student encounters $3 - 8$, the system triggers a \textit{normative crisis}. The predicate \texttt{is\_incoherent} fires. The system recognizes its own boundary: ``I cannot subtract a larger number from a smaller one \textit{here}.'' This recognition is not failure. It is the opening. The crisis drives the transition to $\mathbb{Z}$, where the constraint is lifted, where negative numbers become intelligible.

The Gödelian incompleteness provides the mathematical necessity for this transcendence. Every finite formalization of the ``me'' will generate its own $G$—a statement that points beyond. The ``I'' names the gap that drives learning forward. The horizon is necessarily open because the system is \textit{built to recognize its own limits}.

This is what it means to be \textit{in}finite. Not to possess unlimited knowledge. Not to escape finitude. But to possess the capacity—the necessity—to break the boundaries we build.

\subsection*{Educational and Political Implications: Against the Finite Vessel}

The reductive ideology I oppose assumes finite closure. It assumes that mathematical understanding \textit{is nothing more than} the mastery of a fixed set of procedures. Students are vessels to be filled. Curricula are content to be delivered. Assessment measures retention. The ``science of math education'' seeks optimal transmission protocols.

But elementary arithmetic—the mathematics invented by children from embodied practice—is \textit{necessarily incomplete}. I did not impose incompleteness on the HC. I formalized the strategies, and incompleteness emerged as a structural feature. The origins of mathematical understanding already possess this openness.

The profundity lies in \textit{what} was formalized. Not an arbitrary logical system. Not a mathematician's construction. The HC captures the cognitive moves of students learning to count by grouping objects rhythmically, to add by compressing and elaborating collections, to multiply by chaining iterations. These are the strategies that emerge when children—embodied, situated, socially embedded—engage with quantity. And these strategies, when formalized, yield a system subject to Gödel's theorem.

We are not vessels. We are boundary-recognizers. The normative crises we encounter ($3-8$ in $\mathbb{N}$, $\sqrt{2}$ as a ratio, the need for $i$ to solve $x^2 + 1 = 0$) are not deficiencies to be remediated. They are the engine of mathematical invention. Incompleteness proves there is always ``something more.'' The claim that education should aim for completeness—that students should master a closed curriculum—is mathematically incoherent.

The political stakes are clear. Reductive pedagogies that emphasize rote mastery, that valorize standardized assessment, that frame mathematics as a fixed body of knowledge, rest on a false assumption. They assume closure where incompleteness is necessary. They treat transcendence as deviance. They mistake the recognition of boundaries for failure.

But we are those who break boundaries. The ``I'' that invents a new strategy, that bootstraps a new domain, that grasps Euclid's proof of infinite primes—this ``I'' enacts the self-transcendence formalized by $G$. Mathematics education, if it is to honor what students actually do, must cultivate this capacity. Not the transmission of procedures, but the recognition that every formalization points beyond itself.

\subsection*{Coda: Built to Break}

I return to where I began: to my father, to the song, to the eulogy.

\section{The Sound of Time Revisited: A Final Verse}

As this concludes, this image returns, deepened by the exploration of critical mathematics and enriched by the final verse of the song that has accompanied the journey.

\begin{verse}
A few years back, I eulogized\\
Someone whose death broke a powerful light. \\
An ocean of rainbow listened to me\\
As I read my eulogy.\\

\bigskip

Build something for the breaking\\
Tall thin walls, shivering, quaking\\
Dad, it's been beautiful, breaking with you\\
Build and then break, like you taught me to do\\
\end{verse}

This final verse reveals the deeper meaning that has been implicit throughout our exploration. The ``powerful light'' that was broken by my father's death was not simply extinguished -- it was transformed, refracted into an ``ocean of rainbow'' that encompasses and transcends what was lost. This is the movement of dialectical transcendence that we have traced through mathematical proofs, philosophical arguments, and personal transformations.

The final line -- ``Build and then break, like you taught me to do'' -- captures the essential insight of critical mathematics. We construct systems of understanding not as permanent monuments but as temporary scaffolding that enables further construction. The willingness to break what we have built, to let go of certainties that have served their purpose, is not a rejection of achievement but its highest expression.

Mathematics, like music, is a temporal art. It unfolds in time through the inferential movements of mathematical judgment, through the developmental trajectories of mathematical learning, through the historical evolution of mathematical concepts. Mathematics, like music, involves both determination and transcendence, both the creation of definite forms and the release from those forms into new possibilities.

The sound of time in mathematics is the sound of concepts moving along inference chains, of understanding evolving through the experience of error, of consciousness grasping its own conditions of possibility. It is the sound of the finite resonating with the \textit{in}finite, of the sayable gesturing toward the unsayable, of the representable embracing the unrepresentable.

This sound is not separate from the sounds of everyday life, from the rhythms of work and play, from the melodies of conversation and storytelling, from the harmonies of social interaction. It is not a rarefied music accessible only to mathematical specialists but a variation on the basic rhythms that structure all human experience.

The formalization was an act of building. I constructed the HC with care—state machines, transition rules, grounded semantics, axioms spanning arithmetic and geometry and number theory. I encoded the strategies invented by children. I verified the operations. I proved the system interprets Robinson Arithmetic. I arithmetized the mechanics via Gödel numbering. I demonstrated primitive recursion. I built something rigorous, something mathematically sound.

And then I applied the theorem. I showed that the system I built—this careful, grounded, empirically motivated formalization—necessarily contains a truth it cannot prove. I built the HC \textit{to break it open}. The incompleteness is not a flaw. It is the revelation. The formal system points beyond itself because that is what formal systems \textit{do} when they are rich enough to matter.


The Gödel sentence $G$ is the horizon the HC reveals but cannot cross. The ``ocean of rainbow''—the equivalence class of enabling conditions, the irreducible plurality of lived experience, the diversity that formal systems cannot totalize—remains beyond. The null set $\emptyset$ as the groundless ground, the absence that makes presence possible, cannot be captured by the ``me.'' It is the condition of the ``I.''

This manuscript has been an act of breaking with you, Dad. Breaking in the sense of rupture—the kind of self-transcendence that incompleteness formalizes. Breaking in the sense of recognition—seeing the boundary, naming it, and understanding that the boundary is not a wall but an opening. Breaking in the sense of grief—the recognition that you are gone, that the ``I-You'' is now a recollection, that the enabling conditions can be honored but not recovered.

And breaking in the sense of continuation. The sound of time continues. The formalization reveals its own horizon. The students I teach will encounter their own normative crises, their own $G$, their own need to transcend the ``me'' they have articulated. Mathematics is \textit{in}finite not because it is endless, but because every articulation contains the seeds of its own transcendence.

I built the Hermeneutic Calculator to demonstrate what you taught me: that we are not defined by the systems we construct. We are defined by our capacity to recognize those systems' limits and to move beyond them. The incompleteness theorem is not a conclusion. It is an opening. It says: there is always something more. The horizon is open. We are built to break.

\textit{It's been beautiful.}

\printbibliography

\appendix
\section{Appendix A: Primitive Recursion and the Transition Predicate}

\subsection*{Primitive Recursive Functions: Foundations}

Primitive Recursive (PR) functions are built from a simple base and closed under specific operations:

\textbf{Base Functions:}
\begin{itemize}
    \item Zero: $Z(x) = 0$
    \item Successor: $S(x) = x + 1$
    \item Projection: $P_i^n(x_1, \ldots, x_n) = x_i$
\end{itemize}

\textbf{Closure Operations:}
\begin{itemize}
    \item \textbf{Composition:} If $g, h_1, \ldots, h_m$ are PR, then $f(\vec{x}) = g(h_1(\vec{x}), \ldots, h_m(\vec{x}))$ is PR.
    \item \textbf{Primitive Recursion:} If $g$ and $h$ are PR, then $f$ defined by:
    \begin{align*}
    f(\vec{x}, 0) &= g(\vec{x}) \\
    f(\vec{x}, y+1) &= h(\vec{x}, y, f(\vec{x}, y))
    \end{align*}
    is PR.
\end{itemize}

It is well-established that addition ($x+y$), multiplication ($x \cdot y$), and exponentiation ($x^y$) are all PR. Furthermore, comparisons ($x=y$, $x<y$) and Boolean operations ($\neg$, $\wedge$, $\vee$) are PR.

\textbf{Bounded Minimization:} Crucially, PR functions are closed under bounded minimization. If $P(x, \vec{y})$ is a PR predicate, then the function:
\[
F(\vec{y}, B) = \mu x \leq B [P(x, \vec{y})]
\]
(``the smallest $x \leq B$ such that $P(x, \vec{y})$ is true'') is also PR.

\subsection*{Divisibility, Primality, and Prime Extraction}

\textbf{Divisibility is PR:} The predicate $x | y$ (``$x$ divides $y$'') can be defined using bounded quantification:
\[
x | y \iff \exists k \leq y \, (x \cdot k = y)
\]
Since the search for $k$ is bounded by $y$, and multiplication and equality are PR, divisibility is PR.

\textbf{Primality is PR:} The predicate $\text{Prime}(x)$ checks whether $x$ has divisors other than $1$ and $x$, which involves checking divisibility for all values up to $x$. This is a bounded operation, so primality is PR.

\textbf{The $n^{th}$ Prime is PR:} The function $P_n$ (the $n^{th}$ prime number) is PR. The search for the next prime is bounded (e.g., by $(P_{n-1})! + 1$, though tighter bounds exist).

\subsection*{Extracting Exponents: The Rigorous Step}

We now prove that $\text{exp}_p(N)$—the function returning the exponent of prime $p$ in the factorization of $N$—is Primitive Recursive.

We seek the exponent $e$ such that $p^e$ divides $N$ but $p^{e+1}$ does not. We establish a bound: since $p \geq 2$, we have $2^e \leq N$, which implies $e \leq N$. Therefore:

\[
\text{exp}_p(N) = \mu e \leq N \, [\neg (p^{e+1} | N)]
\]

This reads: ``The smallest $e$ less than or equal to $N$ such that $p^{e+1}$ does not divide $N$.''

Since exponentiation ($p^{e+1}$) is PR, divisibility ($|$) is PR, and negation ($\neg$) is PR, the predicate inside the minimization is PR. Because the minimization is bounded by $N$, the entire function $\text{exp}_p(N)$ is Primitive Recursive.

\subsection*{The Transition Predicate is Primitive Recursive}

Consider the C2C multiplication strategy. A configuration $C$ is encoded as:
\[
g(C) = 2^{g(\text{State})} \cdot 3^{g(G)} \cdot 5^{g(I)} \cdot 7^{g(T)} \cdot 11^{g(N)} \cdot 13^{g(S)}
\]

The Counting Rule states: ``If State $= q_{\text{count}}$ and $I < S$, then $I' = I+1$ and $T' = T+1$.''

We define $\text{Rule}_{\text{Count}}(X, Y)$ as:
\[
\text{Rule}_{\text{Count}}(X, Y) \iff \text{Condition}(X) \wedge \text{Update}(X, Y)
\]

\textbf{Condition is PR:}
\begin{align*}
\text{Condition}(X) \iff \, &(\text{exp}_2(X) = g(q_{\text{count}})) \, \wedge \\
&(\text{exp}_5(X) < \text{exp}_{13}(X))
\end{align*}
Since $\text{exp}_p(N)$, equality, comparison, and conjunction are PR, the Condition is PR.

\textbf{Update is PR:}
\begin{align*}
\text{Update}(X, Y) \iff \, &(\text{exp}_5(Y) = \text{exp}_5(X) + 1) \, \wedge \\
&(\text{exp}_7(Y) = \text{exp}_7(X) + 1) \, \wedge \\
&(\text{exp}_2(Y) = \text{exp}_2(X)) \, \wedge \ldots
\end{align*}
Since $\text{exp}_p(N)$, addition, and equality are PR, the Update is PR.

Therefore, $\text{Rule}_{\text{Count}}(X, Y)$ is PR.

\textbf{The Full Transition Predicate:}
The complete $\text{Transition}(X, Y)$ predicate is the finite disjunction of all rules:
\[
\text{Transition}(X, Y) \iff \text{Rule}_1(X, Y) \vee \text{Rule}_2(X, Y) \vee \cdots \vee \text{Rule}_K(X, Y)
\]

Since PR predicates are closed under finite disjunction, $\text{Transition}(X, Y)$ is Primitive Recursive.

\subsection*{Conclusion: Self-Reference is Possible}

By rigorously demonstrating that the decoding operation $\text{exp}_p(N)$ is Primitive Recursive via bounded minimization, we have established that:

\begin{enumerate}
    \item The mechanics of the HC (the $\text{Transition}$ predicate) are Primitive Recursive.
    \item The HC is sufficiently expressive to represent all Primitive Recursive functions.
    \item Therefore, the HC can represent its own mechanics—self-reference is possible.
\end{enumerate}

This provides the mathematical foundation for applying Gödel's First Incompleteness Theorem to the Hermeneutic Calculator, securing the intellectual and rhetorical payoff: that the formalized system of student-invented strategies is necessarily incomplete.


\printbibliography[heading=subbibliography]