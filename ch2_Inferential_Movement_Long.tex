\chapter{Inferential Movement}

\begin{abstract}
This chapter examines the ethics of inferential movement, drawing on
Robert Brandom's inferentialism. It argues that meaning arises not from
pre-given objects but from their role in inferences. The chapter
explores how concepts gain content through relations of compatibility
and incompatibility, using quadrilateral classification as a case study.
This analysis clarifies the distinction between formal and material
inferences, emphasizing the importance of material inferences in
pedagogical practice. The chapter extends Brandom's account of the
experience of error to misrecognition within interpersonal interactions,
connecting the inferential domain to the ethical. It also connects
geometric figures to the ``I think,'' foreshadowing the later argument
that numerals function as pronouns. Through exploring Brandom's concept
of inferential strength and the inversion of polarity by logical
operators, the chapter demonstrates how meaning emerges from embodied
practices and social norms. The quadrilateral example illustrates the
modal structure of knowledge, the experience of error, and the interplay
of necessity and possibility in conceptual understanding. This chapter
ultimately argues for an inferential approach to meaning, laying the
groundwork for understanding the embodied and ethical dimensions of
knowledge explored in subsequent chapters.

\end{abstract}



\section{Introduction}



In this chapter, I continue to explore ontology (the logic of being) that I began in the previous chapter. However, I now focus on the normative \textit{propriety} of inferential movements, capitalizing on Robert Brandom's work in the philosophy of language, specifically chapter 4 of \textit{Articulating Reasons} \parencite*{Brandom2000}. The guiding question is ``What are mathematical beings?'' That is, what grammatical role do mathematical beings play within a philosophy of language? 

\subsection*{Motivating the Question with Hybridized Models}
To make this question compelling, consider the student work samples (SWS) in Figure \ref{fig:hybridized_not_fractions}. While I was a graduate student, I worked on a research project with Erik Jacobson. We collected a large number of SWS from fourth and fifth graders. I was tasked with analyzing these samples and writing up what the students were doing as high-level algorithms (pseudo-code). There were about 50 samples that could not be easily reconstructed as algorithms. At first, we binned those samples because they were not statistically significant, but I was taking an Arts-Based Educational Research course with Gus Weltsek and so I decided to try and interpret those binned samples as if they were artistic expressions. I was also taking a course on critical theory at the time and reading about \textit{meaning fields} \parencite{Carspecken:1995aa}. I felt drawn to these oddities, as the kids behind the work felt like kindred spirits. 

\begin{figure}[h]
%\centering
\includegraphics[width=0.8\textwidth]{/Users/tio/Documents/GitHub/September_UMEDCA/images/Hybridized_Not_Fractions.pdf}
\caption{\textit{Note. } Two student work samples that purport to illustrate the fraction $\frac{2}{3}$ but that fail two tests.}
\label{fig:hybridized_not_fractions}
\end{figure}

First, let me establish that the fractions in figure \ref{fig:hybridized_not_fractions} are not fractions. The first sample fails a measurement test: folding the circle along the vertical bars yields a figure with overlaps. That indicates that the parts are not equivalent in area. The second test they fail is that iterating one part three times does not return the unit. I call these representations \textit{hybridized models} because they take a respectable unit, like a circle, and a respectable partitioning scheme, like equi-distant vertical bars, but combine them in a way that does not yield a fraction. While not statistically significant, there were several that seemed to follow a pattern (see figure \ref{fig:Chart_Hybridized_Models}).

\begin{figure}[h]
%\centering
\includegraphics[width=0.8\textwidth]{/Users/tio/Documents/GitHub/September_UMEDCA/images/Chart_Hybridized_Models.pdf}
\caption{\textit{Note. } The hybridized models are not fractions. While not statistically significant in the context I found them in, this pattern of reasoning is not uncommon}
\label{fig:Chart_Hybridized_Models}
\end{figure}

While I was not sure exactly what the pattern was, I felt a weight. How many ways are there to draw a fraction? Textbooks in teacher education often present circular models, rectangular models, and set models. Different partitioning rules apply to each of the models. I reasoned that a teacher might think they are adding complexity linearly -- introducing one model at a time -- may, from a student's perspective, be adding complexity exponentially. That is, the complexity function for a teacher may be $C=n$, while the student may experience it as $C=2^n$ (loosely). I represent this relationship in figure \ref{Hybridized_Exponential_Growth}. 

\begin{figure}[h]
%\centering
\includegraphics[width=0.8\textwidth]{/Users/tio/Documents/GitHub/September_UMEDCA/images/Hybridized_Exponential_Growth.pdf}
\caption{\textit{Note. } As the teacher introduces a new model, moving from the number of possible models $n$ to $n+1$, students prone to hybridization may experience exponential growth $2^n \rightarrow 2^{n+1}$. For $n=3$, there are at least 8 possible hybridized models.}
\label{fig:Hybridized_Exponential_Growth}
\end{figure}

I call this loosely exponential relationship the \textit{phenomenology of confusion}. When I feel confused, it is often due to a lack of \textit{information} -- the negation of possibilities.  While everyone else seems to be dealing with $n$ possibilities, I am tangling with $2^n$ possibilities. But, of myself, I know that simply telling me ``no, you can't mix and match'' (providing that information) does not dissipate confusion. I have to understand why good premises lead to bad conclusions. Furthermore, simply naming an inference fallacious, saying ``Tio, you're making a category error,'' for example, does not help me untangle my confusion. Mixing metaphors or categories sometimes manifests what is lauded as ``creativity,'' so it cannot be universally improper. I will return to these hybridized representations later in this chapter to argue that the units drawn as circles, rectangles, squares, etc., do not follow the rules associated with \textit{singular terms}. Instead, I will argue, such units are \textit{anaphoric terms} -- pronouns -- that recollect the \{I think\}. This claim is easily misunderstood as a psychologism, but I will not clarify why it is not until chapter 3. 

\subsection*{Referentialism to Inferentialism}

In chapter 4, Brandom draws on Kant and Frege to invert the common-sense framing that \textit{singular terms} (names like ``Benjamin Franklin'' or definite descriptions like ``the inventor of bifocals'') are expressions that refer to particular objects \parencite*{Brandom2000}. In the common sense approach, their purpose is to enable discourse about the objects that make up the world. Inverting that common-sense understanding reverses this order of explanation. Rather than defining singular terms by appealing to a given notion of ``objects,'' Frege suggests that ``objects'' can be defined as that which is referred to by singular terms. 


If successful, this inversion partially liberates objectivity from the obduracy of a \textit{given} object. However, Frege's \textit{referentialist} project has some downsides in the domain of math education. The problem is with truth. For Hegel, ``The True is thus the Bacchanalian revel in which no member is not drunk; yet because each member collapses as soon as he drops out, the revel is just as much transparent and simple repose'' \parencite[\S 47]{hegel1977phenomenology}. For Hegel, truth is not a static property but a dynamic, self-correcting process that unfolds through history and experience. A falisity isn't a dead end; it is a necessary step on the way to a more complete understanding. But for Frege, ``Every declarative (assertoric) sentence concerned with the referents of its words is therefore to be regarded as a proper name, and its referent, if it exists, is either the true or the false'' \parencite*[\S 34]{frege_frege_1997}. Frege's goal was to create a formal language free of ambiguity, and so he conceptualized truth as static and bivalent -- a timeless \textit{property} of thought. So, what is the true Truth? 

I recently did an ice-breaker with some pre-service teachers. I asked them to write something unique about themselves on a sheet of paper, crumple it up, and throw it into the middle of the room. We each took a `snowball' and read its declarative sentence aloud. While one student spoke something true of themselves, everyone else uttered a false sentence. ``I am a triplet'' is untrue of me, though I read it aloud. This example does not function as a defeasor of Frege's project, as he would say that the indexical ``I'' makes each repetition of the sentence a brand new sentence. Still, Hegel's interpretation of truth is capacious enough to \textit{take} a students' expression as contextually true -- a truth on its way to falsity or a falsity on its way to truth -- which resonates with a developmental trajectory for mathematics education. Frege's interpretation bins an awful lot of good thinking as merely false. 

However, that does not mean I should bin Frege! His distinction between the sense of a term and its referent is very useful \parencite*{frege_frege_1997}. Habermas took Frege's insights and used them to articulate his formal pragmatics, which I synthesize with Brandom to articulate tri-modal conceptual realism. ``I am a triplet'' can be analyzed with respect to my sincerity, the normative goodness of the claim, or the objective (shared access) claim about how many children were born to my mom during the same pregnancy. 



Brandom's project transforms Frege's insights from a referentialist account of truth to an \textit{inferentialist} account. The inversion allows objects to defined through the act of classifying the inferences that can be made about them, rather than by their existence in a world that is simply given. More complete liberation from the `myth of the given' is pursued in Brandom's later work that rationally reconstructs the Perception chapter of Hegel's \textit{Phenomenology of Spirit} \parencite*[Ch. 5]{Brandom:2019aa}. 

In that later work, the cut is made between particular objects, referred to by singular terms, and \textit{universals}, which are repeatable properties presented as predicates that serve as proto-concepts. The problem that Brandom identifies Hegel as working through is how concepts have \textit{determinate content} \parencite{Brandom:2019aa}. For any concept, like \textit{square}, to be determinate, it must stand in two kinds of difference relations. 

First, it must be \textit{incompatibly} different from other concepts it rules out (like \textit{triangle} or \textit{circle}). This is the relation of \textit{determinate negation}. Second, it must be merely or \textit{compatibly} different from other concepts it can coexist with (like \textit{red} or \textit{large}). In fact, ten different types of difference shall emerge from exploring geometric objects and their properties, tracking the differences that Brandom articulates in that later work. 

Another difference is between \textit{formal} and \textit{material} inferences. Many (perhaps most) of my colleagues in math education who do teacher training implicitly understand that the material rules of inference that govern school mathematics must be developed before the formal rules of inference can be taught. However, the political bodies and the teachers-in-training that govern how math is taught or who plan on implementing that math in schools often take only those formal rules to be legitimate. By showing how formal rules of inference can be derived from material ones, readers are invited to dissent from participation in the so-called `math wars.' Neither formality nor materiality is `bad,' but starting with formality does not make pedagogical sense. Falling downhill may result in some bruising, but falling uphill is impossible. 

In the exploration of quadrilaterals developed here, the traditional hierarchies of quadrilaterals give way to a fractal-like structure. % voice-edit: Original first-person sentence below

That traditional activity is extended in three directions. 

First, a material account is offered for the inferences involved in classifying quadrilaterals. Part of Brandom's earlier argument is challenging, specifically his concept of \textit{inferential strength}. The material account is partially formalized to show how quantifying \textit{inferential strength} makes some technical details in Brandom's argument more explicit. This move actualizes the potential expressed in Brandom's claim that ``\textit{formal} proprieties of inference essentially involving logical vocabulary derive from and must be explained in terms of \textit{material} proprieties of inference essentially involving nonlogical vocabulary rather than the other way around'' \parencite*[30]{Brandom2000}. 

The second direction extends Brandom's articulation of the \textit{experience of error}, which focuses on subject-object relations, toward the \textit{experience of misrecognition} that foregrounds a subject-subject relationship. Concepts include the history of their development, like the recollective ``me'' stands to the \{I\}. Those developmental histories include experiences of error/misrecognition and the ways those errors are corrected. Using the concept \textit{square} includes the learning experiences associated with squares. 

Moving toward a subject-subject account of error resolves into the third direction pursued here, which is distinct from Brandom's work. When Ramanujan claimed that all numbers were his friends, he was articulating a subject-subject relationship with numbers, rather than the standard subject-object relationship. While squares do not make or defend identity claims, one can take their position as a friend to `help' them express the negative dimension of their identity. They repel claims like ``No sides of $X$ are equal.'' Just as a vegan friend is honored by not serving butter and only serving plant-based foods, friendship with the square is achieved by knowing both what it is and what it is not. 


A more complete discussion of \textit{apperception} is deferred to the next chapter, but the sound of time metaphor that precedes this chapter is operating here, too. In that chapter, I claimed that spatial thought arises under temporal compression. 



That is to say that geometric figures, like a square, are treated as anaphoric recollections of the ``I think.''




This probably will not make sense until Chapter 7, where numerals are argued to be pronouns that anaphorically recollect the ``I think.'' This seed is planted here without full justification. 

By developing the concept of a square so that it includes the history of its development, I approach geometry through apperceptive self-consciousness. When construed as a friend, the concept is like a You to the \{I\} who thinks it. That You can be respected and honored as an autonomous subject -- someone that can be learned about. That said, the I-feeling, explored in the previous chapter, can \textit{fuse} with the concept when one of the ten differences is recognized.

An account of inference chains (\textbf{Algorithms})\label{def:algorithms} is given that relates to embodied experiences of flow, with various \textit{moments} of the concept metaphorically represented as poses that punctuate yogic flow. 

This chapter will proceed in three stages. 


First, the core concepts of Brandom's inferentialism are unpacked, distinguishing material from formal inference and symmetric from asymmetric substitution.

Second, the classification of quadrilaterals is introduced as a running case study to make these ideas concrete. 


Finally, this case study demonstrates Brandom's most challenging point: how logical operators like negation invert inferential polarity, and what this reveals about the structure of the concepts at issue. 

This chapter is dense.
Each of the differences has to be articulated with care. That care bears an unfortunate burden of jargon.



Technical terms are introduced gradually; readers are encouraged to approach them with ease.
Pounding your head against the chapter will not help. Instead, try to find some spirit of play in the various differences. Doing so may enhance the degree to which the I-feeling fuses with their articulation. 

Despite the chapter's density, the aim is to gesture toward the pleasure many associate with mathematics.
When the difference between two moments of the concept is null, the I-feeling might fuse with the concept. A deep pleasure may accompany that fusion.

\section{Brandom's Early Inferentialism}

Kant noted that judgement is the ``fundamental unit of awareness or cognition, the minimum graspable'' \parencite[125]{Brandom2000}. For Kant, the ``I think'' must be able to accompany all of my representations, but this transcendental ``I think'' enables the representation; it cannot be included in that representation. 

In this context, ``accompany'' can be understood as a kind of implicit harmony, an unheard Voice explored in \ref{def:Voice}.

This necessarily implicit enabling condition can be foregrounded in the assertoric representation of judgment.
``Pokey is a dog'' is not equivalent to ``$\emptyset_{\text{I think}}$ Pokey is a dog,'' as the implicit ``I think'' is discerned by reflecting on the original judgment. 

The previous representation can also be thought, allowing the judgment to be re-recollected again and again  --  a strategy used to define the progressive aspects of counting in section \ref{successor-function}. 

Brandom picks Kant's thought up in the philosophy of language by limiting his analysis to sentences. 

He argues that sentences can perform speech acts; speakers can be held to account for the truth or goodness of such speech in ways not available for isolated subsentential expressions like \{Benjamin Franklin, ambassador, Pokey, is, dog\}. Habermas makes a similar point. While a subsentential expression like a disapproving grunt might be recognized as a speech act in some context, in other contexts such a grunt may not be clear. When asked, ``what did that grunt mean,'' I can put it into assertoric form: ``I do not like how you are treating me.'' Unclear speech can be reconstructed as assertions. Of course, those reconstructions introduce new possibilities for error. 







Brandom is concerned with how those subsentential expressions can be combined in predictable ways to form novel sentences that other people somehow understand. He names this ability \textit{inferential projection}, noting that ``almost every sentence uttered by an adult native speaker is being uttered for the first time -- not just the first time for that speaker, but the first time in human history'' \parencite*[126-127]{Brandom2000}. \textit{Substitution} is the mechanism that he suggests undergirds this ability. Substitution can be formal, as in Russel and Whitehead's \textit{Principia Mathematica} \parencite{nagel2012godel}, or \textit{in}formal (i.e., \textit{material}). 

Brandom proposes a two-part definition for singular terms, considering the syntactic form and semantic content or significance of those roles they play in assertoric judgments. Both roles are understood through substitution. From there, he conducts an ``expressive deduction'' to show why a language capable of using \textit{conditionals} (``if ... then'' statements) or (logical) negation must include singular terms that can be substituted for one another in symmetric ways. 

For readers unfamiliar with Kant's transcendental arguments and how they relate to critical theory, the argument Brandom makes is challenging to follow. Kant instituted a new approach to philosophy to address the problem of skepticism. Descartes meditations deployed a radical form of doubt that questioned the foundations of knowledge. He resolved that doubt by noting that he could not doubt that he was doubting, expressing that conclusion in the famous dictum \textit{cogito ergo sum}, translated as ``I think therefore I am.'' 

The nature of that \{I\} who doubts and the manner of the ``ergo'' -- which does not follow a formal inferential structure -- were explored by Descartes in his meditations, but many readers did not follow the positive implications he drew from that indubitable doubt (the existence of a body and a world of objects). Those positive implications fell, for some, to the method of radical doubt. The paradigm shift that Kant instituted in his \textit{Critique of Pure Reason} -- the title of which informed the founders of the Frankfurt School as they articulated various \textit{critical theories} \parencite{carspecken_limits_2016} -- moves from radical doubt to examining what enables doubt. The transcendental categories arose as the enabling conditions for the kinds of doubts that Descartes expressed in his meditations. 

This method is explored further in the next chapter. 

For Brandom, the logical expressions he analyzes are transcendental-like.

It is possible to imagine a language that does not have those logical expressions, but ``having to do without logical expressions would impoverish linguistic practice in fundamental ways'' \parencite*[48]{Brandom2000}. 

One might go a bit farther and say that contemplating a language bereft of those expressions is contradictory, as it embeds the conditional-free language inside a conditional.
``If our language does not have conditionals, then ...'' sounds a lot like `using the master's hammer to break the master's hammer.'\footnote{There is a technical problem of what constitutes a \textit{language}. Brandom \parencite*{Brandom2008} discusses formal languages that can be written and read by various automatons, like the laughing santa automaton that only writes expressions like ``hohohahaho.'' He claims that such languages do not rise to the level of \textit{autonomous discursive practices}, languages that can be used to make assertions, hold one another to account, and so on. When we think of languages like French or Spanish, those are autonomous discursive practices. Such languages include conditionals and negations, while the laughing santa's language need not. Whether languages like Orca include the conditional is an open question.}

The argument is not that singular terms are necessary for the use of conditionals or negation, but that the ability to substitute one singular term for another in symmetric ways is necessary for the use of conditionals and negation. This is a subtle but important distinction, as it allows a contrast between mathematical and singular terms. 

Later chapters use this argument to determine that \textit{numerals} play an \textit{anaphoric} role in speech  --  functioning as `pronouns,' not singular terms.


This chapter classifies quadrilaterals both to clarify Brandom's argument and to set the stage for that argument. 






He first discusses the syntactic form of assertions, discerning three possible roles that singular terms could play. They could be \textit{substituted-for}, in the sense that they could be the component expression that is replaced in a substitution inference. Or they could be \textit{substituted-in}, in the sense that they could be the compound expression in which the substitution occurs (e.g., the full sentence). Or they could play a derived role he calls a \textit{sentence frame}, that serve as \textit{predicates.} % voice-edit: Original guide-voice sentence with first-person plural below



The discussion will carefully work through how substitution allows those categories to be discerned.




The syntactic categories relate to the \textit{form} of the sentence, while the second part of his definition relates to the semantics or \textit{content} of those roles. He notes that some of them have \textit{symmetric inferential significance} while others have \textit{a}symmetric significance. 

Noting the \textit{a}symmetric significance of some terms, Brandom then argues that a language that can use \textit{conditionals} (``if ... then'', $\models_I$) or logical \textit{negation} ($\neg$) must support symmetric substitution of singular terms. He then shows that the ability to substitute one singular term for another in symmetric ways is necessary for the use of conditionals and negation. % These logical sentence frames \textit{invert inferential polarity} in a way that risks our ability to coherently project inferential consequences.
These logical sentence frames \textit{invert inferential polarity} in a way that risks the ability to coherently project inferential consequences.


\section{Material Inference}

Formal mathematical systems are usually defined by a fixed set of axioms and rules of inference. The axioms are statements that are taken to be true without proof, and the rules of inference are the logical steps that allow us to derive new statements from the axioms. In this sense, formal systems are built on a foundation of \textit{formal inferences} -- inferences that follow purely from the syntactic structure of the system. 

These axioms and inference rules are like the bricks and architectural plans of a castle whose walls keep the hordes of logical contradiction at bay. However, this rigidity means the system itself is static; it cannot learn or adapt. An `error' within a proof is simply a deviation to be discarded, not an experience from which the system itself can grow. Foundational challenges to the system, such as paradoxes or incompleteness, are not resolved by the system adapting, but by mathematicians abandoning it and building a new one. 

Retention of errors within a system of critical mathematics is discussed in \ref{def:admitting_incoherence}. Until then, the focus is on the importance of the experience of error in mathematics education. 


However, formal inferences are not the only kind of inference that we make. In fact, most of our everyday reasoning relies on what Wilfrid Sellars \parencite*{Sellars:1953aa} called \textbf{material inferences}\label{def:material-inference}. A material inference is an inferential step that is licensed not by pure logical form (like \textit{modus ponens} or other formal schemata). Instead, the virtue of being taken as a good inference by a recognitive community lends content to the concepts involved in the inference. 

It may take a few passes at the previous sentence to understand how different material inferences are from formal inferences. By taking a material inference to be good, the terms involved accrue conceptual content.



The meaning of \textit{goodness} in this context will be discussed below; importantly, `taking as good' is not meant to be read subjunctively.


The hypotheticals involved in subjunctive reasoning -- if $x$ were good, then $y$ -- codify temporally antecedent material inferences. These are inferences that are almost spontaneously agreed to or denied. 

For example, consider the inference: `Pokey is a dog so Pokey is a mammal' or ``It's raining; therefore, the streets will be wet'' \parencite[313]{Sellars:1953aa}. This is not formally valid in pure logic (nothing in the form ``$P$; therefore $Q$'' guarantees truth), yet it is a perfectly good inference in practice given our knowledge about rain and wet streets, or dogs and mammals.
Such an inference is widely treated as reasonable, even obvious  --  enough to carry a kind of normative force in ordinary discourse: if someone says ``it's raining'' but refuses to accept ``the streets will be wet,'' that refusal typically signals an incomplete grasp of what ``raining'' means.



Perhaps they do and are being sarcastic. In Carspecken's critical ethnographic methodology \parencite*{Carspecken:1995aa}, meaning is structured as a field of next possible speech acts. A qualitative researcher might hear a participant state ``It is raining.'' Upon analysis, the sentence ``the streets will be wet'' is reconstructed as a next possible act along with others like ``I will bring an umbrella,'' or, if spoken sarcastically, ``it is \textit{not} raining.'' Analyzing the meaning field produces a horizonal structure for what the assertion could mean. Due to the horizonal structure of possible next-acts, there is no way to say with certainty what an assertion means -- especially in everyday empirical speech. That is one reason why I will spend much of the chapter discussing how quadrilaterals are classified, as it offers a relatively regimented set of material consequences. In that regimented context, the speaker is obliged to give some justification for their refusal to accept the consequent or they risk being taken as someone who is playing by a different set of rules. This illustrates Sellars's insight, picked up and extensively developed by Brandom: \textit{material inferences} are those inferences whose normative goodness or acceptability lends content to the terms caught up in them.

For Brandom \parencite*{Brandom2000}, the content of a concept is given by the network of material inferences it participates in. To know what \textit{Dog} means is to know that ``$\textit{Dog}(\textit{Pokey})$'' licenses ``$\textit{Mammal}(\textit{Pokey})$,'' and that it is incompatible with ``$\textit{Reptile}(\textit{Pokey})$''. The web of what follows from, and what is ruled out by, a claim is what gives it meaning. In this way, meaning is not a static mapping from word to world, but the dynamic ability to move in the space of reasons.

A few features of material inferences are worth noting. First, they are often (though not always) \textbf{non-monotonic}\label{def:non_monotonic} \parencite[88]{Brandom2000}. This means that adding new information can flip a good inference to a bad one \textit{and} turn a bad inference into a good one. Formally, adding a premise to the antecedent of an inference doesn't turn a bad inference into a good one. The conjunction operator restricts what sort of consequents can be drawn, it does not open new possibilities. Non-monotonic material inferences are different.  

For example, I was driving $\mathcal{M}$ and $\exists$ to their Aunt Rachel and Uncle Phil's house (my sister and her husband, not Phil Carspecken). One of them said ``We're going to Rachel's!'' The other one corrected her and said ``We're going to Rachel \textit{and} Phil's house.'' I usually ignore this sort of correction as if it represents a quixotic desire for completeness, but I was writing about material inferences right before we got in the car.

Rachel can't make balloon animals, while Phil can. So an inference ``We're going to Rachel's so you might get a balloon animal'' might have felt like a bad inference to the corrector, while the corrected premise ``We're going to Rachel \textit{and} Phil's so you might get a balloon animal'' feels better. What I think matters here is that adding Phil adds to the field of possibilities that the antecedent of the material inference holds. Implicitly, those possibilities are contained within the truncated antecedent, ``We're going to Rachel's,'' but explicating them by adding Phil brings them closer to actualization. 

This contrasts with formal inferences, which are generally monotonic. If the inference $P \rightarrow Q$ is formally true, then (assuming $R$ is true) the inference $P \land R \rightarrow Q$ will still be true because $P$ is a sufficient ground for $Q$. Brandom \parencite*[88]{Brandom2000} provides the following example to illustrate nonmonotonicity (using $\neg$ for negation, rather than \~ -- from \textit{Articulating Reasons} -- or $N$ -- from \textit{Between Saying and Doing} \parencite*{Brandom2008} which is the specifically modal version of negation):
\begin{enumerate}
\item If I strike this dry, well-made match, then it will light. ($p \rightarrow q$)
\item If p and the match is in a very strong electromagnetic field, then it will \textit{not} light. ($p \land r \rightarrow \neg q$)
\item If p and r and the match is in a Faraday cage, then it will light. ($p \land r \land s \rightarrow q$)
\item If p and r and s and the room is evacuated of oxygen, then it will \textit{not} light. ($p \land r \land s \land t \rightarrow \neg q$)
\end{enumerate}
A philosophy of mathematics that is suitable for educational contexts must include nonmonotonic material inferences. 

Recall figure \ref{fig:cartesian} that shows an instance of confusion about whether the orientation of the cartesian plane mattered when computing slope. ``Just remember rise over run'' unless the paper is upsidedown, unless the paper is held up to a strong light, unless the paper is rotated 90 degrees etc. Each additional antecedent can change a materially good inference into a materially bad one. 

Second, material inferences are often \textit{context-sensitive}. The same statement can have different meanings in different contexts. For example, when reading philosophy, the term ``practical'' generally involves ethical or moral reasoning. In other contexts, that term implies that a proposed solution to a problem is viable or efficient. In mathematics, the term ``practical'' might refer to a solution that can be computed in a reasonable amount of time or with available resources. The context in which a statement is made can significantly affect the material inferences we draw from it. This context-sensitivity is crucial for understanding how mathematical concepts are applied in real-world situations.

Finally, material inferences often support a \textit{bi-modal} \parencite[73-74]{Brandom:2019aa} reading. They can express both objective validity claims in the alethic modality of `possibility and natural necessity' and subjective or normative validity claims in the deontic modality of `obligation or practical necessity.' The difference between `practical' and `natural' necessity is encoded in the difference between ``you must brush your teeth now'' and ``you must have teeth in order to brush them.'' The former trades in authority, while the latter is a statement of fact. \textit{Incompatibility} (Brandom's term for determinate negation) is amphibious to either modality. It is impossible to brush teeth one does not have, so the state of tooth-brushing is incompatible with the state of not-having-teeth. This duality allows for a richer understanding of how inferences operate within different contexts.

This analysis of material inference reveals three key features that distinguish mathematical meaning from formal logical manipulation: defeasibility (inferences can be overridden by additional information), context-sensitivity (meaning depends on situational factors), and bi-modality (inferences operate across both factual and normative domains). These properties explain why mathematical concepts cannot be reduced to formal definitions but must be understood through their patterns of use in concrete contexts. Understanding these features prepares us to examine how syntactic structure constrains the substitutional patterns that enable reliable inferential projection.

\section{Syntactic Well-Formedness and Substitution}


Having established the dynamic, context-sensitive nature of material inference, the analysis now turns to the structural constraints that make reliable inferential projection possible. To understand the technical distinction between singular terms and predicates, I begin the analysis with the structure of sentences and the rules for substitution that preserve well-formed sentences. Consider these preliminary sentences about animals:
\begin{enumerate}
 \item Pokey is a dog. \label{pokey_is_dog}
 \item Felix is a cat.
 \item Buddy is a dog.
 \item Pokey and Buddy are distinct dogs.
 \item Slider the Spider only speaks gibberish.
\end{enumerate}

Brandom imagines the substitutional `machinery' can be opened up full-bore. Can ``the'' take the place of ``is''? ``Dog'' for ``a''? $\exists$ (my stepdaughter) recently requested that I tell her a story about Slider the Spider who only speaks gibberish. In the context of that request, the answer to those questions is ``sure!'' ``Is a Spider is a dog, is, is, a cat,'' Slider might say. But mathematics is mostly concerned with well-formed sentences (often called well-formed formulas, or simply WFFs \parencite{Gaifman2005}).

Filtering out `gibberish' is complicated, but an intuitive sense for how such a process might begin implicitly involves discerning syntactic categories by examining which substitutions result in sentences that are implicitly recognized as well-formed.

For instance, replacing ``Pokey'' with ``Felix'' in \ref{pokey_is_dog} (1) yields ``Felix is a dog,'' which, while incompatible with (2), is still a syntactically valid sentence. 
However, consider substitutions that break well-formedness:
\begin{table}[!htp]
 %\centering
 \begin{tabularx}{\textwidth}{lXXX}
Sentence (1): & Pokey && is a dog. \\
Non-preserving substitution: && $\nwarrow$\underline{is a cat}&$ $\\
Sentence (2): & Felix && \underline{is a cat}. \\ \hline
Non-sentence: & \fbox{is a cat} && is a dog.\\ \hline
\end{tabularx}%
\caption{\textit{Note. }Non-preserving substitution}
\label{tab:nonpreserving_sub}
\end{table}%
Here, substituting the predicate ``is a cat'' for the singular term ``Pokey'' results in a non-sentence: ``is a cat is a dog.'' This fails to preserve structure because it violates syntactic roles.
In contrast, consider substitutions that maintain syntactic structure:
\begin{table}[!htp]
%\centering
\begin{tabularx}{\textwidth}{lXXX}
Sentence (1): & \fbox{Pokey}&& \fbox{is a dog}. \\
Preserving substitution: & \underline{Felix }$\uparrow$&&\\
Sentence (2): & \underline{Felix} && is a cat.\\ \hline
New sentence (1*): & \fbox{Felix} && is a dog.\\
&&&\\
Sentence (1): & Pokey&& \fbox{is a dog}. \\
Preserving substitution: & &&\underline{is a cat }$\uparrow$\\
Sentence (2): & Felix && \underline{is a cat}.\\ \hline
New sentence (1**): & Pokey && \fbox{is a cat}.\\
\end{tabularx}%
\caption{\textit{Note. }Preserving substitutions}
\label{tab:preserving_sub}
\end{table}%

The original sentences are substituted-in, repeatedly, and the singular terms (\{Pokey, Felix, Buddy, Slider\}) are substituted-fors. % Upon reflecting on many such substitutions, we can discern \textit{sentence frames} (predicates), as a ``substitutional remainder.''

Upon reflecting on many such substitutions, \textit{sentence frames} (predicates) can be discerned as a ``substitutional remainder.'' Those sentence frames, like ``$\Box$ is a dog'' allow for a transition from simple assertions to algebraic formulas: $f_{\text{dog}}(X)$, where $X$ is a variable that can be replaced by any singular term. 

After discerning those syntactic categories, the subsentential expressions can be classified in terms of their inferential roles. The singular terms are those that can be substituted-for one another in symmetric ways, while the predicates are those that can be substituted-in but may not allow for symmetric substitution.


\section{Symmetric vs. Asymmetric Substitution: Singular Terms and Predicates}




Brandom's insight \parencite*[Chapter 4]{Brandom2000} is that subsentential expressions can be classified by the substitution patterns they participate in. In ordinary language, some substitutions work both ways (\textit{symmetric}), while others work only one way (\textit{asymmetric}). 

I don't know whether a term is a name or a predicate until I reflect on how it can be used. Upon reflection, the behavior of the term in different substitution contexts can be analyzed. This analysis involves positing patterns of use, but further reflection can generate new contexts that break those patterns. The reflection that kicks off analysis tends to reify the term into a lexical `object,' so that whether it was `originally' a name or a predicate becomes demonstrably uncertain: ``it'' -- the term as recollected -- can hold either role. 

That said, syntax plays the role of a recognized forestructure when self-certainty is explored through objective validity. Names do not tend to have argument places (\textit{adicities}), while predicates do. But to recognize a term as differentiated from others requires spacing. Derridean insights into the nature of text can worm their way into those spaces to happily munch their way through any fusion between the I-feeling and the algebraic approach undertaken in this chapter. I do not want to misrecognize self-certainty, which is one reason this text is so moth-eaten and ragged. Still, the pragmatist principle that \textit{meaning is use} holds -- understanding a speech act consists in knowing how to (appropriately) act next. Sometimes, those next actions are inferential substitutions. Sometimes those next actions are deconstructive.

A \textit{singular term} (like a name or definite description) is defined by its role in symmetric substitution. For example, if ``$\textit{Benjamin Franklin}$ was an ambassador to France,'' then ``$\textit{The inventor of bifocals}$ was an ambassador to France'' and vice versa, because the singular term \textit{Benjamin Franklin} and the definite description \textit{the inventor of bifocals} refer to the same person. % We do not need to know every possible appropriate (Fregean) sense for that referent in order for the symmetry to hold.
It is not necessary to know every possible appropriate (Fregean) sense for that referent in order for the symmetry to hold. Perhaps you did not know that Benjamin Franklin invented bifocals. Accruing elements of the symmetrically intersubstitutable (co-referential) terms is often accompanied by the body-feelings associated with ``letting go,'' described in the previous chapter. ``Aha!'' you might say, as that great mystery unfolds as the differentiation between ``Benjamin Franklin'' and ``the inventor of bifocals'' is relaxed. The experience of undifferentiating between those symmetric terms is one way to interpret the ``a-ha!'' moment that math teachers and students often experience when two seemingly quite different expressions are taken to be ``the same.'' % voice-edit: Original first-person reflections below

When I discovered through well-worn proof that $e^{i\pi}$ and $-1$ were symmetrically intersubstitutable, the incommensurable difference between the two terms relaxed. That does not mean that I must teach sixth graders about the expression $e^{i\pi}$ when introducing the concept of $-1$. What I'm trying to get at is the idea that samenesses are not static -- they are the relaxation of difference.

When that undifferentiation is rationally binding, as it often is through mathematical proof, the obligatory aspect of proof can transform into desire.


A \textit{predicate} typically serves as the frame in which singular terms are inserted, but predicates themselves can also be substituted. Crucially, predicates can stand in asymmetric substitution relations. For example, ``Pokey is a dog'' \textit{incompatibility entails} ($\models_I$) ``Pokey is a mammal'' but not the other way around \parencite{Brandom2008}. Everything that is incompatible with the assertion ``Pokey is a mammal,'' like ``Pokey is a spider,'' is also incompatible with ``Pokey is a dog.'' The predicate ``is a dog'' is \textit{inferferentially stronger} than the predicate ``is a mammal'' precisely because it is more exclusive. Incompatibility entailment is how Brandom defines the conditional (``if ... then ...'').

What happens when the material inference about Pokey is embedded in a conditional statement or a negation? Brandom's argument is that the logical operators \textit{invert} the polarity of the inferences, which is incompatible with the idea that singular terms can be used in asymmetric substitution. % If we strengthen the antecedent of a conditional, moving from ``If Pokey is a Dog then Pokey has four legs'' to ``If Pokey is a Retriever then Pokey has four legs'', the resulting conditional is easier to satisfy -- it is a weaker claim than the original conditional.
Strengthening the antecedent of a conditional, moving from ``If Pokey is a Dog then Pokey has four legs'' to ``If Pokey is a Retriever then Pokey has four legs'', makes the resulting conditional easier to satisfy -- it is a weaker claim than the original conditional. % Alternatively, if we weaken the antecedent, with a result like ``If Pokey is an animal then Pokey has four legs,'' the new conditional is harder to satisfy -- it is a stronger claim, in this case a false one, than the original conditional.
Alternatively, weakening the antecedent, with a result like ``If Pokey is an animal then Pokey has four legs,'' makes the new conditional harder to satisfy -- it is a stronger claim, in this case a false one, than the original conditional. The same line of reasoning works with negation: ``Pokey is not a dog'' is a weaker claim than ``Pokey is not an animal.'' More possibilities are ruled out by the latter, so it is a stronger claim. 

The details of Brandom's argument turn schematic at this point, as he has to contemplate a (fantasy) language that has those logical operators but allows substituted-for expressions to have asymmetric inferential significance. He notes that ``any language containing a conditional or negation thereby has the expressive resources to formulate, given any sentence frame, a sentence frame that behaves inferentially in a complementary fashion'' \parencite[146]{Brandom2000}. % We imagine a term $a$ that is stronger than $b$, so that we can generally project from $Q(a)$ to $Q(b)$ by substituting $b$ for $a$. We then discover that those projections become incoherent in polarity-inverting contexts like negation ($Q'$).
Suppose a term $a$ is stronger than $b$, so that, in general, the projection from $Q(a)$ to $Q(b)$ by substituting $b$ for $a$ is good. It then turns out that those projections become incoherent in polarity-inverting contexts like negation ($Q'$). The inference from $Q'(a)$ to $Q'(b)$ (e.g., Pokey is not a dog so Pokey is not an animal) is not a good one. If a good inference can be turned into a bad one with a supposedly good substitution, something has gone awry. % We either can either have a language with logical expressions and reliable projections of subsentential expressions where singular terms have symmetric inferential significance and predicates have asymmetric significance, or we can have an incoherent mess.
Either a language with logical expressions and reliable projections of subsentential expressions is maintained, where singular terms have symmetric inferential significance and predicates have asymmetric significance, or an incoherent mess results. In this way, Brandom's argument shows that singular terms must be used in symmetric substitution classes, while predicates can be used in asymmetric substitution classes. 

For now, I will treat inferential projection as an answer to a metaphysical question: How is it possible for language to be both productive (generating novel sentences from existing parts) and logical? That said, the conversation below about quadrilaterals is mostly \textit{curricular}. How inferential projection works is illuminated through the concept of inferential strength. In part three of the book, where I introduce the \textit{hermeneutic calculator}, I will return to projection to discuss how to make it \textit{methodologically} useful for researchers in math education. 

\section{Quadrilateral Classification as a Study in Substitution}

In the spring of 2025, I taught a class focused on problem-solving in elementary school classrooms. I asked the preservice teachers to design lessons that would be appropriate for my stepdaughters who were in kindergarten at the time. One group's lesson focused on classifying quadrilaterals. They asked $\mathcal{M}$ and $\exists$ to reach into a bag that held several quadrilaterals as well as circles and triangles. Their task was to classify the shapes only by touch. 

As the children reached into the bag of unknowns, their knowledge of what was in the bag was modally structured in relations of \textit{possibility} and \textit{necessity}. In grasping an object, it could have possibly been a square ($\diamond Square(x)$), a rectangle, a circle etc. When a feature like `rounded sides' was discerned, the modal structure of the object changed from being possibly a square to being necessarily not-square ($\Box \not Square(x)$). If they discerned four corners that were perpendicular, the shape became a possible rectangle or square. Discerning that the shape had equal sides would then change the structure of the object to being necessarily a square. While I have introduced the formal modal operators $\diamond$ (possibly) and $\Box$ (necessarily), I do so only for as a shorthand: the children were not using those symbols or explicitly using modal concepts -- their inferences were material, not formal. 

I mention this experiment to realize Brandom's reading of Hegel as ``building modality in at the metaphysical ground floor'' \parencite*[144-145]{Brandom:2019aa}. I assume that few readers take modal logic as a primordial discursive ability -- I did not study the topic until I was already deep into my dissertation, even having taken some symbolic logic courses and getting a bachelor's degree in mathematics. In the history of logic, modal logic is a relative latecomer, reaching a systematic formal description only in the middle of the 20th century. 

An open question is the extent to which this modal structure is primarily psychological. I claim their knowledge of the bag's contents was modal in nature, but were the contents of the bag themselves modal? On the objective side of this modal experiment, it might be helpful to consider the shapes in the bag as superpositioned, like Schr\"odinger's Cat who is both alive and dead.\footnote{In that thought experiment, a cat is placed in a box with a radioactive atom that has a 50\% chance of decaying within an hour. A Geiger counter is placed in the box along with a vial of poison. If the atom decays, it triggers the Geiger counter, which is tied to a release mechanism on the vial of poison, which then kills the cat. Until we open the box and observe the cat, it is in a superposition of being both alive and dead.} That said, it feels a bit outlandish to claim that quantum superpositioning is at work at the mezzo scale of plastic shapes in a bag. The subjective experience is closer to home, as I name it the \textit{phenomenology of confusion}. Commitment to a shape being possibly a square and possibly a circle is to possibly be confused. I might not be confused, and instead recognize the `superpositioned' states as ambiguous. The next act might then be to seek more information that could negate some possibilities. Information negates possibilities, and so can be disambiguating. That said, commitment to a shape necessarily being both a square and a circle is to necessarily be confused. 

The children's experience with the shapes was similar: until they reach into the bag, the shapes are unknown possibilia superpositioned on each other. As the children reach in, those possibilities collapse into determinate actualities. The field of possibilities is \textit{realized}. In this sense, possibility is more primordial than actuality, as the latter is a realization of the former. 


\subsection*{The Experience of Error}
In that teaching (and learning) experiment, the children made some mistakes. They would reach into the bag and pull out a shape that they thought was a square, but it turned out to be a rectangle. They would then have to correct their understanding of the shape, which involved a process of error correction and learning. This is a concrete example of how material inferences work in practice: the children made inferences based on their tactile experiences with the shapes, and those inferences were corrected through further interactions with the objects, for example, by pulling the shape out of the bag and looking at it. 

The experience of error is often discussed as it relates to empirical observations. While foregrounding the objective aspect of the experience of error invites some clarity, I have positioned subject-subject misrecognition as more primordial by introducing this book with Grover, who misrecognized himself. % voice-edit: Original sentence with first-person plural below

For the moment, I stay with the objective pole of error. Brandom \parencite*{Brandom:2019aa} has a lovely example that involves observing a stick that has been partially submerged in water. The stick appears to be bent due to the way light refracts through water, but then when it is removed from the water, it is recognized by the observing consciousness as straight. Whoops! 

There are three structural roles in the experience that are worth teasing out. First, for the experience to be taken as an error, the subject who perceives the bent stick must recognize that it is the same stick whether submerged or removed from the water and, furthermore, it had been a straight stick all along. (If the children dropped a shape back into the bag, any incompatibilities they discerned prior to dropping the shape would lose the informational incompatibilities they had discerned.) The stick, \textit{in-itself} is straight. The stick has some \textit{authority} that representations of the stick -- how the stick appears \textit{for-consciousness} in the moment of perception -- must acknowledge. Being a rational agent, in this reconstruction of the experience of error, involves submitting to the authority of a mutable reality. Merely perceiving the movement from stick-as-bent to stick-as-straight would not involve a change in commitment from ``the stick is bent'' to ``the stick is straight'' in the observing consciousness. Instead, the authority of the object in-itself is what underwrites the change in commitment, as both are normative aspects of reason. Last, what emerges is a ``new, true object'' -- the appearance of the bent-stick becomes, \textit{to-consciousness}, a stick-that-appears-bent-when-submerged-but-is-actually-straight. That is, the appearance of the new object -- what it is for-consciousness -- now has a learning experience compressed into it. The same processes were at work with the quadrilateral experiment. A child who thinks a rectangle is a square is like the person seeing the bent stick. The ``new, true object'' that emerges is the concept of a ``rectangle-that-I-initially-misrecognized-as-a-square,'' which is a richer, more robust concept.

With modality built in the bedrock of a philosophy of mathematics, a role for teaching and learning can be discerned. For, if learning begins in confusion -- essentially beginning with an \textit{incoherent} set of commitments -- then the process involves recognizing those incoherences through the experience of misrecognition and then disambiguating or otherwise repairing them. My kids were not learning about quadrilaterals in the sense of memorizing definitions or properties. Instead, they were learning to disambiguate their modal commitments about the shapes in the bag. They were learning to recognize that a shape could not be both a square and a circle. This process of disambiguation is what allows them to classify the shapes correctly.\footnote{The word `begin' is tricky here. I assume that some intersubjectively constituted commitments are already in place for `us' to be confused about, not backing up to a mystical state of non-intersubjective being and declaring such a state `confused,' and I am definitely not claiming a position in the blank-slate/genetic predisposition debate.} 

\subsection*{The Traditional Approach} 
The classification of quadrilaterals provides an opportunity to explore the nature of substitution inferences, particularly as they apply to \textit{predicates} that ascribe quadrilateral types (e.g., ``$X$ is a Square'', ``$X$ is a Rectangle''). The traditional hierarchy of quadrilaterals (Figure \ref{fig:traditional_quad_hierarchy_ch2}) is often presented as a tree structure reflecting entailments between these predicates. For example, the predicate ``$X$ is a Square'' entails the predicate ``$X$ is a Rectangle'', which in turn entails ``$X$ is a Parallelogram''.

\textbf{Figure}

\begin{figure}[h]
%\centering
\includegraphics[width=0.8\textwidth]{/Users/tio/Documents/GitHub/September_UMEDCA/images/Combined_Traditional_Circular.pdf}
\caption{\textit{Note. }Traditional and Circular Hierarchies of Quadrilaterals. The circular hierarchy (right) is \textit{im}proper until justified.}
\label{fig:traditional_quad_hierarchy_ch2}
\end{figure}

This hierarchical approach can be both clarifying and confusing. Young children typically begin learning about quadrilaterals using exclusive definitions, often disagreeing with statements like ``a square is a rectangle'' because visually, these shapes appear distinct \parencite{van_hiele}. \textit{Canonical} versions of the shapes, like equilateral triangles or non-square rectangles tend to be emphasized early in development. As their geometric understanding develops, they learn inclusive definitions, making inferences such as ``if $X$ is a square, then $X$ is also a rectangle'' (i.e., $\textit{Square}(X) \rightarrow \textit{Rectangle}(X)$), where $\textit{Square}(X)$ and $\textit{Rectangle}(X)$ are predicates.

When examining Table \ref{tab:quad_properties_ch2}, the properties are not mutually exclusive. A square possesses all listed properties, while a trapezoid has only one pair of parallel sides. Fallacious reasoning creeps in if inclusive definitions are grounded on shared properties. Declaring ``$X$ is a square so $X$ is a trapezoid because both have at least one pair of parallel sides'' is akin to declaring that `$X$ is a bird, so $X$ is an airplane because both have wings.'' These features can be helpful for classifying quadrilaterals, but without some modal concepts, they cannot be organized into a hierarchy.

\begin{table}[H]
%\centering
\caption{\textit{Note. }Characteristic Properties of Quadrilateral Families}
\label{tab:quad_properties_ch2}
\resizebox{\textwidth}{!}{%
\begin{tabular}{lcccccc}
\toprule
Quadrilateral Type & \shortstack{$A_1$: 1 Pair of \\ $\parallel$ \ Sides} & $A_2$: \shortstack{2 Pairs of \\ $\parallel$ \ Sides} & $A_3$: \shortstack{2 Pairs of \\ Adjacent Equal \ Sides} & $A_4$: \shortstack{4 Equal \\ Sides} & $A_5$: \shortstack{4 Right \\ Angles} & $A_6$: \shortstack{Diagonals are \\ perpendicular bisectors} \\
\midrule
General quadrilateral & $\times$ & $\times$ & $\times$ & $\times$ & $\times$ & $\times$ \\
Trapezoid & $\checkmark$ & $\times$ & $\times$ & $\times$ & $\times$ & $\times$ \\
Parallelogram & $\checkmark$ & $\checkmark$ & $\times$ & $\times$ & $\times$ & $\times$\\
Kite & $\times$ & $\times$ & $\checkmark$ & $\times$ & $\times$ & $\times$\\
Rectangle & $\checkmark$ & $\checkmark$ & $\times$ & $\times$ & $\checkmark$ & $\times$\\
Rhombus & $\checkmark$ & $\checkmark$ & $\checkmark$ & $\checkmark$ & $\times$ & $\checkmark$\\
Square & $\checkmark$ & $\checkmark$ & $\checkmark$ & $\checkmark$ & $\checkmark$ & $\checkmark$\\
\bottomrule
\end{tabular}%
}
\end{table}

If inclusive definitions are desired: 
\begin{itemize}
 \item \textbf{Necessity}: If $X$ is a Trapezoid, it \emph{must necessarily have} at least one pair of parallel sides. $\Box(Trapezoid(X) \rightarrow A_1(X))$.
 \item \textbf{Possibility}: A Quadrilateral \emph{can possibly have} one pair of parallel sides. $Quadrilateral(X) \rightarrow \Diamond A_1(X)$.
\end{itemize}


If exclusive definitions (canonical shapes) are desired, the last expression is negated: $Quadrilateral(X) \rightarrow \neg\Diamond A_1(X)$. Once that decision is made further modal operators and axioms \parencite[141-175]{Brandom2008} can be introduced to prove that a square is necessarily a rectangle, and that a rectangle is necessarily a parallelogram (under the inclusive definitions). 


In the prior paragraphs, I was using the term ``definition'' in the traditional sense. However, it occurs to me that \textit{the} definition of a square is the `set' (\textit{meaning field}) of inferences that can be made about squares. One reason it is challenging to read authors like myself or those who I cite most prominently is a resistence to glossaries - defining terms in pure Vocabulary-Vocabulary type definitions you might find in the dictionary or a glossary. Pragmatism understands meaning in terms of use. Consequently, one way to `define' terms is to explicate the `sets' of good and bad material substitution inferences the terms are caught up in, along with the contexts that determine the goodness or badness of those inferences. Those `sets' are not, for my project, ever fully explicated. There's always some possible speech act that could be added to the `set' or some reason why one of the elements of the `set' is defeated by the context in which it is claimed to be good or bad. However, the regularized patterns of inferences about geometric shapes, dogs, or whatever also make it unsound to claim that these `sets' are uncountably infinite. I have to reach for the defeasors for localized clusters of material inferences that follow regular patterns. Adding layers of reflection moves the researcher away from the original contexts, and so, as I press against the boundaries of the meaning field, the claimed possible interpretations of a speech act become more and more tenuous and the I-feeling becomes less pronounced. 


\subsection*{Shadows of shape}
The picture so far is that classifying a given shape as a particular instance of a quadrilateral involves understanding what it means to be a particular `thing' as a \textit{medium} in which compatible properties inhere. Doing so amounts to treating the thing as an \textit{also}, rather than as an exclusive \textit{one}. This is the eigth difference that Brandom describes as
\begin{itemize}
\item ``particulars as ``also''s -- that is, as a medium hosting a community of compatible universals -- and
\item particulars as ``exclusive ones'' -- that is as units of account repelling incompatible properties'' \parencite[162]{Brandom:2019aa}.
\end{itemize}-- 
To get at the other side of the object -- treating the thing as an exclusive \textit{one} -- involves discerning its boundaries. What information would preclude stating ``$X$ is a square''? A shape becomes determinate (e.g., as a square) by the set of restrictions it would refuse, like a vegan who will not eat butter. I argue that the negative articulation of the square, its shadow, is a necessary contrast to the positive articulation of the square. That contrast is probably already at work with learners prior to engaging with formal definitions of quadrilaterals, but I have not found anything that resembles my approach in the literature. 

A square object can be conceptualized towards a square \textit{subject} by considering what claims it repels. Following Hegel, Brandom argues that the concept of a particular \textit{object} emerges as a necessary structural feature for tracking such relations. The object is the \textit{unit of account} for incompatibilities and as such are ``of a different ontological category from the features for which they are units of account'' \parencite[150]{Brandom:2019aa}. This difference between universals and particulars is the seventh difference Brandom discerns. The incompatibility between \textit{square} and \textit{triangle} is not a global law of logic; it is the fact that one and the same particular thing cannot be both. The placeholder '$X$' in our predicates, therefore, is not merely a variable; it represents the emergence of the particular object as the locus of property instantiation and exclusion. I repel the claim ``Tio is a dog'' just as a square, $X$, repels the claim ``$R_1$: No sides of $X$ are equal.'' 
\begin{itemize}
\item $R_1$: No sides of $X$ are equal
\item $R_2$: No pair of adjacent sides of $X$ are equal
\item $R_3$: No pair of opposite sides of $X$ are equal
\item $R_4$: Non-parallel sides of $X$ are not congruent
\item $R_5$: No pair of opposite sides of $X$ are parallel
\item $R_6$: No angles of $X$ are right angles
\end{itemize}
The predicate $\textit{Square}(X)$ would entail the refusal of $R_1$ (as a square has four equal sides). A square is \textit{incompatible} with each of the restrictions $R_1, \ldots, R_6$. The collection of $R_i$'s I have listed are sufficient for distinguishing the quadrilaterals I listed in \ref{fig:traditional_quad_hierarchy_ch2}. Other claims are possible, like ``no diagonals of $X$ are perpendicular.'' Exhaustive lists of negatively articulated claims for quadrilaterals are impossible, as we can always say ``$X$ is not a dog'' or ``$X$ is not a triangle.'' Table \ref{tab:quad_incompatibility_matrix_ch2} summarizes the incompatibility relations between the quadrilateral categories and the restrictions $R_1, \ldots, R_6$. A 1 indicates that the shape rejects the restrictive claim. A 0 means it does not necessarily reject it. The inferential strength of each shape is the sum of the restrictions it rejects. The inferential strength of each restriction is the sum of the shapes that reject it.

\begin{table}[ht]
    \centering
    \resizebox{\textwidth}{!}{%
    \begin{tabular}{|>{\raggedright\arraybackslash}p{4cm}|c|c|c|c|c|c|c|c|}
    \hline
    \textbf{Property (Restriction)} & \textbf{Square} & \textbf{Rectangle} & \textbf{Rhombus} & \textbf{Parallelogram} & \textbf{Trapezoid} & \textbf{Kite} & \textbf{Quadrilateral}&\textbf{Strength of $R_i$} \\ \hline
    $R_1$: No sides of $x$ are equal & 1 & 1 & 1 & 1 & 0 & 1 & 0 & 5\\ \hline
    $R_2$: No pair of adjacent sides of $x$ are equal & 1 & 0 & 1 & 0 & 0 & 1 & 0 &3\\ \hline
    $R_3$: No pair of opposite sides of $x$ are equal & 1 & 1 & 1 & 1 & 0 & 0 & 0 & 4\\ \hline
    $R_4$: Non-parallel sides of $x$ are not congruent & 1 & 0 & 1 & 0 & 0 & 1 & 0 &3 \\ \hline
    $R_5$: No pair of opposite sides of $x$ are parallel & 1 & 1 & 1 & 1 & 1 & 0 & 0 & 5\\ \hline
    $R_6$: No angles of $x$ are right angles & 1 & 1 & 0 & 0 & 0 & 0 & 0 & 2\\ \hline
    \textbf{Strength of Shape} & \textbf{6} & \textbf{4} & \textbf{5} & \textbf{3} & \textbf{1} & \textbf{3} & \textbf{0} & \\ \hline
    \end{tabular}%
    }
    \caption{Incompatibility matrix and inferential strength for quadrilateral categories. A 1 indicates the shape rejects the restrictive claim. A 0 means it does not necessarily reject it. The inferential strength of each shape is the sum of the restrictions it rejects.}
    \label{tab:quad_incompatibility_matrix_ch2} 
\end{table}

At the heart of this analysis lies a crucial Hegelian and Brandomian distinction between two kinds of opposition. The first is Brandom's material incompatibility and Hegel's determinate negation. This refers to oppositions that arise from the non-logical content of concepts. For instance, an object cannot be both circular and triangular at the same time; the property `circular' materially excludes the property `triangular'. Brandom defines these as Aristotelian \textit{contraries}. Contraries are the fourth type of difference Brandom discerns. The fifth type of difference is formal contradictoriness, or abstract negation. This is the familiar logical opposition expressed by `not', such as the relationship between `red' and `not-red'. 

One of the delightful (and important for math education) features of Brandom's reconstruction of Hegel's chapter on perception is that Brandom defines formal contradictoriness in terms of material contrariety. He notes that `green' is a contrary of `red' and `not-red' is its contradictory. `Not-red' is the minimal contrary of red in that it is entailed by every contrary of red (green, blue, yellow, etc.). Hegel's key move, which Brandom develops, is to treat material incompatibility (contrariety) as the more fundamental concept, from which formal contradiction can be explained. The sixth type of difference is the metadifference between determinate and abstract negation, which is the distinction between material contrariety and formal contradictoriness. 

The picture of a square as both an exclusive one and an inclusive also (the eigth difference) is a bit like seeing a square with its shadow (see Figure \ref{fig:square_as_medium_and_unit_of_account_ch2}). The shadow is the set of restrictions that the square repels, while the square is equally the set of properties it possesses. The shadow is not a separate entity; it is a necessary aspect of the square's identity. When considering a square pulled out from the bag, other universals besides $A_1 \ldots A_6$ and $R_1 \ldots R_6$ might sit alongside those properties. For example, the shapes for my stepdaughters were plastic, so the square accepts the property of being printed on paper, while the shadow repeled the restriction of being printed on paper. The reverse would be so if this book is being read on paper.

\textbf{Figure}

\begin{figure}[h]
%\centering
\includegraphics[width=0.8\textwidth]{/Users/tio/Documents/GitHub/September_UMEDCA/images/difference_Particulars.pdf}
\caption{\textit{Note. }A particular square is the medium (the Also) in which the properties $A_1, \ldots, A_6$ inhere. The square is also a unit of account for incompatibilities (the One) that repel the restrictions $R_1, \ldots, R_6$.}
\label{fig:square_as_medium_and_unit_of_account_ch2}
\end{figure}

Interestingly, the property of being square exhibits the same duality -- it can be seen as forming a family of co-compatible universals like being rectangular and it can be seen as repelling universals that it cannot co-instantiate with. The first of the ten differences Brandom discerns is the ``mere or `indifferent' [gleichg\"ultig] difference of compatible universals'' while the second difference is the ``exclusive difference of incompatible universals''\parencite*[161]{Brandom:2019aa}. The third difference is the ``metadifference between mere and exclusive difference'' \parencite[161]{Brandom:2019aa}. This third difference is a species of exclusive difference, ``because the universals must be either compatible or incompatible'' \parencite[161-162]{Brandom:2019aa}. 

Figure \ref{fig:differences_Universals_ch2} illustrates this duality for universals/predicates/sentence frames. The universal of being square forms a family with other universals like being rectangular, which are associated with particular rectangles that have properties having four right angles. Likewise, the universal of being square excludes other incompatible universals like being triangular. This is the ninth type of difference -- the ``difference between two roles universal play with respect to particulars'' \parencite[162]{Brandom:2019aa}. In the figure, I try to capture the `with respect to particulars' by clarifying that the universal ``triangular'' is discerned from a particular triangle (same with circular). Category errors arise as we consider other uses of the term `square' that do not involve particular geometric shapes. For example, the particular $-1$, which is not a `square number' like 25 ($5^2$), or the slang term `square' for a person who is not cool (``let's not be L7, come on learn to dance'') divorce ``square'' from the family of co-compatible universals related to geometric shapes by drawing from particulars outside that family. 

\textbf{Figure}

\begin{figure}[h]
%\centering
\includegraphics[width=0.8\textwidth]{/Users/tio/Documents/GitHub/September_UMEDCA/images/difference_Universals.pdf}
\caption{\textit{Note. }Universals (predicates/sentence frames) exhibit the same duality as particulars (objects).}
\label{fig:differences_Universals_ch2}
\end{figure}

It is now possible to summarize nine of the ten types of difference \parencite*[161-162]{Brandom:2019aa}. Figures \ref{fig:square_as_medium_and_unit_of_account_ch2} and \ref{fig:differences_Universals_ch2} illustrate two intracategorial differences within the categories of particulars and universals, respectively. From these nine types of difference, Brandom reaches an interesting -- if somewhat well-trodden -- conclusion: universals have opposites, but particulars do not. This is a ``huge structural difference'' \parencite[155]{Brandom:2019aa} in how objects and properties differ from and exclude one another. Universals have contradictories (like ``not-red'') but objects do not, because the ``opposite'' of an object would need to exhibit mutually incompatible properties, which is impossible. What is the opposite of a red, plastic, square object? The opposite would have to be not-red, not-plastic, and not-square, but not-red encompasses blue, green etc. and not-plastic contains metal or wood. The opposite of such an object would have to exhibit all of those incompatible universals. 

The list below is adapted with very minor revisions from \parencite[161-162]{Brandom:2019aa}.

\begin{enumerate}
\item Mere or ``indifferent'' difference of compatible universals: This refers to properties that can coexist, like ``day'' and ``raining''.

\item Exclusive difference of incompatible universals: This refers to properties that cannot coexist, like ``day'' and ``night,'' or ``red'' and ``green''. This is a ``modally robust exclusion''.

\item Metadifference between mere and exclusive difference: This is the distinction between compatible and incompatible differences. It's an intracategorial metadifference concerning differences relating universals to universals. Universals must be either compatible or incompatible.

\item Material contrariety: This corresponds to determinate negation, where features are materially incompatible due to their nonlogical content. Examples include ``red'' and ``green,'' or ``circular'' and ``triangular''.



\item Formal contradictoriness: This corresponds to abstract logical negation, where features are logically incompatible due to their abstract logical form. Examples include ``red'' and ``not-red,'' or ``circular'' and ``not-circular''.

\item Metadifference between determinate and abstract negation (logical negation): This is the distinction between material contrariety and formal contradictoriness. It's the second intracategorial metadifference between differences relating universals to universals, which are compatibly different. Contradictories are a type of contrary (minimal contraries).

\item Difference between universals and particulars: This is the first intercategorial difference, an exclusive difference between properties and objects.

\item Difference between two roles particulars, or objects, play: 
\begin{itemize}
\item Particulars as ``also's'': Functioning as a medium hosting a community of compatible universals.
\item Particulars as ``exclusive ones'': Functioning as units of account repelling incompatible properties. This is the first intracategorial difference between roles played by particulars, and every particular must play both.
\end{itemize}
\item Difference between two roles universals play with respect to particulars:
\begin{itemize}
\item Universals as related to an inclusive ``One'' in community with other compatible universals.
\item Universals as excluding incompatible universals associated with different exclusive ``One's''.
\end{itemize}
\item Difference between universals and particulars that consists in the fact that universals do and particulars do not have contradictories or opposites: This highlights a ``huge structural difference'' in how objects and properties differ from and exclude one another. Universals have contradictories (like ``not-red''), but objects do not, because the ``opposite'' of an object would need to exhibit mutually incompatible properties, which is impossible.
\end{enumerate}

\subsection*{Difference to-consciousness}

Each of these differences is potentially a difference \textit{to-consciousness}. Each moment where a difference is discerned may be accompanied by apperceptive self-awareness. The I-feeling mode can fuse with these moments of conceptual movement, giving them a palpable quality that we can recollect fondly and desire to revisit. As odd as I may be, that is the driving desire that keeps me teaching and writing about mathematics. It feels good. The inferential chains that flow from stronger to weaker predicates -- the algorithms -- can be experienced with the smoothness of a yogic flow, where each step in a proof is a pose that punctuates the movement. There is a deep pleasure, a fusion of the I-feeling with the concept, that can occur when a difference is discerned to be null  --  when, for instance, the experience of misrecognition resolves into recognition, and the object for-consciousness aligns with the object in-itself.

This experience reaches its peak in the playful but profound dizziness of a logical fixed-point, as in Gaifman's self-referential sentence. Such moments, where a concept or sentence takes itself as its own object, reveal the curious and delightful \textit{iterability} of thought that Derrida's work explores. They are the intellectual equivalent of dwelling in the \textit{in}finite regress between two mirrors. While delightful, these infinities do not capture the Hegelian \textit{in}finite, which requires his concept of self-consciousness to recognize. There is much more to say to break this narcisstic mirror. The fractal-like Venn diagram in Figure \ref{fig:venn_diagram_quads} illustrates the reciprocal definitions between entailments and exclusion. The observation that each shape is, in a sense, both `inside' and `outside' the others dissolves the idea of a simple, linear hierarchy. A square `contains' the general quadrilateral in that it possesses all its properties, yet the general quadrilateral `contains' the square as a specific possibility within its broader conceptual space. Navigating this complex, self-similar landscape is precisely the work of logical operators like negation. % voice-edit: Original first-person plural sentence below

By `relaxing the square'  --  softening the hard negations that define it  -- - the concept is not destroyed; rather, its inferential pathways to neighboring shapes, like the rhombus, are traced. Still, the journey of this chapter from a simple hierarchy to a self-similar, fractal network, approaches conceptual self-awareness. % voice-edit: Original first-person plural sentence below


This is the ultimate goal of sublation: to arrive at a totality that contains its own developmental history, a goal approached as the chapter now turns to a final summary of its findings. But whereas Hegel approached a totality of totalities, I approach a smaller totality, the totality of quadrilaterals. The image from \textit{Breath and Kindling}, ``strung street lights; small sodium moons; whiteknuckled halos cut holes in the gloom,'' captures my ambition here. Rather than clinging to that totality, so easily broken by including an isoceles trapezoid or whatever other properties might turn to relevance as soon as the ink has dried, I encourage readers to let go. The next streetlight beckons. 


\section{Bridging chapter 1 and 2: The Phenomenology of Classification}

In this section, I first offer two explanatory images and then a logical reconstruction of the phenomenology they represent. In figure \ref{fig:relaxing_the_square}, the process of `relaxing the square' is visually represented, showing how the rigid boundaries of the square can be softened to reveal connections to other quadrilateral forms. 

\begin{figure}[h]
	\centering
	\includegraphics[width=\textwidth]{/Users/tio/Documents/GitHub/September_UMEDCA/images/Relaxing_the_Square.pdf}
	\caption{Relaxing the square's restrictive claims deforms it into neighboring quadrilateral predicates, illustrating polarity-inverting contexts that lead toward an inclusive understanding of the general quadrilateral.}
	\label{fig:relaxing_the_square}
\end{figure}

In figure \ref{fig:venn_diagram_quads}, the fractal-like nature of the quadrilateral classification is depicted, illustrating how each shape both contains and is contained by others in a complex web of relationships.

\begin{figure}[h]
	\centering
	\includegraphics[width=\textwidth]{/Users/tio/Documents/GitHub/September_UMEDCA/images/venn_diagram_quads.pdf}
	\caption{The ``Venn Diagram'' for quadrilaterals, extrapolated from Figure~\ref{fig:relaxing_the_square}, is fractal-like. The universal \textit{quadrilaterals} contains many differences. Each shape is both inside and outside the others under polarity-inverting contexts.}
	\label{fig:venn_diagram_quads}
\end{figure}


\subsection*{The Structure of the Hierarchy}
The logic developed in chapter 1 can be extended to understand the two images above, hopefully negating any sense of mysticism that might arise from seeing quadrilaterals represented as fractals. 

\textbf{Moving Down (Specification):} Quadrilateral $\rightarrow$ Parallelogram $\rightarrow$ Rectangle $\rightarrow$ Square. This is a movement of increasing Compression ($\downarrow$). The conceptual hierarchy strengthens by adding constraints (commitments). ``Square'' is the most compressed concept.

\textbf{Moving Up (Generalization):} Square $\rightarrow$ Rectangle $\rightarrow$ Parallelogram. This is a movement of increasing Expansion ($\uparrow$). The conceptual hierarchy weakens by ``Letting Go'' of constraints.

\subsubsection*{The Dynamics of Inference}

Consider the standard inference: ``If it is a Square ($S$), then it is a Rectangle ($R$)'' ($S \Rightarrow R$).

\textbf{Embodied Dynamic:} The analysis starts with the highly compressed concept $S$ ($\downarrow\downarrow$). To move to $R$, the constraint of ``equal sides'' is released. This inference is experienced as an expansive movement ($\uparrow$).

Now consider the contrapositive (Modus Tollens), which relies on polarity inversion: ``If it is NOT a Rectangle ($\neg R$), then it is NOT a Square ($\neg S$)'' ($\neg R \Rightarrow \neg S$).

\textbf{Embodied Dynamic:}

The analysis starts with $\neg R$. Since $R$ is relatively expansive, $\neg R$ introduces a compression ($\downarrow$) -- this closes off the space of rectangles.

The conclusion involves $\neg S$. Since $S$ is compressive, $\neg S$ is expansive ($\uparrow$).

The inference moves from the compression of $\neg R$ to the expansion of $\neg S$. The compression required to exclude the broader category (Rectangle) necessarily forces the exclusion of the narrower category (Square) contained within it.



\section{The Case for the Anaphoric Unit}\label{the-case-for-the-anaphoric-unit}

The central claim of this section, which I will articulate more fully in chapter 7, is that the mathematical ``unit'' (the ``One'')
functions linguistically not as a singular term (like a proper name),
but as an anaphoric term (like a pronoun). The justification for this
rests on the logic of substitution and the evidence provided by the
failure of hybridized models.

\paragraph{1. The Requirement of Singular Terms}\label{the-requirement-of-singular-terms}

The standard interpretation treats mathematical units as
\textbf{singular terms} -- expressions that refer to a specific object.
Singular terms are logically defined by their participation in
\textbf{symmetric substitution inferences}.

If two expressions function as singular terms referring to the same
object (e.g., ``Benjamin Franklin'' and ``The inventor of bifocals''),
they form an equivalence class. They must be substitutable for one
another in any sentence frame without altering the validity of the
assertion. The predicate remains stable under substitution. If I must change the set of predicates that apply to a singular term when the singular term is substituted symmetrically, then it isn't a symmetric substitution. 

If the unit ``One'' is a singular term, then any representation of that
unit -- such as a Circle (\(U_C\)) or a Rectangle (\(U_R\)) -- must also
be symmetrically intersubstitutable. 

\paragraph{2. The Test Case: Spatial Modeling}\label{the-test-case-spatial-modeling}

I test this requirement in the context of modeling fractions. This
involves applying a predicate (a partitioning strategy) to a term (the
shape realizing the unit). The normative constraint is equipartitioning
(creating equal, interchangeable parts).

\begin{itemize}
\item
  A Rectangle Unit is appropriately partitioned using a Vertical
  strategy (\(P_V\)). Speech Act: \(P_V(U_R)\). (Good).
\item
  A Circle Unit is appropriately partitioned using a Radial strategy
  (\(P_R\)). Speech Act: \(P_R(U_C)\). (Good).
\end{itemize}

\paragraph{3. The Failure of Substitution}\label{the-failure-of-substitution}

If \(U_R\) and \(U_C\) were symmetrically intersubstitutable, we should
be able to substitute one for the other while keeping the predicate
stable.

Let's attempt to substitute \(U_C\) for \(U_R\) in the valid model
\(P_V(U_R)\):

\begin{itemize}
\item
  \textbf{Original:} \(P_V(U_R)\) (A Rectangle partitioned Vertically)
  \(\rightarrow\) Valid Model.
\item
  \textbf{Substitution:} \(P_V(U_C)\) (A Circle partitioned Vertically)
  \(\rightarrow\) Invalid Model.
\end{itemize}

The result, \(P_V(U_C)\), is a hybridized model. It is
invalid because vertical partitioning a circle generally fails the normative requirement of equipartitioning.

The inference generated by this substitution -- (\(P_V(U_R)\) is Good) SO
(\(P_V(U_C)\) is Good) -- is a \textbf{Bad Inference}.

\paragraph{4. The Diagnosis: Material Dependency}\label{the-diagnosis-material-dependency}

The failure of the substitution demonstrates that \(U_R\) and \(U_C\)
are \textbf{not} symmetrically intersubstitutable. The predicate is not
stable because the appropriate actualization of ``partitioning'' is
materially \textbf{dependent} on the specific geometry of the term it
modifies.

\begin{quote}
``The contents of this predicate depend on the contents of the term
substituted in. If this were not so, then the partitioning predicate for
circular models\ldots{} would apply to the underlying substitution of
`singular terms'\ldots{} Instead, such a substitution yields the
hybridized model'' \parencite[235]{savich2022}.
\end{quote}

Because the terms fail the fundamental requirement of symmetric
substitutability, the singular term interpretation is untenable.

\paragraph{5. The Anaphoric Resolution}\label{the-anaphoric-resolution}

The \textbf{anaphoric interpretation} resolves this conflict and
explains the dependency structure. Anaphora (like pronouns) function by
referring back to an antecedent. Their key feature is that they possess
\textbf{syntactic sameness without semantic sameness} \parencite[235]{savich2022}.

\begin{enumerate}
\def\labelenumi{\arabic{enumi}.}
\item
  \textbf{Syntactic Sameness:} Both \(U_R\) and \(U_C\) can fulfill the
  syntactic role of ``the unit'' in a fraction model.
\item
  \textbf{Semantic Difference:} Their underlying geometric properties
  (their semantic content) are different. These differences impose
  distinct material constraints on the practices (predicates) that can
  validly apply to them.
\end{enumerate}

Like the pronoun ``He'' -- where the appropriateness of the predicate
``is tall'' depends entirely on the specific referent -- the
appropriateness of the predicate ``is partitioned Vertically'' depends
entirely on the specific actualization of the unit.

The hybridized model error occurs precisely when a student recognizes
the syntactic sameness but ignores the semantic difference. They attempt
to project a predicate (\(P_V\)) across a substitution
(\(U_R \rightarrow U_C\)) that cannot support it, breaking the anaphoric
link that gave the original predicate its validity.

The instability of predicates under the
substitution of different representations of the unit fundamentally
contradicts the singular term interpretation. The anaphoric
interpretation is necessary to account for the material dependency
between mathematical representations and the practices appropriate to
them.

Frege would not accept the argument above as defeating his project. He would likely declare that I am making a category error by confusing the representation of the unit with the unit itself. He would probably say that what I have proved is that circles aren't rectangles. The Unit is the referent (Venus), while the circle and rectangle are different senses (the morning star and the evening star). The fact that circles and rectangles are not symmetrically intersubstitutable does not threaten the underlying sameness in referent.

To respond to that argument, I say ``Yes!'' The act of thought is dynamic and context-specific (the specific material actualization of the unit), but the recollection of that act via the anaphoric term is stable (the formal unit). Anaphora is the second negation, the \cancel{no}. It is \textit{how} terms come to be fixed. 

He would likely also think that I am committing myself to a psychologism, where the validity of a mathematical expression depends on the particular psychology of the person who claims it. The charge of psychologism would be a bitter pill to swallow. I find psychological explanations of the mathematical to be trapped in a vicious circle. Brandom's interpretation of Hegel as introducing a non-psychological conceptualization of the conceptual is a way past that charge. The Square is the collection of positive and negative inferential proprieties listed above. There need be no reliance on the particular psychology of the person who makes claims about squares. While the specific learning experiences of the claimant are temporally compressed into their specific uses of those shapes, the shapes themselves are defined as the unbounded set of material inferences one might make about them. The Square is not a psychological entity; it is a conceptual entity defined by its inferential role. Furthermore, the \{I think\} is not a psychological entity either. It is a transcendental condition of possibility for \textit{any} representation, according to Kant. I will discuss transcendental categories more in the next chapter. 

Until then, note that pronouns follow predictable rules \textit{once they are entered into a context}. For example, ``Tio is a math educator, he is kind of a geek, but his kids are pretty cool.'' He $\rightarrow$ his is a rule. That rule is (by my reckoning) appropriately challenged in folks who assert gender fluidity, but to challenge a norm is to claim a new norm. The rulishness of mathematics is actualized by virtue of how anaphora fixes the semantic contents of varying expressions. We need not assert a platonic heaven or a realm of truth separable from language: we need merely recollect language to find mathematical stability. 



\section{Conclusion}

In this chapter, the discussion has moved from static, representational views of knowledge toward a dynamic, inferential paradigm. Following the Hegelian insight that determinate content requires a field of material incompatibilities, the classification of quadrilaterals served as a concrete case study. This example allowed the exposition to show how the structure of objects and properties emerges from practice -- from the inferences that are endorsed and the claims that are ruled out. The object ($X$) was revealed in its dual role: as the `Also' that unifies compatible properties (a square is also a rhombus), and as the `One' that serves as the unit of account for excluding incompatible ones (a square cannot be a triangle).

The analysis demonstrated how formal logical vocabulary, like negation and the conditional, serves a crucial expressive purpose: it makes explicit the material-inferential proprieties that are already implicit in the use of non-logical concepts. The circular hierarchy of quadrilaterals, which appears under polarity inversion, is not a quirk of formalism. It is a revelation of the holistic, reciprocally-defined nature of the concepts themselves, where each concept's identity is constituted by its relations of difference to all others. By quantifying `inferential strength' and tracking its inversion, the logical is in the service of the material.

Ultimately, by treating knowledge as inferential movement, this highlights the importance of reasoning as a process. Meaning is not a static label attached to a thing but is the dynamic role an expression plays in the game of giving and asking for reasons. The content of `square' is the web of what follows from it and what is incompatible with it. This inferentialist perspective, grounded in the interplay of compatibility and incompatibility, provides the resources to understand not only the classification of shapes, but also serves as a general story for how concepts, objects, and subjects come to be. 

In the next chapter, the discussion turns more explicitly to this interplay, deepening our understanding of how the inferential movement of concepts is always also a movement of self and other, of recognition and transformation.

\printbibliography[heading=subbibliography]
